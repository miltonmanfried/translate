\begin{paper}{1}
\begin{header}
\title{Some remarks on the Dirac theory of the relativistic spinning electron.}
\author{John von Neumann}
\location{Berlin}
\note{Received on March 15th, 1928.}
\makeheader
\end{header}


\begin{abstract}
Some properties of the Dirac spinning electron are analyzed in detail, as well as the nature of monochromatic de Broglie waves, the transformation properties of the four $\psi$-components, and the energy-current vector (whose time component is the probability\footnote{Added by the editor. In a meanwhile-published paper (Proc. Roy. Soc. 118, 351, 1928), Herr Dirac has likewise put forward the aforementioned divergence-free current vector. Since our method is different from his, and at the same time gives a closer analysis of the relativistic transformation properties of the $\psi$, the following explanations may not be without interest.}).
\end{abstract}

\part*{Introduction}
\section*{I.} In a recently-published paper\footnote{Proc. Roy. Soc. 117, 610, 1928.}, Herr P.A.M. Dirac has proposed a new manner of treating the quantum mechanical one-body problem, in which certain deficiencies of the present relativistic one-body equation\footnote{Namely the wave equation put forward by Fock, Gordon, Klein, Kudar and Schr\"odinger
\uequ{
\left.\sum\limits_{k=1}^{4}\left(
\frac{h}{2\pi i}\pX{}\pY{x_k} - \frac{e}{c}\phi_k\right)^2 + m^2c^2\right\}\psi = 0,
}
where $x1,x_2,x3$ are the three spatial coordinates and $x_4=ict$, $phi_1,\phi_2,\phi_3,\phi_4$ is however the electromagnetic four-potential ($\phi_1,\phi_2,\phi_3$ real, $\phi_4$ pure imaginary). Gordon treated the Compton effect with this, ZS. f. Phys. 40, 117, 1926; there it is also discussed in more detail} are resolved in a simple and satisfactory way. It not only enables the resolution of the aforementioned (relativistic) deficiencies, but it even gives -- as he showed in l.c. -- \?{entirely on its own}{ganz von selbst} the well-known spin properties of the electron: namely its mechanical \?{angular momentum}{Drehmoment} $\frac{h}{4\pi}$ as well as its magnetic moments $\frac{he}{8\pi mc}$, $\frac{he}{4\pi mc}$ in the "internal" resp. "external" fields\footnote{Cf Dirac, l.c., resp. p.619, 620, 624.}.

This overwhelming success of the Dirac theory leaves hardly any doubt \?{the relations that connect it to the essential features of the relativistic behavior of "point masses"}{der Beziehung, da\ss mit ihr wesentliche Merkmale des relativistischen Verhaltens von "Massenpunkts" erfa{\ss}t sind}\footnote{So e.g. that the spin of each material object,even the proton, must arise on relativistic grounds.}(though certain difficulties   u   sharply highlighted by Dirac, and soon to be named -- still persist), and it is therefore appropriate to look into its consequences in detail. Since the present "transformation theory" of quantum mechanics 

\end{paper}