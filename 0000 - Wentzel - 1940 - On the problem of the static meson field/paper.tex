\begin{paper}{1}
\begin{header}
\title{On the the problem of the static meson field}
\author{Gregor Wentzel}
\location{Z\"urich}
\note{Received June 21st 1940.}
\makeheader
\end{header}


\begin{abstract}
\textit{Contents}: The Yukawa theory of the interaction of mesons and nuclear particles is discussed for the case of \textit{strong coupling}, and as an example, that of the scalar charged meson field. A procedure will be specified which determines the eigenvalues and eigenfunctions of the static problem in the form of an expansion in \textit{falling} powers of the coupling parameter $g$. A brief discussion of the results with an eye to current problems in meson theory is g]found in \S11.
\end{abstract}

\section{Problem statement.}

The subject of this work is a meson field in interaction with resting protons and neutrons. The problem of the static meson field is of course much more complicated than the corresponding electromagnetic problem: the exact splitting of the Maxwellian field into static and non-static parts cannot be taken over to the meson field, and indeed this is due, as Stueckelberg first showed, to the non-commutability of  the spin and isospin matrices occurring in the interaction operator. This non-commutability must, however, since the exchange character of the Yukawa force arises from it, be seen as an absolutely essential piece of the Yukawa theory. An especially characteristic distinction between meson and light theory is of course the fact that the meson, in contrast to the light quanta, can be scattered on resting (infinitely heavy) particles.

Besides the perturbation method, which only supplies poorly-converging expansions, up to now the only mathematical method for investigating the "mesostatic" field is the method of canonical transformation borrowed from quantum electrodynamics\footnote{}As mathematical method

\end{paper}