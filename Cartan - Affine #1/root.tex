\documentclass{report}
\renewcommand*\rmdefault{ppl}
\usepackage[utf8]{inputenc}
\usepackage{amsmath}
\usepackage{graphicx}
\usepackage{enumitem}
\usepackage{amssymb}
\usepackage{marginnote}
\usepackage{cancel}

\newcommand{\nf}[2]{
\newcommand{#1}[1]{#2}
}
\newcommand{\nff}[2]{
\newcommand{#1}[2]{#2}
}
\newcommand{\rf}[2]{
\renewcommand{#1}[1]{#2}
}
\newcommand{\rff}[2]{
\renewcommand{#1}[2]{#2}
}

\newcommand{\nc}[2]{
  \newcommand{#1}{#2}
}
\newcommand{\rc}[2]{
  \renewcommand{#1}{#2}
}

\nff{\WTF}{#1 (\textit{#2})}

\nf{\translator}{\footnote{\textbf{Translator note:}#1}}

\newcommand{\nequ}[2]{
\begin{equation*}
\begin{split}
#2
\end{split}
\tag{#1}
\end{equation*}
}

\newcommand{\uequ}[1]{
\begin{equation*}
\begin{split}
#1
\end{split}
\end{equation*}
}

\nff{\iffy}{#2}
\nff{\?}{#1}
\rff{\!}{#1}

\newcommand{\sumXY}[2]{\underset{#1}{\overset{#2}{\sum}}}
\newcommand{\sumX}[1]{\underset{#1}{\sum}}
\newcommand{\intXY}[2]{\int_{#1}^{#2}}
\newcommand{\intX}[1]{\underset{#1}{\int}}

\nc{\fluc}{\overline{\delta_s^2}}

\rf{\exp}{e^{#1}}
\nf{\comment}{}

\nc{\grad}{\operatorfont{grad}}
\rc{\div}{\operatorfont{div}}

\nf{\pddt}{\frac{\partial{#1}}{\partial t}}
\nf{\ddt}{\frac{d{#1}}{dt}}

\nf{\inv}{\frac{1}{#1}}
\nf{\Nth}{{#1}^\text{th}}
\nff{\pddX}{\frac{\partial{#1}}{\partial{#2}}}
\nc{\lap}{\Delta}

\rc{\r}{\mathfrak{r}}
\nc{\R}{\mathfrak{R}}
%\nc{\OMEGA}{\mathfrak{\Omega}}
\nc{\OMEGA}{\Omega}
\nc{\f}{\mathfrak{f}}

\nc{\Y}{\psi}
\nc{\y}{\varphi}
\nc{\m}{\mathbf{m}}
\nc{\e}{\mathbf{e}}
\nc{\F}{\mathbf{F}}
\nc{\G}{\mathbf{G}}
\nc{\w}{\omega}
\nc{\wbar}{\overline{\omega}}

% hyperforces
\nc{\hR}{\mathbf{R}}
\nc{\hX}{\mathbf{X}}
\nc{\hY}{\mathbf{Y}}
\nc{\hZ}{\mathbf{Z}}

\nc{\xG}{\mathbf{G}}
\nc{\xPi}{\mathbf{\Pi}}

\nc{\origin}{\mathbb{O}}

\nc{\Tbar}{\overline{T}}


\title{On manifolds with an affine connection and the general theory of relativity (first part)}

\begin{document}

The majority of the sections of this memoir, which shall appear in two parts, is devoted to the purely geometric theory of what I call \textit{manifolds with an affine connection} and which includes, as special cases, manifolds with a \textit{metric} connection and manifolds with a \textit{Euclidian} connection. The term "affine connection" is borrowed from H. Weyl\footnote{In his fine book, \textit{Time, space, metric}, translated from the fourth German edition by Gustave Juvet and Robert Leroy; Paris, A. Blanchard, 1922.}, but it takes on an enlarged significance in this memoir. As I indicated in broad strokes in the notes in \textit{Comptes Rendus de l'Academie des Sciences}\footnote{\textit{C.R.}, v. 174, p. 437, 593, 734, 857, 1104.}, a manifold with an affine connection is a manifold which, in the \textit{immediate} vicinity of each point, has all of the characteristics of an affine space, and for which there is a law for \WTF{identifying}{repérage} the areas surrounding two \textit{infinitesimally-near points}: that is to say that if, at each point, one is given a cartesian coordinate system having that point as its origin, one knows the transformation formulae (of the same natute as in the affine space) which permit one to pass from one system of reference to any other system of reference with an infinitesimally-close origin. In H. Weyl's theory, this pointwise identification is subject \textit{a priori} to a certain restriction, which is not a logical necessity, and which consists in the existence, in the neighborhood of each point, of what he calls a \textit{geodesic} coordinate system. I had indicated in the notes mentioned above that the difference which exists between a manifold with an affine connection and an affine space proper is manifested by an affine displacement associated with any infinitely-small closed contour: this displacement can be decomposed into a translation and a rotation: the translation \WTF{corresponds to}{traduit} the \textit{torsion}, the rotation corresponds to the \textit{curvature} of the manifold. In Weyl's theory, the \textit{torsion} is always zero. All of these notions extend to manifolds with metric or Euclidian connections; the classical theory of spaces defined by a $ds^2$ (Riemann spaces) is basically that of manifolds with Euclidian connections without torsion\footnote{M.D.J. Struik has published an \WTF{overview}{exposé d'ensemble} on the theory of Riemann spaces entitled \textit{"Grundz\"uge der mehrdimensionalen Differentialgeometrie"} (Berlin, Julius Springer, 1922), which contains a quite complete bibliography.}: it is these, as is known, upon which Einstein's general theory of relativity is built\comment{Google says "beyond the theory"}.

The fundamental properties of manifolds with affine connections are studied in chapter II, those of manifolds with metric or Euclidian connections in chapter III. Some summary remarks on the theory of curves and surfaces, in particular on straight lines in manifolds with Euclidian connections, are given in chapter IV.

Apart from these three chapters on pure geometry, the first part of the present memoir contains two chapters of applications to the Newtonian and Einsteinian theories of gravity. Basically, I have taken up the idea, initially due to Einstein, according to which, for an \WTF{observer}{observateur entraîné} in a gravitational field carrying with him a reference system, moving with translatory motion, the laws of physics would be true in his vicinity as if his reference system was immobile and as if there was no gravitational field: his proper motion would satisfy the principle of inertia. This amounts for the observer in regarding as \textit{\WTF{equivalent}{équipollents}} reference systems which are not so in the classical sense, that is to say, as looking at the space-time as a manifold with an affine connection. \textit{Maintaining all of the notions of classical mechanics}, one thus has a reduction to geometry of Newtonian gravity, a reduction which is of \textit{exactly the same nature} as that furnished by Einstein's theory and which shows, like there, the interdependence of the affine connection of the universe and the distribution of matter in space. In all rigor one has, by way of explanation, a simple convention of language; but that just shows the importance in the evolution of science of judiciously-chosen conventions of language.

Newtonian gravitation being thus \textit{explained}, the passage from Newton's theory to that of Einstein is almost intuitive when the laws of Newtonian gravitation are given in their invariant form, that is to say independent of the variable system of reference adopted at different points of the universe: this invariant form only involves the curvature of space-time, and the invariant form of the Einsteinian laws of gravitation is almost a simple transposition, where one passes from the notions of space and time utilized in classical mechanics to the notions of space and time utilized in special relativity.

Before arriving at the exposition of this parallelism, which is given in the last chapter of this first part, a prejudicially-important question must be posed: are physical phenomena compatible with \textit{multiple distinct definitions} of the affine connection of space-time? This question itself presupposes another: how must one formulate physical laws when adopting this or that affine connection for space-time? Physical laws are generally translated into partial differential equations which basically involve numbers which measure certain quantities of the physical state at a point of space-time, and the numbers which measure the same quantities at an infinitesimally-near point; to be able to analytically formulate the law, one must compare the two series of measurements in a common system of reference, in the case where the measurements were made in two different reference systems: said another way, one must know the affine connection associated with the space time. But supposing this connection is known, the problem is far from being solved: its solution depends in large part on the physical interpretation of the analytical formulae which represent the laws in question. For example, certain laws may be expressed by the property that a certain integral, over a closed domain of space-time in two or three dimensions, is zero\footnote{\WTF{Examples}{Citons}: in classical mechanics, the law of conservation of mass; in electromagnetism, the two first groups of Maxwell equations.}: one such law is, to a very large degree, independent of the affine connection associated with space-time. In others, like the principle of inertia in mechanics, it surely is involved. In chapter I, I show that, whether in classical mechanics or special relativity, the dynamics of \WTF{continuous media}{milieux continus}, whose equations are established from the same point of view based on the conception of space-time as composed of four dimensions, contains an infinity of \textit{equivalent} affine connections: put otherwise, the mechanical phenomena are only able to reveal a \textit{class} of affine connections, between which the choice is \textit{a priori} arbitrary.

It is in chapter V that, thanks to the study of geometry made in chapters II and III, I show that between all of the mechanically-equivalent affine connections, there is one which is distinguished from all the others by its intrinsic properties. The consideration of Lorentz's laws of electromagnetism leads to the idea that this affine connection has \textit{no torsion}, which is not necessarily required by the laws of mechanics. Thus we shall justify Einstein's \textit{a priori} idea that the physical properties of the universe\footnote{At least those which are the object of the theory of general relativity} are all entirely contained in its $ds^2$. Nonetheless, the conclusion may be incorrect if one adopt a larger view of the mechanics of continuous media than usual, by admitting the possibility of material elements with \WTF{angular momenta}{moments cinétiques} not infinitesimally small with respect to their momenta. In this case, it would be appropriate to generalize Einstein's theory; I indicate the most natural of these generalizations: in this new theory, the torsion of the universe continues to be zero in the vacuum.

In the second part of this memoir, I shall study in-depth the torsion and curvature tensors. This study was not indispensable before; in fact it is quite remarkable that, in the applications of the theory of relativity, the rotation which \WTF{is caused by}{traduit} by the curvature of the universe enters \textit{globally}, without needing to explicitly include the coefficients which define it analytically\footnote{This is not the case in the usual exposition of the theory.}.

This memoir does not suppose knowledge of absolute differential calculus: on the other hand, it supposes knowledge of the fundamental rules of the calculus of multiple integrals, in particular those which \?{take an integral extending over a closed domain to an integral extending over a domain of a more-limited dimension than the first\footnote{I permit myself in this regard to refer the reader to my \textit{Leçons sur les Invariants intégraux}; Paris, Hermann, 1922.}.} Basically, the laws of the dynamics of continuous media and those of electromagnetism are expressed by equations analogous to Stokes's formula or its generalized form.

Although the absolute differential calculus doesn't need to be used, I have nonetheless employed some of its notation in utilizing, for example, upper and lower indices; I have also, despite the inconveniences that this may present, suppresses the summation signs when there is no ambiguity possible. Perhaps this will entice those familiar with the absolute differential calculus (and now who is not, to some extent?) to engage in reading this memoir with less suspicion.

\chapter{Dynamics of continuous media and the notion of the affine connection in space-time}

\section*{The principle of inertia and Newtonian gravitation}

\subsection*{1}
Classical mechanics, as it was founded by Newton, rests on the notion of an absolute time and an absolute space; all events are localized analytically in time by the choice of a time-origin and a unit of time, in space by the choice of a fixed coordinate system, for example having as its origin the center of gravity of the solar system and the axes directed towards the fixed stars; it is naturally permissible to take any other system of axes, provided that it is invariably \WTF{linked to the first}{lié au premier}. As is known, the laws of mechanics remain true if, always maintaining the notion of absolute time, one admits coordinate axes that are \textit{mobile} with respect to Newton's absolute space, with the condition that the axes move, with respect to the absolute space, with a uniform rectilinear translation. One thus arrives at the notion of the \textit{Galilean reference system}.

The principle of inertia is enunciated in the following manner: \textit{A material point, \WTF{neglecting}{soustrait à} the action of all other bodies, moves in such a manner that its velocity, determined with respect to a Galilean reference system, is constantly equivalent to itself}. If the principle of inertia is valid for one Galilean reference system, it is valid for all others.

Analytically, this is evident, if one uses the formulae for passing from one Galilean reference system to another. In the general case, to localize a point in space, we make use of arbitrary cartesian coordinates, rectangular or not\footnote{The laws of classical mechanics retain exactly the same form in oblique coordinates as in rectangular coordinates, and the formulae of \textit{theoretical} mechanics are exactly the same.}. The transformation formulae are
\nequ{1}{
x' &= a_1 x + b_1 y + c_1 z + g_1 t + h_1, \\
y' &= a_2 x + b_2 y + c_2 z + g_2 t + h_2, \\
z' &= a_3 x + b_3 y + c_3 z + g_3 t + h_3, \\
t' &= t + h,
}

with constant coefficients.

One may enunciate the principle of inertia in another manner. By "\textit{equivalent} reference  systems" we mean two reference systems consisting of two triads of coordinates which are equivalent in the ordinary geometric sense of the word, and supposed to be immobile with respect to one another; analytically, the formulae for passing from one reference system to an equivalent system are of the form
\uequ{
x' &= x + h_1,\\
y' &= y + h_2,\\
z' &= z + h_3,\\
t' &= t + h.
}

Given this, we attach to a mobile material point, at each instance of its motion, a Galilean reference system having as its origin the point itself\footnote{This means that the point is the origin of the coordinate axes and that the instant that one observes it is taken as the origin of time.}. The principle of inertia is then enunciated as:

\textit{If a material point is isolated from the action of all other bodies and one attaches to it at each instant a Galilean reference system always equivalent to the point itself, the components of its velocity, determined at each instant with respect to the corresponding reference system, are constant.}

% 
\subsection*{2}
It is seen that the structure of classical mechanics rests on two notions:
\begin{enumerate}
    \item The notion of a Galilean reference system (which permits the definition of the velocity of a moving point);
    \item The notion of equivalent Galilean reference systems (which permits the statement of the principle of inertia).
\end{enumerate}

It is essential to remark that the advantage of the second formulation of the principle of inertia is that \textit{it only requires utilization of the notion of equivalent Galilean systems for two systems with infinitesimally-close origins}.

In all of the generalizations which have been made in classical and relativistic mechanics, the notion of a Galilean reference system has remained unshaken; it is the notion of equivalent reference frames that has suffered fundamental modifications.

We shall always remain in the realm of classical mechanics, with the notion of absolute time (measured with a unit of time fixed once and for all). We shall see that a modification of the notion of equivalent reference systems permits us to extend the principle of inertia not only to a material point isolated from the action of all other bodies, \textit{but also to a material point located in a gravitational field}. We relate the position of a point in space and time to a fixed Galilean system where we call the triad of components $T_0$, and consider a field of forces analogous to a gravitational field, that is to say, essentially, \textit{an acceleration field} $(X, Y, Z)$.

If, with respect to a fixed Galilean system, the velocity at the instant $t$ is
\uequ{
u,v,w,
}
its velocity at the instant $t+dt$ will be
\uequ{
u + Xdt, v + Ydt, w + Zdt.
}
We attach \?{to the motion} at the instant $t$ a triad of components $T$, equivalent to $T_0$
in the ordinary geometric sense of the word, and similarly attach at the instant $t+dt$ a triad $T'$ equivalent to $T_0$. These triads only define Galilean reference systems if their motion is given by uniform rectilinear translation with respect to $T_0$; if they are \?{so} chosen, we will define two Galilean reference systems with origins
\uequ{
x,y,z,t;\\
x+dx,y+dy,z+dz,t+dt.
}

We respectively denote the translatory velocities of the triads $T$ and $T'$ with respect to $T_0$ by
\uequ{
a,b,c,\\
a',b',c'.
}
Given this, the velocity of the point with respect to a Galilean system attached to it at the instant $t$ has the components
\uequ{
u-a, v-b, w-c;
}
the velocity with respect to the Galilean system which is attached at the instant $t+dt$ has the components
\uequ{
u + Xdt - a', v + Ydt - b', w + Zdt - c'.
}

The components will not change if one has
\uequ{
a'-a = Xdt, b'-b = Ydt, c' - c = Zdt.
}

Therefore, \textit{the motion of an arbitrary material point located in the force field under consideration will again satisfy the principle of inertia if one agrees to regard as equivalent two Galilean reference systems with infinitesimally-close origins}
\uequ{
z,y,z,t;\\
x+dx,y+dy,z+dz,t+dt,
}
\textit{where their coordinate triads $T$ and $T'$ are equivalent in the ordinary geometric sense of the word, and where additionally the triad $T'$ moves with respect to the triad $T$ with uniform rectilinear translatory velocity $(Xdt, Ydt, Zdt)$.}

We remark that, again, only the mutual relations between the two infinitesimally-close reference systems are involved.

%
\subsection*{3}
One may present these in another manner which is perhaps more intuitive and which is closer to the point of view which served as a point of departure for Einstein in his theory of gravitation. We imagine a material point in the force field under consideration and a triad $T$ having this point for its origin, \WTF{moving}{entraîné} with a translatory motion. At each instant $t$ we consider the Galilean reference system consisting of a triad $\overline{T}$ coinciding with $T$ at the instant under consideration and moving with uniform rectilinear translatory motion, with its velocity the current velocity of the moving point\footnote{In reality, the triad $\overline{T}$ may be replaced \WTF{for our purposes}{pour l'usage qui est fait} by the triad $T$ itself, which thus plays, from the point of view of the measurement of the velocity of a point at the instant $t$, the role of a Galilean triad.}; with respect to this Galilean system, this material point evidently has zero velocity: its motion then satisfies the principle of inertia (constancy of velocity) if the successive Galilean reference systems defined by the triads $\overline{T}$ are considered as equivalent \textit{\WTF{at each step}{de proche en proche}}. One easily sees that the constant velocity of the triad $\Tbar'$ corresponding to the instant $t+dt$ with respect to the triad $\Tbar$ corresponding to the instant $t$ is
\uequ{
Xdt, Ydt, Zdt.
}

We consider, for example, the uniform field consisting of the gravity at the surface of the earth, supposing for the moment that we can neglect the motion of the earth with respect to absolute space. Taking the axis $z$ as vertical and ascending, two parallel triads $T$ and $T'$, one considered at an instant $t$, the other at the instant $t+dt$, would be considered as defining two equivalent Galilean reference systems if the second is moving with respect to the first with a constant vertical velocity $g dt$. In this case, one may attach to every space-time event $(x,y,z,t)$ a Galilean reference system \textit{such that all these systems are equivalent to one another}; it would suffice to be given the Galilean system attached to a particular event $(x_0,y_0,z_0,t_0)$; all the others would be perfectly determined.

This circumstance is no longer present in the general case. The equivalence relation, being only defined at each step, only permits saying whether two systems are equivalent if one is given the space-time path followed going from the origin of one to the origin of the other; two systems that are equivalent for a certain path cease to be equivalent for another path. We return later to this fundamental question.


\subsection*{4}
\WTF{We have arrived at}{Convenons de dire que} the equivalence conditions of two Galilean reference systems with infinitesimally-close origins defining the \textit{geometric} properties\footnote{In reality, these properties are at the same time geometric and kinematic} of space-time. The gravitational phenomena then pass from physics into geometry\footnote{Essentially, this is another way to say that the inertial mass is identical with the gravitational mass, or again that the gravitational field is a kinematic field (an acceleration field) rather than a dynamic field (a force field).}. The components $X,Y,Z$ of the gravitational field essentially constitute the elements of the geometrical structure of space-time\footnote{In reality, as one will see later, this structure not only involves the functions $X,Y,Z$ but also \WTF{nearly arbitrary constants}{constants arbitraires près}. This corresponds to the fact that, given a material system subject to a uniform gravitational field, it is impossible, by means of mechanical experiments conducted from the interior of the system, to detect the gravitational field; in particular, if one supposes that the stars produce a \textit{uniform} gravitational field \WTF{throughout the solar system}{dans l'étendue du système solaire}, absolutely nothing changes in what the laws of celestial mechanics predict for the motions of the sun and the planets.}. The relations
\nequ{1}{
\pddX{Z}{y} - \pddX{Y}{z} = 0, \quad
\pddX{X}{z} - \pddX{Z}{x} = 0, \quad
\pddX{Y}{x} - \pddX{X}{y} = 0,
}
which are true in \textit{rectangular} coordinates, carry the properties of this structure.

Finally, Poisson's fundamental equation
\nequ{2}{
\pddX{X}{x} + \pddX{Y}{y} + \pddX{Z}{z} = -4\pi\rho,
}
which, with the preceding, furnishes the complete laws of Newtonian gravitation\footnote{One must add the complementary condition that the functions $X,Y,Z$ vanish at infinity.}, shows that \textit{the density of matter in a continuous medium is the physical manifestation of a local geometrical property of space-time}.

Thus we recover, without leaving the realm of classical mechanics, some of the traits of Einstein's theory of gravitation. The only essential trait which is lacking concerns the connection between gravitational phenomena and electromagnetic phenomena. But here we work out that interdependence of the geometric structure of space-time and of the matter which fills the space.

\subsection*{5}
The preceding considerations call for \WTF{elaboration}{compléments}. In the first place, is the reduction of gravitation to geometry only possible with one definition of the equivalence of two infinitesimally-close Galilean systems? We examine this question later; we are content for the moment to remark that it must be answered in the negative. We imagine, in fact, two Galilean reference systems with infinitesimally-close origins, equivalent in the sense of section 3. The corresponding triads $(T)$ and $(T')$ with origins $M$ and $M'$ are equivalent in the ordinary geometric sense of the word and the triad $(T')$ is moving with respect to the triad $(T)$ with a certain uniform rectilinear translation. Then we imagine that we replace the triad $(T')$ by the triad $(T'')$ with the same origin which is derived from $(T)$ by a \WTF{helical}{hélicoïdal} movement of the axis $MM'$, \?{of a given direction and fixed once and for all}, this triad $(T'')$ being fixed with respect to the triad $(T')$. We consider a material point moving in the gravitational field, that is at $M$ at the instant $t$, at $M'$ at the instant $t+dt$. Its velocity at the instant $t+dt$
being \WTF{quite well-approximated}{très sensiblement portée} by the line $MM'$,will evidently have the same components with respect to the reference system $(S')$ defined by the moving triad $(T')$ as with respect to the reference system $(S'')$ defined by the moving triad $(T'')$. Thus, if the motion of the point verifies the principle of inertia when one supposes the systems $(S)$ and $(S'')$ to be equivalent, \textit{the principle of inertia will also be verified with the modified definition of equivalence}.

We conclude from this example that \textit{there exists, as long as one confines oneself to the dynamics of a material point, an infinity of possible definitions of equivalence for two Galilean reference systems with infinitesimally-close origins.}

\subsection*{6}

One may think that these conclusions must be modified when dealing with the dynamics of material  \textit{systems}, since point dynamics neglects an important element, the rotation of the material element about itself. We consider a small spherical ball moving with an absolutely constant rotation; the axis of its rotation, which \WTF{carries}{porte} the angular momentum of the ball, must be regarded as constantly equivalent to itself. It then seems that our primitive convention, in which the directions of space remain equivalent to themselves in the usual sense, is the only admissible one. We leave this question aside for the moment, \?{reserving for later} the demonstration that the preceding conclusions are quite premature and that \textit{on the contrary, the indeterminacy revealed earlier remains in its entirely when one takes account of the laws of the dynamics of systems}\footnote{Except for one possible restriction, see later section 16.}. But for the question to be usefully studied, it is important to remark that the new point of view which we have \WTF{adopted}{nous nous sommes placé} obliges us to enunciate the laws of mechanics in an \textit{exclusively local} form, that is to day to reduce everything to the mechanics of continuous media; indeed, we do not know whether two reference systems are equivalent when their origins are not infinitesimally close.

But, to facilitate the passage from Newtonian mechanics to relativistic mechanics, we have to show how the conception of four-dimensional space-time allows us to write the equations of classical mechanics for continuous media.



\section*{Four-dimensional space-time and the classical dynamics of continuous media}

\subsection*{7}
We adopt the point of view of classical mechanics. The space-time or the universe is an \textit{affine manifold}. Here is what is meant by that.

By \textit{world-vector} we denote the ensemble of two events (each localized in space and in time) where one will be the origin, the other the extremity of the vector. With respect to a Galilean reference system the components of a world-vector are the four quantities
\uequ{
t'-t,x'-x,y'-y,z'-z
}
obtained by subtracting the coordinates of the extremity and the origin. If two vectors have the same components with respect to a certain Galilean system, they also have the same components with respect to every other Galilean system; \?{this is} a property, for two world vectors, independent of the Galilean reference system by means of which one analytically defines the two vectors; we say that the two world vectors are \textit{equivalent}. It is evident that if two world vectors equivalent to a third, they are equivalent to one another. It is the existence of this notion of equivalence of two vectors that we express in saying that the universe is affine.

Of the four quantities
\uequ{
t'-t,x'-x,y'-y,z'-z
}
which analytically define a world vector, we say that the first is its \textit{time component}, the three others its \textit{space components}. We remark that \textit{the time component is independent of the chosen reference system}. It is not the same as a \textit{space vector} which has for its components, with respect to a triad of coordinates $T$ which serves to define a Galilean reference frame, the three quantities $x'9x,y'-y,z'-z$; this space vector depends, not only on the given world vector and on the triad $T$, but also on the velocity of this triad with respect to absolute space.

If we now consider a point moving with respect to a Galilean reference system, the world vector which has for its origin the point taken at the instant $t$ and for its extremity the point taken at the instant $t+dt$, has for its components
\uequ{
dt,dx,dy,dz;
}
it is, \textit{intrinsically}, independent of the reference system; it is the same as the world vector
\uequ{
1,\frac{dx}{dt},\frac{dy}{dt},\frac{dz}{dt}
}
which is derived by dividing by $dt$. If finally one denotes by $m$ the mass of the point, the world vector
\uequ{
m,m\frac{dx}{dt},m\frac{dy}{dt},m\frac{dz}{dt}
}
has an \textit{intrinsic} meaning, independent of the chosen reference system. This is the "\textit{mass-momentum}" vector. Its time component, the mass, is independent of the reference system; on the contrary, its space components, the momentum, depend on it.

One then sees that the fundamental principle of point dynamics may be stated thus:

\textit{The derivative with respect to time of a the mass-momentum world vector is equal to the "force" space-vector}.

This statement contains at the same time the principle of conservation of mass and the law which relates force to acceleration.


\subsection*{8}
We now consider a continuous medium with a given Galilean reference frame, and take a three-dimensional space-time volume. I have shown elsewhere\footnote{E. Cartan, \textit{Leçons sur les Invariants intégraux}, p. 35-37; Paris, Hermann, 1922.} that the total mass of the material elements which are found in this volume is represented by the integral
\uequ{
\int\int\int \rho\,dx\,dy\,dz - \rho u \,dy\,dz\,dt - \rho v\,dz\,dx\,dt - \rho w\,dx\,dy\,dt,
}
designating by $\rho$ the density and by $u,v,w$ the components of the velocity of each material element.

\textit{If one initially assumes that there is no pressure or tension in the medium}, the momentum along the $x$-axis of the same volume will be given by the integral
\uequ{
\int\int\int \rho u\,dx\,dy\,dz - \rho u^2\,dy\,dz\,dt - \rho uv\,dz\,dx\,dt - \rho uw\,dx\,dy\,dt
}
and similarly for the components along the $y$-axis and $z$-axis. We respectively designate the elements under the preceding integral sign $\int\int\int$ by
\uequ{
\Pi, \Pi_x, \Pi_y, \Pi_z:
}
these are the components of the "mass-momentum" for a material element of the medium.

Finally we designate by $X,Y,Z$ the components of the force per unit volume.

To obtain the equations of mechanics for continuous media, we may proceed in the following manner: we consider a four-dimensional space-time domain. We decompose this domain in the following manner: take a specific material element; at each instant $t$, one of the components it contains has as its coordinates $(x,y,z)$; the world-point $(x,y,z,t)$ is either never part of the domain under consideration, or it is a part of this domain in a certain time interval $(t_1,t_2)$: the ensemble of world-points made up of the different material points of the material element under consideration, taken between the instant $t_1$ and the instant $t_2$, make up the \textit{world tubes} into which the world-domain can be decomposed. The boundary of this domain is made up of the material elements taken at the extremities $t_1$ and $t_2$ of the time interval.

Given this, the geometrical variation of the "mass-momentum" world-vector of each material element, between the instant $t_1$, where it enters the world-domain and the instant $t_2$, where it leavea, is equal to a space vector whosw components are
\uequ{
\intXY{t_1}{t_2}(X\,dx\,dy\,dz)dt,\quad
\intXY{t_1}{t_2}(Y\,dx\,dy\,dz)dt,\quad
\intXY{t_1}{t_2}(Z\,dx\,dy\,dz)dt.
}
Put another way, \textit{the integral over the domain of the "mass-momentum" world-vector is equal to the four-dimensional integral of the "force" space-vector over the same domain.} This is expressed by the formulae
\nequ{5}{
\Pi' &= 0,\\
\Pi_x' &= X\,dt\,dx\,dy\,dz,\\
\Pi_y' &= Y\,dt\,dx\,dy\,dz,\\
\Pi_z' &= Z\,dt\,dx\,dy\,dz,
}
by putting
\uequ{
\Pi' &= \left[
\pddX{\rho}{t} + \pddX{(\rho u)}{x} + \pddX{(\rho v)}{y} + \pddX{(\rho w)}{z}
\right]dt\,dx\,dy\,dz,\\
\Pi_x' &= \left[
\pddX{(\rho u)}{t} + \pddX{(\rho u^2)}{x} + \pddX{(\rho uv)}{y} + \pddX{(\rho u w)}{z}
\right]dt\,dx\,dy\,dz,\\
\Pi_y' &= \left[
\pddX{(\rho v)}{t} + \pddX{(\rho vu)}{x} + \pddX{(\rho v^2)}{y} + \pddX{(\rho v w)}{z}
\right]dt\,dx\,dy\,dz,\\
\Pi_z' &= \left[
\pddX{(\rho w)}{t} + \pddX{(\rho wu)}{x} + \pddX{(\rho wv)}{y} + \pddX{(\rho w^2)}{z}
\right]dt\,dx\,dy\,dz,\\
}

In performing this calculation and simplifying the three last equations by means of the first, one obtains the classical equations
\uequ{
\pddX{\rho}{t} + \pddX{(\rho u)}{x} + \pddX{(\rho v)}{y} + \pddX{(\rho w)}{z} = 0,\\
\rho\left(\pddX{u}{t} + u\pddX{u}{x} + v\pddX{u}{y} + w\pddX{u}{z}\right) = X,\\
\rho\left(\pddX{v}{t} + u\pddX{v}{x} + v\pddX{v}{y} + w\pddX{v}{z}\right) = Y,\\
\rho\left(\pddX{w}{t} + u\pddX{w}{x} + v\pddX{w}{y} + w\pddX{w}{z}\right) = Z.\\
}

In calling the operation which permits \?{passing from an element of the integral over a closed manifold with $p$ dimensions to the equal integral element over the manifold with $p+1$ dimensions which encloses the first} the \textit{exterior derivative}, one sees that one may state the fundamental principle of the mechanics of continuous media in the following form:

\textit{The exterior derivative of the elementary "mass-momentum vector" is equal to the product of $dt$ multiplied by the force of the elementary volume.}

\subsection*{9}
We have supposed in the preceding that there is neither pressure nor tension. One gets back to the general case from the preceding case by agreeing to call the (generalized) momentum of the material element the vector whose components are obtained by adding respectively to the original components the quantities
\uequ{
-p_{xx}\,dy\,dz\,dt - p_{xy}\,dz\,dx\,dt - p_{xz}\,dx\,dy\,dt,\\
-p_{yx}\,dy\,dz\,dt - p_{yy}\,dz\,dx\,dt - p_{yz}\,dx\,dy\,dt,\\
-p_{zx}\,dy\,dz\,dt - p_{zy}\,dz\,dx\,dt - p_{zz}\,dx\,dy\,dt.
}

In the kinetic theory of gas, the pressure is effectively considered as a momentum flux due to the irregularity of the velocities of different molecules; the quantities designated earlier as $u,v,w$ only represent a \textit{mean} velocity.

One retrieves the classical equations of the mechanics of continuous media by developing (5), which gives
\uequ{
\pddX{\rho}{t} + \pddX{(\rho u)}{x} + \pddX{(\rho v)}{y} + \pddX{(\rho w)}{z} = 0,\\
\rho\left(\pddX{u}{t} + u\pddX{u}{x} + v\pddX{u}{y} + w\pddX{u}{z}\right) 
+ \pddX{p_{xx}}{x} + \pddX{p_{xy}}{y} + \pddX{p_{xz}}{z} = X,\\
\rho\left(\pddX{v}{t} + u\pddX{v}{x} + v\pddX{v}{y} + w\pddX{v}{z}\right) 
+ \pddX{p_{yx}}{x} + \pddX{p_{yy}}{y} + \pddX{p_{yz}}{z} = Y,\\
\rho\left(\pddX{w}{t} + u\pddX{w}{x} + v\pddX{w}{y} + w\pddX{w}{z}\right) 
+ \pddX{p_{zx}}{x} + \pddX{p_{zy}}{y} + \pddX{p_{zz}}{z} = Z.\\
}

\subsection*{10}
These equations are not complete. This is in effect because we have not taken account of the \WTF{angular momentum theorem}{théorème des moments cinétiques}\footnote{It may be remarked that the analytic formulation of thus theorem does not necessarily presuppose rectangular axes.}; here it translates to the equalities
\uequ{
\int\int\int y\Pi_z - z\Pi_y = \int\int\int\int (yZ - zY)dt\,dx\,dy\,dz,\\
\int\int\int z\Pi_x - x\Pi_z = \int\int\int\int (zX - xZ)dt\,dx\,dy\,dz,\\
\int\int\int x\Pi_y - y\Pi_x = \int\int\int\int (xY - yX)dt\,dx\,dy\,dz;
}
the integrals of the second members are taken over an arbitrary four-dimensional domain, those of the first over the three-dimensional boundary of that domain. These equations give
\uequ{
[dy\,\Pi_z] - [dz\,\Pi_y] = 0,\\
[dz\,\Pi_x] - [dx\,\Pi_z] = 0,\\
[dx\,\Pi_y] - [dy\,\Pi_x] = 0;
}
these are identically satisfied if there is no pressure; in the general case they give
\uequ{
p_{zy} - p_{yz} = 0,\\
p_{xz} - p_{zx} = 0,\\
p_{yx} - p_{xy} = 0.
}

\subsection*{11}
One may represent the preceding results by means of a simple vectorial notation. We designate by the letters
\uequ{
\e_0,\,\,\e_1\,\,\e_2\,\,\e_3
}
the four world-vectors which have respectively for their components
\uequ{
1,\,\,0,\,\,0,\,\,0;\\
0,\,\,1,\,\,0,\,\,0;\\
0,\,\,0,\,\,1,\,\,0;\\
0,\,\,0,\,\,0,\,\,1.
}
the last four\translator{sic} are space vectors. With this notation, the "mass-momentum" at a material point of mass $m$ is represented by
\uequ{
m\left(\e_0 + \frac{dx}{dt}\e_1 + \frac{dy}{dt}\e_2 + \frac{dz}{dt}\e_3\right).
}

If we again agree to designate by the letter $\m$ a world-point $(t,x,y,z)$, the derivative $\frac{d\m}{dt}$ of this point with respect to time is the world-vector with components
\uequ{
1,\frac{dx}{dt}, \frac{dy}{dt}, \frac{dz}{dt};
}
one sees that the "mass-momentum" of a material point is represented by the notation
\uequ{
m\frac{d\m}{dt}.
}
The points and the (\?{free}) vectors are \textit{geometric forms} of the first degree. One may also consider second-degree geometric forms, which represent \textit{sliding} vectors. One designates by $[\m\,\m']$ the sliding vector which has for its origin the world vector $\m$ and for its extremity the world vector $\m'$. This sliding vector of six \?{Plückerian} components which are the determinants of the second-order formed with the table

\begin{tabular}{ccccc}
  $1$ & $t$ & $x$ & $y$ & $z$ \\
  $1$ & $t'$ & $x'$ & $y'$ & $z'$
\end{tabular}
evidently one has
\uequ{
[\m'\,\m] = -[\m\,\m'].
}

Similarly, one designates by $[\m\,\e]$ the sliding vector obtained by carrying from the world-point $m$ a vector equivalent to a given vector $\e$; the six Pl\"uckerian coordinates of this sliding vector are formed with the table
\begin{tabular}{ccccc}
  $1$ & $t$ & $x$ & $y$ & $z$ \\
  $0$ & $\theta$ & $\xi$ & $\eta$ & $\zeta$,
\end{tabular}
where the second line contains the coordinates of the vector $\e$. Finally, the notation $[\e\,\e']$ will designate the \textit{bivector} whose six components are formed with the table
\begin{tabular}{ccccc}
  $0$ & $\theta$ & $\xi$ & $\eta$ & $\zeta$, \\
  $0$ & $\theta'$ & $\xi'$ & $\eta'$ & $\zeta'$,
\end{tabular}
the components of the two free vectors $\e$ and $\e'$.

In each of the preceding cases, the sliding vector or the bivector under consideration may be regarded as the (exterior) product of two factors, which are first-degree geometric forms (points or free vectors). The product of two arbitrary first-degree geometric forms satisfies the distributive laws, but the sign changes with the order of factors.


\subsection*{12}
The sliding vector which has as its origin the world-point $\m$ which represents a given material point at a given instant, and which one obtains \?{by carrying from this point its} "mass-momentum", is expressed by
\uequ{
m\left[\m\,\frac{d\m}{dt}\right],
}
and the equatiob
\uequ{
\frac{d}{dt}\left\{
m\left[\m\,\frac{d\m}{dt}\right]
\right\} = [\F],
}
where $\F$ represents the sliding "force" vector, contains, at the same time as the fundamental principle of dynamics, the angular momentum theorem; it condenses into six equations
\uequ{
&\frac{dm}{dt} = 0.\\
&\frac{d}{dt}\left(m\frac{dx}{dt}\right) = X,\\
&\frac{d}{dt}\left(m\frac{dy}{dt}\right) = Y,\\
&\frac{d}{dt}\left(m\frac{dz}{dt}\right) = Z,\\
&\frac{d}{dt}\left(mt\frac{dx}{dt} - mx\right) = tX\\
&\frac{d}{dt}\left(mt\frac{dy}{dt} - my\right) = tY\\
&\frac{d}{dt}\left(mt\frac{dz}{dt} - mz\right) = tZ\\
&\frac{d}{dt}\left(my\frac{dz}{dt} - mz\frac{dy}{dt}\right) = yZ - zY,\\
&\frac{d}{dt}\left(mz\frac{dx}{dt} - mx\frac{dz}{dt}\right) = zX - xZ,\\
&\frac{d}{dt}\left(mx\frac{dy}{dt} - my\frac{dx}{dt}\right) = xY - yX.
}


\subsection*{13}
We return to the mechanics of continuous media. We designate by $\G$ the sliding vector which represents the three-dimensional "mass-momentum" element and by $\F$ the sliding vector which represents the force \?{of the} elementary volume. \textit{The equations of mechanics are condensed in the formula}
\nequ{6}{
\G'=[dt\,\F].
}

\WTF{In fact, here one has}{On a du reste ici}
\uequ{
\G &= [\m\,\e_0]\Pi + [\m\,\e_1]\Pi_x + [\m\,\e_2]\Pi_y + [\m\,\e_3]\Pi_z,\\
\F &= [\m\,\e_1]X\,dx\,dy\,dz + [\m\,\e_2]Y\,dx\,dy\,dz + [\m\,\e_3]Z\,dx\,dy\,dz.
}
$\F$ may be calculated by taking account of the equation
\uequ{
d\m = \e_0 dt + \e_1 dx + \e_2 dy + \e_3 dz;
}
it gives
\uequ{
\G' = &[\m\,\e_0]\Pi' + [\m\,\e_1]\Pi_x' + [\m\,\e_2]\Pi_y' + [\m\,\e_3]\Pi_z'\\
      & + [\e_0 \, \e_1][ dt \Pi_x - dx \Pi]
        + [\e_0 \, \e_2][ dt \Pi_y - dy \Pi] 
        + [\e_0 \, \e_3][ dt \Pi_z - dz \Pi] \\
      & + [\e_2 \, \e_3][ dy \Pi_z - dz \Pi_y]
        + [\e_3 \, \e_1][ dz \Pi_x - dx \Pi_z] 
        + [\e_1 \, \e_2][ dx \Pi_y - dy \Pi_x].
}

One easily verifies that the coefficients of $[\e_0 \, \e_1]$, $[\e_0 \, \e_2]$, $[\e_0 \, \e_3]$ \WTF{vanish identically}{sont nuls d'eux-mêmes}.

If one supposes that each material element is acted upon \textit{not only by a force, but by a couple}, the fundamental equation (6) would not change, but one would need to add to the expression for $\F$ terms of the form
\uequ{
[\e_2 \, \e_3] L\,dx\,dy\,dz + [\e_3 \, \e_1]M\,dx\,dy\,dz + [\e_1 \, \e_2]N\,dx\,dy\,dz.
}

One would then have
\uequ{
p_{zy} - p_{yz} &= L,\\
p_{xz} - p_{xz} &= M,\\
p_{yx} - p_{xy} &= N.
}

It is quite evident that from the equation (6) one may deduce the fundamental equation of point dynamics and supposing that the matter is condenses in a very small portion of the space, one then obtains
\uequ{
d\G = dt\,\F.
}

\subsection*{14}
The equation (6) will permit us to easily write the equations of mechanics of continuous media in attaching to each point a variable Galilean reference system. We again designate by $\e_0$, $\e_1$, $\e_2$, $\e_3$ the free world-vectors which define the Galilean reference system attached to the point $\m$. In passing from a point $\m$ to an infinitesimally-close point $\m'$, these free vectors will vary, but the time component of the vector $\e_0$ will remain constantly equal to $1$, and the time-components of the vectors $\e_1$, $\e_2$, $\e_3$ remain constantly equal to $0$. One will thus have formulae of the form\footnote{We suppose, as in section 1, that the axes of the coordinate triads are not necessarily rectangular.}
\nequ{7}{
d\e_0 &= \omega^1_0 \e_1 + \omega^2_0 \e_2 + \omega^3_0 \e_3,\\
d\e_1 &= \omega^1_1 \e_1 + \omega^2_1 \e_2 + \omega^3_1 \e_3,\\
d\e_2 &= \omega^1_2 \e_1 + \omega^2_2 \e_2 + \omega^3_2 \e_3,\\
d\e_3 &= \omega^1_3 \e_1 + \omega^2_3 \e_2 + \omega^3_3 \e_3,
}
the $\omega^j_i$ being linear combinations of the differentials of the four quantities which serve to define, in an arbitrary manner, the different world-points. Thus we designate by
\nequ{8}{
d\m = \omega^0 \e_0 + \omega^1 \e_1 + \omega^2 \e_2 + \omega^3 \e_3,
}
the free vector with the origin $\m$ and extremity $\m'$; $\omega^0$ is simply the corresponding elementary time-interval. One will again have here expressions of the form
\uequ{
\G &= [\m\, \e_0]\Pi + [\m\,\e_1]\Pi_x [\m\,\e_2]\Pi_y + [\m \e_3]\Pi_z,\\
\F &= [\m\, \e_1]X \omega^1 \omega^2 \omega^3 + [\m \e_2]Y\omega^1 \omega^2 \omega^3
      [\m\, \e_3]Z \omega^1 \omega^2 \omega^3;
}
only the expression for $\G'$ is more complicated, since the free vectors $\e_0$,$\e_1$,$\e_2$,$\e_3$ are not fixed. One has
\uequ{
\G' = [\m\,\e_0]\Pi' 
&+ [\m\,\e_1][\Pi_x' + \omega^1_0 \Pi + \omega^1_1 \Pi_x + \omega^1_2 \Pi_y + \omega^1_3 \Pi_z]\\
&+ [\m\,\e_2][\Pi_y' + \omega^2_0 \Pi + \omega^2_1 \Pi_x + \omega^2_2 \Pi_y + \omega^2_3 \Pi_z]\\
&+ [\m\,\e_3][\Pi_z' + \omega^3_0 \Pi + \omega^3_1 \Pi_x + \omega^3_2 \Pi_y + \omega^3_3 \Pi_z]\\
&+ [\e_0 \, \e_1][\omega^0 \Pi_x - \omega^1 \Pi]
+ [\e_0 \, \e_2][\omega^0 \Pi_y - \omega^2 \Pi]
+ [\e_0 \, \e_3][\omega^0 \Pi_z - \omega^3 \Pi]\\
&+ [\e_2 \, \e_3][\omega^2 \Pi_z - \omega^3 \Pi_y]
+ [\e_3 \, \e_1][\omega^3 \Pi_x - \omega^1 \Pi_z]
+ [\e_1 \, \e_2][\omega^1 \Pi_y - \omega^2 \Pi_x].
}

One immediately deduces the desired equations.



\section*{The affine connection of the universe of classical mechanics}

\subsection*{15}
In the preceding we have uses the ordinary definition of equivalence of world-vectors. But the formulae obtained are again valid with an arbitrary definition of the equivalence of two world vectors \textit{with infinitesimally-close origins}. With this new definition, the formulae (7) preserve their form \textit{but with modified coefficients}\footnote{It is important to note that if one attaches to each world-point \textit{another} Galilean reference system, for example defined by
\uequ{
\overline{\e}_0 = \e_0 + u\e_1,\quad
\overline{\e}_1 = \e_1,\quad\overline{\e}_2 = \e_2, \quad \overline{\e}_3 = \e_3,
}
the formulae (7) should be modified as a consequence; in particular, $\omega^1_0$ will be replaced by $\overline{\omega}^1_0 = \omega^1_0 + u\omega^1_1 + du$.}.

We suppose that the only volume forces are those due to gravitation, \?{which is effectively the case in practice}; defining the equivalence of two Galilean systems with infinitesimally-close origins evidently reduces to defining the equivalence between two world vectors with infinitesimally-close origins. If one choses this definition in such a manner that the gravitational forces are suppressed, the dynamical equations reduce to
\uequ{
\G'=0.
}

Or, we take for $\e_0,\e_1,\e_2,\e_3$ vectors equivalent \textit{in the ordinary sense} to unit vectors of a \textit{fixed} Galilean reference system and take
\uequ{
\omega^1_0 = -X\,&dt,\quad \omega^2_0 = -Y\,dt,\quad \omega^3_0n= -Z\,dt,\\
& \omega^j_i = 0\quad (i,j=1,2,3).
}

The equations of mechanics become
\uequ{
\Pi' = 0,\quad
\Pi_x' - X[dt\,\Pi] = 0,\quad
\Pi_y' - Y[dt\,\Pi] = 0,\quad
\Pi_z' - Z[dt\,\Pi] = 0,
}
or again
\uequ{
  \Pi' &= 0,\\
\Pi_x' &= \rho X [dt\,dx\,dy\,dz],\\
\Pi_y' &= \rho Y [dt\,dx\,dy\,dz],\\
\Pi_z' &= \rho Z [dt\,dx\,dy\,dz];
}
these will be the classical dynamical equations of a continuous medium subject to a volume force \textit{proportional to the mass}. It is sufficient to take for $X,Y,Z$ the components of the acceleration due to gravity to \WTF{bring gravitation into}{pour faire rentrer...dans} the geometry.

The result just obtained is identical with that which was furnished directly by the dynamics of material points. The formulae
\uequ{
d\e_0 = -X\,dt\,\e_1 - Y\,dt\,\e_2 - Z\,dt\,\e_3,\\
d\e_1 = d\e_2 = d\e_3 = 0
}
mean in effect that two Galilean reference systems with origins
\uequ{
t,\,\,x,\,\,y,\,\,z,\\
t+dt,\quad x+dx,\quad y+dy,\quad z+dz
}
are regarded as equivalent when the corresponding triads $T$ and $T'$ are equivalent in the ordinary sense, the triad $T'$ moving with respect to the triad $T$ with a rectilinear translation and a uniform velocity $(X\,dt, Y\,dt, Z\,dt)$.


\subsection*{16}
We agree to say that a given definition of equivalence for two Galilean systems with infinitely-close origins defines an \textit{affine connection} on space-time. It is now simple to see if the phenomena of gravitation are compatible with several distinct \?{world affine connections}. Beforehand we make the important remark that \textit{although the world affine connection depends on the distribution of matter in space, nonetheless, it is not perceptibly modified by the introduction of small masses in a given region of the universe}. If we reflect on the system formed by one of these small masses, the world affine connection at each corresponding world-point does not depend on the state of these masses. Any imaginable modification of the world affine connection would lead in the expression for $\G'$ to the addition in the terms
\nequ{9}{
  &[\m\,\e_1][\wbar^1_0 \Pi + \wbar^1_1 \Pi_x + \wbar^1_2 \Pi_y + \wbar^1_3 \Pi_z]\\
+ &[\m\,\e_2][\wbar^2_0 \Pi + \wbar^2_1 \Pi_x + \wbar^2_2 \Pi_y + \wbar^2_3 \Pi_z]\\
+ &[\m\,\e_3][\wbar^3_0 \Pi + \wbar^3_1 \Pi_x + \wbar^3_2 \Pi_y + \wbar^3_3 \Pi_z]\\
}
where one designates by $\wbar^i_0 , \wbar^j_i$ the modifications suffered by the components $\w^i_0,\w^j_i$ of the affine connection. The only possible modifications are then those which cancel the three quantities in the brackets, and that \textit{\WTF{regardless of}{quelles que soient}the numerical values which characterize the state of the material medium}.

We initially adopt the most general point of view possible, the affine connection permitting the equivalence of two Galilean reference systems where one has an \WTF{orthogonal}{trirectangle} triad $(T)$, the other a \textit{non}-orthogonal triad $(T')$. We additionally suppose that it is allowed that the vectors $\e_0,\e_1,\e_2,\e_3$ used in the formulae (7) are always equivalent to one another \textit{in the usual sense}, that is to say that everything is referred to a fixed Galilean system. We put
\uequ{
\wbar^i_0 &= \gamma^i_{00}\,dt + \gamma^i_{01}\,dx + \gamma^i_{02}\,dy + \gamma^i_{03}\,dz,\\
\wbar^j_i &= \gamma^j_{i0}\,dt + \gamma^j_{i1}\,dx + \gamma^j_{i2}\,dy + \gamma^j_{i3}\,dz.
}

The coefficients of the forms $\Pi,\Pi_x,\Pi_y,\Pi_z$ are
\uequ{
\rho,\,\,\rho u,\,\,\rho v,\,\,\rho w,\\
\rho u^2 + p_{xx},\,\, \rho uv + p_{xy},\,\, \rho uw + p_{xz},\,\, \dots, \rho w^2 + p_{zz};
}
one may, to cancel the three expressions in the brackets in the formulae (7), regard them as independent. In turning our attention to each in turn, the others being regarded as zero, one arrives at the formulae
\nequ{10}{
\gamma^i_{00}=0,\quad \gamma^i_{0} =\gamma^i_{j0},\quad \gamma^k_{ij} = \gamma^k_{ji};
}
\textit{they simply express that the three quadratic differential forms}
\uequ{
\wbar^i_0 \, dt + \wbar^i_1 \, dx + \wbar^i_2 \, dy + \wbar^i_3 \,dz\quad (i=1,2,3)
}
\textit{are identically zero}.

One would arrive at the same result by simply utilizing the dynamics of material points; the equality
\uequ{
\frac{d}{dt}\left(
\e_0 + \e_1 \frac{dx}{dt} + \e_2 \frac{dy}{dt} \e_3 \frac{dz}{dt}
\right) = 0,
}
where one takes account of the relations (7), in fact remains valid by modifying the coefficients $\w^i_0, \w^j_i$ of these relations by the addition of the terms $\wbar^i_0 , \wbar^j_i$ if one has
\uequ{
\wbar^i_0 + \wbar^i_1 \frac{dx}{dt} + \wbar^i_2 \frac{dy}{dt} + \wbar^i_3 \frac{dz}{dt} = 0,
}
\textit{and that regardless of the mutual relations between $dx,dy,dz,dt$}: one arrives at exactly the same conclusions.

On the contrary the conclusions would be different, if one did not suppose the symmetry of the components of the pressure; this symmetry would necessarily disappear if one admits the possibility of \textit{couples}\footnote{This case would be present in the case of a magnet placed in a magnetic field} acting on a material element. In this case, the expression (9) would be identically zero, even assuming
\uequ{
p_{xy} \neq p_{yx}, \quad p_{yz} \neq p_{zy}, \quad p_{zx} \neq p_{xz}.
}
One the easily sees that \textit{all the coefficients $\gamma^k_{ik}$ must be zero}: only arbitrary nine coefficients remain instead of 18. In this case, \textit{and in this case only}, the dynamics if continuous media impose on the world affine connection mkre restrictive conditions than the dynamics of material points.


\subsection*{17}
We now suppose that the affine connection preserves the metric character of the space, that is to say that a reference system composed of an orthogonal triad can only be equivalent to another reference system also composed of an orthogonal triad. In this case, if one attaches a Galilean reference system to each world point, such that the space vectors $\e_1, \e_2, \e_3$ are orthogonal and of length one (once a unit of length is fixed once and for all), the formulae (7) still hold, but one has between the $\w^j_i$ the relations which are derived from the formulae
\uequ{
(\e_i)^2,\quad \e_i \e_j = 0, \quad (i \neq j = 1,2,3);
}
these relations are
\uequ{
\w^i_i = 0, \quad \w^j_i + \w^i_j = 0;
}
the three quantities $\w^3_2 = -\w^2_3$, $\w^1_3 = -\w^3_1$, $\w^2_1 = -\w^1_2$ are the components of the rotation which brings the triad $T$ to be equivalent to the triad $T'$.

The modifications which are permitted to be made to the world affine connection are then defined by the same conditions as in section 16; just the fact that one has
\uequ{
\wbar^i_i = \wbar^j_i + \wbar^i_j = 0
}
reduces to four the number of arbitrary coefficients; one has\footnote{The geonetric interprtation of these formulae will be trivial.}
\uequ{
\wbar^1_0 = r\, dy - q\,dz,\quad \wbar^2_0 = p\,dz - r\,dx,\quad \wbar^3_0 = q\,dx - p\,dy,\\
\wbar^3_2 = -\wbar^2_3 = p\, dt + h\,dx,\quad \wbar^1_3 = -\wbar^3_1 = q\,dt + h\, dy, \quad
\wbar^2_1 = -\wbar^1_2 = r\, dt + h\, dz.
}

If one did not suppose the symmetry of the components of the pressure, the coefficient $h$ would necessarily be zero.


\subsection*{18}
We shall see later on how, following adopting the point of view of section 16 or 17, among all affine connections compatible with experience, there is one which is distinguished from the others by its intrinsic properties. We note however that one may imagine a mechanical theory in which the angular momentum of a material element with respect to a point on the interior of the element will not be infinitesimally small with respect to the momentum of the same element; one may also suppose that the state of tension of the medium is manifested by the \textit{couples} and not only the \textit{forces} acting on each plane element. In these conditions the analytic expression for $\G$ would contain the terms $[\e_0 \, \e_i]$ and $[\e_i \, \e_j]$, and thus \textit{the world affine connection would be \WTF{uniquely determined}{relèverait uniquement} by experience}; the mechanical facts would be compatible with just one definition of equivalence between tow Galilean systems with infinitesimally close origins.


\section*{The space-time of special relativity and its affine connection}

\subsection*{19}
The special theory of relativity admits the same Galilean reference systems as classical mechanics; the only difference, essentially, resides in the formulae which allow passing from the coordinates $(t,x,y,z)$ of an event with respect to a first Galilean reference system to the coorinates $(t_1,x_1,y_1,z_1)$ of the same event with respect to another Galilean reference system. These formulae are again linear, that is to say the universe of special relativity is affine, but the time component $t$ of a world vector is no longer \WTF{constant}{invariable}. The following expression is invariant
\uequ{
c^2(t' - t)^2 - (x' - x)^2 - (y' - y)^2 - (z' - z)^2,
}
where $c$ designates the speed of light in vacuum. Two vectors then admit an invariant, their \textit{scalar product}, whixh is
\uequ{
c^2 \theta \theta' - \xi\xi' - \eta\eta' - \zeta\zeta',
}
designating the components of the two vectors respectively by
\uequ{
\theta,\, \xi,\, \eta,\, \zeta,\\
\theta',\, \xi',\, \eta',\, \zeta'.
}

If in particular one designates as above by $\e_0, \e_1, \e_2, \e_3$ the unit vectors attached to a Galilean reference system, one has the formulae
\nequ{11}{
(\e_0)^2 = c^2, \quad (\e_1)^2 = (\e_2)^2 = (\e_3)^2 = -1,
\e_0 \e_i = 0, \quad \e_i \e_j = 0 (i \neq j = 1,2,3).
}

\subsection*{20}
If one considers a Galilean reference system depending on several parameters, one will have, for an infinitely-small total variation of the parameters, the formulae

\nequ{12}{
d\e_0 = \omega_0^0 \e_0 + \omega_0^1 \e_1 + \omega_0^2 \e_2 + \omega_0^3 \e_3,\\
d\e_1 = \omega_1^0 \e_0 + \omega_1^1 \e_1 + \omega_1^2 \e_2 + \omega_1^3 \e_3,\\
d\e_2 = \omega_2^0 \e_0 + \omega_2^1 \e_1 + \omega_2^2 \e_2 + \omega_2^3 \e_3,\\
d\e_3 = \omega_3^0 \e_0 + \omega_3^1 \e_1 + \omega_3^2 \e_2 + \omega_3^3 \e_3,\\
}

where the $\omega_i^j$ are linear with respect to the differentials of the parameters. These expressions are not absolutely arbitrary, since the relations (11) must always be verified. Differentiating them, one easily obtains

\nequ{13}{
\omega_0^0 =0, \quad \omega_0^i =c^2 \omega_j^i, \quad \omega_i^j + \omega_j^i =0 \quad (i,j = 1, 2, 3).
}

Thus there remain six independent expressions, and six is actually the number of parameters which determine \textit{orientation} of a Galilean reference system.

In the preceding formulae, the expressions $\omega_0^1, \omega_0^2, \omega_0^3$ represent the (infinitely small) changed velocity of the \?{sign}{signe} of the rectilinear translational motion where the axes of the second reference system are moved with respect to the first.

It is remarked that one recovers the law of the dependence of two infinitely-close Galilean systems in classical mechanics\footnote{It has Galilean systems with tri-rectangular triads, as in number 17.} by taking $c$ to be infinite; the formulae (13) in effect define

\uequ{
\omega_0^0 = 0, \quad \omega_i^0, \quad \omega_i^j + \omega_j^i = 0.
}

\subsection*{21. The dynamics of a point particle.}

In special relativity, the notion of a "momentum-mass" vector may also be put at the base of the dynamics of point particles: this is the vector

\uequ{
m\left(\e_0 + \e_1 \frac{dx}{dt} + \e_2 \frac{dy}{dt} + \e_3 \frac{dz}{dt} \right).
}

\textit{The rest mass} $\mu$ of a point particle is, up to a constant factor, the square root of the scalar product of the "momentum-mass" vector with itself; more precisely, one has

\uequ{
\mu = m\sqrt{1 - 
\frac{\left(\frac{dx}{dt}\right)^2 + \left(\frac{dy}{dt}\right)^2 + \left(\frac{dz}{dt}\right)^2
}{c^2} = m\sqrt{1-\frac{V^2}{c^2}}};
}
this number $\mu$ is attached to each point particle, the same as the ordinary mass in classical mechanics.

The analytic expression for the "momentum-mass" vector takes a more symmetrical form if one introduces the \textit{proper time} at a point defined by

\uequ{
d\tau = \sqrt{dt^2 - \frac{dx^2+dy^2+dz^2}{c^2}} = \sqrt{1-\frac{V^2}{c^2}}dt = \frac{\mu}{m}dt.
}

The momentum-mass vector then becomes
\uequ{
\mu\left(\e_0\frac{dt}{d\tau} + \e_1\frac{dx}{d\tau} + \e_2\frac{dy}{d\tau} + \e_3\frac{dz}{d\tau}\right);
}
in this expression, the quantities $\mu$ and $d\tau$ are independent of the reference system.

The fundamental principle of dynamics may be put thusly:

The derivative with respect to the proper time of a "momentum-mass" world-vector is equal to the "hyperforce" world-vector

\uequ{
\e_0 \hR + \e_1 \hX + \e_2 \hY + \e_3 \hZ.
}

This hyperforce vector has a meaning independent of the reference system, and one has

\uequ{
\frac{dm}{dt} &= \frac{dm}{d\tau}\frac{d\tau}{dt} = \hR\sqrt{1-\frac{V^2}{c^2}},\\
\frac{d}{dt}\left(m\frac{dx}{dt}\right) &= \frac{d}{d\tau}\left(m\frac{dx}{dt}\right)\frac{d\tau}{dt} = \hX\sqrt{1-\frac{V^2}{c^2}},\\
\frac{d}{dt}\left(m\frac{dy}{dt}\right) &= \frac{d}{d\tau}\left(m\frac{dy}{dt}\right)\frac{d\tau}{dt} = \hY\sqrt{1-\frac{V^2}{c^2}},\\
\frac{d}{dt}\left(m\frac{dz}{dt}\right) &= \frac{d}{d\tau}\left(m\frac{dz}{dt}\right)\frac{d\tau}{dt} = \hZ\sqrt{1-\frac{V^2}{c^2}}.
}

The force, in the usual sense of the word, is the space components of the hyperforce multiplied by $\sqrt{1-\frac{V^2}{c^2}}$.

On the other hand, there is a necessary relation between $\hR, \hX, \hY, \hZ$, which expresses the constancy of the rest mass of the point:
\uequ{
c^2 m \frac{dm}{dt} 
- m\frac{dx}{dt}\frac{d}{dt}\left(m\frac{dx}{dt}\right)
- m\frac{dy}{dt}\frac{d}{dt}\left(m\frac{dy}{dt}\right)
- m\frac{dz}{dt}\frac{d}{dt}\left(m\frac{dz}{dt}\right) = 0,
}
where
\uequ{
c^2 \hR dt = \hX dx + \hY dy + \hZ dz;
}
this relation says that the elementary work of the flrce is equal to the differential of the quantity $mc^2$; this quantity is the \textit{energy} of the point particle
\uequ{
mc^2 = \frac{\mu c^2}{\sqrt{1-\frac{V^2}{c^2}}};
}
to a first approximation, it is equal to
\uequ{
\mu c^2 + \frac{1}{2} \mu V^2
}
or again
\uequ{
\mu c^2 + \frac{1}{2} m V^2
}
if one supposes that $V$ is very small with respect to $c$.
%
\subsection*{22. The dynamics of continuous media.}

If we take the particular case where there are no volume forces, the dynamical equations of continuous media are essentially the same as in classical mechanics. One forms the \?{sliding vector}{vecteur glissant} which forms the momentum-mass of a material element
\uequ{
\xG = \left[\m\e_0\right]\xPi + \left[\m\e_1\right]\xPi_x + \left[\m\e_2\right]\xPi_y + \left[\m\e_3\right]\xPi_z
}
and writes that the exterior derivative $\xG'$ is identically zero.

If the medium is related to a fixed Galilean system, one may again put the components $\xPi, \xPi_x, \xPi_y, \xPi_z$ in the form
\uequ{
\xPi &= \rho dx\,dy\,dz - \rho u dy\,dz\,dt - \rho\nu dz\,dx\,dt - \rho w dx\,dy\,dt;\\
\xPi_x &= u\xPi - p_{xx}dy\,dz\,dt - p_{xy}dz\,dx\,dt - p_{xz}dx\,dy\,dt,\\
\xPi_y &= \nu\xPi - p_{yx}dy\,dz\,dt - p_{yy}dz\,dx\,dt - p_{yz}dx\,dy\,dt,\\
\xPi_z &= w\xPi - p_{zx}dy\,dz\,dt - p_{zy}dz\,dx\,dt - p_{zz}dx\,dy\,dt.
}

The \textit{rest density} of the material element under consideration is given by
\uequ{
\rho_0\left[dt\,dx\,dy\,dz\right] = \left[dt\xPi\right] 
- \frac{1}{c^2}\left[dx\xPi_x\right]
- \frac{1}{c^2}\left[dy\xPi_y\right]
- \frac{1}{c^2}\left[dz\xPi_z\right],
}
the formula in the second member of that occurring in the scalar product of the vector $(dt, dx, dy, dz)$ and the vector $(\xPi, \xPi_x, \xPi_y, \xPi_z)$. Expanding, one finds\footnote{It is to be remarked that since the volume element $[dt\,dx\,dy\,dz]$ is independent of the chosen reference system, the quantity $\rho_0$ has a meaning independent of this reference system. It is naturally not the same with the apparent density $\rho$.}
\uequ{
\rho_0 = \rho\left(1 - \frac{u^2 + \nu^2 + w^2}{c^2}\right) - \frac{1}{c^2}(p_xx + p_yy + p_zz).
}

The dynamical equations of continuous media are thus identical to those of classical mechanics, in the case where there are no given forces.

If one attaches a variable Galilean reference system to each world-point, one will have
\uequ{
\xG' = &[\m\e_0][\xPi' + \omega_1^0 \xPi_x + \omega_2^0 \xPi_y + \omega_3^0 \xPi_z]\\
     +& [\m\e_1][\xPi_x' + \omega_1^1 \xPi_x + \omega_2^1 \xPi_y + \omega_3^1 \xPi_z]\\
     +& [\m\e_2][\xPi_y' + \omega_1^2 \xPi_x + \omega_2^2 \xPi_y + \omega_3^2 \xPi_z]\\
     +& [\m\e_3][\xPi_z' + \omega_1^3 \xPi_x + \omega_2^3 \xPi_y + \omega_3^3 \xPi_z].
}
%. m

\subsection*{23. Investigating whether several distinct affine connections are compatible with experience.} We pass from one affine connection to another by subjecting the components $\omega_0^i, \omega_i^j$ to the variations $\wbar_0^i, \wbar_i^j$ naturally satisfying
\uequ{
\wbar_0^i = c^2 \wbar_0^i,\quad \wbar_i^j + \wbar_j^i = 0,
}
and these variations must cancel the four expressions
\uequ{
[\omega_1^0 \xPi_x] &+ [\omega_2^0 \xPi_y] + [\omega_3^0 \xPi_z],\\
[\omega_0^1 \xPi] &+ [\omega_2^1 \xPi_y] + [\omega_3^1 \xPi_z],\\
[\omega_0^2 \xPi] &+ [\omega_1^2 \xPi_x] + [\omega_3^2 \xPi_z],\\
[\omega_0^3 \xPi] &+ [\omega_1^3 \xPi_x] + [\omega_2^3 \xPi_y],
}
\textit{regardless of the state of the material element under consideration}. If we take the medium with respect to a fixed Galilean reference system, we find, as in (16), that the quadratic form
\uequ{
\wbar_0^i dt + \wbar_1^i dx + \wbar_2^i dy + \wbar_3^i dz \quad (i=0,1,2,3)
}
must be identically zero. One recovers for the $\wbar_0^i$ and the $\wbar_j$ exactly the same expressions as in (17), with \textit{four} arbitrary coefficients $p, q, r, h$.

If one admits a larger conception of the mechanics of continuous media, the elementary "momentum-mass" containing the terms in $[\e_0\,\e_i]$ and $[\e_i\,\e_j]$, there are no longer any arbitrary coefficients and \textit{the affine space-time connection uniquely \?{matches}{relève} experiment}\footnote{This is so if one simply supposes the possibility of couples acting on the material elements, because then the coefficient $h$ is necessarily zero, and, as the quantities $p, q, r, h$ are transformed by changing a Galilean reference system (these are the components of a world-vector), \textit{one is obliged to suppose as well that $p, q, r$ are zero}.}.

%that 
\subsection*{24. Gravitation in special relativity.}

In classical mechanics, the equations
\uequ{
\omega_0^1 = -\hX dt, \quad \omega_0^2 = -\hY dt, \quad \omega_0^3 = -\hZ dt,\\
\omega_i^j = 0 \quad (i,j = 1, 2, 3),
}
define the affine connection permitting the gravitation to be pulled back into the geometry, conserving their form regardless of the Galilean reference system chosen as the fixed system.

In special relativity, if one admits the Newtonian laws of gravitation for a Galilean system formed of axes having as their origin the center of gravity of the solar system and directed towards the fixed stars, these laws no longer conserve the same form for another Galilean reference system. If one postulates, with Einstein, that the form of the laws of gravitation must be the same, regardless of the Galilean reference system adopted\footnote{We shall see later on the exact meaning of this phrase}, one is obliged to modify these laws; but, what is important to note now is that Einstein's reduction of gravitation to geometry is essentially of the same nature as that indicated at the start of the chapter.

% Anus Frey
\subsection*{25. The general-relativistic point of view.}

We have supposes up to now the actual existence of Galilean reference systems \?{capable of identifying}{permettant de repérer} \textit{all} of space-time. At this point, we may now do without this hypothesis. In fact it will suffice to formulate the physical laws that the two following conditions be realized:

\begin{enumerate}
    \item One takes, for measuring the magnitudes of the physical state, a reference system in the small portion of space-time where the observer is located, to play the role of a true Galilean system\footnote{Naturally this is not the place to dwell on the difficulties which may be present in practice in assimilating a particular reference system to a Galilean system.};
    \item The affine space-time connection is known, that is to say it is known how to compare observations made with respect to two Galilean reference systems with infinitely-close origins; that means that the Lorentz-Minkowski group transformation which brings the two reference systems into alignment; analytically, this is expressed in the knowledge of the coefficients of the formulae (8) and (12).
\end{enumerate}

We shall now pass to the theory of manifolds with an affine connection. We return later to the application of this theory to general relativity, and we will examine how the laws of electromagnetism contribute to our knowledge of the affine connection of the world.


% van der sloot

\chapter{The fundamental properties of manifolds with an affine connection.}

\section*{Affine space.}

\subsection*{26.}

In ordinary geometry, ther are properties which are called \textit{affine properties}: these are those which are conserved when a homographic transformation is effected which preserves the \?{plane at infinity}{le plan de l'infini}. The notions of \textit{vector}, \textit{equivalence of two vectors}, the \textit{geometric sum of two vectors}, are affine notions; this is not the case with the notion of the \textit{length of a vector}, which is a metric notion. In affine geometry, one may only compare the lengths of parallel vectors. The theory of systems of sliding vectors, of their equivalence, of their reduction to a vector and a couple, is also a purely affine theory, despite the metric form in which it is usually \?{formulated}{enseignée}.

In affine geometry, the normal coordinate system is the system of cartesian coordinates; taking an origin $\origin$ and three vectors (not coplanar) $\e_1, \e_2, \e_3$ issuing from $\origin$, all vectors may be cast in the form
\uequ{
x^1\e_1 + x^2\e_2 + x^3\e_3,
}
and each point $\m$ may also be written in the form
\uequ{
\m = \origin + x^1\e_1 + x^2\e_2 + x^3\e_3,
}
and designating by $\m'-\m$ the (free) vector with origin $\m$ and endpoint $\m'$ (or all equivalent vectors).

We imagine that at each point $\m$ of space a corresponding cartesian reference point with origin $\m$ is fixed; let $\e_1, \e_2, \e_3$ be the three vectors which define, with $\m$, this reference system. One could likewise imagine that at each point $\m$ there is an infinity of such reference systems. Thus we will have an ensemble of reference systems depending on a number of parameters, up to 12; we will call these parameters $u_i$.

If one makes an infinitely small variation on these parameters, the point $\m$ and the vectors $\e_1, \e_2, \e_3$ are subject to the infinitely small variations which are vectors, and which are expressable linearly by means of $\e_1, \e_2, \e_3$. Thus
\nequ{1}{
d\m &= \omega^1 \e_1 + \omega^2 \e_2 + \omega^3 \e_3,\\
d\e_1 &= \omega_1^1 \e_1 + \omega_1^2 \e_2 + \omega_1^3 \e_3,\\
d\e_2 &= \omega_2^1 \e_1 + \omega_2^2 \e_2 + \omega_2^3 \e_3,\\
d\e_3 &= \omega_3^1 \e_1 + \omega_3^2 \e_2 + \omega_3^3 \e_3.
}

The $\omega^i$ and $\omega_i^j$ are linear with respect to the differentials $du_i$; these dozen Pfaffian forms together permit \?{the comparison of}{de repérer} the reference system with origin $\m + d\m$ with respect to a reference system with origin $\m$. One may also say that they define the small affine displacement which permits passing from one to the other.

The forms $\omega^i$ and $\omega_i^j$ are not arbitrary. The integrals
\uequ{
\int d\m, \quad \int d\e_1, \quad \int d\e_2, \quad \int d\e_3
}
extend around any arbitrary closed contour are evidently zero. Or, transforming them into surface integrals, one obtains
\uequ{
\int d\m = \int\int(\omega^1)'\e_1 + (\omega^2)'\e_2 + (\omega^3)'\e_3
 + d\e_1\omega_1 + d\e_2\omega_2 + d\e_3\omega_3,
}
oe, taking account of the formulae (1) themselves and making the reductions,
\uequ{
\int d\m = \int\int 
  \left[(\omega^1)' - \omega^1\omega_1^1 - \omega^2\omega_2^1 - \omega^3 \omega_3^1\right]\e_1
+ \left[(\omega^2)' - \omega^1\omega_1^2 - \omega^2\omega_2^2 - \omega^3 \omega_3^2\right]\e_2
+ \left[(\omega^3)' - \omega^1\omega_1^3 - \omega^2\omega_2^3 - \omega^3 \omega_3^3\right]\e_3.
}
Similarly,
\uequ{
\int d\e_1 &= \int\int
   \left[(\omega_1^1)' - \sum\omega_1^i\omega_i^1\right]\e_1
 + \left[(\omega_1^2)' - \sum\omega_1^i\omega_i^2\right]\e_2
 + \left[(\omega_1^3)' - \sum\omega_i^i\omega_i^3\right]\e_3,\\
\int d\e_2 &= \int\int
   \left[(\omega_2^1)' - \sum\omega_2^i\omega_i^1\right]\e_1
 + \left[(\omega_2^2)' - \sum\omega_2^i\omega_i^2\right]\e_2
 + \left[(\omega_2^3)' - \sum\omega_2^i\omega_i^3\right]\e_3,\\
\int d\e_3 &= \int\int
   \left[(\omega_3^1)' - \sum\omega_3^i\omega_i^1\right]\e_1
 + \left[(\omega_3^2)' - \sum\omega_3^i\omega_i^2\right]\e_2
 + \left[(\omega_3^3)' - \sum\omega_3^i\omega_i^3\right]\e_3.
}

\?{Cancelling}{En annulant} the second members and remarking that these second members, being vectors, must have their three components be zeroes, one obtains the formulae
\nequ{2}{
(\omega^i)' &= \sumXY{k=1}{k=3}\left[\omega^k \omega_k^i\right] \quad (i=1,2,3)\\
(\omega_i^j)' &= \sumXY{k=1}{k=3}\left[\omega_i^k \omega_k^j\right] \quad (i,j=1,2,3).
}

These define what is called the \textit{structure} of the affine space; \textit{all} of the properties are condensed in them.

%
\subsection*{27.}

\nc{\xa}{\mathbf{a}}

One may retrieve these formulae by a slightly different procedure, though one identical in substance. Consider a fixed point $\xa$; let $x^1, x^2, x^3$ be its coordinates with respect to a reference system attached to the point $\m$; we have
\uequ{
\xa = \m + x^1\e_1 + x^2\e_2 + x^3\e_3,
}
from which, differentiating and taking account of (1),
\uequ{
d\xa &= \left(dx^1 + \omega^1 + x^1\omega_1^1 + x^2\omega_2^1 + x^3 \omega_3^1\right)\e_1 \\
     &+ \left(dx^2 + \omega^2 + x^1\omega_1^2 + x^2\omega_2^2 + x^3 \omega_3^2\right)\e_2
      + \left(dx^3 + \omega^3 + x^1\omega_1^3 + x^2\omega_2^3 + x^3 \omega_3^3\right)\e_3.
}

The point $\xa$ being fixed, one thus has
\nequ{3}{
dx^1 + \omega^1 + x^1\omega_1^1 + x^2\omega_2^1 + x^3\omega_3^1 &= 0\\
dx^2 + \omega^2 + x^1\omega_1^2 + x^2\omega_2^2 + x^3\omega_3^2 &= 0\\
dx^3 + \omega^3 + x^1\omega_1^3 + x^2\omega_2^3 + x^3\omega_3^3 &= 0.
}

Expressing that the equations (3) are completely integrable, the structure equations (2) are recovered.

%


\section*{The notion of a manifold with an affine connection.}

\subsection*{28.}

We now consider a \?{numerical three-dimensional manifold}{variété numérique à trois dimensions}, where each point $\m$ is supposed to be defined by three numbers $u^1, u^2, u^3$. According to this idea, we attach to each point $\m$ an affine space containing this point, and the three vectors $\e_1, \e_2, \e_3$ forming with $\m$ a reference system for this space. The manifold will be called an "affine connection" when one defines, in an otherwise-arbitrary manner, a law permitting the identification of the affine spaces attached to any two \textit{infinitely-close} points $\m$ and $\m'$ on the manifold; this law permits saying that each point of the affine space attached to a point $\m'$ corresponds with each point of the affine space attached to the point $\m$, that each vector in the first space is parallel or equivalent to each vector in the second space. In particular, the point $\m'$ itself will be identified with respect to the affine space at the point $\m$ and we admit the law of continuity according to which the coordinates of $\m'$ with respect to the affine reference system with origin $\m$ are infinitely small; this permits saying in a certain sense that the affine space attached to $\m$ is the affine space \textit{tangent} to the given manifold\footnote{One could of course, as is usually done, remark that in in the vicinity of the point $\m$ there is an affine \?{neighborhood}{repérage} for the points of the manifold, \WTF{which}{ne serait-ce que celui qui} consists in attributing to the point defined by $u^i + du^i$ the cartesian coordinates $du^i$; in this sense, the affine space attached to $\m$ is indeed \textit{tangent} to the manifold. One may suppose as well that the manifold is submerged in an affine space with more or less dimensions and that the affine space attached to $\m$ is essentially the plane space tangent to this manifold. Finally one may regard the affine space attached to the point $\m$ as the manifold itself which is perceived in an affine manner by an observer located at $\m$. All of these points of view are compatible with the point of view of the text, which seems logically preferable to me.}:
\nequ{1}{
d\m &= \omega^1 \e_1 + \omega^2 \e_2 + \omega^3 \e_3,\\
d\e_i &= \omega_i^1 \e_1 + \omega_i^2 \e_2 + \omega_i^3 \e_3\quad (i=1,2,3),
}
in which the $\omega^i$ and $\omega_i^j$ are linear forms with respect to the differentials of the parameters of the variable reference system, the $\omega^i$ only depend on the differentials $du^1, du^2, du^3$. They are interpreted by saying that each point $\m'$ infinitely-close to $\m$ on the manifold must be regarded as the point
\uequ{
\m + \omega^1 \e_1 + \omega^2 \e_2 + \omega^3 \e_3
}
of the affine space tangent at $\m$\footnote{One sees that changing as needed the reference system chosen as the affine space tangent to $\m$, the cartesian coordinates of the point $\m + d\m$ are $du^1, du^2, du^3$, since the $\omega^i$ are linear combinations of the $du^i$}: similarly for the vector $\e_i$ attached to $\m'$ is equivalent to the vector
\uequ{
\e_i + \omega^1_i \e_1 + \omega^2_i \e_2 + \omega^3_i \e_3
}
of the affine space tangent to $\m$. Naturally, this equivalence must only be considered as having only \?{an infinitely-small second order meaning}{un sens qu'aux infiniment petits du second ordre près}.

To the formulae (1) can be adjoined those which permit passing from the coordinates $x^i$ of a point of the affine space of $\m$ to the coordinates $x^i + dx^i$ of the corresponding point (one could say equal) of the affine space $\m'$. These formulae are identical to the formulae (3) and are obtained in the same manner:
\nequ{3}{
dx^i \omega^i + x^i \omega_1^i + x^2 \omega_2^i + x^3 \omega_3^i = 0 \quad (i=1,2,3).
}
To this we add the formulae which permit the same passage for a vector of projections $\xi^i$, and which are
\nequ{4}{
d\xi^i +\xi^1 \omega_1^i + \xi^2 \omega_2^i + \xi^3 \omega_3^i = 0 \quad (i=1,2,3).
}
% 
\subsection*{29.}

The laws of the affine connection define what type of \?{\textit{connection}}{raccord} there is between the affine spaces tangent to two infinitely-close points $\m$ and $\m'$. What will happen if we consider two arbitrary points on the manifold?

One may answer this question by giving a specific path containing $\m_0$ and $\m_1$; the connection of the two tangent spaces may then be made step-by-step. In fact, if one chooses at each point in the path of an affine reference system, the $\omega^i$ and the $\omega_j^i$ are expressed in the form $p^i dt$ and $p_i^j dt$, by naming $t$ the parameter which defines the variable point on the path, the $p^i$ and $p_i^j$ being known functions of $t$. The equations (1) may then be regarded as ordinary differential equations giving, at each point on the path, $\m, \e_1, \e_2, \e_3$ as functions of their initial values. Then, integrating these functions, one will get a rigorously exact \?{correspondence}{repérage} of the affine space at $\m_1$ with respect to the affine space at $\m_0$.

It comes to the same thing, and it is possibly easier to understand for those used to calculating with vectors, to integrate the differential equations (3), which here become
\uequ{
\frac{dx^i}{dt} + p^i + p_i^i x^1 + p_2^i x^2 + p_3^i x_3 = 0.
}

Taking for the constants of integration $(x^i)_0$ the values of the unknowns for the initial value of $t$, the formulae
\nequ{5}{
x^i = a^i + a_1^i(x^1)_0 + a_2^i(x^2)_0 + a_3^i(x^3)_0\quad (i=1,2,3)
}
define the change of the cartesian coordinates which \?{describe}{repère} the affine space at the variable point $\m$ on the path with respect to the affine space at the original point $\m_0$. This change of coordinates, applied to a vector, would give
\uequ{
\xi^i = a_1^i(\xi^1)_0 + a_2^i(\xi^2)_0 + a_3^i(\xi^3)_0 \quad (i=1,2,3),
}
where
\nequ{6}{
(\e_i)_0 = a_i^1 \e_1 + a^i_2 \e_2 + a^i_3\e_3 \quad (i=1,2,3).
}

These last formulae show what becomes of the vector $(\e_i)_0$ as it is transported so as to remain step-wise equivalent to itself.

It is similarly seen that the point $\m_0$ has as its coordinates $a^1, a^2, a^3$ in the affine space tangent to $\m$,
\nequ{6'}{
\m_0 = \m + a^1 \e_1 + a^2 \e_2 + a^3 \e_3.
}

The formulae (6) and (6') define the general solution of the differential equations (1), as the formulae (5) define those of the equations (3).
%
\subsection*{30.}

Can the \?{comparison}{raccord} of the affine spaces tangent to two arbitrary points $\m$ and $\m'$ be defined independently of the path, for all $\m$ and $\m'$? To make it so, it is necessary and sufficient that the equations of the total differentials (1) or (3) be completely integrable. \?{The calculation had been made when we were dealing with the affine space proper}{Le calcul a été fait quand nous nous sommes occupés de l'espace affine proprment dit}: it gives the necessary and sufficient conditions (2):
\uequ{
(\omega^i)' &= [ \omega^k \, \omega_k^i ],\\
(\omega_i^j)' &= [ \omega_i^k \, \omega_k^j ];
}
we have suppressed the summation sign over $k$, following a usage which \?{has been popularized in / has vulgarized}{qu'a vulgarisè} the absolute differential calculus.

If the conditions (2) are realized, and if one attaches to a particular point $\m_0$ a reference system $(\e_i)_0$, nothing stops us from choosing an at arbitrary point $\m$, for $\e_1, \e_2, \e_3$, the vectors respectively equivalent to $(\e_1)_0, (\e_2)_0, (\e_3)_0$, since the property of equivalence now has an absolute meaning. With this choice of reference system, we have
\uequ{
d\e_i = 0,
}
and therefore the forms $\omega_i^j$ are identically zero; the formulae (2) then show that the $\omega^i$ have their bilinear covariants identically zero: these are thus exact differentials, and no generality is lost by supposing that $\omega^i = du^i$. Thus the formulae 
\uequ{
d\m = du^i \e_i
}
show that $u^1, u^2, u^3$ may be taken as the affine coordinates for the point $\m$ for a variable reference system for the whole manifold. Put otherwise, \textit{the manifold is the same as an affine space}.


%


\section*{The structure of a manifold with an affine connection.}

\subsection*{31.}

We return to the general case. Can formulae on the manifold with an affine connection under consideration be found which play the same role as the formulae (2) in an affine space proper? They can be obtained by following a procedure analogous to that which led, at the start of the chapter, to the formulae (2).

We consider in the manifold a closed contour starting at a point $\m_0$ and returning there, and consider the integral $\int d\m$ extending over this closed contour. We can make sense of this integral \?{as a way of identifying all the infinitely-small vectors which it represents the geometric sum with respect to the same affine space}{à condition de repérer tous les vecteurs infiniment petits dont elle représente la somme géométrique par rapport à un même espace affine}. The considerations of the preceding sections permit making this identification step-by-step with respect to the affine space tangent to $\m_0$. We return later on to this procedure, which is valid for any contour, but is not without serious-enough difficulties in practice, if one wants to be rigorous.

We now rather apply a second procedure, which will have the advantage of extending to multiple integrals with vectors, \textit{but which essentially assumes an infinitely-small contour}. We take a fixed point $\xa$ once and for all, and which is infinitely close to all of the points of the contour, \textit{and we \?{measure}{repérons} all of the infinitely-small vectors $d\m$ with respect to the affine space tangent to $\xa$}, which is possible to an infinitely-small second-order distance: this circumstance, as is known, has no influence on the calculation of a definite integral.

Thus let $u_0^i$ be the (curvilinear) coordinates of the point $\xa$, and $u^i$ those of a point $\m$ on the contour. In addition, we posit
\uequ{
\omega^i = \gamma_k^i du^k, \quad \omega_i^j = \gamma_{ik}^j du^k.
}

Calling $(\e_i)_0$ the reference vectors at $\xa$, we have
\uequ{
\e_i = (\e_i)_0 + \omega_i^j(\e_j)_0 = (\e_i)_0 + (\gamma_{ik}^j)_0(u^k -u_0^k)(\e_j)_0,
}
and finally
\uequ{
d\m = \omega^i \e_i = [ \omega^i + (\omega_{jk}^i)_0(u^k-u_0^k)\omega^j](\e_i)_0.
}

The components of the vector $d\m$ in the affine space tangent to $\xa$ are thus
\uequ{
\omega^i + (\gamma_{jk}^i)_0(u^k - u^k_0)\omega^j,
}
calling the value of the variable coefficient $\gamma_{jk}^i$ at $\xa$ $(\gamma_{jk}^i)$.

Then, the integral of $d\m$ of the length of the closed contour will have for the $i^\text{th}$ component, in the affine space tangent to $\xa$,
\uequ{
\int\omega^i + (\omega_{jk}^i)_0(u^k - u_0^k)\omega^j
 = \int\int (\omega^i)' + (\omega_{jk}^i)_0 du_k \omega^j + (\gamma_{jk}^i)_0(u^k - u_0^k)(\omega^j)'.
}

The second part is modified by suppressing the third part, which infinitely-small compared to the rest, and by replacing in the second the constant coefficient $(\gamma_{jk}^i)_0$ by its variable value $\gamma_{jk}^i$. We finally arrive, \textit{for an infinitely-small closed contour}, at the formula
\uequ{
\int d\m = \int\int \left[(\omega^i)' - \omega^j \omega_j^i \right]\e_i,
}
identical to the formula found in the case of a proper affine space. Here the vectors $\e_i$ of the second part are relative to an arbitrary point $\xa$, provided that it is infinitely close to the contour; replacing the point $\xa$ by another only alters the result by an infinitely-small quantity.

The coefficients of $\e_1, \e_2, \e_3$ in the second part are the elements of the double integral which, according to the nature of the question, only involve $du^1, du^2, du^3$, even if the reference system depends on arbitrary parameters other than $u^1, u^2, u^3$. We will put
\uequ{
\Omega^i = (\omega^i)' - [\omega^k \omega_k^i] = \Lambda_{jk}^i [\omega^j \omega^k],
}
where the last part is to be regarded as a sum over all pairs $(jk)$ of the indices $1,2,3$.

With this notation,
\uequ{
\int d\m = \int\int \Omega^1 \e_1 + \Omega^2 \e_2 + \Omega^3 \e_3,
}
which may also be written
\uequ{
(d\m)' = \Omega^1 \e_1 + \Omega^2 \e_2 + \Omega^3 \e_3,
}
designating as usual by $(d\m)'$ the bilinear covariant of the Pfaffian expression $d\m$: \textit{this last expression is thus not to be regarded in general as an exact differential}.

The vector $\Omega^1 \e_1 + \Omega^2 \e_2 + \Omega^3 \e_3$ defines what may be called the \textit{torsion} of the given manifold with affine connection. This torsion is zero, as it must be, if the geometric sum of an infinitely-small closed contour is zero.
%
\subsection*{32.}

The integral $\int d\e_i$ over an infinitely-small closed contour may be calculated in the same manner, and it is found that
\uequ{
\int d\e_i = \int\int \left[(\omega_i^1)' - \omega_i^k \omega_k^1\right]\e_1
 + \left[(\omega_i^2)' - \omega_i^k \omega_k^2\right]\e_2
 + \left[(\omega_i^3)' - \omega_i^k \omega_k^1\right]\e_3.
}
The result is formulae such as
\uequ{
(\omega_i^j)' - \left[\omega_i^k \omega_k^j\right] = \Omega_i^j
 = \Lambda_{ikl}^j \left[\omega_k \omega_l \right],
}
equivalent to the formulae
\uequ{
(d\e_i)' = \Omega_i^1 \e_1 + \Omega_i^2 \e_2 + \Omega_i^3 \e_3.
}

The forms define what is called the \textit{curvature} of the given manifold with an affine connection. This curvature is zero, as is immediately apparent, if the total differential equations (4) are completely integrable, put another way if the equivalence of two vectors has an absolute meaning, independent of the path chosen from the origin of first to the origin of the second. In a manifold without curvature, different equivalent reference systems can be attached to different points; the components $\omega_i^j$ are then zero and we are left with
\uequ{
\Omega^i = (\omega^i)';
}
it is seen that, in the case where there is torsion, the components $\omega^i$ of the vector $d\m$ with respect to axes with fixed directions \textit{are not exact differentials}.
% the new texpad fucking erases my shit when it crashes
\subsection*{33}

In summary, the structure equations (2) must be replaced in the case of an arbitrary manifold with an affine connection by the formulae
\nequ{2'}{
(\omega^i)' &= [\omega^k \omega_k^i] + \Omega^i = [\omega^k \omega_k^i] + \Lambda_{jk}^i [\omega^j \omega^k],\\
(\omega_i^j)' &= [\omega_i^k \omega_k^j] + \Omega_i^j = [\omega_i^k \omega_k^j] + \Lambda_{ikl}^j [\omega^k \omega^l].
}
If at each point $\m$, the most general reference system possible is chosen, which introduces 9 parameters other than $u^1, u^2, u^3$, these formulae retain their validity; only, the $\Lambda_{jk}^i$ and $\Lambda_{ikl}^j$ depend on the 12 parameters.

% peppydick

\section*{The affine displacement associated with a closed contour.}

\subsection*{34.}

\nc{\areaA}{\mathcal{A}}
\nc{\overx}{\overline{x}}

The route we have followed to arrive at the structure formulae (2') and the notions of curvature and torsion are not amenable to generalization to manifolds with more complicated connections, for example manifolds with a projective or conformal connection, etc. A more satisfactory route consists in returning to the originally-adopted point of view of transferring a point and a vector along a given path.

Thus we consider an infinitely-small closed contour starting from a point $\m_0$ and returning to it. The pointwise identification of the affine spaces tangent to the different points of the contour with respect to the affine space tangent to $\m_0$ will, as we have seen, by differentiating the ordinary differential equations (3), lead to formulae such as (5). But the equations (3) may be regarded as defining, in the affine space at point $\m$, a certain \textit{affine displacement} which permits passing from a reference system attached to $\m$ to a reference system attached to an infinitely-close point $\m + d\m$; the quantities $\omega^j, \omega_i^j$  are \?{in some sense}{en quelque sorte} the components, with respect to the reference system attached to $\m$, of the affine displacement. \textit{Integrating the equations (3) basically consists in the successive infinitely-small affine displacements which permit passing from the reference system attached to each point on the contour to the reference system on an infinitely-close point, starting in the affine space at the point $\m_0$.} When the closed contour has been completely described, we arrive at a final affine displacement which will naturally be very small if the closed contour is very small. It is this affine displacement \textit{associated} with the contour \?{that must be determined}{qu'il s'agit de déterminer}.

We imagine a surface ($S$) containing the contour; we may, without loss of generality, suppose that the numerical coordinates $u^3$ remain constant on this surface, so that it only involves two independent variables $u^1$ and $u^2$, which we will designate by $u$ and $\nu$. We may regard $u$ and $\nu$ as the rectangular coordinates of a point in an auxiliary plane; in this plane, the closed contour have as its image a certain closed line enclosing a certain area $\areaA$. We will suppose, \textit{obviously at a loss of generality}, that the closed curve is rectifiable, that its length $l$ is infinitely small, and finally that \?{the area of the square of the boundary $l$ is finite compared to the area $\areaA$}{l'aire du carré de côté est finie par rapport à l'aire A}\footnote{It will not be the case, for example, if the closed contour has as its image a rectangle with wides $\epsilon$ and $\epsilon^2$. All of the mentioned restrictions, \?{which are obviously not fundamental}{qui ne sont évidemment pas dans le nature des choses}, is designed to permit a simple demonstration of the result that we want to achieve. A rigorous demonstration would be necessary, though there is indeed no reason to doubt the general validity of the result. To this one may compare the text of the demonstration by M.J. Pérès in his Note: \textit{Le parallélisme de M. Levi-Civita et la courbure riemanienne [Rend. Accad. Lincei, (5a) 28 I, p. 425-428].}}.

Given this, if at each point of the surface ($S$) we chose a definite reference system, the $\omega^i$ and $\omega_i^j$ will become the definite Pfaffian expressions linear in $du$, $d\nu$; let
\uequ{
\omega^i = \alpha^i du + \beta^i d\nu, \quad \omega_i^j  = \alpha_i^j du + \beta d\nu,
}
where the coefficients definite functions of $u$ and $\nu$, which we will suppose to be differentiable. Equations (3) become
\uequ{
dx^i + \alpha^i du + \beta^i d\nu &+ x^1(\alpha_1^i du + \beta_1^i d\nu)\\
&+ x^2(\alpha_2^i du + \beta_2^i d\nu) + x^3(\alpha_3^i du + \beta_3^i d\nu) = 0.
}

If we move along the closed curve ($C$), we may express $u$ and $\nu$ as a function of the arc $s$ of the image curve ($0 \leq s \leq l$).

We designate by $(\alpha^i)_0$, $(\beta^i)_0$,$(\alpha_i^j)_0$, $(\beta_i^j)_0$ the numerical values taken by the functions $\alpha^i, \beta^i, \alpha_i^j, \beta_i^j$ at the starting point of the curve ($C$) ($s=0$); we put
\uequ{
(\omega^i)_0 = (\alpha^i)_0 du + (\beta^i)_0 d\nu, \quad
(\omega_i^j)_0 = (\alpha_i^j)_0 du + (\beta_i^j)_0 d\nu;
}
these are then the Pfaffian expressions with constant coefficients.

That being so, $\epsilon$ being a given sufficiently-small positive number, the contour may be supposed to be small-enough that, at every point of the contour, the differences
\uequ{
\alpha^i - (\alpha^i)_0,\quad \beta^i - (\beta^i)_0,\quad 
\alpha_i^j - (\alpha_i^j)_0,\quad \beta_i^j - (\beta_i^j)_0,\quad 
}
are smaller than $\epsilon$ in absolute value.

In the second place the differential equations which give $x^i$ show that we have
\uequ{
|x^i - (x^i)_0| < As,
}
a being a fixed positive number.

We then define the auxiliary functions $\overx^i$ of $u$ and $\nu$, taking as the origin of the contour the values $(x^i)_0$ and satisfying the completely integral equations
\uequ{
d\overx^i + (\omega^i)_0 + (x^k)_0(\omega_k^i)_0 = 0;
}
these functions are linear in $u$ and $\nu$. We will evaluate an upper limit for $x^i - \overx^i$ at an arbitrary point of the contour.

One has
\uequ{
dx^i - d\overx^i + \omega^i - (\omega^i)_0 + x^k\left[\omega_k^i - (\omega_k^i)_0\right] + \left[x^k - (x^k)_0\right](\omega_k^i)_0 = 0,
}
which gives, integrating from $0$ to $s$ and taking account of the previously-established inequalities,
\uequ{
|x^i - \overx^i| < B\epsilon s + Cs^2,
}
designating by $B$ and $C$ two fixed positive numbers.

Finally, one may write
\uequ{
dx^i +\omega^i + \overx^k \omega_k^i + (x^k - \overx^k)\omega_k^i = 0,
}
where, integrating from $0$ to $l$,
\uequ{
|(x^i)_1 - (x^i)_0 + \int(\omega^i + \overx^k \omega_k^i)| < 
\left(\frac{1}{2} B\epsilon l^2 + \frac{1}{3}Cl^3\right),
}
$H$ designating a new fixed number. The second member of this integral is, according to our hypothesis, \textit{infinitely small compared to the area $\areaA$}.

Limiting ourselves to the principal part of the increase
\uequ{
(x^i)_1 - (x^i)_0 = \Delta(x^i)_0,
}
we then have
\uequ{
\Delta(x^i)_0 + \int\int\left[(\omega^i)' + \overx^k(\omega_k^i)'
 + d\overx\omega_k^i\right] = 0,
}
one has, replacing $d\overx^k$ by its value and regarding the double integral as reduced to a single element,
\uequ{
\Delta(x^i)_0 + (\omega^i)' - (\omega^k)_0\omega^i_k
 + \overx^k (\omega_k^i)' - (x^k)_0(\omega_k^h)_0\omega_h^i = 0.
}

We may finally, without changing the principal part, rplace $\overx^k$ by $(x^k)_0$, $(\omega^k)_0$ and $(\omega_k^h)_0$ by $\omega^k$ and $\omega_k^h$; in the last place, suppressing the $0$ indices, we will obtain the definitive formula
\nequ{5}{
\Delta x^i + (\omega^i)' - [\omega^k \omega^i_k]
 + x^k \lbrace (\omega_k^i)' - [\omega_k^h \omega_h^i] \rbrace = 0.
}
or finally, taking account of the previously-introduced notation,
\nequ{5'}{
\Delta x^i + \Omega^i + x^k \Omega_k^i = 0.
}

% pisspits
\subsection*{35.}

We take the particular case of a vector with components $\xi^i$, which shall be transported in a manner such that it remains pointwise equivalent to itself;\textit{when the origin of the vector has described a closed contour, its initial components will have been subjected to the infinitely-small variations $\Delta\xi^i$ given by the formulae}
\nequ{6}{
\Delta\xi^i = -\xi^k\Omega_k^i.
}

For example, the vector chosen at the starting point $\m$ for the $i^\text{th}$ reference vector will become, \textit{if it is transported \?{maintaining equivalence}{par équipollence} along the closed contour},
\uequ{
\e_i - \Omega^k_i \e_k;
}
it will have been subject to a \textit{geometric diminution} of $\Omega_i^k \e_k$. This result is not a contradiction, despite appearances, with the preceding formula it is seen that
\uequ{
\underset{(C)}{\int}d\e_i = \int \int \Omega_i^k \e_k,
}
since $\int d\e_i$, in this last formula, \?{is not related back to the variation of a vector transported equivalent to itself}{ne se rapporte pas à la variation d'un vecteur transporté par équipollence}, but to that of a vector which, \textit{at each point $\m$ of the contour, is taken for the $i^\text{th}$ reference vector.}

% smooty fruit
\subsection*{36.}

Ultimately, to any infinitely-small closed contour starting at a point $\m$ and returning to it is associated an infinitely-small (to second order) affine displacement  whose components with respect to a reference system attached to the starting poi t $\m$ are the elements of the double integrals $\Omega^i, \Omega_i^j$. The first components $\Omega^i$ define a \textit{translation}, the others an (affine) \textit{rotation} around $\m$.

The translation is due to the \textit{torsion}, the rotation to the curvature of the given manifold.

\section*{The theorem of conservation of curvature and torsion.}

\subsection*{37.}

\nc{\overOmega}{\overline{\Omega}}

We return to the formulae (2')
\nequ{2'}{
(\omega^i)' &= [\omega^k \omega_k^i] + \Omega^i,\\
(\omega_i^j)' &= [\omega_i^k \omega_k^j] + \Omega_i^j,
}
which define the structure of a manifold with an affine connection. the differential forms $\Omega^i$ and $\Omega_i^j$ satisfy the remarkable identities obtained by simply expressing that the exterior derivatives of the second parts of equations the (2') are identically zero. The calculation may be simplified by remarking that the exterior derivatives are zero themselves when the $\Omega^i$ and $\Omega_i^j$ are zero. Then
\nequ{7}{
(\Omega^i)' + [\Omega^k\omega_k^i] - [\omega^k \Omega_k^i] &= 0,\\
(\Omega_i^j)' + [\Omega_i^k\omega_k^j] - [\omega_i^k \Omega_k^j] &= 0.
}

We will give a geometric interpretation for these formulae.

To this end we consider an arbitrary volume in the manifold delimited by a closed surface. To each element of the closed surface surrounding a point $\m$ of this surface there is associated an infinitely-small affine displacement symbolized by the formulae (5')
\nequ{5'}{
\Delta x^i + \Omega^i + x^k \Omega^i_k =0,
}
whose components $\Omega^i, \Omega_i^k$ are \?{referred back to}{rapportées au} a reference system at the point $\m$. We cannot contemplate \textit{composing} these infinitesimal transformations, since the composition depends on the \textit{order} in which they are applied, and the elements of a surface cannot be ordered like those of a line. But we may \textit{add} these infinitesimal transformations, in the same sense that one adds two instantaneous rotations in kinematics; more precisely, we define the sum of the two infinitely-small affine displacements
\uequ{
\delta x^i + a^i + a_k^i x_k &= 0,\\
\delta x^i + b^i + b_k^i x_k &= 0,
}
as being the affine displacement
\uequ{
\delta x^i + (a^i + b^i) + (a_k^i + b_k^i) x_k &= 0.
}

Here there is still the difficulty that the different affine displacements are all in different affine spaces: they must thus all be reduced to the same affine space. \textit{This will only be possible if the closed surface is infinitely small.}

In this case, in fact, we will take, following a procedure already employed, a fixed point $\xa$ on the interior of the small volume. as the affine space tamgent to $\m$ may be identified with respect to the affine space tangent to $\xa$, we may determine, with respect to this last space, the components of the affine displacement (5'). We designate by $u_0^k$ the fixed numerical coordinates of $\xa$, $u^k$ those of $\m$; finally, we put
\uequ{
\omega^i = \gamma_k^i du^k, \quad \omega_i^j = \gamma_{ik}^j du ^k,
}
and designate by
\uequ{
(\gamma_k^i)_0, \quad (\gamma_{ik}^j)_0
}
the numerical values in $\xa$ of the coefficients of these forms. Naming $\overx^i$ the coordinates in the affine space tangent to $\xa$ of the corresponding point $(x^i)$ in the affine space tangent to $\m$, one has the formulae
\uequ{
x^i - \overx^i + (\gamma_k^i)_0(u^k - u_0^k) + \overx^k(\gamma_{kh}^i)_0 (u^h - u_0^h) = 0,
}
which may be solved for $\overx^i$,
\uequ{
\overx^i = x^i + (\gamma_k)_0(u^k - u_0^k) + x^k(\gamma_{kh}^i)_0 (u^h - u_0^h).
}

Given that, substituting the variables $\overx^i$ for the variables $x^i$ according to the preceding relations,
\uequ{
\Delta\overx^i &+ \Omega^i + (\gamma_{jk}^i)_0(u^k - u_p^k)\Omega^j -(\omega_h^k)_0(u^k - u_0^k)\Omega_k^i\\
& + \overx^k\left[\Omega_k^i + (\gamma_{jh}^i)_0(u^h - u_0^h)\Omega_k^j - (\gamma_{kh}^j)_0(u^h - u_0^h)\Omega_j^i \right] = 0.
}

As a result the components of the infinitely-small displacement (5'), \textit{evaluated with respect to the reference system attached to $\xa$}, are
\uequ{
\overOmega^i &= \Omega^i + (\gamma_{jk}^i)_0(u^k - u_0^k)\Omega^j - (\gamma_h^k)_0(u^h - u_0^h)\Omega_k^i,\\
\overOmega_k^i &= \Omega_k^i + (\gamma_{jh}^i)_0(u^k - u_0^k)\Omega_k^j - (\gamma_{kh}^j)_0(u^h - u_0^h)\Omega_j^i.
}

The sum of all the infinitely-small displacements associated with the different elements of the closed surface thus have for their components the surface integrals of $\Omega^i$ and $\Omega_k^i$, which gives the elements of the volume integrals
\uequ{
(\overOmega^i)' &= (\Omega^i)' + (\gamma_{jk}^i)_0[du^k \Omega^j] 
 - (\gamma_h^k)_0[du^h \Omega_k^i],\\
(\overOmega_k^i)' &= (\Omega_k^i)' + (\gamma_{jh}^i)_0[du^k \Omega_k^j]
 - (\gamma_{kh}^j)_0[du^h \Omega_j^i],
}
or finally, without changing the principal parts,
\uequ{
(\overOmega^i)' &= (\Omega^i)' + [\omega_j^i \Omega^j] - [\omega^k \Omega_k^i],\\
(\overOmega_k^i)' &= (\Omega_k^i)' + [\omega_j^i \Omega_k^j] - [\omega_k^j \Omega_j^i].
}

In this way the first part of the formulae (7), which is identically zero, is recovered.

One thus obtains the \textit{theorem of the conservation of curvature and torsion}:

\textit{The geometric sum of the infinitely-small displacements associated with the different elements of a closed surface is zero when the closed surface is infinitely-small.}

\section*{\?{Integral invariants}{Invariants integraux} attached to a manifold.}

\subsection*{38.}

In the preceding we have supposed that the manifold has three dimensions; but it is evident that the results obtained are valid for an arbitrary number of dimensions.

We have considered certain simple or multiple integrals whose element is a vector $d\m$ or $d\e_i$. More generally, we suppose that to each surface element of the manifold one may attach in an intrinsic manner a vector
\uequ{
\e_i \Pi^i;
}
this integral, extending over the infinitely-small closed surface, will be defined by \?{relating}{rapportant} each element of the integral to the affine space tangent to a point $\xa$ interior to the surface, which permits creating the sum of the different components\footnote{This is the same manner that, in chapter I, we have implicitly generalized the dynamical equations of continuous media for an arbitrary space-time affine connection.} In the same fashion it is shown above that the desired integral is equal to a volume integral whose element is
\uequ{
[d\e_i \Pi^i] + \e_i(\Pi^i)',
}
where $d\e_i$ is replaced by $\omega_i^k \cancel{d}e_k$. It would be the same if the $\Pi^i$ were the elements of triple or quadruple etc integrals, and if the integration was over a \textit{closed} manifold with 3, 4, ... dimensions.

One may also consider geometric forms of the second, third degree etc. A second degree form would be
\uequ{
[\m\e_i]\Pi^i + [\e_i \e_j] \Pi^{ij};
}
it would represent a system of (sliding) vectors. The exterior derivative of this form, which permits the reduction of an integral of $p+1$ dimensions to an integral over the form over a closed manifold with $p$ dimensions, would be
\uequ{[\m\e_1] \lbrace(\Pi^i) &+ [\omega_k^l \Pi^k]\rbrace \\
&+ [\e_i \e_j] \lbrace (\Pi^{ij}) + [\omega^i \Pi^j] - [\omega^j \Pi^i]
 + [\omega_k^l \Pi^{kj}] + [\omega_k^j \Pi^{ik}]\rbrace.
}
\subsection*{39.}

As an example, we take the form
\uequ{[\m d\m] = \omega^i [\m \e_i],
}
which represents the sliding vector with origin at $\m$ and endpoint $\m + d\m$. the integral of this vector extended over an infinitely-small closed contour is
\uequ{
\int [\m d\m] = \int\int [\m d\m]' = \int\int [\m\e_i]\Omega^i + 2 [\e_i \e_j][\omega^i \omega_j].
}

\textit{In the case $n+3$} one thus finds a vector with components $\Omega^i$ and a \textit{couple} whose moment is twice the area bounded by the contour; this is the generalization of the classical theorem.

Similarly,
\uequ{
\int\int  [\e_i \e_j][\omega^i \omega_j] = \int\int\int [\e_i \e_j] \left\{
[\Omega^i \omega^j] - [\omega^i \Omega^j]\right\};
}
this formula, in the affine space proper, is the condensation of the formulae
\uequ{
\int\int dx^i dx^j = 0,
}
the integrals stretching over a closed surface; manifolds with affine connections for which these formulae are carried over without further modification are those for which one has
\uequ{
[\omega^i \Omega^j] - [\omega^j \Omega^i]  = 0 \quad (i,j = 1,2,...,n).
}

It is easily shown that if $n\geq 4$, these relations entail $\Omega^i=0$.

We again note the formulae
\uequ{
\int\int \e_i\Omega^i &= \int\int\int \e_i [\omega^k \Omega_k^i],\\
\int\int [\m\e_i]\Omega^i &= \int\int\int [\m\e^i][\omega^k\Omega_k^i] 
+ [\e_i \e_j] \left\{ [\omega^i \Omega^j] - [\omega^j \Omega^i] \right\},\\
\int\int\int \e_i [\omega^k \Omega_k^i] &= \int\int\int\int \e_i[\Omega^k\Omega_k^i],\\
\int\int\int [\e_i \e_j]  \left\{ [\Omega^i \omega^j] - [\omega^i \Omega^j] \right\} &= 
\int\int\int\int [\e_i \e_j]  \left\{ [\omega^j \omega^k Omega_k^i] - [\omega^i \omega^k \Omega_k^j] \right\}.
}

These formulae have the natural advantage of leading to the formation of \?{\textit{integral invariants}}{invariants intégraux} of higher and higher degree attached to the manifold.

\section*{Isomorphism of two manifolds\footnote{Sections 40 to 43 may be skipped without harming the understanding of the following chapters}.}

\subsection*{40.}

\nc{\overomega}{\overline{\omega}}
\nc{\overA}{\overline{A}}

We say that two manifolds with affine connections with $n$ dimensions are \textit{isomorphic} if between the two manifolds a pointwise correspondence can be established with the following properties: for any choice of reference systems attached to the points of the first manifold, one may make a corresponding choice of reference system attached to corresponding points on the second manifold such that the components $\omega^i$ and $\omega_i^j$ relative to the first manifold become, for the pointwise correspondence under consideration, equal to the components $\omega^i$ and $\omega_i^j$ relative to the second manifold. one also says that the two manifolds are \?{\textit{applicable}}{applicables} upon one another.

We consider two isomorphic manifolds and attribute to an arbitrary point $\m$ of each of the manifolds as general a reference system as possible; on each manifold this reference system depends on $n(n+1)$ parameters, \?{of which $n$ coordinates are for the origin $\m$}{dont les n coordonées de l'origine m}. Between the $n(n+1)$ parameters of the first manifold and the $n(n+1)$ parameters of the second, there exists a correspondence such that one has the identities
\nequ{8}{
\overomega^i = \omega^i,\quad \overomega_i^j = \omega_i^j, 
}
designating by $\overomega^i$ and $\overomega_j^i$ the components relative to the second manifold. The reciprocal is true, since the first $n$ in the preceding equations  entails a necessary correspondence between the point coordinates of a point on the second manifold and the point coordinates of a point on the first.

The relations (8) imply, after exterior derivation,
\uequ{
\overOmega^i = \Omega^i, \quad \overOmega_i^j = \Omega_i^j,
}
and finally
\uequ{
\overA_{jk}^i = A_{jk}^i, \quad &\overA_{ikl}^j = A_{ikl}^j,\\
d\overA_{jk}^i = dA_{jk}^i, \quad &d\overA_{ikl}^j = dA_{ikl}^j,
}
so again the equality in the two members of the last formulae, taken to be expanded in functions linear in $\omega^i$ and $\omega_i^j$, of the respective coefficients, and thus of the whole.
\subsection*{41.}

The \?{development}{développement} of the differentials $dA_{jk}^i$ and $dA_{ikl}^j$ may be obtained by the following geometric procedure:

We consider a point $\m$ in the first manifold, and a two-dimensional element passing through this point; to this element corresponds an infinitely-small displacement of components $\Omega^i$ and $\Omega_i^j$. In particular, the translation associated with the element may be represented by the vector $\e_i\Omega^i$. We now consider a point $\m'$ infinitely close to $\m$ and the two-dimensional element passing through $\m'$ which is \textit{equivalent} to the element passing through $\m$; to this element corresponds a certain vector attached to $\m'$; the geometric difference between the second vector and the first, the geometric difference which is meaningful since it is known how to relate the affine space tangent to $\m'$ to the affine space tangent to $\m$, is a vector attached to $\m$ which has a geometric meaning independent of the chosen reference system.

It see the analytic expression, we imagine that the two-dimensional element passing through $\m$ is the parallelogram constructed from two infinitely-small vectors $(\xi^i)$ and $(\eta^i)$ issuing from $\m$; by
\uequ{
\e_i \Omega^i(\xi, \eta)
}
we designate the first vector representing the translation associated with this parallelogram. We now designate by $d$ the symbol for differentiation corresponding to passing from $\m$ to $\m'$. The vector which will represent the translation associated with the equivalent element issuing from $\m'$ may be designated by
\uequ{
\e_i \Omega^i + d\left[\e_i \Omega^i(\xi, \eta)\right],
}
and, if we apply the following formulae, which analytically translate the geometric conventions made above:
\uequ{
d\e_i &= \omega_i^k \e_k,\\
d\xi^i &= -\xi^k S\omega_k^i,\\
d\eta^i &= -\eta^k\omega_k^i.
}

The final vector attached to the point $\m$, \textit{or rather to the thee vectors $(\xi^i), (\eta^i), d\m$} issuing from $\m$, will thus be
\uequ{
d\left[\e_i\Omega^i(\xi, \eta)\right],
}
and its components will obviously be trilinear forms of the components of the three vectors, that is to say the $\xi^i$, the $\eta^i$ and the $\omega^i$, so
\uequ{
d\left[\e_i\Omega^i(\xi, \eta)\right] = A_{jklh}^i(\xi^j \eta^k - \eta^k \eta^j)\omega^h.
}

Performing the calculation and equating the terms in $\e_i(\xi^j \eta^k - \eta^k \eta^j)$, it is found that
\nequ{9}{
dA_{jk}^i + A_{jk}^\rho \omega_\rho^i - A_{\rho k}^i \omega_j^\rho - A_{j\rho}^i \omega_k^\rho = A_{jkl\rho}^i \omega^\rho,
}
which shows the form of the differential $d_{jk}^i$.

What we have done for the translation components can be done for the rotation components. We consider an elementary parallelogram at the point $\m$ constructed from two vectors $(\xi^i)$ and $(\eta^i)$ and an arbitrary vector $(u^i)$. The vector
\uequ{
\e_i u^j \Omega_j^i(\xi, \eta)
}
represents, up to a sign, the \WTF{geometrical increase}{accroissement géométrique} experienced by the vector $(u^i)$ when it is \?{transported equivalently}{transporté par équipollence} along the contour of the parallelogram. $\m'$ is a point infinitely close to $\m$; at this point we consider the parallelogram \textit{equivalent} to the first and the vector $(\overline{u}^i)$ equivalent to the vector $(u^i)$; the increase experienced by this latter vector when it is transported equivalently along the contour of the second parallelogram is a second vector
\uequ{
\overline{\e_i}\overline{u^j}\overline{\Omega_j^i}(\overline{\xi}, \overline{\eta});
}
the geometric distance between this vector and the first, which may be designated by
\uequ{
p(\e_i u^j \Omega_j^i(\xi, \eta),
}
is a vector attached to the point $\m$, depending on the $(u^i)$ and the three vectors $(\xi^i), (\eta^i), d\m$ in an intrinsic manner. This vector is of the form
\uequ{
\e_i u^j A_{jkl| h}^i (\xi^k \eta^l - \xi^l \eta^k) \omega^h.
}

Performing the calculation, one finds
\nequ{10}{
dA_{jkl}^i + A_{jkl}^\rho \omega_\rho^i - A_{\rho kl}^i \omega_j^\rho 
 - A_{j\rho l}^i \omega_k^\rho - A_{jk\rho}^i\omega_l^\rho 
= A_{jkl|\rho}^i \omega^\rho.
}
\subsection*{42.}

It follows from the preceding that differentiating the components $A_{jk}^l$ and $A_{jkl}^i$ introduces as the only new coefficients the coefficients $A_{jkl\rho}^i, A_{jkl | \rho}^i$ if the form
\uequ{
d\left(\e_i \Omega^i(\xi, \eta)\right), d\left(\e_i u^j \Omega_j^i(\xi, \eta)\right).
}

If now one replaces in these \?{expanded forms}{formes développées} the $\omega^i$ by the components $\zeta^i$ of an arbitrary vector, one obtains the forms
\uequ{
\e_i \Pi^i (\xi, \eta, \zeta), \e_i u^j \Pi_j^i(\xi, \eta, \zeta),
}
which together define a displacement associated with an elementary parallelepiped issuing from $\m$\footnote{Or rather a parallelogram and a vector.}; \textit{only the \?{ridges}{arêtes} of this parallelepiped do not play the same role}. In any case one may, by the same procedure as earlier, deduce the new forms
\uequ{
d\left(\e_i \Pi^i(\xi, \eta, \zeta)\right), d\left(\e_i u^j \Pi_j^i(\xi, \eta, \zeta)\right).
}
which, \?{expanded}{développées}, will be of the form
\uequ{
\e_i A_{jk | hl}^i \xi^j \eta^k \zeta^h \omega^l,
\e_i u^j A_{jkl | hm}^i \xi^k \eta^l \zeta^h \omega^m.
}

The new coefficients which are introduced in this way permit writing the complete expressions of the differentials of $A_{jk| h}^i$ and $A_{jkl| h}^i$:
\nequ{11}{
dA_{jk | h}^i + A_{jk|h}^\rho \omega_\rho^i - A_{\rho k|h}^i \omega_j^\rho
 - A_{j\rho|h}^i \omega_k^\rho - A_{jk|\rho}^i\omega_h^\rho
 = A_{jk|h\rho}^i \omega^\rho,\\
dA_{jkl | h}^i + A_{jkl|h}^\rho \omega_\rho^i - A_{\rho kl|h}^i \omega_j^\rho
 - A_{j\rho l|h}^i \omega_k^\rho - A_{jk\rho|h}^i\omega_l^\rho
 - A_{jkl|\rho}^i \omega_h^\rho = A_{jkl|h\rho}^i \omega^\rho.
}

Obviously these operations may be pursued indefinitely; the new coefficients which are introduced by each differentiation are the coefficients of the forms which one may give a geometrical significance and which represent the displacements associated with the arbitrary vectors of larger and larger numbers.  
\subsection*{43.}

If two manifolds are isomorphic, the correspondence which realizes the isomorphism of these two manifolds makes all of the coefficients $A_{jk}^i, A_{jkl}^i$ and their derivatives of all orders equal to one another. I do not intend to pursue further the study of this problem which is treated following the method that I have employed in a preceding memoir\footnote{\textit{Sur les équations de la gravitation d'Einstein} (\textit{Journal de Math.}, 1922, p. 141-203).}. I will be content to indicate the relations that necessarily exist between the quantities which have bene considered, omitting the demonstration of the fact that the relations that I have indicated are the only ones which exist in the general case.

First of all, the formulae which give the theorems of conservation of curvature and torsion
\nequ{7}{
(\Omega^i)' + [\Omega^k \omega_k^i] - [\omega^k \Omega_k^i] &= 0,\\
(\Omega_i^j)' + [\Omega_i^k \omega_k^j] - [\omega_i^k \Omega_k^i] &= 0
}
give us, expanding and ignoring all the terms which contain the $\omega_i^j$,
\uequ{
[dA_{jk}^i\omega^j\omega^k] + A_{jk}^i[\Omega^j\omega^k - \omega^j\Omega^k]
 - A_{k\rho\sigma}^i[\omega^k\omega^\rho\omega^\sigma] = 0,\\
[dA_{ikl}^j\omega^k\omega^l] + A_{jkl}^i[\Omega^k\omega^l - \omega^k\Omega^l] = 0,
}
from which, taking the ensemble of terms in $[\omega^\alpha \omega^\beta \omega^\gamma]$,
\nequ{12}{
A_{\alpha\beta|\gamma}^i &+ A_{\beta\gamma|\alpha}^i + A_{\gamma\alpha|\beta}^i + A_{\rho\gamma}^i A_{\alpha\beta}^i\\
 &+ A_{\rho\alpha}^i A_{\beta\gamma}^\rho + A_{\rho\beta}^i A_{\gamma\alpha}^\rho
  - A_{\alpha\beta\gamma}^i - A_{\beta\gamma\alpha}^i 
 - A_{\gamma\alpha\beta}^i = 0,\\
A_{i\alpha\beta|\gamma}^j &+ A_{i\beta\gamma|\alpha}^j
 + A_{i\gamma\alpha|\beta}^j + A_{i\rho\gamma}^i A_{\alpha\beta}^\rho
 + A_{j\rho\alpha}^i A_{\beta\gamma}^\rho + A_{j\rho\beta}^i A_{\gamma\alpha}^\rho = 0.
}

As regards the $A_{jk|lh}^i$ and the $A_{jkl|hm}^i$, one will have between these quantities and the preceding relations of two different types. The first comes from the differentiation of the relations (12), only conserving the terms in $\omega^\delta$; they are of the form
\nequ{13}{
A_{\alpha\beta|\gamma\delta}^i + A_{\beta\gamma|\alpha\delta}^i
 + A_{\gamma\alpha|\beta\delta}^i = ...,\\
A_{i\alpha\beta|\gamma\delta}^j + A_{i\beta\gamma|\alpha\delta}^j
 + A_{i\gamma\alpha|\beta\delta}^j = ...,
}
the unwritten terms depending only upon the quantities previously considered. The relations of the second type come from equating the bilinear covariants of the first parts of (9) and (10); by only conserving the terms $[\omega^\alpha \omega^\beta]$ in the resulting equality, one finds
\nequ{14}{
A_{jk|\beta\alpha}^i - A_{jk|\alpha\beta}^i = A_{jk}^\rho A_{\rho\alpha\beta}^i
 - A_{\rho k}^i A_{j\alpha\beta}^\rho - A_{j\rho}^i A_{k\alpha\beta}^\rho
 - A_{jk|\rho}^i A_{\alpha\beta}^\rho,\\
A_{jkl|\beta\alpha}^i - A_{jkl|\alpha\beta}^i = A_{jkl}^\rho A_{\rho\alpha\beta}^i
 - A_{\rho kl}^i A_{j\alpha\beta}^\rho - A_{j\rho l}^i A_{k\alpha\beta}^\rho
 - A_{jk\rho}^i A_{i\alpha\beta}^\rho - A_{jkl|\rho}^i A_{\alpha\beta}^\rho.
}

The relations between the $A_{jk|lhm}^i$ amd the $A_{jklh|mn}^i$ are similarly deduced:
\begin{enumerate}
\item The relations (13) and (14), differentiated conserving only the terms in $\omega^\epsilon$;
\item The exterior derivatives of relations (11), conserving only the terms in $[\omega^\alpha \omega^\beta]$.
\end{enumerate}

One proceeds this way step by step.

It is easily seen that the number of components of the curvature and the torsion is $\frac{n^2(n^2-1)}{2}$ and that of their first-order derivatives is $\frac{n^2(n^2-1)(n+1)}{3}$.

\textit{Case of manifolds without torsion.} -- If the manifold is without torsion, the preceding relations are simplified; they are reduced to the following:
\uequ{
A_{i\alpha\beta|\gamma}^j + A_{i\beta\gamma|\alpha}^j + A_{i\gamma\alpha|\beta}^j &= 0,\\
A_{i\alpha\beta|\gamma\delta}^j + A_{i\beta\gamma|\alpha\delta}^j + A_{i\gamma\alpha|\beta\delta}^j &= 0,\\
A_{jkl|\beta\alpha}^i - A_{jkl|\alpha\beta}^i = A_{jkl}^\rho A_{\rho\alpha\beta}^i
 - A_{\rho kl}^i A_{j\alpha\beta}^\rho - A_{j\rho l}^i A_{k\alpha\beta}^\rho
 - A_{jk\rho}^i A_{i\alpha\beta}^\rho.
}

\section*{Generalization}\footnote{The end of this chapter assumes that the reader has certain knowledge about the theory of groups; it may be skipped without harming the comprehension of the following chapters.}

\subsection*{44.}

The equations (3) and (5') define the infinitesimal transformations of the group of affine displacements. It is simple to generalize the considerations developed in the preceding sections, which will permit a better understanding of the general case.

We consider an arbitrary finite continuous group $G$ with $n$ variables $x_1, x_2, ..., x_n$, this group being, for example, defined by $r$ independent infinitesimal transformations
\uequ{
X_1 f, X_2 f, ..., X_r f.
}

We may regard the $x_i$ as the coordinates of a point in a certain soace $(E)$. If in this space we only pay attention to those properties of figures which are not altered by transformations of the group $G$, we may say that the space $(E)$ admits the group $G$ as the fundamental group. There one may substitute for the primitive coordinates $x_1, ..., x_n$ those that are induced by a transformation $T$ in the group; with this new coordinate system the properties of figures are analytically carried over in the same way as in the old; the two coordinate systems are equivalent. The passage from one coordinate system to the other equivalent one is thus carried out by a transformation from the fundamental group.

With this, we imagine a continuous ensemble of \WTF{observers}{observateurs}, reduced to points, and each has adopted a coordinate system for the study of the space $(E)$, these systems being naturally completely equivalent to one another. The manifold formed by these observer-points has, I suppose, $p$ dimensions, each point being defined in an arbitrary manner by $p$ coordinates $u_1, ..., u_p$. If one passes from a point $\m$ on the manifold to an infinitely-near point $\m'$, one passes in the space $(E)$ from a certain coordinate system to another which we suppose to be infinitely close; put another way, one passes from the coordinates $x_i$ utilized for the observer $\m$ to the coordinates $x_i'$ utilized by the observer $\m'$ by effecting a certain infinitesimal transformation in the group $G$, \WTF{so}{soit}
\uequ{
\omega_1 X_1 f + \omega_2 X_2 f + \dots + \omega_r X_r f,
}
designating by $\omega_1, \dots, \omega_r$ the expressions linear in $du_1, \dots, du_p$, with the coefficients being functions of $u_1, \dots, u_p$. Put another way, if $(x_i)$ and $(x_i + dx_i)$ are respectively the coordinates for \textit{one} point in $(E)$ by the observer are $\m$ and $\m'$, one will have
\nequ{15}{
dx_i = \omega_1 X_1 (x_1) + \omega_2 X_2 (x_2) + \dots + \omega_r X_r (x_r);
}
or again, if $f$ is a \textit{definite} function of the coordinates, one will have
\nequ{15'}{
df = \omega_1 X_1 f + \omega_2 X_2 f + \dots + \omega_r X_r f.
}

Suppose, for example, that $G$ is the group with one variable and one parameter
\uequ{
x' = x + a;
}
then
\uequ{
dx = \omega,
}
$\omega$ being a Pfaffian expression in $u_1, u_2, \dots, u_p$. 
\subsection*{45.}

That being the case, imagine that in the manifold $(V)$ of observers, one goes from an arbitrary point $\m_0$ to an arbitrary point $\m_1$ by a certain path. Passing from an arbitrary point $\m$ of this latter to an infinitely-close point $\m'$, one will have effected in the space $(E)$ an infinitesimal transformation whose components, \textit{with the coordinates adopted by the observer $\m$}, are $\omega_1, \omega_2, \dots, \omega_r$. The composition of all these successive infinitesimal transformations will permit passing from the coordinates adopted by the observer $\m_0$ to the coordinates adopted by the observer $\m_1$; this will be a certain finite transformation in the group $G$. It will obviously be furnished by integrating the differential equations (15'), where the $\omega_i$ are now of the form $p_i(t) dt$, taking $t$ as a parameter in a function with which the coordinates $u_1, \dots, u_p$ of a variable point on the path.

If now a \textit{closed} contour is described on the manifold $(V)$, ones not necessarily retrieve the same coordinate system when returning to the point $\m_0$ as at the start. In order to recover the same, it would be necessary (and sufficient) for the equations (15) to be completely integrable. We take these equations in the condensed form (15'). The condition for complete integrability being given, designating the bilinear covariant of $\omega_i$ by $\omega'$,
\uequ{
X_i f \omega_i' + \left[ d(X_i f) \omega_i \right] \equiv
X_i f \omega_i' + \left[ X_h (X_k f) - X_k (X_h f) \right] [ \omega_h \omega_k] = 0.
}

Or, following a classical theorem by S. Lie,
\uequ{
(X_h X_k) \equiv X_h (X_k f) - X_k ( X_h f) = c_{hks} X_s f,
}
the coefficients $c_{hks}$ being constant. One must then have
\uequ{
X_s f \left\{ \omega'_s + c_{hks} [\omega_h \omega_k] \right\} = 0,
}
and finally
\nequ{16}{
\omega_s' + c_{hks} [\omega_h \omega_k] = 0.
}
\subsection*{46.}
\nc{\opT}{\mathbb{T}}

The particular circumstance being studied will come forth particularly if we take as the manifold $(V)$ \textit{the manifold of parameters} of the group. If
\uequ{
x_i' = f_i(x_1, \dots, x_n; a_1, \dots, a_r)
}
are the \?{final}{finies} equations of the most general transformation $T_a$ of the group $G$, $a_1, \dots, a_r$ may be regarded as the coordinates of a point $\m$ of a manifold $(V)$ with $r$ dimensions. If $a_0^1, \dots, a_r^0$ are the parameters of the identity transformation and if $\m_0$ is the corresponding point of $(V)$, we assume that the system of coordinates adopted by the observer $\m$ \?{is recovered}{se déduit} by the transformation $\opT_a$ of the coordinate system adopted by the observer $\m_0$. One then passes from the coordinate system at $\m$ to that at the infinitely close point $\m'$ by the infinitesimal transformation $\opT_a^{-1}\opT_{a+da}$. It is then obvious enough that, \textit{whatever path is followed} to get from a point $\m$ with coordinates $(a_1, \dots, a_r)$ to a point $\m'$ with coordinates $(b_1, \dots, b_r)$, the resulting change of coordinates in the space $(E)$ will be $\opT_a^{-1}\opT_b$.

Reciprocally, if the relations (16) are verified for an arbitrary manifold $(V)$ of observers, and if we arbitrarily choose in this manifold a point $\m_0$, each point $\m$ will correspond to a definite transformation $\opT_a$ of the group $G$, which permits passing from the chosen coordinates for $\m_0$ to the coordinates chosen for $\m$; finally, the $u_i$ are the \?{determined}{déterminées} functions of $a_i$. It may naturally also be \?{the same}{un même} transformation $\opT_a$ corresponding to an infinity of points $\m$; it may also be, particularly if $p<r$, that only a part of the transformations correspond to different points of the manifold.
\subsection*{47. Return to the general case.}

\nc{\opS}{\mathbb{S}}

So far we have interpreted a transformation of the group $G$ as defining a change of coordinates in the space $(E)$, the primitive variables $x_i$ and the transformed variables $x_i'$ relating to \textit{the same point} in the space $(E)$. One could adopt a different, and in a certain sense more intuitive point of view and regard the $x_i$ and $x_i$ as the coordinates of \textit{two different points} in \textit{the same coordinate system}. We designate this transformation by the letter $\opS$, interpreted in this manner: this transformation is obviously only meaningful if one has made once and for all a prior choice of coordinate system. We say that this is a \textit{displacement} of the space $(E)$.

Taking this point of view, the infinitesimal transformation $\omega_1 X_1 f + \dots + \omega_r X_r f$ that we have associated to a couple of two infinitely close points $\m$ and $\m'$ on the manifold may be regarded as an \textit{infinitely small displacement} in the space $(E)$. The Pfaffian expressions $\omega_1, \dots, \omega_r$ thus together define a continuous (ideal) motion with $p$ parameters on the space $(E)$. Employing an awkward expression from the \?{theory of links} in mechanics, we may say that in general \textit{this motion is not holonomic}; it is holonomic if the conditions (16) hold. For example, in the case of the group
\uequ{
x' = x + a,
}
we are dealing with a continuous translation on a line; this translation is holonomic if the component $\omega$ of the infinitely small translation is an exact differential; it is not in the contrary case.
\subsection*{48.}

We return to an \textit{infinitely-small} closed contour on the manifold $(V)$. The approximate integration of equations (15) may be made by a procedure analogous to that which served in the study of manifolds with affine connections. Calling $\Delta x_i$ the infinitely-small variation experienced by $x_i$ , it is seen that this variation may be obtained by taking the bilinear covariant of the second part, replacing there the differentials $dx_k$ by their expressions furnished by the equations (15) themselves. One thus obtains, in a condensed form,
\nequ{17}{
\Delta f = \Omega_1 X_1 f + \Omega_2 X_2 f + \dots \Omega_r X_r f,
}
by. putting
\nequ{18}{
\Omega_s = \omega_s' + c_{hks}[\omega_h \omega_k].
}

\textit{To each closed contour is thus associated an infinitely-small displacement of the space $(E)$, a displacement whose components $\Omega_1,\dots,\Omega_r$ are the elements of a double integral.} This displacement measures in some sense the \textit{non-holonomy} of motion in the $(E)$.

Taking the exterior derivative of the formulae (18) gives, taking account of the relations which exist between the constants $c_{khs}$\footnote{These relations result from the fact that, for the manifold $(V)$ of parameters, the formulae (13) are true and thus those that can be deduced from exterior derivation. This means that, in taking the exterior derivative of the formulae (15), one may suppress all of the terms that contain neither the $\Omega_i$ nor the $\Omega_i$.},
\nequ{19}{
\Omega'_s = c_{khs}\left\{[\Omega_h \omega_k - \omega_h \Omega_k ]\right\}.
}

The interpretation of these formulae is analogous to that which we have obtained in the theory of manifolds with an affine connection.
\subsection*{49.}

\nc{\opU}{\mathbb{U}}
\nc{\opV}{\mathbb{V}}
\nc{\overu}{\overline{u}}

For this we return to our first point of view. We imagine in the manifold $(V)$ a volume delimited by an infinitely-small closed surface. Associated to each element of this surface enclosing a cergain point $\m$ is the infinitesimal transformation $\Theta$, \?{in symbols}{de symbole}
\uequ{
\Omega_1 X_1 f + \dots + \Omega_r X_r,
}
which we may interpret as defining a \textit{displacement} $S$ in the space $(E)$, defined analytically by using the coordinates adopted by the observer at $\m$. Then let $\xa$ be a fixed point inside the volume under consideration. The \textit{same} displacement of the space $(E)$ may be expressed analytically by means of the coordinates adopted by the obsver $\xa$, \?{since it is known how to pass}{puisqu'on sait passer} from one to the other when the two observers are infinitely close. Let
\uequ{
x_1, \dots, x_n, \textit{ and } \overx_1, \dots, \overx_n,
}
be the coordinates adopted by $\m$ and $\xa$ respectively. The passage from the latter to the former is made via an infinitesimal transformation $\opT$.

Finally we designate by primed letters the coordinates of the point transformed by the displacement $S$. It is immediately seen that the passage from $\overx$ to $\overx'$ will be obtained by applying the transformations $\opT, \Theta, \opT^{-1}$ successively. Put another way, the displacement $S$ will be defined by adopting the coordinates of the fixed observer $\xa$, via the transformation $\opT\Theta\opT^{-1}$.

Then it is easily demonstrated that if $\opU f$ is the symbol of the infinitesimal transformation $\opT$, $\opV f$ that of $\Theta$, the symbol for $\opT\Theta\opT^{-1}$ is
\uequ{
\opV f + (\opU\opV) = \opV f + \opU(\opV f) - \opV(\opU f).
}

If one calls $u_h$ and $\overu_h$ the respective coordinates of $\m$ and $\xa$ on the manifold $(V)$, and putting
\uequ{
\omega_i = \gamma_{ih} du_h,
}
we have
\uequ{
\opU f = \gamma_{ih} (u_h - \overu_h)X_i f,\\
\opV f = \Omega_i X_i f,
}
and, finally.
\uequ{
\opV + (\opU\opV) = \left\{\Omega_s + c_{iks} \left[
\gamma_{ih}(u_h - \overu_h)\Omega_k -
 \gamma_{kh}(u_h - \overu_h)\Omega_i\right]\right\} X_s f.
}
This is the analytic expression for the displacement $S$ when it is related back to the reference system of the point $\xa$.

Taking the \textit{sum} of the components of the displacement, over all the elements of the closed surface under consideration in the manifold $(V)$, it is found, for the $s^\text{th}$ sum,
\uequ{
\Omega_s' + c_{iks} \left([\omega_i \Omega_k] - [\omega_k \Omega_i] \right),
}
\subsection*{50.}

In the particular case where the group $G$ is the group
\uequ{
x'=x+a,
}
the preceding general conservation theorem leads to a classical result. If $\omega$ is any Pfaffian expression on the manifold $(V)$, and if by $\omega'$ is designated the element of the double integral given by the application of Stokes' formula, the integral $\int\int\omega'$ extending over a closed surface is zero. Here there is no need to relate the infinitesimal translation of magnitude $\omega'$ on the reference system of the observer $\xa$, since the magnitude if a translation of a line is independent of the origin chosen for the axes, the transformation of the group $G$ being \textit{commutative}. In this particular case, it is again seen that the translation associated with any closed contour, \?{even a finite one}{même fini}, is always given by the integral $\int\int\omega'$ stretching over an area delimited by the contour.

We now return to the notion of a manifold with an affine connection. The space $(E)$ is here the affine space proper, the fundamental group is the group of affine transformations generated by the infinitesimal transformations
\uequ{
\frac{\partial f}{\partial x^i}, \quad x^i \frac{\partial f}{\partial x^j}.
}

There is now however a complete identity between the notion of a manifold with an affine connection and the notion of a manifold $(V)$ of observers introduced in the preceding considerations, since to each point $\m$ of the manifold with an affine connection there may correspond an infinity of observers adopting different reference systems, \textit{but with the same origin $\m$}.

\chapter{Manifolds with a metric connection.}

\subsection*{51.}

We imagine a Euclidian space at each point $\m$ on which a rectangular coordinate system is adopted, with a length unit chosen at each point and which may vary from one point to another. Naming $\e_1$, $\e_2$, $\e_3$ equal to the unit of length \?{along}{portés sur} the axes, we have, passing from a point $\m$ to the infinitely close point $\m+d\m$, the generalized formulae (I) (chapter II)
\nequ{1}{
d\m &= \omega^1\e_1 + \omega^2 \e_2 + \omega^3 \e_3,\\
d\e_1 &= \omega_1^1\e_1 + \omega_1^2 \e_2 + \omega_1^3 \e_3,\\
d\e_2 &= \omega_2^1\e_1 + \omega_2^2 \e_2 + \omega_2^3 \e_3,\\
d\e_3 &= \omega_3^1\e_1 + \omega_3^2 \e_2 + \omega_3^3 \e_3.
}

Defining the scalar product of two vectors by taking a certain \textit{absolute} unit of length, we have
\uequ{
\e_1 \e_1 = \e_2 \e_2 = \e_3 \e_3, \quad \e_2 \e_3 = 0, \quad \e_3 \e_1 = 0, \quad \e_1 \e_2 = 0.
}

Differentiating these relations, the following formulae are obtained
\uequ{
\omega_1^1 = \omega_2^2 = \omega_3^3, \quad 
\omega_2^3 + \omega_3^2 = 0, \quad
\omega_1^3 + \omega_3^1 = 0, \quad
\omega_1^2 + \omega_2^1 = 0.
}

In place of $\omega_i^i$ we simply write $\omega$. The formulae (I) thus become
\nequ{1'}{
d\m &= \omega^k \e_k,\\
d\e_i &= \omega \e_i + \omega_i^k \e_k \quad (\omega^\text{unintelligible}_\text{unintelligible} + \omega_j^i) = 0,
}
where in the second relation the summation is taken for the values of the index $k$ different from $i$.
\subsection*{52.}

These formula permit us to define \textit{manifolds with a metric connections}. These are manifolds with affine connections where each at point $\m$ the tangent space is a Euclidian space. The mutual identification of two tangent Euclidian spaces at two infinitely close points is made by means of the Pfaffian expressions
\uequ{
\omega^i, \quad \omega, \quad \omega_i^j = -\omega_j^i;
}
the $\omega^i$ are the components of the \textit{translation}, $\omega$ is the component of the \?{\textit{dilation}}{homothétie}\footnote{\?{The relation}{Le rapport} of the dilation is $1 + \omega$.} and the $\omega_i^j$ are the components of the \textit{rotation} which brings the reference system attached to the point $\m$ into coincidence with the reference system attached to the point $\m'$.

It may also be imagined that there exists an absolute unit of length valid for all of the Euclidian spaces tangent to the manifold; the unit vectors $\e_1, \e_2, \e_3$ attached to each point being by convention all equal to one another. In this case, the form $\omega$ will be null. In this case, we say that we have a \textit{manifold with a Euclidian connection}. 
\subsection*{53.}

All of the preceding may be generalized to the case of an arbitrary number of dimensions. It may also be supposed that a non-rectangular reference system has been chosen, or where the rectangular reference systems with the vectors $\e_i$ are not equal to one another. Without treating the general case, we confine ourselves to indicating what happens when, a unit of length being chosen at each point $\m$, the reference vectors $\e_i$ are taken to satisfy
\uequ{
(\e_i)^2=g_{ij}, \quad \e_i \e_j = 0, \quad (i\neq j = 1,2,\dots,n),
}
the $g_{ij}$ being \textit{constant} coefficients chosen once and for all and the same once and for all the points of the manifold\footnote{It is obvious that this convention does not harm this generality at all; it may even be supposed that these coefficients are equal to $\pm 1$.}. Since the unit of length varies from one point to the other on the manifold, \?{one may simply differentiate}{one aura simplement le droit de différentier} the relations
\uequ{
\frac{(\e_1)^2}{g_{11}} = \frac{(\e_2)^2}{g_{22}} = \dots = \frac{(\e_n)^2}{g_{nn}}, \quad \e_i \e_j = 0,
}
which will give, taking account to equations (1),
\nequ{2}{
\omega_1^1 = \omega_2^2 = \dots = \omega_n^n, \quad
g_{ii} \omega_j^i + g_{jj} \omega_i^j = 0.
}

The square of the length of a vector $(\xi^i)$ issuing from $\m$ is, under these conditions,
\uequ{
g_{ii} (\xi^i)^2;
}
it varies with the unit of length $\m$ chosen at $\m$. The scalar product of two vectors $(\xi^i), (\eta^i)$ is similarly
\uequ{
g_{ii}\xi^i \eta^i.
}
\subsection*{54.}

Being given a vector $(\xi^i)$, we designate by $\xi_i$ the scalar product of this vector with the vector $\e_i$,
\uequ{
\xi_i = \xi \e_i = g_{ii} \e_i;
}
we say that the $\xi_i$ are the \textit{covariant} components of the vector\footnote{It is necessary to remark that the $g_{ii}$, being constants fixed once and for all, the covariant components of a vector are tied to their contravariant components in a manner independent of the choice of reference system.}, the $\xi^i$ being called its contravariant components. With this notation, the square of the length of a vector is $\xi_i \xi^i$ and the scalar product of two vectors is $\xi_i \eta^i$ or $\xi^i \eta_i$. The $ds^2$ of the manifold (square of the distance between two infinitely-near points) is similarly $\omega_i \omega^i$, putting
\uequ{
\omega_i = g_{ii} \omega^i.
}

We naturally designate by $\Omega_i$ the covariant components of the vector $\Omega^i$.

The form $\omega_i^j$ may be regarded as the $j^\text{th}$ (contravariant) component of the vector $d\e_i$; we put
\uequ{
\omega_{ij} = d\e_i \dot \e_j = g_{jj} \omega_i^j,
}
and similarly
\uequ{
\omega_{ji} = d\e_j \dot \e_i = g_{ii} \omega_j^i;
}
the relations (2) show that
\nequ{2'}{
\omega_{ij} + \omega_{ji} = 0.
}

We similarly put
\uequ{
\Omega_{ij} = g_{jj} \Omega_i^j = (d\e_i)' \e_j, \quad &
\Omega_{ji} = g_{ii} \Omega_j^i = (d\e_j)' \e_i, \quad;\\
\Omega^{ij} = \frac{1}{g_{ii}} \Omega^j_i, \quad &
\Omega^{ji} = \frac{1}{g_{jj}} \Omega^i_j.
}
\subsection*{55. Structure equations.}

These are immediately deduced from the general equations established in Chapter I. These are written
\nequ{3}{
(\omega^i)' &= [\omega^i \omega] + [\omega^k \omega_k^i] + \Omega^i,\\
\omega' &= \Omega,\\
(\omega_i^j)' &= [\omega_i^k \omega_k^j] + \Omega_i^j,
}
and between these $\Omega_i^j$ one has the relations
\uequ{
g_{ii} \Omega_j^i + g_{jj} \Omega_i^j =0,
}
or
\nequ{4}{
\Omega_{ij} + \Omega_{ji}=0.
}
The equations (3) may also be written
\nequ{3'}{
(\omega_i)' &= [\omega_i \omega] + [\omega_i^k \omega_k] + \Omega_i,\\
\omega' &= \Omega,\\
(\omega_{ij})' &= [\omega_i^k \omega_{kj}] + \Omega_i^j
 = [\omega_{ki} \omega_j^k] + \Omega_{ij}.
}

Finally, the equations \?{bring over}{traduisent} the conservation theorem are
\nequ{5}{
(\Omega^i)' + [\omega\Omega^i] - [\omega^i \Omega]
 + [\omega_k^i \Omega^k] - [\omega^k \Omega_k^i] = 0,\\
\Omega'=0,\\
(\Omega_i^j)' + [\omega_k^j \Omega_i^k] - [\omega_i^k \Omega_k^j] = 0,
}
or
\nequ{5'}{
(\Omega_i)' + [\omega\Omega_i] - [\omega_i \Omega]
 + [\omega_i^k \Omega^k] - [\omega_k \Omega_i^k] = 0,\\
\Omega'=0,\\
(\Omega_{ij})' + [\omega_j^k \Omega_{ik}] - [\omega_i^k \Omega_{kj}] = 0,
}
\subsection*{56. Torsion and curvature.}

To each infinitely-small closed contour is associated:

\begin{enumerate}
	\item A translation $\e_i\Omega^i$;
	\item A dilation by $1 + \Omega$;
	\item A rotation with components $\Omega_i^j$.
\end{enumerate}

On the manifold, the translation defines the \textit{torsion}, the dilation the \textit{dilation curvature}, the rotation the \textit{rotation curvature}.

\textit{If the dilation curvature is zero, the manifold has a Euclidian connection.} One may satisfactorily dispose of the arbitrariness depending on the choice of reference system with the equation
\uequ{
\omega = 0;
}
this equation is in fact completely integrable, since the first part, having its bilinear covariant $\omega'$ identically zero, is an exact differential. The reference system attached to the different points $\m$ of the manifold may be chosen in such a manner that the connection becomes Euclidian: this is possible in an infinity of ways, the choice of a unit of length (which is now absolute) being arbitrary\footnote{Analytically, replacing $\e_i$ by $u\e_i$,where $u$ is an arbitrary parameter, $\omega$ is replaced by $\omega + \frac{du}{u}$. To make the connection Euclidian, just take $u=C\exp{-\int\omega}$.}
\subsection*{57. Geometric representation of tge rotation associated with an infinitely-small closed contour.}

The rotation of the components $\Omega_i^j$ may be represented geometrically. For this we take the formulae which give the geometric variation $\Delta\xi$ that performs this rotation on a vector $\xi$; there is
\uequ{
\Delta \xi^i = \xi^k \Omega_k^i,
}
or, what amounts to the same thing,
\uequ{
\Delta\xi^i = \xi_k \Omega^{ki}.
}
This variation is the geometric sum of the $\frac{n(n-1)}{2}$ variations due to the different components $\Omega^{ij}$ of the total rotation, which again permits regarding the total rotation as the sum of $\frac{n(n-1)}{2}$ component rotations. We take for example the rotation component $\Omega^{12}$; one has
\uequ{
\Delta\xi^1 &= \xi_2 \Omega^{21},\\
\Delta\xi^2 &= \xi_1 \Omega^{12},\\
\Delta\xi^3 &= \dots = \Delta\xi^n = 0;
}
this component of the rotation does not alter the vectors perpendicular to the plane $\e_1\e_2$; for a vector that is not perpendicular to this plane, its normal component is not altered, the rotation's effect is only felt on the projection of the vector onto the plane.

Consider in particular the vector $\e_1$ with components $(1,0)$; after the rotation it becomes the vector with components $(1, g_{11}\Omega_{12})$
\uequ{
\e_1' = \e_1 + g_{11}\Omega^{12}\e_2;
}
let $\theta$ be the angle this vector has rotated, taken to be positive in the direction from $\e_1$ towards $\e_2$; then
\uequ{
\e_1'\e_2 = |\e_1||\e_2|\theta = g_{11}\Omega^{12}|\e_2|^2,
}
from which
\uequ{
\theta = g_{11}\Omega^{12} \frac{|\e_2|}{|\e_1|}.
}

Finally, consider the bivector on the $\e_1\e_2$ plane with algebraic measure $\theta$; this bivector is\footnote{If in particular one considers an infinitely small area $d\sigma$ in the $\e_1\e_2$ plane, the contour everywhere being in the \?{positive direction}{sens direct}, the ratio $\frac{\Omega^{12}}{d\sigma}$ defines the \textit{curvature} of the plane element $\e_1\e_2$.}
\uequ{
\frac{[\e_1\e_2]\theta}{|\e_1\e_2|} = g_{11}\frac{\Omega^{12}}{|\e_1|^2}[\e_1\e_2]\Omega^{12}.
}

Thus, \textit{the rotation $\Omega_i^j$ may be represented by the system of bivectors}
\nequ{6}{
[\e_i\e_j]\Omega^{ij},
}
each term defining one of the $\frac{n(n-1)}{2}$ component rotations.

It is essential to remark that this representation has an intrinsic value if the manifold has a \textit{Euclidian} connection; if on the contrary the manifold has a metric connection, this system of bivectors depends on the choice of unit of length attached to the point $\m$.

One could say that, in the case of manifolds with Euclidian connections, the expression (6) is a \textit{bivectorial integral invariant} attached to each two-dimensional element on the manifold.
\include{subfiles/58}

\end{document}

%