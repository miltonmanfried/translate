\subsection*{1}
Classical mechanics, as it was founded by Newton, rests on the notion of an absolute time and an absolute space; all events are localized analytically in time by the choice of a time-origin and a unit of time, in space by the choice of a fixed coordinate system, for example having as its origin the center of gravity of the solar system and the axes directed towards the fixed stars; it is naturally permissible to take any other system of axes, provided that it is invariably \WTF{linked to the first}{lié au premier}. As is known, the laws of mechanics remain true if, always maintaining the notion of absolute time, one admits coordinate axes that are \textit{mobile} with respect to Newton's absolute space, with the condition that the axes move, with respect to the absolute space, with a uniform rectilinear translation. One thus arrives at the notion of the \textit{Galilean reference system}.

The principle of inertia is enunciated in the following manner: \textit{A material point, \WTF{neglecting}{soustrait à} the action of all other bodies, moves in such a manner that its velocity, determined with respect to a Galilean reference system, is constantly equivalent to itself}. If the principle of inertia is valid for one Galilean reference system, it is valid for all others.

Analytically, this is evident, if one uses the formulae for passing from one Galilean reference system to another. In the general case, to localize a point in space, we make use of arbitrary cartesian coordinates, rectangular or not\footnote{The laws of classical mechanics retain exactly the same form in oblique coordinates as in rectangular coordinates, and the formulae of \textit{theoretical} mechanics are exactly the same.}. The transformation formulae are
\nequ{1}{
x' &= a_1 x + b_1 y + c_1 z + g_1 t + h_1, \\
y' &= a_2 x + b_2 y + c_2 z + g_2 t + h_2, \\
z' &= a_3 x + b_3 y + c_3 z + g_3 t + h_3, \\
t' &= t + h,
}

with constant coefficients.

One may enunciate the principle of inertia in another manner. By "\textit{equivalent} reference  systems" we mean two reference systems consisting of two triads of coordinates which are equivalent in the ordinary geometric sense of the word, and supposed to be immobile with respect to one another; analytically, the formulae for passing from one reference system to an equivalent system are of the form
\uequ{
x' &= x + h_1,\\
y' &= y + h_2,\\
z' &= z + h_3,\\
t' &= t + h.
}

Given this, we attach to a mobile material point, at each instance of its motion, a Galilean reference system having as its origin the point itself\footnote{This means that the point is the origin of the coordinate axes and that the instant that one observes it is taken as the origin of time.}. The principle of inertia is then enunciated as:

\textit{If a material point is isolated from the action of all other bodies and one attaches to it at each instant a Galilean reference system always equivalent to the point itself, the components of its velocity, determined at each instant with respect to the corresponding reference system, are constant.}

% 