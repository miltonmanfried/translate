\subsection*{2}
It is seen that the structure of classical mechanics rests on two notions:
\begin{enumerate}
    \item The notion of a Galilean reference system (which permits the definition of the velocity of a moving point);
    \item The notion of equivalent Galilean reference systems (which permits the statement of the principle of inertia).
\end{enumerate}

It is essential to remark that the advantage of the second formulation of the principle of inertia is that \textit{it only requires utilization of the notion of equivalent Galilean systems for two systems with infinitesimally-close origins}.

In all of the generalizations which have been made in classical and relativistic mechanics, the notion of a Galilean reference system has remained unshaken; it is the notion of equivalent reference frames that has suffered fundamental modifications.

We shall always remain in the realm of classical mechanics, with the notion of absolute time (measured with a unit of time fixed once and for all). We shall see that a modification of the notion of equivalent reference systems permits us to extend the principle of inertia not only to a material point isolated from the action of all other bodies, \textit{but also to a material point located in a gravitational field}. We relate the position of a point in space and time to a fixed Galilean system where we call the triad of components $T_0$, and consider a field of forces analogous to a gravitational field, that is to say, essentially, \textit{an acceleration field} $(X, Y, Z)$.

If, with respect to a fixed Galilean system, the velocity at the instant $t$ is
\uequ{
u,v,w,
}
its velocity at the instant $t+dt$ will be
\uequ{
u + Xdt, v + Ydt, w + Zdt.
}
We attach \?{to the motion} at the instant $t$ a triad of components $T$, equivalent to $T_0$
in the ordinary geometric sense of the word, and similarly attach at the instant $t+dt$ a triad $T'$ equivalent to $T_0$. These triads only define Galilean reference systems if their motion is given by uniform rectilinear translation with respect to $T_0$; if they are \?{so} chosen, we will define two Galilean reference systems with origins
\uequ{
x,y,z,t;\\
x+dx,y+dy,z+dz,t+dt.
}

We respectively denote the translatory velocities of the triads $T$ and $T'$ with respect to $T_0$ by
\uequ{
a,b,c,\\
a',b',c'.
}
Given this, the velocity of the point with respect to a Galilean system attached to it at the instant $t$ has the components
\uequ{
u-a, v-b, w-c;
}
the velocity with respect to the Galilean system which is attached at the instant $t+dt$ has the components
\uequ{
u + Xdt - a', v + Ydt - b', w + Zdt - c'.
}

The components will not change if one has
\uequ{
a'-a = Xdt, b'-b = Ydt, c' - c = Zdt.
}

Therefore, \textit{the motion of an arbitrary material point located in the force field under consideration will again satisfy the principle of inertia if one agrees to regard as equivalent two Galilean reference systems with infinitesimally-close origins}
\uequ{
z,y,z,t;\\
x+dx,y+dy,z+dz,t+dt,
}
\textit{where their coordinate triads $T$ and $T'$ are equivalent in the ordinary geometric sense of the word, and where additionally the triad $T'$ moves with respect to the triad $T$ with uniform rectilinear translatory velocity $(Xdt, Ydt, Zdt)$.}

We remark that, again, only the mutual relations between the two infinitesimally-close reference systems are involved.

%