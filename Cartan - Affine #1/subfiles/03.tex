\subsection*{3}
One may present these in another manner which is perhaps more intuitive and which is closer to the point of view which served as a point of departure for Einstein in his theory of gravitation. We imagine a material point in the force field under consideration and a triad $T$ having this point for its origin, \WTF{moving}{entraîné} with a translatory motion. At each instant $t$ we consider the Galilean reference system consisting of a triad $\overline{T}$ coinciding with $T$ at the instant under consideration and moving with uniform rectilinear translatory motion, with its velocity the current velocity of the moving point\footnote{In reality, the triad $\overline{T}$ may be replaced \WTF{for our purposes}{pour l'usage qui est fait} by the triad $T$ itself, which thus plays, from the point of view of the measurement of the velocity of a point at the instant $t$, the role of a Galilean triad.}; with respect to this Galilean system, this material point evidently has zero velocity: its motion then satisfies the principle of inertia (constancy of velocity) if the successive Galilean reference systems defined by the triads $\overline{T}$ are considered as equivalent \textit{\WTF{at each step}{de proche en proche}}. One easily sees that the constant velocity of the triad $\Tbar'$ corresponding to the instant $t+dt$ with respect to the triad $\Tbar$ corresponding to the instant $t$ is
\uequ{
Xdt, Ydt, Zdt.
}

We consider, for example, the uniform field consisting of the gravity at the surface of the earth, supposing for the moment that we can neglect the motion of the earth with respect to absolute space. Taking the axis $z$ as vertical and ascending, two parallel triads $T$ and $T'$, one considered at an instant $t$, the other at the instant $t+dt$, would be considered as defining two equivalent Galilean reference systems if the second is moving with respect to the first with a constant vertical velocity $g dt$. In this case, one may attach to every space-time event $(x,y,z,t)$ a Galilean reference system \textit{such that all these systems are equivalent to one another}; it would suffice to be given the Galilean system attached to a particular event $(x_0,y_0,z_0,t_0)$; all the others would be perfectly determined.

This circumstance is no longer present in the general case. The equivalence relation, being only defined at each step, only permits saying whether two systems are equivalent if one is given the space-time path followed going from the origin of one to the origin of the other; two systems that are equivalent for a certain path cease to be equivalent for another path. We return later to this fundamental question.

