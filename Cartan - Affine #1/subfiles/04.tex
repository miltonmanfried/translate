\subsection*{4}
\WTF{We have arrived at}{Convenons de dire que} the equivalence conditions of two Galilean reference systems with infinitesimally-close origins defining the \textit{geometric} properties\footnote{In reality, these properties are at the same time geometric and kinematic} of space-time. The gravitational phenomena then pass from physics into geometry\footnote{Essentially, this is another way to say that the inertial mass is identical with the gravitational mass, or again that the gravitational field is a kinematic field (an acceleration field) rather than a dynamic field (a force field).}. The components $X,Y,Z$ of the gravitational field essentially constitute the elements of the geometrical structure of space-time\footnote{In reality, as one will see later, this structure not only involves the functions $X,Y,Z$ but also \WTF{nearly arbitrary constants}{constants arbitraires près}. This corresponds to the fact that, given a material system subject to a uniform gravitational field, it is impossible, by means of mechanical experiments conducted from the interior of the system, to detect the gravitational field; in particular, if one supposes that the stars produce a \textit{uniform} gravitational field \WTF{throughout the solar system}{dans l'étendue du système solaire}, absolutely nothing changes in what the laws of celestial mechanics predict for the motions of the sun and the planets.}. The relations
\nequ{1}{
\pddX{Z}{y} - \pddX{Y}{z} = 0, \quad
\pddX{X}{z} - \pddX{Z}{x} = 0, \quad
\pddX{Y}{x} - \pddX{X}{y} = 0,
}
which are true in \textit{rectangular} coordinates, carry the properties of this structure.

Finally, Poisson's fundamental equation
\nequ{2}{
\pddX{X}{x} + \pddX{Y}{y} + \pddX{Z}{z} = -4\pi\rho,
}
which, with the preceding, furnishes the complete laws of Newtonian gravitation\footnote{One must add the complementary condition that the functions $X,Y,Z$ vanish at infinity.}, shows that \textit{the density of matter in a continuous medium is the physical manifestation of a local geometrical property of space-time}.

Thus we recover, without leaving the realm of classical mechanics, some of the traits of Einstein's theory of gravitation. The only essential trait which is lacking concerns the connection between gravitational phenomena and electromagnetic phenomena. But here we work out that interdependence of the geometric structure of space-time and of the matter which fills the space.
