\subsection*{5}
The preceding considerations call for \WTF{elaboration}{compléments}. In the first place, is the reduction of gravitation to geometry only possible with one definition of the equivalence of two infinitesimally-close Galilean systems? We examine this question later; we are content for the moment to remark that it must be answered in the negative. We imagine, in fact, two Galilean reference systems with infinitesimally-close origins, equivalent in the sense of section 3. The corresponding triads $(T)$ and $(T')$ with origins $M$ and $M'$ are equivalent in the ordinary geometric sense of the word and the triad $(T')$ is moving with respect to the triad $(T)$ with a certain uniform rectilinear translation. Then we imagine that we replace the triad $(T')$ by the triad $(T'')$ with the same origin which is derived from $(T)$ by a \WTF{helical}{hélicoïdal} movement of the axis $MM'$, \?{of a given direction and fixed once and for all}, this triad $(T'')$ being fixed with respect to the triad $(T')$. We consider a material point moving in the gravitational field, that is at $M$ at the instant $t$, at $M'$ at the instant $t+dt$. Its velocity at the instant $t+dt$
being \WTF{quite well-approximated}{très sensiblement portée} by the line $MM'$,will evidently have the same components with respect to the reference system $(S')$ defined by the moving triad $(T')$ as with respect to the reference system $(S'')$ defined by the moving triad $(T'')$. Thus, if the motion of the point verifies the principle of inertia when one supposes the systems $(S)$ and $(S'')$ to be equivalent, \textit{the principle of inertia will also be verified with the modified definition of equivalence}.

We conclude from this example that \textit{there exists, as long as one confines oneself to the dynamics of a material point, an infinity of possible definitions of equivalence for two Galilean reference systems with infinitesimally-close origins.}
