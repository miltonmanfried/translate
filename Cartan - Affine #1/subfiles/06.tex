\subsection*{6}

One may think that these conclusions must be modified when dealing with the dynamics of material  \textit{systems}, since point dynamics neglects an important element, the rotation of the material element about itself. We consider a small spherical ball moving with an absolutely constant rotation; the axis of its rotation, which \WTF{carries}{porte} the angular momentum of the ball, must be regarded as constantly equivalent to itself. It then seems that our primitive convention, in which the directions of space remain equivalent to themselves in the usual sense, is the only admissible one. We leave this question aside for the moment, \?{reserving for later} the demonstration that the preceding conclusions are quite premature and that \textit{on the contrary, the indeterminacy revealed earlier remains in its entirely when one takes account of the laws of the dynamics of systems}\footnote{Except for one possible restriction, see later section 16.}. But for the question to be usefully studied, it is important to remark that the new point of view which we have \WTF{adopted}{nous nous sommes placé} obliges us to enunciate the laws of mechanics in an \textit{exclusively local} form, that is to day to reduce everything to the mechanics of continuous media; indeed, we do not know whether two reference systems are equivalent when their origins are not infinitesimally close.

But, to facilitate the passage from Newtonian mechanics to relativistic mechanics, we have to show how the conception of four-dimensional space-time allows us to write the equations of classical mechanics for continuous media.

