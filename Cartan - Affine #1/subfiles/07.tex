\subsection*{7}
We adopt the point of view of classical mechanics. The space-time or the universe is an \textit{affine manifold}. Here is what is meant by that.

By \textit{world-vector} we denote the ensemble of two events (each localized in space and in time) where one will be the origin, the other the extremity of the vector. With respect to a Galilean reference system the components of a world-vector are the four quantities
\uequ{
t'-t,x'-x,y'-y,z'-z
}
obtained by subtracting the coordinates of the extremity and the origin. If two vectors have the same components with respect to a certain Galilean system, they also have the same components with respect to every other Galilean system; \?{this is} a property, for two world vectors, independent of the Galilean reference system by means of which one analytically defines the two vectors; we say that the two world vectors are \textit{equivalent}. It is evident that if two world vectors equivalent to a third, they are equivalent to one another. It is the existence of this notion of equivalence of two vectors that we express in saying that the universe is affine.

Of the four quantities
\uequ{
t'-t,x'-x,y'-y,z'-z
}
which analytically define a world vector, we say that the first is its \textit{time component}, the three others its \textit{space components}. We remark that \textit{the time component is independent of the chosen reference system}. It is not the same as a \textit{space vector} which has for its components, with respect to a triad of coordinates $T$ which serves to define a Galilean reference frame, the three quantities $x'9x,y'-y,z'-z$; this space vector depends, not only on the given world vector and on the triad $T$, but also on the velocity of this triad with respect to absolute space.

If we now consider a point moving with respect to a Galilean reference system, the world vector which has for its origin the point taken at the instant $t$ and for its extremity the point taken at the instant $t+dt$, has for its components
\uequ{
dt,dx,dy,dz;
}
it is, \textit{intrinsically}, independent of the reference system; it is the same as the world vector
\uequ{
1,\frac{dx}{dt},\frac{dy}{dt},\frac{dz}{dt}
}
which is derived by dividing by $dt$. If finally one denotes by $m$ the mass of the point, the world vector
\uequ{
m,m\frac{dx}{dt},m\frac{dy}{dt},m\frac{dz}{dt}
}
has an \textit{intrinsic} meaning, independent of the chosen reference system. This is the "\textit{mass-momentum}" vector. Its time component, the mass, is independent of the reference system; on the contrary, its space components, the momentum, depend on it.

One then sees that the fundamental principle of point dynamics may be stated thus:

\textit{The derivative with respect to time of a the mass-momentum world vector is equal to the "force" space-vector}.

This statement contains at the same time the principle of conservation of mass and the law which relates force to acceleration.

