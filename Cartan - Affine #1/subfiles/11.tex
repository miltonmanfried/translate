\subsection*{11}
One may represent the preceding results by means of a simple vectorial notation. We designate by the letters
\uequ{
\e_0,\,\,\e_1\,\,\e_2\,\,\e_3
}
the four world-vectors which have respectively for their components
\uequ{
1,\,\,0,\,\,0,\,\,0;\\
0,\,\,1,\,\,0,\,\,0;\\
0,\,\,0,\,\,1,\,\,0;\\
0,\,\,0,\,\,0,\,\,1.
}
the last four\translator{sic} are space vectors. With this notation, the "mass-momentum" at a material point of mass $m$ is represented by
\uequ{
m\left(\e_0 + \frac{dx}{dt}\e_1 + \frac{dy}{dt}\e_2 + \frac{dz}{dt}\e_3\right).
}

If we again agree to designate by the letter $\m$ a world-point $(t,x,y,z)$, the derivative $\frac{d\m}{dt}$ of this point with respect to time is the world-vector with components
\uequ{
1,\frac{dx}{dt}, \frac{dy}{dt}, \frac{dz}{dt};
}
one sees that the "mass-momentum" of a material point is represented by the notation
\uequ{
m\frac{d\m}{dt}.
}
The points and the (\?{free}) vectors are \textit{geometric forms} of the first degree. One may also consider second-degree geometric forms, which represent \textit{sliding} vectors. One designates by $[\m\,\m']$ the sliding vector which has for its origin the world vector $\m$ and for its extremity the world vector $\m'$. This sliding vector of six \?{Plückerian} components which are the determinants of the second-order formed with the table

\begin{tabular}{ccccc}
  $1$ & $t$ & $x$ & $y$ & $z$ \\
  $1$ & $t'$ & $x'$ & $y'$ & $z'$
\end{tabular}
evidently one has
\uequ{
[\m'\,\m] = -[\m\,\m'].
}

Similarly, one designates by $[\m\,\e]$ the sliding vector obtained by carrying from the world-point $m$ a vector equivalent to a given vector $\e$; the six Pl\"uckerian coordinates of this sliding vector are formed with the table
\begin{tabular}{ccccc}
  $1$ & $t$ & $x$ & $y$ & $z$ \\
  $0$ & $\theta$ & $\xi$ & $\eta$ & $\zeta$,
\end{tabular}
where the second line contains the coordinates of the vector $\e$. Finally, the notation $[\e\,\e']$ will designate the \textit{bivector} whose six components are formed with the table
\begin{tabular}{ccccc}
  $0$ & $\theta$ & $\xi$ & $\eta$ & $\zeta$, \\
  $0$ & $\theta'$ & $\xi'$ & $\eta'$ & $\zeta'$,
\end{tabular}
the components of the two free vectors $\e$ and $\e'$.

In each of the preceding cases, the sliding vector or the bivector under consideration may be regarded as the (exterior) product of two factors, which are first-degree geometric forms (points or free vectors). The product of two arbitrary first-degree geometric forms satisfies the distributive laws, but the sign changes with the order of factors.

