
\subsection*{13}
We return to the mechanics of continuous media. We designate by $\G$ the sliding vector which represents the three-dimensional "mass-momentum" element and by $\F$ the sliding vector which represents the force \?{of the} elementary volume. \textit{The equations of mechanics are condensed in the formula}
\nequ{6}{
\G'=[dt\,\F].
}

\WTF{In fact, here one has}{On a du reste ici}
\uequ{
\G &= [\m\,\e_0]\Pi + [\m\,\e_1]\Pi_x + [\m\,\e_2]\Pi_y + [\m\,\e_3]\Pi_z,\\
\F &= [\m\,\e_1]X\,dx\,dy\,dz + [\m\,\e_2]Y\,dx\,dy\,dz + [\m\,\e_3]Z\,dx\,dy\,dz.
}
$\F$ may be calculated by taking account of the equation
\uequ{
d\m = \e_0 dt + \e_1 dx + \e_2 dy + \e_3 dz;
}
it gives
\uequ{
\G' = &[\m\,\e_0]\Pi' + [\m\,\e_1]\Pi_x' + [\m\,\e_2]\Pi_y' + [\m\,\e_3]\Pi_z'\\
      & + [\e_0 \, \e_1][ dt \Pi_x - dx \Pi]
        + [\e_0 \, \e_2][ dt \Pi_y - dy \Pi] 
        + [\e_0 \, \e_3][ dt \Pi_z - dz \Pi] \\
      & + [\e_2 \, \e_3][ dy \Pi_z - dz \Pi_y]
        + [\e_3 \, \e_1][ dz \Pi_x - dx \Pi_z] 
        + [\e_1 \, \e_2][ dx \Pi_y - dy \Pi_x].
}

One easily verifies that the coefficients of $[\e_0 \, \e_1]$, $[\e_0 \, \e_2]$, $[\e_0 \, \e_3]$ \WTF{vanish identically}{sont nuls d'eux-mêmes}.

If one supposes that each material element is acted upon \textit{not only by a force, but by a couple}, the fundamental equation (6) would not change, but one would need to add to the expression for $\F$ terms of the form
\uequ{
[\e_2 \, \e_3] L\,dx\,dy\,dz + [\e_3 \, \e_1]M\,dx\,dy\,dz + [\e_1 \, \e_2]N\,dx\,dy\,dz.
}

One would then have
\uequ{
p_{zy} - p_{yz} &= L,\\
p_{xz} - p_{xz} &= M,\\
p_{yx} - p_{xy} &= N.
}

It is quite evident that from the equation (6) one may deduce the fundamental equation of point dynamics and supposing that the matter is condenses in a very small portion of the space, one then obtains
\uequ{
d\G = dt\,\F.
}
