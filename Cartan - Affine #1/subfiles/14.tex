\subsection*{14}
The equation (6) will permit us to easily write the equations of mechanics of continuous media in attaching to each point a variable Galilean reference system. We again designate by $\e_0$, $\e_1$, $\e_2$, $\e_3$ the free world-vectors which define the Galilean reference system attached to the point $\m$. In passing from a point $\m$ to an infinitesimally-close point $\m'$, these free vectors will vary, but the time component of the vector $\e_0$ will remain constantly equal to $1$, and the time-components of the vectors $\e_1$, $\e_2$, $\e_3$ remain constantly equal to $0$. One will thus have formulae of the form\footnote{We suppose, as in section 1, that the axes of the coordinate triads are not necessarily rectangular.}
\nequ{7}{
d\e_0 &= \omega^1_0 \e_1 + \omega^2_0 \e_2 + \omega^3_0 \e_3,\\
d\e_1 &= \omega^1_1 \e_1 + \omega^2_1 \e_2 + \omega^3_1 \e_3,\\
d\e_2 &= \omega^1_2 \e_1 + \omega^2_2 \e_2 + \omega^3_2 \e_3,\\
d\e_3 &= \omega^1_3 \e_1 + \omega^2_3 \e_2 + \omega^3_3 \e_3,
}
the $\omega^j_i$ being linear combinations of the differentials of the four quantities which serve to define, in an arbitrary manner, the different world-points. Thus we designate by
\nequ{8}{
d\m = \omega^0 \e_0 + \omega^1 \e_1 + \omega^2 \e_2 + \omega^3 \e_3,
}
the free vector with the origin $\m$ and extremity $\m'$; $\omega^0$ is simply the corresponding elementary time-interval. One will again have here expressions of the form
\uequ{
\G &= [\m\, \e_0]\Pi + [\m\,\e_1]\Pi_x [\m\,\e_2]\Pi_y + [\m \e_3]\Pi_z,\\
\F &= [\m\, \e_1]X \omega^1 \omega^2 \omega^3 + [\m \e_2]Y\omega^1 \omega^2 \omega^3
      [\m\, \e_3]Z \omega^1 \omega^2 \omega^3;
}
only the expression for $\G'$ is more complicated, since the free vectors $\e_0$,$\e_1$,$\e_2$,$\e_3$ are not fixed. One has
\uequ{
\G' = [\m\,\e_0]\Pi' 
&+ [\m\,\e_1][\Pi_x' + \omega^1_0 \Pi + \omega^1_1 \Pi_x + \omega^1_2 \Pi_y + \omega^1_3 \Pi_z]\\
&+ [\m\,\e_2][\Pi_y' + \omega^2_0 \Pi + \omega^2_1 \Pi_x + \omega^2_2 \Pi_y + \omega^2_3 \Pi_z]\\
&+ [\m\,\e_3][\Pi_z' + \omega^3_0 \Pi + \omega^3_1 \Pi_x + \omega^3_2 \Pi_y + \omega^3_3 \Pi_z]\\
&+ [\e_0 \, \e_1][\omega^0 \Pi_x - \omega^1 \Pi]
+ [\e_0 \, \e_2][\omega^0 \Pi_y - \omega^2 \Pi]
+ [\e_0 \, \e_3][\omega^0 \Pi_z - \omega^3 \Pi]\\
&+ [\e_2 \, \e_3][\omega^2 \Pi_z - \omega^3 \Pi_y]
+ [\e_3 \, \e_1][\omega^3 \Pi_x - \omega^1 \Pi_z]
+ [\e_1 \, \e_2][\omega^1 \Pi_y - \omega^2 \Pi_x].
}

One immediately deduces the desired equations.

