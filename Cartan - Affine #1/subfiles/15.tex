\subsection*{15}
In the preceding we have uses the ordinary definition of equivalence of world-vectors. But the formulae obtained are again valid with an arbitrary definition of the equivalence of two world vectors \textit{with infinitesimally-close origins}. With this new definition, the formulae (7) preserve their form \textit{but with modified coefficients}\footnote{It is important to note that if one attaches to each world-point \textit{another} Galilean reference system, for example defined by
\uequ{
\overline{\e}_0 = \e_0 + u\e_1,\quad
\overline{\e}_1 = \e_1,\quad\overline{\e}_2 = \e_2, \quad \overline{\e}_3 = \e_3,
}
the formulae (7) should be modified as a consequence; in particular, $\omega^1_0$ will be replaced by $\overline{\omega}^1_0 = \omega^1_0 + u\omega^1_1 + du$.}.

We suppose that the only volume forces are those due to gravitation, \?{which is effectively the case in practice}; defining the equivalence of two Galilean systems with infinitesimally-close origins evidently reduces to defining the equivalence between two world vectors with infinitesimally-close origins. If one choses this definition in such a manner that the gravitational forces are suppressed, the dynamical equations reduce to
\uequ{
\G'=0.
}

Or, we take for $\e_0,\e_1,\e_2,\e_3$ vectors equivalent \textit{in the ordinary sense} to unit vectors of a \textit{fixed} Galilean reference system and take
\uequ{
\omega^1_0 = -X\,&dt,\quad \omega^2_0 = -Y\,dt,\quad \omega^3_0n= -Z\,dt,\\
& \omega^j_i = 0\quad (i,j=1,2,3).
}

The equations of mechanics become
\uequ{
\Pi' = 0,\quad
\Pi_x' - X[dt\,\Pi] = 0,\quad
\Pi_y' - Y[dt\,\Pi] = 0,\quad
\Pi_z' - Z[dt\,\Pi] = 0,
}
or again
\uequ{
  \Pi' &= 0,\\
\Pi_x' &= \rho X [dt\,dx\,dy\,dz],\\
\Pi_y' &= \rho Y [dt\,dx\,dy\,dz],\\
\Pi_z' &= \rho Z [dt\,dx\,dy\,dz];
}
these will be the classical dynamical equations of a continuous medium subject to a volume force \textit{proportional to the mass}. It is sufficient to take for $X,Y,Z$ the components of the acceleration due to gravity to \WTF{bring gravitation into}{pour faire rentrer...dans} the geometry.

The result just obtained is identical with that which was furnished directly by the dynamics of material points. The formulae
\uequ{
d\e_0 = -X\,dt\,\e_1 - Y\,dt\,\e_2 - Z\,dt\,\e_3,\\
d\e_1 = d\e_2 = d\e_3 = 0
}
mean in effect that two Galilean reference systems with origins
\uequ{
t,\,\,x,\,\,y,\,\,z,\\
t+dt,\quad x+dx,\quad y+dy,\quad z+dz
}
are regarded as equivalent when the corresponding triads $T$ and $T'$ are equivalent in the ordinary sense, the triad $T'$ moving with respect to the triad $T$ with a rectilinear translation and a uniform velocity $(X\,dt, Y\,dt, Z\,dt)$.

