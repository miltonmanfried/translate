\subsection*{16}
We agree to say that a given definition of equivalence for two Galilean systems with infinitely-close origins defines an \textit{affine connection} on space-time. It is now simple to see if the phenomena of gravitation are compatible with several distinct \?{world affine connections}. Beforehand we make the important remark that \textit{although the world affine connection depends on the distribution of matter in space, nonetheless, it is not perceptibly modified by the introduction of small masses in a given region of the universe}. If we reflect on the system formed by one of these small masses, the world affine connection at each corresponding world-point does not depend on the state of these masses. Any imaginable modification of the world affine connection would lead in the expression for $\G'$ to the addition in the terms
\nequ{9}{
  &[\m\,\e_1][\wbar^1_0 \Pi + \wbar^1_1 \Pi_x + \wbar^1_2 \Pi_y + \wbar^1_3 \Pi_z]\\
+ &[\m\,\e_2][\wbar^2_0 \Pi + \wbar^2_1 \Pi_x + \wbar^2_2 \Pi_y + \wbar^2_3 \Pi_z]\\
+ &[\m\,\e_3][\wbar^3_0 \Pi + \wbar^3_1 \Pi_x + \wbar^3_2 \Pi_y + \wbar^3_3 \Pi_z]\\
}
where one designates by $\wbar^i_0 , \wbar^j_i$ the modifications suffered by the components $\w^i_0,\w^j_i$ of the affine connection. The only possible modifications are then those which cancel the three quantities in the brackets, and that \textit{\WTF{regardless of}{quelles que soient}the numerical values which characterize the state of the material medium}.

We initially adopt the most general point of view possible, the affine connection permitting the equivalence of two Galilean reference systems where one has an \WTF{orthogonal}{trirectangle} triad $(T)$, the other a \textit{non}-orthogonal triad $(T')$. We additionally suppose that it is allowed that the vectors $\e_0,\e_1,\e_2,\e_3$ used in the formulae (7) are always equivalent to one another \textit{in the usual sense}, that is to say that everything is referred to a fixed Galilean system. We put
\uequ{
\wbar^i_0 &= \gamma^i_{00}\,dt + \gamma^i_{01}\,dx + \gamma^i_{02}\,dy + \gamma^i_{03}\,dz,\\
\wbar^j_i &= \gamma^j_{i0}\,dt + \gamma^j_{i1}\,dx + \gamma^j_{i2}\,dy + \gamma^j_{i3}\,dz.
}

The coefficients of the forms $\Pi,\Pi_x,\Pi_y,\Pi_z$ are
\uequ{
\rho,\,\,\rho u,\,\,\rho v,\,\,\rho w,\\
\rho u^2 + p_{xx},\,\, \rho uv + p_{xy},\,\, \rho uw + p_{xz},\,\, \dots, \rho w^2 + p_{zz};
}
one may, to cancel the three expressions in the brackets in the formulae (7), regard them as independent. In turning our attention to each in turn, the others being regarded as zero, one arrives at the formulae
\nequ{10}{
\gamma^i_{00}=0,\quad \gamma^i_{0} =\gamma^i_{j0},\quad \gamma^k_{ij} = \gamma^k_{ji};
}
\textit{they simply express that the three quadratic differential forms}
\uequ{
\wbar^i_0 \, dt + \wbar^i_1 \, dx + \wbar^i_2 \, dy + \wbar^i_3 \,dz\quad (i=1,2,3)
}
\textit{are identically zero}.

One would arrive at the same result by simply utilizing the dynamics of material points; the equality
\uequ{
\frac{d}{dt}\left(
\e_0 + \e_1 \frac{dx}{dt} + \e_2 \frac{dy}{dt} \e_3 \frac{dz}{dt}
\right) = 0,
}
where one takes account of the relations (7), in fact remains valid by modifying the coefficients $\w^i_0, \w^j_i$ of these relations by the addition of the terms $\wbar^i_0 , \wbar^j_i$ if one has
\uequ{
\wbar^i_0 + \wbar^i_1 \frac{dx}{dt} + \wbar^i_2 \frac{dy}{dt} + \wbar^i_3 \frac{dz}{dt} = 0,
}
\textit{and that regardless of the mutual relations between $dx,dy,dz,dt$}: one arrives at exactly the same conclusions.

On the contrary the conclusions would be different, if one did not suppose the symmetry of the components of the pressure; this symmetry would necessarily disappear if one admits the possibility of \textit{couples}\footnote{This case would be present in the case of a magnet placed in a magnetic field} acting on a material element. In this case, the expression (9) would be identically zero, even assuming
\uequ{
p_{xy} \neq p_{yx}, \quad p_{yz} \neq p_{zy}, \quad p_{zx} \neq p_{xz}.
}
One the easily sees that \textit{all the coefficients $\gamma^k_{ik}$ must be zero}: only arbitrary nine coefficients remain instead of 18. In this case, \textit{and in this case only}, the dynamics if continuous media impose on the world affine connection mkre restrictive conditions than the dynamics of material points.

