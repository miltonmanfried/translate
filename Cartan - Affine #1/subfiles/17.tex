\subsection*{17}
We now suppose that the affine connection preserves the metric character of the space, that is to say that a reference system composed of an orthogonal triad can only be equivalent to another reference system also composed of an orthogonal triad. In this case, if one attaches a Galilean reference system to each world point, such that the space vectors $\e_1, \e_2, \e_3$ are orthogonal and of length one (once a unit of length is fixed once and for all), the formulae (7) still hold, but one has between the $\w^j_i$ the relations which are derived from the formulae
\uequ{
(\e_i)^2,\quad \e_i \e_j = 0, \quad (i \neq j = 1,2,3);
}
these relations are
\uequ{
\w^i_i = 0, \quad \w^j_i + \w^i_j = 0;
}
the three quantities $\w^3_2 = -\w^2_3$, $\w^1_3 = -\w^3_1$, $\w^2_1 = -\w^1_2$ are the components of the rotation which brings the triad $T$ to be equivalent to the triad $T'$.

The modifications which are permitted to be made to the world affine connection are then defined by the same conditions as in section 16; just the fact that one has
\uequ{
\wbar^i_i = \wbar^j_i + \wbar^i_j = 0
}
reduces to four the number of arbitrary coefficients; one has\footnote{The geonetric interprtation of these formulae will be trivial.}
\uequ{
\wbar^1_0 = r\, dy - q\,dz,\quad \wbar^2_0 = p\,dz - r\,dx,\quad \wbar^3_0 = q\,dx - p\,dy,\\
\wbar^3_2 = -\wbar^2_3 = p\, dt + h\,dx,\quad \wbar^1_3 = -\wbar^3_1 = q\,dt + h\, dy, \quad
\wbar^2_1 = -\wbar^1_2 = r\, dt + h\, dz.
}

If one did not suppose the symmetry of the components of the pressure, the coefficient $h$ would necessarily be zero.

