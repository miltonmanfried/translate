\subsection*{19}
The special theory of relativity admits the same Galilean reference systems as classical mechanics; the only difference, essentially, resides in the formulae which allow passing from the coordinates $(t,x,y,z)$ of an event with respect to a first Galilean reference system to the coorinates $(t_1,x_1,y_1,z_1)$ of the same event with respect to another Galilean reference system. These formulae are again linear, that is to say the universe of special relativity is affine, but the time component $t$ of a world vector is no longer \WTF{constant}{invariable}. The following expression is invariant
\uequ{
c^2(t' - t)^2 - (x' - x)^2 - (y' - y)^2 - (z' - z)^2,
}
where $c$ designates the speed of light in vacuum. Two vectors then admit an invariant, their \textit{scalar product}, whixh is
\uequ{
c^2 \theta \theta' - \xi\xi' - \eta\eta' - \zeta\zeta',
}
designating the components of the two vectors respectively by
\uequ{
\theta,\, \xi,\, \eta,\, \zeta,\\
\theta',\, \xi',\, \eta',\, \zeta'.
}

If in particular one designates as above by $\e_0, \e_1, \e_2, \e_3$ the unit vectors attached to a Galilean reference system, one has the formulae
\nequ{11}{
(\e_0)^2 = c^2, \quad (\e_1)^2 = (\e_2)^2 = (\e_3)^2 = -1,
\e_0 \e_i = 0, \quad \e_i \e_j = 0 (i \neq j = 1,2,3).
}
