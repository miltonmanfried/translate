\subsection*{20}
If one considers a Galilean reference system depending on several parameters, one will have, for an infinitely-small total variation of the parameters, the formulae

\nequ{12}{
d\e_0 = \omega_0^0 \e_0 + \omega_0^1 \e_1 + \omega_0^2 \e_2 + \omega_0^3 \e_3,\\
d\e_1 = \omega_1^0 \e_0 + \omega_1^1 \e_1 + \omega_1^2 \e_2 + \omega_1^3 \e_3,\\
d\e_2 = \omega_2^0 \e_0 + \omega_2^1 \e_1 + \omega_2^2 \e_2 + \omega_2^3 \e_3,\\
d\e_3 = \omega_3^0 \e_0 + \omega_3^1 \e_1 + \omega_3^2 \e_2 + \omega_3^3 \e_3,\\
}

where the $\omega_i^j$ are linear with respect to the differentials of the parameters. These expressions are not absolutely arbitrary, since the relations (11) must always be verified. Differentiating them, one easily obtains

\nequ{13}{
\omega_0^0 =0, \quad \omega_0^i =c^2 \omega_j^i, \quad \omega_i^j + \omega_j^i =0 \quad (i,j = 1, 2, 3).
}

Thus there remain six independent expressions, and six is actually the number of parameters which determine \textit{orientation} of a Galilean reference system.

In the preceding formulae, the expressions $\omega_0^1, \omega_0^2, \omega_0^3$ represent the (infinitely small) changed velocity of the \?{sign}{signe} of the rectilinear translational motion where the axes of the second reference system are moved with respect to the first.

It is remarked that one recovers the law of the dependence of two infinitely-close Galilean systems in classical mechanics\footnote{It has Galilean systems with tri-rectangular triads, as in number 17.} by taking $c$ to be infinite; the formulae (13) in effect define

\uequ{
\omega_0^0 = 0, \quad \omega_i^0, \quad \omega_i^j + \omega_j^i = 0.
}
