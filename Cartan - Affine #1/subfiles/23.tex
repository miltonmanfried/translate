\subsection*{23. Investigating whether several distinct affine connections are compatible with experience.} We pass from one affine connection to another by subjecting the components $\omega_0^i, \omega_i^j$ to the variations $\wbar_0^i, \wbar_i^j$ naturally satisfying
\uequ{
\wbar_0^i = c^2 \wbar_0^i,\quad \wbar_i^j + \wbar_j^i = 0,
}
and these variations must cancel the four expressions
\uequ{
[\omega_1^0 \xPi_x] &+ [\omega_2^0 \xPi_y] + [\omega_3^0 \xPi_z],\\
[\omega_0^1 \xPi] &+ [\omega_2^1 \xPi_y] + [\omega_3^1 \xPi_z],\\
[\omega_0^2 \xPi] &+ [\omega_1^2 \xPi_x] + [\omega_3^2 \xPi_z],\\
[\omega_0^3 \xPi] &+ [\omega_1^3 \xPi_x] + [\omega_2^3 \xPi_y],
}
\textit{regardless of the state of the material element under consideration}. If we take the medium with respect to a fixed Galilean reference system, we find, as in (16), that the quadratic form
\uequ{
\wbar_0^i dt + \wbar_1^i dx + \wbar_2^i dy + \wbar_3^i dz \quad (i=0,1,2,3)
}
must be identically zero. One recovers for the $\wbar_0^i$ and the $\wbar_j$ exactly the same expressions as in (17), with \textit{four} arbitrary coefficients $p, q, r, h$.

If one admits a larger conception of the mechanics of continuous media, the elementary "momentum-mass" containing the terms in $[\e_0\,\e_i]$ and $[\e_i\,\e_j]$, there are no longer any arbitrary coefficients and \textit{the affine space-time connection uniquely \?{matches}{relève} experiment}\footnote{This is so if one simply supposes the possibility of couples acting on the material elements, because then the coefficient $h$ is necessarily zero, and, as the quantities $p, q, r, h$ are transformed by changing a Galilean reference system (these are the components of a world-vector), \textit{one is obliged to suppose as well that $p, q, r$ are zero}.}.

%that 