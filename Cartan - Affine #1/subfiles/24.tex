\subsection*{24. Gravitation in special relativity.}

In classical mechanics, the equations
\uequ{
\omega_0^1 = -\hX dt, \quad \omega_0^2 = -\hY dt, \quad \omega_0^3 = -\hZ dt,\\
\omega_i^j = 0 \quad (i,j = 1, 2, 3),
}
define the affine connection permitting the gravitation to be pulled back into the geometry, conserving their form regardless of the Galilean reference system chosen as the fixed system.

In special relativity, if one admits the Newtonian laws of gravitation for a Galilean system formed of axes having as their origin the center of gravity of the solar system and directed towards the fixed stars, these laws no longer conserve the same form for another Galilean reference system. If one postulates, with Einstein, that the form of the laws of gravitation must be the same, regardless of the Galilean reference system adopted\footnote{We shall see later on the exact meaning of this phrase}, one is obliged to modify these laws; but, what is important to note now is that Einstein's reduction of gravitation to geometry is essentially of the same nature as that indicated at the start of the chapter.

% Anus Frey