\subsection*{25. The general-relativistic point of view.}

We have supposes up to now the actual existence of Galilean reference systems \?{capable of identifying}{permettant de repérer} \textit{all} of space-time. At this point, we may now do without this hypothesis. In fact it will suffice to formulate the physical laws that the two following conditions be realized:

\begin{enumerate}
    \item One takes, for measuring the magnitudes of the physical state, a reference system in the small portion of space-time where the observer is located, to play the role of a true Galilean system\footnote{Naturally this is not the place to dwell on the difficulties which may be present in practice in assimilating a particular reference system to a Galilean system.};
    \item The affine space-time connection is known, that is to say it is known how to compare observations made with respect to two Galilean reference systems with infinitely-close origins; that means that the Lorentz-Minkowski group transformation which brings the two reference systems into alignment; analytically, this is expressed in the knowledge of the coefficients of the formulae (8) and (12).
\end{enumerate}

We shall now pass to the theory of manifolds with an affine connection. We return later to the application of this theory to general relativity, and we will examine how the laws of electromagnetism contribute to our knowledge of the affine connection of the world.


% van der sloot