\subsection*{30.}

Can the \?{comparison}{raccord} of the affine spaces tangent to two arbitrary points $\m$ and $\m'$ be defined independently of the path, for all $\m$ and $\m'$? To make it so, it is necessary and sufficient that the equations of the total differentials (1) or (3) be completely integrable. \?{The calculation had been made when we were dealing with the affine space proper}{Le calcul a été fait quand nous nous sommes occupés de l'espace affine proprment dit}: it gives the necessary and sufficient conditions (2):
\uequ{
(\omega^i)' &= [ \omega^k \, \omega_k^i ],\\
(\omega_i^j)' &= [ \omega_i^k \, \omega_k^j ];
}
we have suppressed the summation sign over $k$, following a usage which \?{has been popularized in / has vulgarized}{qu'a vulgarisè} the absolute differential calculus.

If the conditions (2) are realized, and if one attaches to a particular point $\m_0$ a reference system $(\e_i)_0$, nothing stops us from choosing an at arbitrary point $\m$, for $\e_1, \e_2, \e_3$, the vectors respectively equivalent to $(\e_1)_0, (\e_2)_0, (\e_3)_0$, since the property of equivalence now has an absolute meaning. With this choice of reference system, we have
\uequ{
d\e_i = 0,
}
and therefore the forms $\omega_i^j$ are identically zero; the formulae (2) then show that the $\omega^i$ have their bilinear covariants identically zero: these are thus exact differentials, and no generality is lost by supposing that $\omega^i = du^i$. Thus the formulae 
\uequ{
d\m = du^i \e_i
}
show that $u^1, u^2, u^3$ may be taken as the affine coordinates for the point $\m$ for a variable reference system for the whole manifold. Put otherwise, \textit{the manifold is the same as an affine space}.


%
