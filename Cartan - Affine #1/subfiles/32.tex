\subsection*{32.}

The integral $\int d\e_i$ over an infinitely-small closed contour may be calculated in the same manner, and it is found that
\uequ{
\int d\e_i = \int\int \left[(\omega_i^1)' - \omega_i^k \omega_k^1\right]\e_1
 + \left[(\omega_i^2)' - \omega_i^k \omega_k^2\right]\e_2
 + \left[(\omega_i^3)' - \omega_i^k \omega_k^1\right]\e_3.
}
The result is formulae such as
\uequ{
(\omega_i^j)' - \left[\omega_i^k \omega_k^j\right] = \Omega_i^j
 = \Lambda_{ikl}^j \left[\omega_k \omega_l \right],
}
equivalent to the formulae
\uequ{
(d\e_i)' = \Omega_i^1 \e_1 + \Omega_i^2 \e_2 + \Omega_i^3 \e_3.
}

The forms define what is called the \textit{curvature} of the given manifold with an affine connection. This curvature is zero, as is immediately apparent, if the total differential equations (4) are completely integrable, put another way if the equivalence of two vectors has an absolute meaning, independent of the path chosen from the origin of first to the origin of the second. In a manifold without curvature, different equivalent reference systems can be attached to different points; the components $\omega_i^j$ are then zero and we are left with
\uequ{
\Omega^i = (\omega^i)';
}
it is seen that, in the case where there is torsion, the components $\omega^i$ of the vector $d\m$ with respect to axes with fixed directions \textit{are not exact differentials}.
% the new texpad fucking erases my shit when it crashes