\subsection*{35.}

We take the particular case of a vector with components $\xi^i$, which shall be transported in a manner such that it remains pointwise equivalent to itself;\textit{when the origin of the vector has described a closed contour, its initial components will have been subjected to the infinitely-small variations $\Delta\xi^i$ given by the formulae}
\nequ{6}{
\Delta\xi^i = -\xi^k\Omega_k^i.
}

For example, the vector chosen at the starting point $\m$ for the $i^\text{th}$ reference vector will become, \textit{if it is transported \?{maintaining equivalence}{par équipollence} along the closed contour},
\uequ{
\e_i - \Omega^k_i \e_k;
}
it will have been subject to a \textit{geometric diminution} of $\Omega_i^k \e_k$. This result is not a contradiction, despite appearances, with the preceding formula it is seen that
\uequ{
\underset{(C)}{\int}d\e_i = \int \int \Omega_i^k \e_k,
}
since $\int d\e_i$, in this last formula, \?{is not related back to the variation of a vector transported equivalent to itself}{ne se rapporte pas à la variation d'un vecteur transporté par équipollence}, but to that of a vector which, \textit{at each point $\m$ of the contour, is taken for the $i^\text{th}$ reference vector.}

% smooty fruit