\subsection*{40.}

\nc{\overomega}{\overline{\omega}}
\nc{\overA}{\overline{A}}

We say that two manifolds with affine connections with $n$ dimensions are \textit{isomorphic} if between the two manifolds a pointwise correspondence can be established with the following properties: for any choice of reference systems attached to the points of the first manifold, one may make a corresponding choice of reference system attached to corresponding points on the second manifold such that the components $\omega^i$ and $\omega_i^j$ relative to the first manifold become, for the pointwise correspondence under consideration, equal to the components $\omega^i$ and $\omega_i^j$ relative to the second manifold. one also says that the two manifolds are \?{\textit{applicable}}{applicables} upon one another.

We consider two isomorphic manifolds and attribute to an arbitrary point $\m$ of each of the manifolds as general a reference system as possible; on each manifold this reference system depends on $n(n+1)$ parameters, \?{of which $n$ coordinates are for the origin $\m$}{dont les n coordonées de l'origine m}. Between the $n(n+1)$ parameters of the first manifold and the $n(n+1)$ parameters of the second, there exists a correspondence such that one has the identities
\nequ{8}{
\overomega^i = \omega^i,\quad \overomega_i^j = \omega_i^j, 
}
designating by $\overomega^i$ and $\overomega_j^i$ the components relative to the second manifold. The reciprocal is true, since the first $n$ in the preceding equations  entails a necessary correspondence between the point coordinates of a point on the second manifold and the point coordinates of a point on the first.

The relations (8) imply, after exterior derivation,
\uequ{
\overOmega^i = \Omega^i, \quad \overOmega_i^j = \Omega_i^j,
}
and finally
\uequ{
\overA_{jk}^i = A_{jk}^i, \quad &\overA_{ikl}^j = A_{ikl}^j,\\
d\overA_{jk}^i = dA_{jk}^i, \quad &d\overA_{ikl}^j = dA_{ikl}^j,
}
so again the equality in the two members of the last formulae, taken to be expanded in functions linear in $\omega^i$ and $\omega_i^j$, of the respective coefficients, and thus of the whole.