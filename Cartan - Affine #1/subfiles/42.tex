\subsection*{42.}

It follows from the preceding that differentiating the components $A_{jk}^l$ and $A_{jkl}^i$ introduces as the only new coefficients the coefficients $A_{jkl\rho}^i, A_{jkl | \rho}^i$ if the form
\uequ{
d\left(\e_i \Omega^i(\xi, \eta)\right), d\left(\e_i u^j \Omega_j^i(\xi, \eta)\right).
}

If now one replaces in these \?{expanded forms}{formes développées} the $\omega^i$ by the components $\zeta^i$ of an arbitrary vector, one obtains the forms
\uequ{
\e_i \Pi^i (\xi, \eta, \zeta), \e_i u^j \Pi_j^i(\xi, \eta, \zeta),
}
which together define a displacement associated with an elementary parallelepiped issuing from $\m$\footnote{Or rather a parallelogram and a vector.}; \textit{only the \?{ridges}{arêtes} of this parallelepiped do not play the same role}. In any case one may, by the same procedure as earlier, deduce the new forms
\uequ{
d\left(\e_i \Pi^i(\xi, \eta, \zeta)\right), d\left(\e_i u^j \Pi_j^i(\xi, \eta, \zeta)\right).
}
which, \?{expanded}{développées}, will be of the form
\uequ{
\e_i A_{jk | hl}^i \xi^j \eta^k \zeta^h \omega^l,
\e_i u^j A_{jkl | hm}^i \xi^k \eta^l \zeta^h \omega^m.
}

The new coefficients which are introduced in this way permit writing the complete expressions of the differentials of $A_{jk| h}^i$ and $A_{jkl| h}^i$:
\nequ{11}{
dA_{jk | h}^i + A_{jk|h}^\rho \omega_\rho^i - A_{\rho k|h}^i \omega_j^\rho
 - A_{j\rho|h}^i \omega_k^\rho - A_{jk|\rho}^i\omega_h^\rho
 = A_{jk|h\rho}^i \omega^\rho,\\
dA_{jkl | h}^i + A_{jkl|h}^\rho \omega_\rho^i - A_{\rho kl|h}^i \omega_j^\rho
 - A_{j\rho l|h}^i \omega_k^\rho - A_{jk\rho|h}^i\omega_l^\rho
 - A_{jkl|\rho}^i \omega_h^\rho = A_{jkl|h\rho}^i \omega^\rho.
}

Obviously these operations may be pursued indefinitely; the new coefficients which are introduced by each differentiation are the coefficients of the forms which one may give a geometrical significance and which represent the displacements associated with the arbitrary vectors of larger and larger numbers.  