\subsection*{44.}

The equations (3) and (5') define the infinitesimal transformations of the group of affine displacements. It is simple to generalize the considerations developed in the preceding sections, which will permit a better understanding of the general case.

We consider an arbitrary finite continuous group $G$ with $n$ variables $x_1, x_2, ..., x_n$, this group being, for example, defined by $r$ independent infinitesimal transformations
\uequ{
X_1 f, X_2 f, ..., X_r f.
}

We may regard the $x_i$ as the coordinates of a point in a certain soace $(E)$. If in this space we only pay attention to those properties of figures which are not altered by transformations of the group $G$, we may say that the space $(E)$ admits the group $G$ as the fundamental group. There one may substitute for the primitive coordinates $x_1, ..., x_n$ those that are induced by a transformation $T$ in the group; with this new coordinate system the properties of figures are analytically carried over in the same way as in the old; the two coordinate systems are equivalent. The passage from one coordinate system to the other equivalent one is thus carried out by a transformation from the fundamental group.

With this, we imagine a continuous ensemble of \WTF{observers}{observateurs}, reduced to points, and each has adopted a coordinate system for the study of the space $(E)$, these systems being naturally completely equivalent to one another. The manifold formed by these observer-points has, I suppose, $p$ dimensions, each point being defined in an arbitrary manner by $p$ coordinates $u_1, ..., u_p$. If one passes from a point $\m$ on the manifold to an infinitely-near point $\m'$, one passes in the space $(E)$ from a certain coordinate system to another which we suppose to be infinitely close; put another way, one passes from the coordinates $x_i$ utilized for the observer $\m$ to the coordinates $x_i'$ utilized by the observer $\m'$ by effecting a certain infinitesimal transformation in the group $G$, \WTF{so}{soit}
\uequ{
\omega_1 X_1 f + \omega_2 X_2 f + \dots + \omega_r X_r f,
}
designating by $\omega_1, \dots, \omega_r$ the expressions linear in $du_1, \dots, du_p$, with the coefficients being functions of $u_1, \dots, u_p$. Put another way, if $(x_i)$ and $(x_i + dx_i)$ are respectively the coordinates for \textit{one} point in $(E)$ by the observer are $\m$ and $\m'$, one will have
\nequ{15}{
dx_i = \omega_1 X_1 (x_1) + \omega_2 X_2 (x_2) + \dots + \omega_r X_r (x_r);
}
or again, if $f$ is a \textit{definite} function of the coordinates, one will have
\nequ{15'}{
df = \omega_1 X_1 f + \omega_2 X_2 f + \dots + \omega_r X_r f.
}

Suppose, for example, that $G$ is the group with one variable and one parameter
\uequ{
x' = x + a;
}
then
\uequ{
dx = \omega,
}
$\omega$ being a Pfaffian expression in $u_1, u_2, \dots, u_p$. 