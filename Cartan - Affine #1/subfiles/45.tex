\subsection*{45.}

That being the case, imagine that in the manifold $(V)$ of observers, one goes from an arbitrary point $\m_0$ to an arbitrary point $\m_1$ by a certain path. Passing from an arbitrary point $\m$ of this latter to an infinitely-close point $\m'$, one will have effected in the space $(E)$ an infinitesimal transformation whose components, \textit{with the coordinates adopted by the observer $\m$}, are $\omega_1, \omega_2, \dots, \omega_r$. The composition of all these successive infinitesimal transformations will permit passing from the coordinates adopted by the observer $\m_0$ to the coordinates adopted by the observer $\m_1$; this will be a certain finite transformation in the group $G$. It will obviously be furnished by integrating the differential equations (15'), where the $\omega_i$ are now of the form $p_i(t) dt$, taking $t$ as a parameter in a function with which the coordinates $u_1, \dots, u_p$ of a variable point on the path.

If now a \textit{closed} contour is described on the manifold $(V)$, ones not necessarily retrieve the same coordinate system when returning to the point $\m_0$ as at the start. In order to recover the same, it would be necessary (and sufficient) for the equations (15) to be completely integrable. We take these equations in the condensed form (15'). The condition for complete integrability being given, designating the bilinear covariant of $\omega_i$ by $\omega'$,
\uequ{
X_i f \omega_i' + \left[ d(X_i f) \omega_i \right] \equiv
X_i f \omega_i' + \left[ X_h (X_k f) - X_k (X_h f) \right] [ \omega_h \omega_k] = 0.
}

Or, following a classical theorem by S. Lie,
\uequ{
(X_h X_k) \equiv X_h (X_k f) - X_k ( X_h f) = c_{hks} X_s f,
}
the coefficients $c_{hks}$ being constant. One must then have
\uequ{
X_s f \left\{ \omega'_s + c_{hks} [\omega_h \omega_k] \right\} = 0,
}
and finally
\nequ{16}{
\omega_s' + c_{hks} [\omega_h \omega_k] = 0.
}