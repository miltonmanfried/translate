\subsection*{46.}
\nc{\opT}{\mathbb{T}}

The particular circumstance being studied will come forth particularly if we take as the manifold $(V)$ \textit{the manifold of parameters} of the group. If
\uequ{
x_i' = f_i(x_1, \dots, x_n; a_1, \dots, a_r)
}
are the \?{final}{finies} equations of the most general transformation $T_a$ of the group $G$, $a_1, \dots, a_r$ may be regarded as the coordinates of a point $\m$ of a manifold $(V)$ with $r$ dimensions. If $a_0^1, \dots, a_r^0$ are the parameters of the identity transformation and if $\m_0$ is the corresponding point of $(V)$, we assume that the system of coordinates adopted by the observer $\m$ \?{is recovered}{se déduit} by the transformation $\opT_a$ of the coordinate system adopted by the observer $\m_0$. One then passes from the coordinate system at $\m$ to that at the infinitely close point $\m'$ by the infinitesimal transformation $\opT_a^{-1}\opT_{a+da}$. It is then obvious enough that, \textit{whatever path is followed} to get from a point $\m$ with coordinates $(a_1, \dots, a_r)$ to a point $\m'$ with coordinates $(b_1, \dots, b_r)$, the resulting change of coordinates in the space $(E)$ will be $\opT_a^{-1}\opT_b$.

Reciprocally, if the relations (16) are verified for an arbitrary manifold $(V)$ of observers, and if we arbitrarily choose in this manifold a point $\m_0$, each point $\m$ will correspond to a definite transformation $\opT_a$ of the group $G$, which permits passing from the chosen coordinates for $\m_0$ to the coordinates chosen for $\m$; finally, the $u_i$ are the \?{determined}{déterminées} functions of $a_i$. It may naturally also be \?{the same}{un même} transformation $\opT_a$ corresponding to an infinity of points $\m$; it may also be, particularly if $p<r$, that only a part of the transformations correspond to different points of the manifold.