\subsection*{47. Return to the general case.}

\nc{\opS}{\mathbb{S}}

So far we have interpreted a transformation of the group $G$ as defining a change of coordinates in the space $(E)$, the primitive variables $x_i$ and the transformed variables $x_i'$ relating to \textit{the same point} in the space $(E)$. One could adopt a different, and in a certain sense more intuitive point of view and regard the $x_i$ and $x_i$ as the coordinates of \textit{two different points} in \textit{the same coordinate system}. We designate this transformation by the letter $\opS$, interpreted in this manner: this transformation is obviously only meaningful if one has made once and for all a prior choice of coordinate system. We say that this is a \textit{displacement} of the space $(E)$.

Taking this point of view, the infinitesimal transformation $\omega_1 X_1 f + \dots + \omega_r X_r f$ that we have associated to a couple of two infinitely close points $\m$ and $\m'$ on the manifold may be regarded as an \textit{infinitely small displacement} in the space $(E)$. The Pfaffian expressions $\omega_1, \dots, \omega_r$ thus together define a continuous (ideal) motion with $p$ parameters on the space $(E)$. Employing an awkward expression from the \?{theory of links} in mechanics, we may say that in general \textit{this motion is not holonomic}; it is holonomic if the conditions (16) hold. For example, in the case of the group
\uequ{
x' = x + a,
}
we are dealing with a continuous translation on a line; this translation is holonomic if the component $\omega$ of the infinitely small translation is an exact differential; it is not in the contrary case.