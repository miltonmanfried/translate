\subsection*{48.}

We return to an \textit{infinitely-small} closed contour on the manifold $(V)$. The approximate integration of equations (15) may be made by a procedure analogous to that which served in the study of manifolds with affine connections. Calling $\Delta x_i$ the infinitely-small variation experienced by $x_i$ , it is seen that this variation may be obtained by taking the bilinear covariant of the second part, replacing there the differentials $dx_k$ by their expressions furnished by the equations (15) themselves. One thus obtains, in a condensed form,
\nequ{17}{
\Delta f = \Omega_1 X_1 f + \Omega_2 X_2 f + \dots \Omega_r X_r f,
}
by. putting
\nequ{18}{
\Omega_s = \omega_s' + c_{hks}[\omega_h \omega_k].
}

\textit{To each closed contour is thus associated an infinitely-small displacement of the space $(E)$, a displacement whose components $\Omega_1,\dots,\Omega_r$ are the elements of a double integral.} This displacement measures in some sense the \textit{non-holonomy} of motion in the $(E)$.

Taking the exterior derivative of the formulae (18) gives, taking account of the relations which exist between the constants $c_{khs}$\footnote{These relations result from the fact that, for the manifold $(V)$ of parameters, the formulae (13) are true and thus those that can be deduced from exterior derivation. This means that, in taking the exterior derivative of the formulae (15), one may suppress all of the terms that contain neither the $\Omega_i$ nor the $\Omega_i$.},
\nequ{19}{
\Omega'_s = c_{khs}\left\{[\Omega_h \omega_k - \omega_h \Omega_k ]\right\}.
}

The interpretation of these formulae is analogous to that which we have obtained in the theory of manifolds with an affine connection.