\subsection*{50.}

In the particular case where the group $G$ is the group
\uequ{
x'=x+a,
}
the preceding general conservation theorem leads to a classical result. If $\omega$ is any Pfaffian expression on the manifold $(V)$, and if by $\omega'$ is designated the element of the double integral given by the application of Stokes' formula, the integral $\int\int\omega'$ extending over a closed surface is zero. Here there is no need to relate the infinitesimal translation of magnitude $\omega'$ on the reference system of the observer $\xa$, since the magnitude if a translation of a line is independent of the origin chosen for the axes, the transformation of the group $G$ being \textit{commutative}. In this particular case, it is again seen that the translation associated with any closed contour, \?{even a finite one}{même fini}, is always given by the integral $\int\int\omega'$ stretching over an area delimited by the contour.

We now return to the notion of a manifold with an affine connection. The space $(E)$ is here the affine space proper, the fundamental group is the group of affine transformations generated by the infinitesimal transformations
\uequ{
\frac{\partial f}{\partial x^i}, \quad x^i \frac{\partial f}{\partial x^j}.
}

There is now however a complete identity between the notion of a manifold with an affine connection and the notion of a manifold $(V)$ of observers introduced in the preceding considerations, since to each point $\m$ of the manifold with an affine connection there may correspond an infinity of observers adopting different reference systems, \textit{but with the same origin $\m$}.