\subsection*{51.}

We imagine a Euclidian space at each point $\m$ on which a rectangular coordinate system is adopted, with a length unit chosen at each point and which may vary from one point to another. Naming $\e_1$, $\e_2$, $\e_3$ equal to the unit of length \?{along}{portés sur} the axes, we have, passing from a point $\m$ to the infinitely close point $\m+d\m$, the generalized formulae (I) (chapter II)
\nequ{1}{
d\m &= \omega^1\e_1 + \omega^2 \e_2 + \omega^3 \e_3,\\
d\e_1 &= \omega_1^1\e_1 + \omega_1^2 \e_2 + \omega_1^3 \e_3,\\
d\e_2 &= \omega_2^1\e_1 + \omega_2^2 \e_2 + \omega_2^3 \e_3,\\
d\e_3 &= \omega_3^1\e_1 + \omega_3^2 \e_2 + \omega_3^3 \e_3.
}

Defining the scalar product of two vectors by taking a certain \textit{absolute} unit of length, we have
\uequ{
\e_1 \e_1 = \e_2 \e_2 = \e_3 \e_3, \quad \e_2 \e_3 = 0, \quad \e_3 \e_1 = 0, \quad \e_1 \e_2 = 0.
}

Differentiating these relations, the following formulae are obtained
\uequ{
\omega_1^1 = \omega_2^2 = \omega_3^3, \quad 
\omega_2^3 + \omega_3^2 = 0, \quad
\omega_1^3 + \omega_3^1 = 0, \quad
\omega_1^2 + \omega_2^1 = 0.
}

In place of $\omega_i^i$ we simply write $\omega$. The formulae (I) thus become
\nequ{1'}{
d\m &= \omega^k \e_k,\\
d\e_i &= \omega \e_i + \omega_i^k \e_k \quad (\omega^\text{unintelligible}_\text{unintelligible} + \omega_j^i) = 0,
}
where in the second relation the summation is taken for the values of the index $k$ different from $i$.