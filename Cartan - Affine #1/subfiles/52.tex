\subsection*{52.}

These formula permit us to define \textit{manifolds with a metric connections}. These are manifolds with affine connections where each at point $\m$ the tangent space is a Euclidian space. The mutual identification of two tangent Euclidian spaces at two infinitely close points is made by means of the Pfaffian expressions
\uequ{
\omega^i, \quad \omega, \quad \omega_i^j = -\omega_j^i;
}
the $\omega^i$ are the components of the \textit{translation}, $\omega$ is the component of the \?{\textit{dilation}}{homothétie}\footnote{\?{The relation}{Le rapport} of the dilation is $1 + \omega$.} and the $\omega_i^j$ are the components of the \textit{rotation} which brings the reference system attached to the point $\m$ into coincidence with the reference system attached to the point $\m'$.

It may also be imagined that there exists an absolute unit of length valid for all of the Euclidian spaces tangent to the manifold; the unit vectors $\e_1, \e_2, \e_3$ attached to each point being by convention all equal to one another. In this case, the form $\omega$ will be null. In this case, we say that we have a \textit{manifold with a Euclidian connection}. 