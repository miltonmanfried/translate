\subsection*{53.}

All of the preceding may be generalized to the case of an arbitrary number of dimensions. It may also be supposed that a non-rectangular reference system has been chosen, or where the rectangular reference systems with the vectors $\e_i$ are not equal to one another. Without treating the general case, we confine ourselves to indicating what happens when, a unit of length being chosen at each point $\m$, the reference vectors $\e_i$ are taken to satisfy
\uequ{
(\e_i)^2=g_{ij}, \quad \e_i \e_j = 0, \quad (i\neq j = 1,2,\dots,n),
}
the $g_{ij}$ being \textit{constant} coefficients chosen once and for all and the same once and for all the points of the manifold\footnote{It is obvious that this convention does not harm this generality at all; it may even be supposed that these coefficients are equal to $\pm 1$.}. Since the unit of length varies from one point to the other on the manifold, \?{one may simply differentiate}{one aura simplement le droit de différentier} the relations
\uequ{
\frac{(\e_1)^2}{g_{11}} = \frac{(\e_2)^2}{g_{22}} = \dots = \frac{(\e_n)^2}{g_{nn}}, \quad \e_i \e_j = 0,
}
which will give, taking account to equations (1),
\nequ{2}{
\omega_1^1 = \omega_2^2 = \dots = \omega_n^n, \quad
g_{ii} \omega_j^i + g_{jj} \omega_i^j = 0.
}

The square of the length of a vector $(\xi^i)$ issuing from $\m$ is, under these conditions,
\uequ{
g_{ii} (\xi^i)^2;
}
it varies with the unit of length $\m$ chosen at $\m$. The scalar product of two vectors $(\xi^i), (\eta^i)$ is similarly
\uequ{
g_{ii}\xi^i \eta^i.
}