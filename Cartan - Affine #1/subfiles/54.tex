\subsection*{54.}

Being given a vector $(\xi^i)$, we designate by $\xi_i$ the scalar product of this vector with the vector $\e_i$,
\uequ{
\xi_i = \xi \e_i = g_{ii} \e_i;
}
we say that the $\xi_i$ are the \textit{covariant} components of the vector\footnote{It is necessary to remark that the $g_{ii}$, being constants fixed once and for all, the covariant components of a vector are tied to their contravariant components in a manner independent of the choice of reference system.}, the $\xi^i$ being called its contravariant components. With this notation, the square of the length of a vector is $\xi_i \xi^i$ and the scalar product of two vectors is $\xi_i \eta^i$ or $\xi^i \eta_i$. The $ds^2$ of the manifold (square of the distance between two infinitely-near points) is similarly $\omega_i \omega^i$, putting
\uequ{
\omega_i = g_{ii} \omega^i.
}

We naturally designate by $\Omega_i$ the covariant components of the vector $\Omega^i$.

The form $\omega_i^j$ may be regarded as the $j^\text{th}$ (contravariant) component of the vector $d\e_i$; we put
\uequ{
\omega_{ij} = d\e_i \dot \e_j = g_{jj} \omega_i^j,
}
and similarly
\uequ{
\omega_{ji} = d\e_j \dot \e_i = g_{ii} \omega_j^i;
}
the relations (2) show that
\nequ{2'}{
\omega_{ij} + \omega_{ji} = 0.
}

We similarly put
\uequ{
\Omega_{ij} = g_{jj} \Omega_i^j = (d\e_i)' \e_j, \quad &
\Omega_{ji} = g_{ii} \Omega_j^i = (d\e_j)' \e_i, \quad;\\
\Omega^{ij} = \frac{1}{g_{ii}} \Omega^j_i, \quad &
\Omega^{ji} = \frac{1}{g_{jj}} \Omega^i_j.
}