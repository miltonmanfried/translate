\subsection*{56. Torsion and curvature.}

To each infinitely-small closed contour is associated:

\begin{enumerate}
	\item A translation $\e_i\Omega^i$;
	\item A dilation by $1 + \Omega$;
	\item A rotation with components $\Omega_i^j$.
\end{enumerate}

On the manifold, the translation defines the \textit{torsion}, the dilation the \textit{dilation curvature}, the rotation the \textit{rotation curvature}.

\textit{If the dilation curvature is zero, the manifold has a Euclidian connection.} One may satisfactorily dispose of the arbitrariness depending on the choice of reference system with the equation
\uequ{
\omega = 0;
}
this equation is in fact completely integrable, since the first part, having its bilinear covariant $\omega'$ identically zero, is an exact differential. The reference system attached to the different points $\m$ of the manifold may be chosen in such a manner that the connection becomes Euclidian: this is possible in an infinity of ways, the choice of a unit of length (which is now absolute) being arbitrary\footnote{Analytically, replacing $\e_i$ by $u\e_i$,where $u$ is an arbitrary parameter, $\omega$ is replaced by $\omega + \frac{du}{u}$. To make the connection Euclidian, just take $u=C\exp{-\int\omega}$.}