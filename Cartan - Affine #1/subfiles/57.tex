\subsection*{57. Geometric representation of tge rotation associated with an infinitely-small closed contour.}

The rotation of the components $\Omega_i^j$ may be represented geometrically. For this we take the formulae which give the geometric variation $\Delta\xi$ that performs this rotation on a vector $\xi$; there is
\uequ{
\Delta \xi^i = \xi^k \Omega_k^i,
}
or, what amounts to the same thing,
\uequ{
\Delta\xi^i = \xi_k \Omega^{ki}.
}
This variation is the geometric sum of the $\frac{n(n-1)}{2}$ variations due to the different components $\Omega^{ij}$ of the total rotation, which again permits regarding the total rotation as the sum of $\frac{n(n-1)}{2}$ component rotations. We take for example the rotation component $\Omega^{12}$; one has
\uequ{
\Delta\xi^1 &= \xi_2 \Omega^{21},\\
\Delta\xi^2 &= \xi_1 \Omega^{12},\\
\Delta\xi^3 &= \dots = \Delta\xi^n = 0;
}
this component of the rotation does not alter the vectors perpendicular to the plane $\e_1\e_2$; for a vector that is not perpendicular to this plane, its normal component is not altered, the rotation's effect is only felt on the projection of the vector onto the plane.

Consider in particular the vector $\e_1$ with components $(1,0)$; after the rotation it becomes the vector with components $(1, g_{11}\Omega_{12})$
\uequ{
\e_1' = \e_1 + g_{11}\Omega^{12}\e_2;
}
let $\theta$ be the angle this vector has rotated, taken to be positive in the direction from $\e_1$ towards $\e_2$; then
\uequ{
\e_1'\e_2 = |\e_1||\e_2|\theta = g_{11}\Omega^{12}|\e_2|^2,
}
from which
\uequ{
\theta = g_{11}\Omega^{12} \frac{|\e_2|}{|\e_1|}.
}

Finally, consider the bivector on the $\e_1\e_2$ plane with algebraic measure $\theta$; this bivector is\footnote{If in particular one considers an infinitely small area $d\sigma$ in the $\e_1\e_2$ plane, the contour everywhere being in the \?{positive direction}{sens direct}, the ratio $\frac{\Omega^{12}}{d\sigma}$ defines the \textit{curvature} of the plane element $\e_1\e_2$.}
\uequ{
\frac{[\e_1\e_2]\theta}{|\e_1\e_2|} = g_{11}\frac{\Omega^{12}}{|\e_1|^2}[\e_1\e_2]\Omega^{12}.
}

Thus, \textit{the rotation $\Omega_i^j$ may be represented by the system of bivectors}
\nequ{6}{
[\e_i\e_j]\Omega^{ij},
}
each term defining one of the $\frac{n(n-1)}{2}$ component rotations.

It is essential to remark that this representation has an intrinsic value if the manifold has a \textit{Euclidian} connection; if on the contrary the manifold has a metric connection, this system of bivectors depends on the choice of unit of length attached to the point $\m$.

One could say that, in the case of manifolds with Euclidian connections, the expression (6) is a \textit{bivectorial integral invariant} attached to each two-dimensional element on the manifold.