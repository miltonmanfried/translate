\header{July 7th 1924}

The political condition in Europe, especially in Germany-France, is pressing towards a violent upheaval. It is hardly comprehensible how the mood of the people could turn around in the exact opposite direction so soon after 1918. The evil forces are still at work today. How long yet? Who can say? In the end the great ray of light of our freedom will yet shine once again. One must not lose courage. The idea lives and\WTF{marches on into}{maschiert[!]...hinein} the future. Hail and Victory! For the new man! I am reading Bebel's memoirs. The man also started with nothing and later became the great, feared leader of the socialists. I believe that in his young years he was an ambitious idealist, later he reversed, i.e. he was a socialistic capitalist. The \?{leader}{Führer} who comes from the \?{people}{Volke}! Ach, god, the acclaimed auto-didact! \?{There are so many vermin running around}{Es läuft so viel Pack darin}. \WTF{Word-pudding!}{Phrasenbrei!} \WTF{We go miserably to the phrases of the half-educated}{Wir gehen an den Phrasen der Halbgebildeten elend kaput}. One is soon shy to bring his ideas out into the open: after a few days he finds them again as the most trivial phrases. Bebel's socialism was a sound development against the (at the time, all-powerful) liberalism. He was also patriotic towards the Fatherland. Proof: The struggle against Lasalle, perhaps from instinct. Later this socialism was contaminated by Judaism. How does a German petit-bourgeois get involved with the bloodthirsty world-catastrophic ideas of a Karl Marx, a Lenin and a Trotsky? The Russian is sufficiently fantastical, with him Bolshevism is mixed with all of the ideas of mysticism, fantasy, ecstasy etc; perhaps without the leader wanting and knowing. With that alone can Bolshevism hang on for long in Russia. Here in Germany it would have long ago been recognized and rectified. (see the Municb Räterepublik and the Berlin days in 1918 and the start of 1919.) Bolshevism is sound in its core. What we see of it today js \WTF{growing pains}{Krippenjagd}, incompetence, immaturity and cowardice. This fantastic, extremist leader of German communism will be destroyed by the German petit-bourgeois. By the German stupidity -- or insight, depending on how one looks at it. Bebel had sympathetic traits. He was seen as an upright, straigtforward character. But he gave intellectual people nothing -- nothing at all. He had no culture, wrote in an awful style, \?{preferred to speak and was himself unnerving}{spricht gern und auf die Nerven fallend von sich selbst} (even Noske did that, it also seemes to be the fashion with Red Rosa), he was unbearable except for his fine head. Indeed, if, instead of spitting out phrases in their big meetings they wrote a book instead, they would fail completely. Then the false magic of the outer effect would fall away and the man woukd stand before you in his total intellectual mediocrity. It is more difficult to unravel Karl Marx. Overall, the Jews are craftier. They speak fluently, educatedly, interestingly, avoid the obstacles of culturelessness and speak around the matter. The German workers are too \WTF{biderb}{???}-honest -- good for us, we understand them earlier and better. \WTF{Another Kost.}{Eine andere Kost.} Graf Eduard von Keyserling \?{night clubs}{abendliche Häuser}. That lack of culture, this refined and expensively worked-up culture; perhaps its end -- civilization. A rather more tired, decadent Graf recounts in his quiet, splendid history of the decline if his morbid race. And this he does with a love, with a secret hidden nostalgia, with a painfully smiling grief, that it pains the heart to read. \?{Mood of decline}{Untergangsstimmung}. Spengler, the bourgeois, the strong and unspent, \WTF{he creates from them the strong will for decline}{münzt sie aus zum starken Wollen für den Untergang}, this morbid Graf no longer has the courage and no longer has the power -- and perhaps no longer has the desire either. He recounts calmly -- and yet full of hidden melancholy - the sadness of the \WTF{houses in which it becomes evening}{der Häuser, in denen es Abend wird}. Thus a delightful culture goes to its end. We bourgeois may not hear too much of these things, we may not let it stick to us. We must go forth and work for the new race. Thomas Mann has shown in his work how dangerous it is to play with decline. Fritz from Unruh is precisely the polar opposite of Keyserling, the noble rebel, \WTF{who the new people also struggled against his race and against the tradition of his class underneath him}{der den neuen Menschen sucht auch gegen sein Geschlecht und gegen die Tradition seines Standes unter ihm}. He will not be content. Keyserling may despise that. \WTF{So he remains in his frame and gives the last style of his class}. But Unruh must struggle with the style and defeats it. What a culture, in Keyserling's language. Fine, engraved, a filigree. His irony remains noble, agreeable, not accusatory, understood, represented, excused. Who would be so bourgeois as to laugh about this wistfulness and feel himself above it? Keyserling's people are in their morbid uselessness still noble peple, people who one grows fond of. But we must overcome them. We do not need be tied to them. We still have -- against them and over and beyond them -- a task, an function, a mission. The best part of the nobility perhaps still has one. Maybe not all of them have lost theit usefulness. But \WTF{we put forward}{...stellen wir} the quintessence of the new man, we young men without tradition and \WTF{without race}{ohne Geschlecht}. We are the salt of the earth. Beyond the nobility and the bourgeois a new race. We must not despair, that is not the way and is too easy, no task for Europa's youth, who have lived through the worst times in human memory. \WTF{Keyserling only scribbles}{Keyserling strichelt nur}. But so urgently, that his people stand fully and completely before you. His Fastrade is a delightful dream-form. Bitter, sweet, strong and surrounded with all the wistful magic of a declining world. Dieter von Egloff, an arrogant good-for-nothing. Died from despair at the fact that he was put into the world without a calling and without a task. The scene, where Fastrade finds the fiance abandoned in the hut, is shattering in the frugal, succinct simplicity. "All alone, he had to die all alone, I was not there, I have abandoned him, I hadn't helped him, so he has died alone, no one was with him when he was in need." There is a moving lament for the dead fiance who the pure maiden abandoned, since the law demands it! My future lies in the impenetrable dark. I have nothing to hope for and everything to fear. Nothing to make me happy when I wake up in the morning. I live from day to day. All paths are closed to me. The brest is filled with desire, -- and completely superfluous. Where will I find salvation? In Berlin, there was an election for the student committee. 100 deputies were elected. Of them 60 Radical-Volkish. They youth did not fail and this time see purer and clearer than the proud elders. \?{One likes to hope when in despair}{Man hofft so gerne, wenn man in der Verzweiflung steht}. Here at home I am gradually beginning to be understood. That gives me pleasure and satisfaction. Now I have omce again spoken out. That makes me free and secure. \?{I gather only the future in myself}{Ich sammle in mir nur die Zukunft}. A good word can from time to time do wonders. We people are the servants of the mood and temper. Else helps me valiantly. That good girl. I owe her infinitely much. I would like to once again flap my wings! To fly in the blue distance! Why do we modern people love all that sickness? Are we ourselves sick? We have suffered too much! Decadence is sweet and bitter and the same time. But the mixture is seductive for our contemporaries. Pass it up, friends! Don't think of it! Sacrifice! Fulfill your mission!

 %