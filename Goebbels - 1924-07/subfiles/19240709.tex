\header{July 9th 1924}

More can be learned from Bebel than I had thought at first. Above all: that one must not lost heart when things go awry for a time. But the \?{noise}{Poltern}, the dimwitted polemizing against friends and enemies, \?{this smug smoke-blowing}{diese selbstgefällig Beräucherung} (all virtues to a social-democratic functionary) grates on the nerves. The typical signs of the half-educated are there in these men. Bloody autodidacts. Like to brag about their \WTF{cobbled-together}{angeknobelten} knowledge. Disguised bourgeois. Enemies of capital out of envy, not from a deep yearning and from sympathy with the poor. Negative capitalists. Enemies of the agrarians, since they themselves would like to be feudal lords. Socialism is not at all beautiful in theory. (That is the least that one can ask of a worldview.) They have no momemtum, no verve, no enthusiasm. Without idealism. The \WTF{theorizing}{enttheoretisierte} materialism. The eventual mechananization of all thought, feeling, work and interaction. As a person, Bebel is very sympathetic, as a memoirist simply impossible. Perhaps I will have something more to say about him after I have finished reading his second volume. The high politics of the day no longer consumes me as much. It is all so mindless. Phrases and \?{mush}{Brei}. The only thing that I enjoy is reading between the lines, to look behind the scenes. The French nationalists are arming again for an attack, in order to neutralize Herriot. Perhaps that is just as well. With Poincaré, every German knows where he stands. With Herriot, not so much. And the goal of each is the same. Open politics. With that we will get a united front in Germany. Perhaps England, with its hidden business sense, has harmed us more since 1918 than France, with its open desire for annihilation. State socialism has the future. I put my hopes in Russia. \?{Who knows what good shall come of the fact}{Wer weiß, wozu es gut ist} that just this holy land has to pass through the most crass bolshevism? Out feeling for the state must be steeped in responsibility and joy. We will have to overcome the present weariness of the state. I think of Anka Stalherm a lot these days. Strange. I cannot get rid of these types of people. \?{We still have so much to give eachother}{Wir hatten uns noch so viel zu geben}. She to me, \?{thoughts about}{an X...zum zu-Ende-Denken} nature, love and good, and me to her about strength, self-confidence, courage. I often dream of her. Then I see her as a lovely, proud lady, who takes life \?{as it is}{wie es nun einmal ist}. Is that true? What might she think and do? I believe that we need only a day together, and we will understand eachother. Dear, dear Anka! How often I yearn for You! Fidelity to the memory of you gives me a feeling of courage and strength. Then I always think that I must still fulfill what we both desired. I must become \WTF{"One as well"}{"auch Einer"} for her. Without the woman, I will never be ready. She does not give me much immediately. But she awoke a strength in me which would have been slumbering otherwise. Else is dear to me. Today she came, beaming with joy, in new Summer clothes. She had sewn it herself in one day in order to surprise me with it. If I am not enthusiastic, then she is deathl unhappy. She would rather die. This plaything is dangerous. Not for the strong! For them the woman is a delightful plaything. I am often depressed, am remorseful about something, and do not know why. Man is put in the world to suffer. We always have pain and remorse -- and the feeling of shame. Perhaps we carry the shame of others who came before us, or some shame from another life. In any case, there is a mysterious power which always drives us to do something to lessen the shame. The shame is is what stands between the last desire and the last pain. For that we must atone and sacrifice. Only do not forget that we are poor, poor people.

%