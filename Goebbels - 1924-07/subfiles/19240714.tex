\header{July 14th 1924}

Hot, sultry summer day. Yearning for Baltrum. There to lie in the hot dune sand and look over the infinite sea. And forget everything. No thinking. Politically hopeless and despairing. Struggle between industry and the market in France, i.e. between Poincaré, the exponent of national industry, and Herriot, the exponent of international exchange. The first will simply drive us into the ground, or better have us completely vanish (20 million Germans are too many), the second has no intention of annihilating us. On the contrary, they want to let us work for their moneybags. And yet the "perpetual contracts" are both of a short duration, since a race of 60 million cannot be made into slaves for eternity. The people, the unleashed beast in this beautiful world! A gruesome dissonance. Done with Bebel's memoirs. Thank god. They were rather boring and terribly uncultured. His phrases about the International fit him perfectly. The internationalists in communism are Marx, Liebknecht, Radek, Schdanek, and thus the Jews. The actual workers are in fact nationalist down to their bones, \?{even if they act like internationalists}{wenn sie sich auch noch so international gebärden}. It crushes them that the Jews are considered to be so mentally superior to them, and that \WTF{they destroy them with their fast talking}{sie mit ihrem Phrasenbrei vernichten}. A worker could never come arouns to the idea of the international. We never find internationalism from above, i.e. by denying nations. That contradicts all the laws of nature. Through a strong nationalist feeling for European thought. Only in this way do we balance the contradictions. One does not seek a Volk for a king, but rather a king for a Volk. All through Bebel's memoirs the old man from Sachsenwald\footnote{Bismarck} shines like a distant tempest. "He was a great hater, and as such he had always impressed me", Bebel says of him. A Kleistian hero. Just as great in hate as in love to the enemies of his fatherland, and to this fatherland itself. We need this old man now in these times of pain, more than back then when everything proceedes tolerably well. An errant butterfly is lying in the sun on my windowsill. Has \WTF{flown itself to death}{sich totgeflogen}. Poor creature! I now like playing chess with Else and love when, after a long strategic preparation, can mate her. But she has a certain sophistication in games. \WTF{Wife's cunning}{Weiberschlauheit}! She has been so needy latelt, and yet sometimes so real and sober, almost businesslike. A rare blend of lust and caution. She can never be carried away wholeheartedly. On top of that she is too reasonable. How totally different from Anka Stahlherm. She was ready to accept eternal damnation for a minute of bliss. A godly woman. But not a woman for me to marry. We would tear eachother down. We would -- without phrases -- die of love. I must at the first favorable opportunity see Anka Stahlherm once again. Today the bishop is in Rheydt. "The citizenry of Rheydt offers the worthy lord reverential greetings." (Nausea.) There is nothing like a healthy mash of lies and phrases.