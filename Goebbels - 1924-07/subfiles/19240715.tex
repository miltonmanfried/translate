\header{July 15th 1924}

Graf Eduard von Keyserling, "Tides". The old Keyserling. Not quite so tired as in "Twilight", but at the same time often piquant and charming. I liked the other book much better. In "Tides" the tired Graf is already rather too clear. He had already said too much. He is no longer so delicate. Or have I already become more accustomed to his type? The Geheimrat von Knospelius is a strange caricature. A soothsayer and a prophet. His closing words are distressing. \?{I have often thought to myself}{Ich habe oft an mich gedacht}. Destiny hasn't intended anything all that bad for me. Doralise again a piquant female. Similar to Fastrade. Must he always repeat himself? Even Hilmar is Egloff \#2. The same young, fresh, defiant little chap. And even the purely humane resolution is similar. Fastrade and Doralise remain alone with their pain. Wonderful images of the ocean. New. Peculiar. Also rather sickly. The sea is seen with tired aestheticist's eyes. These books are piquant, charming in their tired decadence, a refreshment for those with refined tastes, a handbook of good form and a refinedcway of life; but one must not read too much into that. It is just as with a sweet desert. Not even close to the everyday. Good, but not an elixer of life. \?{Not much substance}{Wenig Eiweiß}. Strange that the sickness and inner rottenness is still holding us captive. We probably still have some of this sickness in us. Or is it the woe and the pity that grips us when we see that the beautiful must die? Are we then all actually decadent? We are missing a supply of fresh blood. We have become sterile \?{in daring and creativity}{im Wagen und Schöpfen}. We must pull ourselves up. Not to bemoan what is lost, but to shape the future with joy. I live completely in the thought that my Michael will get the prize from the Kölnische Zeitung and travel in spirit around the continent as a scholar hungry for learning and knowledge. To Italy! Ach god! To Italy! Dostoevsky's "Nettchen Neswanow"\footnote{Actually: Netotschka Neswanowna}. Be joyful. The Russian philosophy is so plausible, since it is so clear and simple. The Russian seeks no problems outside of himself, since he carries it in his breast. Russia, when shall you awaken? The old world yearns for your redemption! Russia, you are the hope of a dying world! When will the day come?