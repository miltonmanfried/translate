The \?{creme de la creme}{Die hohe Korona} of the persistent friends of the fatherland were gathered at my home the day before yesterday. The circle is ever larger and \?{less significant}{unbedeutender}. Hartmann from M Gladbach spoke of grand politics. I fully share his ideas about Russia and its relation to us. Ex oriente lux. In spirit, in the state, in business and in the grand politics. The Western powers are already corrupt. Our ruling circle have a Westward impulse, since the Western powers are the classical states of liberalism. And under which liberalism is for those that have lived well (either money or connections or the obligatory recklessness and unscrupulousness). Out of the East comes the idea of the new state, of \?{individual connection and responsible discipline with regard to the state}{individuellen Gebundenheit und verantwort� lichen Zucht dem Staate gegen�ber}. And that does not please the gentlemen liberals. Hence, the pull to the West. Bank and stock market, big industry, big capitalism, agriculture is nonsense, making money is \?{all that matters}{des Pudels Kern}. The National Liberals and the Zentrum have in their deeper spiritual attitudes very much in common. Above all, this: the first \?{trades in}{machen ihrer Gesch�fte in} patriotism, the second in Catholicism. Both are equally dangerous for the idea of the Volkish community. Whoever in the first imagines he is \?{preserving}{gut aufgehoben} love of the fatherland, in the second love of the Catholic church, has made a bloody error. Business, and the nervus rerum, both go always and everywhere before patriotism and Christendom. Thus the instinctive hare of the lower classes against patriotism and church. (Here there is a contradiction. The National Liberals and the Zentrum claim to protect fatherland and church, though the worker only hates the members of these parties, not what they hypocritically  pretend to defend.) Put briefly: we must again make it clear to the worker that love and respect for fatherland and church has nothing to do with these vulgar parties. That the idea of the national Volkish community can actually be the idea of social leveling. 
In London, everything still the same. These gentlemen ministers don't take any steps forward. "That will lead to war", mother says. Could she be right? For the moment, probably not yet. In Russia signs of a renewed and much fore frightful outbreak of famine. Shall this bring the solution? Shall the most gruesome destiny of these people should to a realization. Men of Russia, chase the pack of Jews to hell, and reach your hand out to Germany. \?{To the coming people}{Zum kommenden Menschen}. The key to the European question lies in Russia. How can you put your hope in England and America? What is more precious, people or money? You gentlemen diplomats, read Spengler, Dostoevsky, and not Rathenau and the French. Depressed from inactivity. Mother helps me out as much as she can about that. Father is taciturn. Business is going poorly. Credit crisis is the new swindle. No child believes in its honesty anymore. The gentlemen up there will probably soon need to be told what new name to give the old swindle now. Inflation, sales crisis, credit crisis, overtime, good God, how we are being deceived. Idiots, open up your eyes! I see it now: the battle with stupidity and the slogan is the most difficult. A hundred army corps arrayed in vain against stupidity in the field. What is the point? A test? Purification? Do we have guilt to atone for? Is there justice in a life after this one? Is the old god asleep? Or does he trust his people too much? I doubt it all! Where am I supposed to get the strength to believe? If I had been born as a stronger person, that I could work, I would throw the whole plunder away and become Bergmann or Schlosser. Richard Wagner. Does not bring as much joy anymore. Now has enough money and little struggle against injustice. Hence one becomes soft and entitled. I love the fighters and martyrs. Beautiful, rather foggy July day. Else drives in some days to the Black Forest. The kid is joyful. I don't begrudge her it. If only I could go! I spend my days moping around. I hardly go outside. At my desk I chew over my annoyance with myself. I am filled up with it. When does the hour of \WTF{Platzens} come?
If a great revolution were to break out today, I would be ready to climb the barricades with a pistol. No artistic problem occupies me. I am dried out. The day and its trivialities rob me of too much strength. I am unsatisfied with myself and everything what concerns me. No excitement, no enthusiasm, no belief. Wait! Wait!! \WTF{If one knew in what - W��te man noch worauf}. Out of self-deception I send my "Michael" from one publisher to another. No one accepts it. Surprising? The whole thing is the history of the world in which one lives. What will our grandchildren say about these times? Keep quiet and hope! Only don't despair! Hold out! From self-preservation!

