It is still peculiar that with creative people, after a period of highest and most creative productivity there always follows a period of most withering drought and sterility. How it was with me last Autumn and Winter. Prometheus, Wanderer, Michael, and was not ready to formulate the \WTF{anst�rmenden} problems.Every day, indeed every hour new ideas, new deepening, new vistas. And now? Withered, drought, dejection, despair, no belief and no more hope. I read yesterday that Richard Wagner once went five years without composing a single note. Is this not similar? And though after he had already written Rienzi, Tannhauser, Hollander and Lohengrin. I read his biography to the end yesterday evening until deep into the night. What a story. At the end one is moved. In this life, a fate prevailed that was only appeared to be evil, but was in reality good.

With joy and excitement we are following the output of this great man. He has tact. How wonderfully keenly and delicately he has understood it, to introduce us into the deeper reasons of his relationship to Cosima.

\WTF{blah blah blah}

He is actually the prototype of the modern musician (as Lessing is of the modern writers), poor, homeless, exiled, without connections, without family, like the eternal Jew driven from land to land, full of the eerie, demonic restlessness that will not let him find a home anywhere. Wagner has already started becoming rootless. His struggle for the art of the future, for German ideas, is on the deepest level a struggle for something that he has perhaps already lost. The wellspring of the modern people is the yearning for something which has now been irretrievably lost. Yearning alone is not creative. \?{In addition, there is}{Zu ihr geh�rt noch}  love and strength. We lack both. Modern art is an art of yearning. There is rutting and heat, without lighting and kindling the flame of enthusiasm. Besides all of this, Wagner still has an \WTF{unpleasant virtue}{unangenehme Tugend}: wastefulness and a tendency to pomp and luxury. That usually does not happen with artists, especially not the last. His complaints about lack of money are no longer moving. He is also very generous with money, and one cannot reproach his relatives, friends and acquaintances if they leave his requests unanswered. Above all it hurts to see this wonderful person go begging. And how clearly he writes about it. That is probably due to genius. -- But Beethoven? Wagner probably needs pomp for his art (Lohengrin, Tannhauser, Tristan). It is probably difficult to feast in a whitewashed cellar.

Who knows? Perhaps it is more an elemental force that acts in this soul. Perhaps \WTF{he must give in and work on his strength to make it work for him}. The persistent fighters in honor. But I have no love for egoists and those who waste other people's money. And above all not the demanding types. But those are probably all small issues that cast little shame on the greatness of this man. In any cases, it muddies the picture a bit. Above all, since he himself is has written so uninhibitedly about all these things. It is true: the smallest mistakes obscure the greatest character. That with an artist, one can never separate the person from the artist! Why Wagner is always put first as a great person. Why not Beethoven? He was still incomparably greater in character. \!{But with Beethoven, the pure humanity is missing}{Aber bei Beethoven h�rt wohl das rein Menschliche schon auf}. There we find ourselves in the icy cold of the titan, of the Ubermensch, the visionaries -- the Sonderlings. With Wagner it always stays in principle normal and cosy. And purely human. There we can understand everything. With Beethoven it is often damnably uninviting and austere. "I hate Mammon!" Who has said thar? Beethoven or Wagner?
Else has celebrated today. Day after tomorrow she drives to the Black Forest. Then I am completely alone. A month of recuperation will do me good as well. I am nervous and exhausted. Most of the time the inner excitement and agitation makes me sick. I am irritable and in a foul mood. I sit up here in my little house the whole day, read, ponder, get annoyed, think over all kinds if dumb things and am glad when I am left in peace. I can undertake nothing for the future and consequently I have no bourgeois hopes. A \WTf{strengthening of my existence} is as far as I can see not thinkable in the least. A miracle would have to happen. I can't do anything anymore.

