Yesterday: in the night heard the gripping rally from Vienna over the radio. Tears well in the eyes. The January 30th 1933 for Austria. Seiss-Inquart named Bund chancellor. Miklas has given way to the power of events. A completely national-socialistic cabinet. Unending jubilation from the populace. In between speeches the Horst Wessel song. I listen until 3:00 in the morning and then can't sleep from joy. The new government has just taken over the office. All constraints are gone. This is the revolution for Austria. London and Paris register sharp protests. But what will all that do. They must bend to the facts. The Italian press has completely turned around. They welcome the developments. Mussolini does not take part in the protest. Italy will show that it dan also be loyal. There is consternation in Prague.  We breath in the morning air. \missing And in Paris there is a crisis. We don't want to wail, but rather feel pure joy. I give a decree that flags go out for three days. In a flash Berlin is transformed into a sea of flags. Frick has already worked out the law for Austria. Election announced for April 10th. Austria under Germany's protection. Fuehrer Bund president. He sets up the constitution. That will be adopted without further ado. And then we could carry forward its development how we want. The foreign press is in part very sharp, above all in London, otherwise resigned, above all in Paris. The protests from London and Paris are insignificant at the moment. The first reports are coming in. At 5:30 in the morning the march begins. Our troops have been greeted with indescribable enthusiasm. Austria is transported into a single paroxysm of joy. \missing. Then I make the proclamation on the radio.
Down below on the Wilhelmplatz the people go wild. Everyone is in an uproar. A glorious enthusiasm for the struggle. \missing The impression in the world is totally outrageous. But in the afternoon the press is essentially as calm as in the morning. The Volk receives the marching in of the German troops with an indescribable enthusiasm. Austria has never experienced this yet. \?{They have come to terms with it in Prague}{In Prag hat man sich abgefunden}. They view the German entry as legal. So what! They are in agreement in Rome. With some reservations, they declare their assent. The fuehrer has written a personal letter to Mussolini. There he offers the Italians their long-coveted military alliance. But the public still doesn't know that. In any case, Mussolini is apparently totally satisfied. The Italian press gives its strongest critique to Schuschnigg. They are totally beaten down in Paris. And in London it is announced that no tanks and no soldiers will be sacrificed for this hopeless cause. Bravo! The Viennese press is in chaos. The Jewish sheets are banned. Great shortage of journalists. Berndt has arrived. He has already acted. But that costs much effort and patience. The Jews have for the most part fled. To where? Like the eternal Jew, into nothingness. The fuehrer has arrived in Austria. Greeted with indescribable jubilation in Braunau. He is on the road to Linz and will continue to Vienna. That will be an entrance. I am so happy.

