\header{August 13th, 1932}

At noon, the fuehrer with Schleicher and Papen. He is encouraged to be content with the vice-chancellorship. Thus the goal is to wear him and the party out. An impossible request. If the fuehrer joins, he is lost. Thus, it is out of the
question. He flatly refused. Dr Frick enthusiastically seconded him. Thus these conversations went off to no result. The opposition claims that they want to leave the decision to the Reich President himself. Midday we sat at the Reichskanzlerplatz and awaited things to come. The fuehrer is strong and \?{decisive}{entschlußkräftig}. Whatever happens, happens; the decisive thing is only that we remain steadfast. At 3 in the afternoon State Secrtary Planck called from the Reich Chancellery. Question: "Has the decision already been made? Then there is no longer any reason for the fuehrer to come back." Answer: "The Herr Reich President wants to speak with him first." Vague hope. The fuehrer went with Dr Frick and Chief of Staff Roehm to discuss with the Reich President. Painful, tortured waiting. In a measly half hour he is there again. So it came to nothing. Everything rejected. Papen shall remain Chancellor, the fuehrer will be satisfied with the vice-chancellorship. A solution which leads to no results. An acceptance of this proposal is out of the question. There is nothing left to do but to refuse. The fuehrer had also done that immediately. He is completely clear about tje consequences with all of us. It will mean a difficult struggle, but we will nonetheless win it. An official communiqué on the decisive discussions was mistakenly put out, that the fuehrer had demanded the full power. In reality, he had only, and indeed with justification, requested the chancellorship; since that was refused him, we again adopt the opposition position. I wrote, while the conversation participants were agreed on a protocol, a very sharp essay in which the demand for power is again raised with all urgency. In the back room the SA leaders met under the chief of staff. They were brought up to speed by him and the fuehrer. It is  the hardest on them. Who knows whether their formations will be able to be kept. Nothing is more difficult than telling victorious troops that the victory has been \WTF{taken out of their hands}{aus den Händen geronnen}. A bitter task, but also one that must be resolved. It cannot go otherwise. The proposition that the fuehrer be the Vice-Chancellor of a bourgeois cabinet, is too grotesque to be taken seriously. Better to struggle for ten more years than to accept this offer. The fuehrer is marvellous in his calm clarity. He stands unwavering over all the vacillations, the hopes, vague opinions and suspicions. \WTF{A calming influence in the apparent maddness}{Ein ruhender Pol in der Erscheinungen Flucht}. A first chance spent! Struggle! Wilhelmstrasse will soon crumble. A Cromwell does not sit in the cabinet, and in the end the strong and the tenacious are always victorious over all obstacles. Feverishly in all rooms until the evening. Prepared records, put together speeches, and dictated appeals. On the whole floor, typewriters are clacking. A fantastic picture of calm and purposeful unity. The worse it gets, the stronger the party gets. Wilhelmstrasse will wonder how we \?{managed this probe}{dieser Probe bestehen werden}. All possible paths are considered. At the moment, no decision can be fixed. In every case there is one constant: this cabinet gets no majority in the Reichstag. Though that would not be bad if it was supported by the Volk, but that is precisely not the case here. A dictatorship of bayonets would end in chaos. This playing with fire can lead to monstrous results. General Schleicher is forced to not let his bridges be completely burned. That fits entirely with his character. After 2 or 3 hours \WTF{our initial astonishment has worn off}{der ersten Verblüffung ist bei uns wieder in Form}. The fuehrer has above all not lost his cool. Already last evening in Caputh he repeatedly stressed that the situation was not yet ripe. He was once again right. In the late evening he departed from Berlin back for Obersalzburg. He will need strong nerves more than ever in the coming weeks. In front of the apartment there were dense masses of people cheering him. Always the call: "Adolf, remain tough!" \WTF{The opposition will soon be pulled down}{Wir werden die Gegenseite schon von den Pferden herunterholen}. The Chief of Staff remained here with us for a long time. He has serious worries about the SA. He now has to carry out the uncomfortable task. But we all want to help him, and where there is a will, a way will be found. At midnight, everything is finished. The loud whirlwind falls silent. I read from the Letters of Fredrick the Great.
%ua fits enre