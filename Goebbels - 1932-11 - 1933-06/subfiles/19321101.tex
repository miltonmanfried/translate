\header{November 1st 1932 (Kaiserhof)}

We have challenged the German Nationals to a second big debate in the Sportspalast; but they seem to have no desire to disgrace themselves in public yet again. They rejected our invitation. That is such a shame; since on the historical stage of the sports palace, where we are at home, they would experience a yet more frightful defeat. Instead of that, I spoke in Braunschweig and Schoneberg. The nights are ever longer, since one needs an ever-expanding time to explain the so-tangled political facts, and the last meeting usually ends after midnight. The mood has visibly improved. In the party, they are again at full strength. Only the question remains, whether we will succeed in sweeping away the greater part of the electorate on the day of the vote. But it is not even all too bad if we lose a few million votes, since the decisive thing for the outcome of the is not this or that battle, but rather the question as to who has the last batallion to throw onto the battlefield. The tiring and grueling thing lies in the everlasting repetition, in the excess of work and in the minimum of sleep. And on top of that, the electoral battle is not over this year. If one only had to speak, then I could go on; but besides that one muat deal with the repulsivenesses of organization, with money questions, personal rivalries, and similar challenges. The yearning for November 6th and the conclusion of the electoral struggle is undescribable.