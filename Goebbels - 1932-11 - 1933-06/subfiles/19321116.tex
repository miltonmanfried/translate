\header{November 16th 1932 (Kaiserhof)}

Wilhelmstrasse's stocks are sinking. All parties have declared war on Papen. I was on the telephone with the fuehrer until nighttime and oriented him as to the individual phases of development. He cancelled a meeting with Papen in a very precise letter. It is claimed that Papen and Schleicher will step down today. With that, this cabinet is finished. The government is left without any choice. We are already arming for the next battle. Better safe than sorry. The cabinet is going to the Reich President in the evening. There it will register its \?{formal resignation}{Gesamtdemission}\footnote{Calculating that a parliamentary majority cannot be found and Hindenburg will not appoint them again, Papen stepped down with his cabinet on 17.11.1932, but nonetheless remained in office as caretaker}. The fueherer will appear to Hindenburg via telegraph. I telephoned him as well. It is very calm and serene. He wants to come to Berlin via airplane. Always repeating: \WTF{No expedient optimism!}{kein Zweckoptimismus!} I learned from a government middle-man that the fuehrer is to be \?{appointed}{beauftragt}. We are advised that he should \?{tighten up the presidential cabinet}{sich auf das Präsidalkabinett versteifen} and not attempt a majority solution. The intrigues continue all around. But we must keep our nerve and preserve our standpoint. The situation must be assessed calmly and cooly. We are still not at the finish. And if we again fail this time then we must be determined to continue fighting.

% cabineAdd one a day women's 50+