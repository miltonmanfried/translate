Midday around one o'clock another caucus meeting. Frick gave the members the rules of conduct. The Reichstag will probably be adjourned until the middle of January. In the organization it is now hard to work, since you can't predict what will happen in the next hour. The Reichstag will open. General Litzmann proves himself well as the senior president. The communists don't have the effrontery to attack him crudely and vilely. Hopefully soon the occasion will come to pay them back. Presidium election: Goering again goes through smoothly. For the vice president there is a tie between Loebe and a Volkspartei member. \?{While the lots for the Volkspartei member decided}{W�hrend das Los f�r den Volksparteiler entschei�det}, it arose from the final vote count that Loebe nevertheless was elected. Goering gave a cutting inaugural speech, with strongest zeal he put himself beside the fuehrer. That also makes a very good impression on the public. Then followed excited debates. The situation in the Reich is catastrophic. In Thuringia we have lost nearly 40 percent since July 31st. We must work more and negotiate less. In the evening, the fuehrer was at my house. We again spoke calmly about the whole situation. The fuehrer is at core an artistic person. With his certain instinct he instantaneously grasps every situation, and his decisions always have absolute clarity and penetrating logic. No one can come near him with regards to tactical moves. Even the Schleicher cabinet will smash headfirst against his efficiency.


