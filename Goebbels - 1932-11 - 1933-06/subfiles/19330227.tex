Kaiserhof

The huge propaganda action at the "Day of the awakening nation" is now set in all its particulars. It will unfold like a glorious show in all of Germany. The fuehrer is back from Munich. He had spoken there and in Nuremberg with huge success. He is enthusiastic about the course of our electoral campaign up to this point. I give the press instructions in preparation for the "Day of the awakening nation". We now concentrate all public interest on this single point. We will succeed in \?{bring everyone around}{herauszurei�en} with this day. In the evening I sit at home and work. At 9 o'clock the fuehrer comes by for supper. We make music and talk. Suddenly a call from Dr Hanfstaengl: "The Reichstag is burning!" I took that to be a wild and fantastic report and refuse to tell the fuehrer about it. \?{I look all around me}{Ich orientiere mich nach allen Seiten} and then get the frightful confirmation: it is true. The flames pour out of the great dome. Arson! I immediately inform the fuehrer, and then we race down the Charlottenburger Chaussee at 100km/hr to the Reichstag. The whole structure in flames. Over thick firehoses we arrived through door 2 in the great foyer. On the way there we met Goering and soon thereafter Papen was there too. In many places arson was already confirmed. There is no doubt that this is a last attempt by the communists to sow confusion through fire and terror, in order to seize power in the general panic. Now the decisive moment has arrived. \WTF{Goering is totally huge at the drive}; Hitler does not lose his cool for an instant; remarkable to see him here giving his orders, the same man who half an hour ago was sitting carefree with us at supper, chatting. The Plenum offers a singular picture of devastation. The flames reach up to the ceilings, which threaten to cave in any moment. Now, however, is the time for action. Goering immediately banned the whole Communist and Social-Democratic press. The Communist functionaries were arrested in the night. The SA is alarmed, to be ready for any eventuality. I race to the Gau to inform everyone there and to be prepared for every possibility. The fuehrer gathers a hastily-called-together cabinet meeting. We meet again briefly in the Kaiserhof and discuss the situation. One culprit is already apprehended, a young Dutch Communist named van der Lubbe. I drive with the fuehrer to the editorial offices of the "Volkischer Beobachter". We both go right to work, writing the lead article and appeals. I withdraw back to the Gau, in order to be able to dictate undisturbed. In the middle of the night, Diels, from the Prussian Interior Ministry, appears and gives me an in-depth report on the measures taken so far. The arrests went smoothly. The whole Communist and Social-Democratic press is already banned. If they attempt any opposition, the streets will be free for the SA. It is already morning, I meet the fuehrer again in the Kaiserhof. As far as the press goes, everything is in order. The line of our agitation is fixed by the event itself. Now we can go all the way. The KPD have deceived themselves. They think they have destroyed us, in reality they have dealt themselves their own death-blow. Two SA men were shot in Berlin overnight. They will not go unavenged. Worn down and exhausted I come home at 8 o'clock in the morning. \missing

