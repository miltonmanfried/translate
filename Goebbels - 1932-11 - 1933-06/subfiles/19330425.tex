Kaiserhof

Sunday: Plane to Cologne. A very stormy flight. In three hours however we are already there. My trip in the home city Rheydt was like my own triumphal procession. I accept the great reception in my home city as an honor to my mother. She has been slandered for a year in this little city, slapped, disrespected and hounded, and has had unending pain over it. Indeed one knows how that goes in these petit-bourgeois circles. Socially ostracized, that hits a vital nerve. It is torture for an old woman to hear for a year only regret and indignation about her unruly son, who lives in opposition to the church, state and society.
When such a defenseless person then on top of that has a sensitive disposition, then he can be completely destroyed by this meanness. I always still had friends and comrades-in-arms in Berlin, I could defend myself. I had a press, halls in which I could speak, an audience that would listen to me: I was never totally abandoned. My mother however was defenseless against the intrigue and malice of a cowardly, stupid and vulgar bourgeois community. A year-long torment and agonies of conscience shall now be repaid \WTF{nach meinem Willen} with a great triumph. That is why I have come to Rheydt, to show her that everything that she had suffered in those countless days for my sake and for our cause has not been in vain. That is a rehabilitation that would be unthinkable for a simple woman: that the people stands in their tens of thousands in the steets, that a whole city has been turned on its head, that the quarters are bathed in a sea of a single flag. On Monday afternoon I go to my old school, where I lived and worked with my old comrades day-in day-out for nine years and spoke from the same podium at which I first spoke as an Abiturient, the farewell speech of my class. At the time, that is, sixteen years ago now, after my performance, our old Ordinarius, who has been buried for a long time now, came to me, clapped me warmly and jovially on the shoulder and said: "You are talented, yes, but unfortunately not a born speaker." Proof how lovingly and understandingly he looked after my character and my talent for those 9 years. In the afternoon I spoke twice more, in Cologne on the radio and before the press, and again found my whole father-city in indescribable giddiness. It still hasn't been a year since they threw stones at me. I am glad when I can again escape the bustle and fly back to Berlin. In the cabinet the new Jewish law is adopted. A decisive step forward.
