Kaiserhof

The great day of the German people has broken. Yesterday there was still a threat of rain, today the sun is shining. Proper Hitler weather! Now everything will proceed perfectly. It doesn't so much come down to the execution of the day; what is important is the content and the meaning that comes through here. In the morning the Berlin school children march through the Lustgarten. Already the departure is overwhelming. Wherever one looks, under the lindens, in the whole Lustgarten, children, children, children. I speak to them from the depths of my heart. \WTF{Vor Kindern spricht's sich gut}, when one understands children's souls. Then the storm of jubilation: in the car, sitting next to one another, the Reich President and the fuehrer. Old and young united. A wonderful symbol of the new Germany that we have set up. Harald presents the Reich President with a big bouquet of roses. He is completely proud and happy. The Reich President speaks to the youth, as if he were \?{a man our age}{einer aus unserem Alter}. He exhorts fidelity, persistence, diligence and respect for the past. The fuehrer \?{gives them three cheers}{bringt auf ihn ein dreifaches Hoch}, with which the youth agree with enthusiasm and fervor. The drive through the masses of boys and girls is like a triumphal procession. The Tempelhof field is dominated by an indescribable throng of people. The Berliner is already on the way with a large entourage. Workers and citizens, high and low. Employers and subordinates, the distinctions are now blurred, only one German Volk marches. A couple years ago Berlin still rattled with machinegun fire. At the airfield we received the workers' delegations from the whole Reich, who are coming to Berlin on planes. On the faces of these earnest, hard men is the purest joy.
Midday they are invited to eat with the fuehrer and after that the Reich President is received. This reception is touching in its monumental simplicity. The great soldier of three wars, the faithful Ekkehard of the German nation, stands among the poorest of the countryside and forms a bond with them. Now the masses writhe through Berlin. An unending, unbroken stream of men, women and children pours into the Tempelhof field. In the evening towards 7 o'clock the announcement comes that one and a half million people had marched here. It was similar throughout the cities and villages of the Reich. I sit behind the fuehrer in the car as he drives his triumphal procession through the masses of workers, who fill the streets from Tempelhof field to the Reich Chancellery. It is indescribable. On the Tempelhof field one can no longer see past this enormous sea of people. The headlights shine over them. One can only see the heads of the gray masses. I get out briefly and ask for a moment of silence for the miners who had an accident that same day in Essen. Now the whole nation stands in silence. The loudspeaker carries the silence over city and country.
A moving moment of mutuality and closeness of all layers and all classes. Then the fuehrer speaks. Once more he summarized what we are and what we want. \WTF{He gives back to the work their new ethos}. \WTF{Labor now surrounds all good Germans}. The nation has meaning again. Now we will work and not despair. It is for Germany, its future and the future of our children. A great paroxysm of enthusiasm has come over the people. The Horst Wessel song rings out true and strong into the eternal night sky. The ether carries the voices of one and a half million people standing united here in Berlin on the Tempelhof field over all of Germany, through cities and villages, and now everywhere they agree. The workers in the Ruhr region, the boatmen of the Hamburg harbors, the lumberjacks from Upper Bavaria and the lonesome farmer at Masuren's lakes. Here nobody can be excluded, here we all belong, and it is no longer a slogan: we have become a single Volk if brothers. And he who showed us the way, who now leads, upright in the car, back to his place of work on Wilhelmstrasse, through a via triumphalis formed about him from a mass of living human bodies. Tomorrow we shall occupy the union houses. No opposition is expected anywhere. The struggle goes on! Up in the Reich Chancellery we stand at the window of the fuehrer's residence. The songs and chants of Heil from the marching masses ring out from the Tempelhof field all the way here. Berlin does not want to go to sleep, and with this metropolis the whole Reich shudders in joyous palpitations and will recognize the great hour of the turning of two ages. A marching column on Vossstrasse turns down Wilhelmstrasse. Beneath the Reich Chancellery the crooked-cross standards glow, the red flags bow before the fuehrer and greet him silently, full of reverence for him and his work. And from the throats of the young men rang Horst Wessel's eternal song: "Nun flattern die Hitlerfahnen �ner allen Stra�en..." We sit together until morning breaks. The long night is at an end. The sun is again rising over Germany!

Oben in der Reichskanzlei stehen wir in des F�hrers Wohnung mit ihm am Fenster. Von ferne klingen bis hier herauf die Ges�nge und Heilrufe der vom Tempelhofer Feld abmar� schierenden Massen. Berlin will nicht schlafen gehen, und mit dieser Riesenstadt zittert noch das ganze Reich in seligem Erbeben und wird sich der gro�en Stunde, die die Wende zweier Zeiten in sich schlie�t, bewu�t. Eben biegt eine Marschkolonne von der Vo�-