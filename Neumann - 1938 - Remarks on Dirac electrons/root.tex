\documentclass{article}
\usepackage[utf8]{inputenc}
\renewcommand*\rmdefault{ppl}
\usepackage{amsmath}
\usepackage{graphicx}
\usepackage{enumitem}
\usepackage{amssymb}
\usepackage{marginnote}
\newcommand{\nf}[2]{
\newcommand{#1}[1]{#2}
}
\newcommand{\nff}[2]{
\newcommand{#1}[2]{#2}
}
\newcommand{\rf}[2]{
\renewcommand{#1}[1]{#2}
}
\newcommand{\rff}[2]{
\renewcommand{#1}[2]{#2}
}

\newcommand{\nc}[2]{
  \newcommand{#1}{#2}
}
\newcommand{\rc}[2]{
  \renewcommand{#1}{#2}
}

\nff{\WTF}{#1 (\textit{#2})}

\nf{\translator}{\footnote{\textbf{Translator note:}#1}}
\nc{\sic}{{}^\text{(\textit{sic})}}

\newcommand{\nequ}[2]{
\begin{align*}
#1
\tag{#2}
\end{align*}
}

\newcommand{\uequ}[1]{
\begin{align*}
#1
\end{align*}
}

\nf{\sskip}{...\{#1\}...}
\nff{\iffy}{#2}
\nf{\?}{#1}
\nf{\tags}{#1}

\nf{\limX}{\underset{#1}{\lim}}
\newcommand{\sumXY}[2]{\underset{#1}{\overset{#2}{\sum}}}
\newcommand{\sumX}[1]{\underset{#1}{\sum}}
%\newcommand{\intXY}[2]{\int_{#1}^{#2}}
\nff{\intXY}{\underset{#1}{\overset{#2}{\int}}}

\nc{\fluc}{\overline{\delta_s^2}}

\rf{\exp}{e^{#1}}

\nc{\grad}{\operatorfont{grad}}
\rc{\div}{\operatorfont{div}}
\nc{\spur}{\operatorfont{spur}}

\nf{\pddt}{\frac{\partial{#1}}{\partial t}}
\nf{\ddt}{\frac{d{#1}}{dt}}

\nf{\inv}{\frac{1}{#1}}
\nf{\Nth}{{#1}^\text{th}}
\nff{\pddX}{\frac{\partial{#1}}{\partial{#2}}}
\nf{\rot}{\operatorfont{rot}{#1}}

\nf{\Elt}{\operatorfont{#1}}

\nff{\MF}{\nc{#1}{\mathfrak{#2}}}

\nc{\wta}{\widetilde{a}}

\MF{\fr}{r}
\MF{\fV}{V}
\MF{\fp}{p}

\nc{\fYm}{\fY^{(m)}}
\nc{\fXm}{\fX^{(m)}}
\nc{\fZm}{\fZ^{(m)}}

\nff{\MV}{\nc{#1}{\vec{#2}}}

\MV{\vgamma}{\gamma}
\MV{\valpha}{\alpha}

\nc{\Y}{\psi}
\nc{\y}{\varphi}

%niemals aufs Sofa

\title{Some remarks on the Dirac theory of the relativistic spinning electron}
\author{John von Neumann}
\date{1928}

\begin{document}

\maketitle

\begin{abstract}
Some properties of Dirac's spinning electrons are analyzed further, as well as the nature of the monochromatic de Broglie waves, the transformation characteristics of the four $\Y$-components, and the energy-current vector whose time component is the probability\footnote{Added in corrections. In a meanwhile-published paper (Proc. Roy. Soc. \textbf{118}, 351, 1928) Herr Dirac has likewise established a divergence-free current vector. Since our method is different from his, and since at the same time it gives a closer analysis of the relativistic transformation characteristics of the $\Y$, the following observations concerning this are perhaps not yet without interest.}.
\end{abstract}

\section*{Introduction}

\subsection*{I}

In a recently-published paper\footnote{Proc. Roy. Soc. \textbf{117}, 610, 1928}, Herr P.A.M. Dirac has proposed a new way if treating the quantum-mechanical one-body problem, in which certain deficiencies of the previous relativistic one-body equation\footnote{Namely the wave equation formulated by Fock, Gordon, Klein, Kudar and Schr\"odinger,
\uequ{
\left\{\sumX{k=1}{4}\left(\frac{h}{2\pi i} \pddX{}{x_k} - \frac{e}{c}\Phi_k\right)^2 + m^2 c^2\right\}\Y = 0,
}
where $x_1$, $x_2$, $x_3$ are tge three spatial components and $x_4=ict$, $\Phi_1,\Phi_2,\Phi_3,\Phi_4$ are the electromagnetic four-potential ($\Phi_1,\Phi_2,\Phi_3$ real, $\Phi_4$ purely imaginary). Gordon treated the Compton effect with it, ZS. f. Phys., \textbf{40}, 117, 1926; \WTF{this will also be discussed in more detail}{dort wird sie auch näher diskutiert}} are removed in a simple and satisfactory manner. His approach however does not only enable the removal of the aforementioned (relativistic) deficiencies, it also provides -- as he shows in l.c. -- entirely automatically the well-known spin-characteristics of the electron: namely the mechanical \WTF{angular momentum}{Drehmoment} $\frac{h}{4\pi}$, as well as its magnetic moments $\frac{he}{8\pi mc}$, $\frac{he}{4\pi mc}$ in the "interior" resp. "exterior" fields\footnote{C.f. Dirac, loc. cit., resp. p.619, 620, 624.}

This overwhelming success of the Dirac theory leaves little doubt

in the connection, that in its essential characteristics it takes on the the relativistic
behavior of "point masses"\footnote{\?{So that e.g. the spin of every form of matter, even the proton, must come from relativistic grounds.}} (though some difficulties highlighted by Dirac still exist)

\end{document}
