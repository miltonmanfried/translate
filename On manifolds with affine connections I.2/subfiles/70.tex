\subsection*{70.}

\nc{\overomega}{\overline{\omega}}

In chapter I we have seen that it is possible, in an infinite number of ways, to relate Newtonian gravitation back to geometry by attributing the universe an affine connection. In this conception the universe is a manifold with four dimensions whose affine connection satisfies \textit{a priori} the conditions expressed by the formulae
\nequ{1}{
\omega^0 = dt, \quad \omega_i^0 = 0,
}
$t$ designating the universal time. To these conditions may be added those expressing that the universe is metric, which is given by the relations
\nequ{2}{
\omega_i^j + \omega_j^i = 0 \quad (i,j = 1,2,3).
}

We say that a manifold satisfying only the conditions (1) has a \textit{Galilean connection}, and that it has a \textit{Newtonian connection} if it additionally satisfies the conditions (2).

As we have seen (sections 16-17), mechanical phenomena are compatible with an infinity of distinct affine connections: to $\omega_0^i$ and $\omega_i^j$ may be added the quantities $\overomega_0^i, \overomega_i^j$ on the sole condition that the three forms quadratic in $\omega^0, \omega^1, \omega^2, \omega^3$
\nequ{3}{
\omega_0\overomega_0^i + \omega^1\overomega_1^i + \omega^2\overomega_2^i + \omega^3\overomega_3^i
}
are identically zero.