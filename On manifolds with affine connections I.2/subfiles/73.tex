\subsection*{73. Invariant character of the relations (I), (II), (III).}

Time having an absolute meaning, the sections of the universe $t=\text{const.}$ also have an absolute meaning: they define the \textit{space} (manifold with three dimensions) at different instants of time. The structure of space, \textit{at a given instant}, is furnished by the formulae
\uequ{
&(\omega^i)' = [\omega^1 \omega_1^i] + [\omega^2 \omega_2^i] + [\omega^3 \omega_3^i]\quad (i=1,2,3),\\
&(\omega_i^j)' = [\omega_i^k\omega_k^j] + \Omega_i^j.
}

\textit{The relations (I) then express that at each instant the space is Euclidian}: they thus have an invariant character.

Consider now the three forms $\Omega_0^1, \Omega_0^2, \Omega_0^3$. To each infinitely-small closed contour in the space-time is associated ab infinitely-small affine displacement of the reference system attached to a point on this contour; the components $\Omega_2^3, \Omega_3^1, \Omega_1^2$ are zero, which means that the trirectangular triad which, in the reference system attached to the point under consideration, serves to localize the points in the space, does not change, but that its velocity of uniform rectilinear translation is augmented by the spatial vector $(\Omega_0^i)$.

Put another way, \textit{the spatial vector $\e_i \Omega_0^i$ is a vectorial integral attached to the universe}.

On the other hand, given that the torsion is zero, one has
\uequ{
[\omega^0 \Omega_0^i] + [\omega^1 \Omega_1^i] + [\omega^2 \Omega_2^i] + [\omega^3 \Omega_3^i] = 0,
}
that is to say, by virtue of (I),
\nequ{IV}{
[\omega^0 \Omega_0^i] = 0.
}\footnote{Thanks to these three equations, the equations (I) may be interpreted by saying that \textit{at each instant $t$}, the equivalence of two \textit{world-vectors} has an absolute meaning; one could also say that the equivalence of two \textit{space vectors} has an absolute meaning \textit{valid for the whole duration}. It is easily demonstrates that these two statements are equivalent in a universe with no torsion.}

That being so, considering the world vector
\uequ{
d\m = \e_0 \omega^0 + \e_1 \omega^1 + \e_2 \omega^2 + \e_3 \omega^3,
}
\textit{its space components} depend on the chosen reference system; but regardless of this reference system, the scalar product
\uequ{
[\omega_1 \Omega_0^1] + [\omega_2 \Omega_0^2] + [\omega_3 \Omega_0^3]
}
of these space components with the vector $\e_i \Omega_0^i$ always remains the same, since a change of reference system alters the three projections of these space components in terms of $\omega^0$ and the $[\omega^0 \Omega_0^i]$ are zero according to (IV). The relation (II) thus has an invariant character.

Finally, the first member of the relation (III) may be obtained by taking the space component of the system of bivectors
\uequ{
\inv{2}[d\m d\m] = [\e_0 \e_1] [\omega^0 \omega^1] + [\e_0 \e_2] [\omega^0 \omega^2] + [\e_0 \e_3] [\omega^0 \omega^3] + [\e_2 \e_3] [\omega^2 \omega^3] + [\e_3 \e_1] [\omega^3 \omega^1] + [\e_1 \e_2] [\omega^1 \omega^2],
}
and performing an exterior multiplication with the slace vector $\e_i \Omega_0^i$, which gives the space trivector (parallelepiped)
\uequ{
[\e1 \e_2 \e_3][\omega^2 \omega^3 \Omega_0^1 + \omega^3 \omega^1 \Omega_0^2 + \omega^1 \omega^2 \Omega_0^3];
}
the measure of this trivector does not depend on the choice of reference system, since a \?{change of reference}{changement de repère} only adds terms containing $\omega^0$ and which will be zero according to (IV).

We have thus indeed demonstrated the invariant character of the relations (I), (II), (III) for a manifold with a Newtonian connection and no torsion.