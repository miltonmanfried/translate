\subsection*{74. The relations (I), (II), (III) characterizing the universe of Newtonian gravitation.}

Assume in fact a torsion-free universe with a Newtonian connection; the relations (I) show that the space is always Euclidian. A space vector transported maintaining equivalence along any path will have the coordinates $\xi^1, \xi^2, \xi^3$ varying in a manner satisfying the relations
\uequ{
d\xi^i + \xi^1\omega_1^i + \xi^2\omega_2^i \xi^3\omega_3^i = 0;
}
now these relations are completely integrable by virtue of (I): thus one could, once choosing the spatial axes $\e_1, \e_2, \e_3$ at a particular point in the universe, choose them at every other point in a manner such that they are always equivalent to themselves, whatever path is taken. Put another way, one might suppose
\uequ{
\omega_2^3 = \omega_3^1 = \omega_1^2 = 0.
}

The formulae
\uequ{
(\omega^i)' = [dt\omega_0^i] + [\omega^1\omega_1^i] + [\omega^2\omega_2^i] + [\omega^3\omega_3^i]
}
then show that, if $t$ is supposed constant, $\omega^i$ is an exact differential. Thus one may put
\uequ{
\omega^1 = dx - a\,dt, \quad \omega^2 = dy - b\,dt, \quad \omega^3 = dz - c\,dt.
}

It can be supposed that the coefficients $a, b, c$ are zero by conveniently choosing the time vector $\e_0$, which gives simply
\uequ{
\omega^0=dt, \quad \omega^1 = dx, \quad \omega^2 dy, \omega^3 dz.
}

The torsion being zero, one has
\uequ{
(\omega^i)' = [dt\,\omega_0^i],
}
from which
\uequ{
\omega^1_0 = -X\,dt, \quad \omega_0^2 = -Y\,dt, \quad \omega_0^3 = -Z\,dt;
}
and consequently
\uequ{
\Omega_0^1 = -[dX\,dt],\quad \Omega_0^2 = -[dY\,dt], \quad \Omega_0^3 = -[dZ\,dt.]
}

The formulae (II), which are written
\uequ{
[(dx\,dX + dy\,dY + dz\,dZ) dt] = 0,
}
showing that $X\,dx + Y\,dy + Z\,dz$ is, \textit{for constant $t$}, an exact differential $dV$; thus one has
\uequ{
X = \pddX{V}{x}, \quad Y = \pddX{V}{y}, \quad Z = \pddX{V}{z}.
}

Finally, the relation (III) gives
\uequ{
\frac{\partial^2 V}{\partial x^2} + \frac{\partial^2 V}{\partial y^2} + \frac{\partial^2 V}{\partial z^2} 
 = -4\pi f \rho.
}
\textit{We thus retrieve all of the laws of Newtonian gravitation\footnote{There is naturally the additional condition that the potential $V$ vanishes at infinity. An analogous remark will be made   qconcerning Einsteinian gravitation.}} The momentum of a point with mass $m$ is
\uequ{
m\frac{dx}{dt}, \quad m\frac{dy}{dt}, \quad m\frac{dz}{dt},
}
and the equations of its motion, supposing this point is subject to the action of the entire given force, are
\uequ{
\frac{d}{dt}\left(\e_0 + \e_1\frac{dx}{dt} + \e_2\frac{dy}{dt} + \e_3\frac{dz}{dt}\right) = 0,
}
that is to say,
\uequ{
\frac{d^2 x}{dt^2} - \pddX{V}{x} = 0,
\quad\frac{d^2 y}{dt^2} - \pddX{V}{y} = 0, \quad\frac{d^2 z}{dt^2} - \pddX{V}{z} = 0.
}