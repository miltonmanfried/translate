\subsection*{76.}

In Einstein's theory of gravitation, the universe is regarded as a manifold whose reference systems are \?{related to one another}{se repèrent entre eux} as in special relativity. We say that \textit{this is a manifold with an Einsteinian connection}, which we state with the formulae
\nequ{6}{
\omega_i^0 = c^2 \omega_0^i,\quad \omega_i^j + \omega_j \quad (i,j=1,2,3),
}
where $c$ designates, in C.G.S. units, the number $3\times 10^10$.

The $ds^2$ of this manifold is
\uequ{
c^2(\omega^0)^2 - (\omega^1)^2 - (\omega^2)^2 - (\omega^3)^2;
}
thus, following the conventions set forward earlier (section 54),
\uequ{
\omega_0 = c^2 \omega^0, \quad \omega_1 = -\omega^1, \quad \omega_2 = -\omega^2, \quad \omega_3 = -\omega^3.
}

\textit{Supposing $c$ to be infinite, the manifold with a Newtonian connection is recovered.}

It is demonstrated as in section 71 that \textit{among all the mechanically-equivalent affine connections on a universe with an Einsteinian connection, there is one and only one which nullifies the scalar form $[\omega^i \Omega_i]$}.

It will be natural to say that this is \textit{just this one} which will constiture the affine connection of the universe. The torsion of the universe may theoretically then be nonzero.