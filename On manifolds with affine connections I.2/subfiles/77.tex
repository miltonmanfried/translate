\subsection*{77.}

In Einstein's theory of gravitation, the torsion of the universe is zero. Let us consider this hypothesis, which consists in attributing to the universe, from the point of view of torsion, the same character as in classical mechanics.

As has already been remarked, the laws of Newtonian gravitation no longer have an invariant character. \?{This is how they should be modified, respecting as much as possible their invariant structure.}{Il s'agit de savoir comment il convient de les modifier tout en respectant le plus possible leur structure général.}

First, \textit{the relations (1) must be abandoned}. The components $\Omega^{ij}$ of the system of bivectors $[\e_i \e_j]\Omega^{ij}$ which represent the rotation associated with a planar element of the universe are transformed linearly between those by a change of reference system, and it may be shown that the only real and homogenous system of linear relations between these components which are conserved by this change in system of reference is that which annuls all of the $\Omega^{ij}$. This is impossible, since it falls back on the definition of equivalence of Galilean reference systems utilized in special relativity and which is not valid when there is a gravitational field.

The relation (II)
\uequ{
[\omega_1 \Omega_0^1] + [\omega_2 \Omega_0^2] + [\omega_3 \Omega_0^3] = 0
}
\textit{is here self-evidently true}\footnote{For those admitted by Einstein's theory, this law (II) of Newtonian gravitation (existence of a gravitational potential) is a residue of the absence of torsion in Einstein's universe.}, since the component $\Omega^0$ of the torsion being zero by hypothesis, one has
\uequ{
[\omega^i \Omega_i^0] = 0,
}
that is to say, since $\Omega_i^0 = c^2\Omega_0^i$,
\uequ{
[\omega_i \Omega_0^i] = 0.
}

As for the relation (III), it no longer has an invariant character, but it recovers this invariant character of we write\footnote{The second part has had its sign changed since $\omega_1, \omega_2, \omega_3, \Omega_{01}, \Omega_{02}, \Omega_{03}$ are equal and opposite to $\omega^1, \omega^2, \omega^3, \Omega_0^1, \Omega_0^2, \Omega_0^3$. }
\uequ{
[\omega_2 \omega_3 \Omega_{01}] + [\omega_3 \omega_1 \Omega_{02}] + [\omega_1 \omega_2 \Omega_{03}] + [\omega_0 \omega_1 \Omega_{23}] + [\omega_0 \omega_2 \Omega_{31}] + [\omega_0 \omega_3 \Omega_{12}] = -4\pi f \rho [\omega^1 \omega^2 \omega^3 \omega^0],
}
since \textit{the first part is. as we have seen, an integral invariant attached to the manifold.}