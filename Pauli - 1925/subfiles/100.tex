\letter{100-UNFINISHED}
\rcpt{Kronig}
\date{October 9, 1925}
\location{Hamburg}

Dear Kronig!

I have now pretty much come to terms with the interpretation of the intercombination lines via the model - so far as this is at all possible in the present general \WTF{madness}{Wirrwarr} of atomic physics - and for this reason I shall now fulfill my promise to write you again on the matter.

I start out by considering an atom with two valence electrons. As you shall probably recall, the naive application of my \WTF{trick}{Schimmel} (two-valuedness of the light electrons, uniqueness of the \WTF{rest of the atom}{Atomrestes}) leads to two pairs of states and should e.g. have in the $p$-terms, the terms $p_0$ and $p_1$ the same relativistic corrections, corresponding to $k_2=1$, while the pair $P_{(1)}, p_2$ should get the relativistic correction with $k_2=2$. (The indices under the letters specify the $j$-values.) It is known however that under normal conditions the grouping of 1+3 terms (singlet and triplet) is found instead of the grouping in 2+2 terms.

In contrast, I would now like to \?{show that it is also plausible} that the latter grouping can be realized under suitable conditions. Namely one considers \?{homologous} atoms with two valence electrons, with the electron number remaining constant and an increasing nuclear charge. How then will the triplet-splitting and the distance of the triplet- from the singlet-terms vary with the atomic number? From the triplet's $\delta\nu$ one knows that they initially go with $\alpha^2 Z^4_{\text{eff}}$ and from the distance $\Delta\nu$ from the singlet and triplet system one will have to assume that it behaves as a \WTF{shielding}{Abschirmung} doublet, so that $\Delta\sqrt{\nu}$ is initially roughly independent of $Z$. (According to Heisenberg's schema III this would in fact be the case, since according to him the singlet-triplet energy difference should be due to the electrical interaction forces.) Since now $\delta\nu$ increases much more quickly with $Z$ than $\Delta\nu$ a point will finally arrive, where the singlet-triplet difference will be the same order of magnitude as the triplet splitting. And now I claim that \textit{with these $Z$-values a transformation of the term-formation takes place, where with greater $Z$, in place of the 1+3 grouping, the above-mentioned 2+2 grouping sets in.} Then with greater $Z$, $(p_0, p_1)$ and $(p_2, P)$ will each form a \?{shielding} doublet, while the distance between these pairs forms a relativistic doublet. The \WTF{graph}{Verlauf} will look something like this,
TODO: IMAGE: p243
according to whether the singlet term lies above or below the triplet term in the optical domain (both of which can happen). Incidentally in Hg the transition phase is already realized in the optical domain.

In Millikan and Bowen one is still in the domain of the 1+3 grouping and the postulated transition is not yet practically realized. That my postulate is however a natural one seems to emerge already from the case where \?{an electron is far away from the $L$-shell (and indeed $L_{II}$ or $L_{III}$) of a heavy element, in addition to a valence electron in an "orbit" with $k=1$ ($s$-term)}. According to the reciprocity rule indeed the ionized $L_{II \text{ or } III}$-shell is equivalent with an electron in a $2_2$-"orbit", and we could regard the configuration as analogous to those which correspond to the $p$-terms of the alkaline earth elements. It is clear that one cannot here speak of a 1+3 grouping. Rather the outer valence electron will have only a minute (and in the Roentgen domain practically imperceptible) \WTF{splitting}{Zweiteiling} of each of the two levels of the relativistic doublets $L_{II}$, $L_{III}$ (and thus 2+2 grouping). Incidentally similar considerations have also been found by Jordan.

If we now attempt to describe the specified conversion in more detail, \skipped{...rest of the paragraoh about term zoology}

\nc{\kbar}{\overline{k}}

Apparently the following ersatz model is suitable for the representation of the facts. One assigns the inner electron a vector $i_1$, but the outer electron however gets \textit{two} vectors $\kbar$ and $i_2$. Then it is convenient to say that $i_1$ and $i_2$ both have the length $\inv{2}$, $\kbar$ has the length $0$ for the $s$-term, $1$ for the $p$-term, etc. Then one namely obtains the $j$-value by simple vector addition of these three vectors. Naturally one could make all of the following so that the $\kbar$ is assumes greater than $\inv{2}$ and then \?{averages between these two positions}. Now we propose the Ansatz

the \WTF{orientation forces}{Orientierungskräfte} (which belong to those parts of the potential energy that depend on the angle between $i_1$ and $i_2$) acting between $i_1$ and $i_2$ are of the same order of magnitude as the electrical interaction forces and determine the difference $\Delta\nu$ between the singlet and triplet system for small $Z$. The interaction energy between $\kbar$ and $i_2$ is relativistic, but is reinterpreted in the ersatz model in the well-known fashion as depending on the angle between $\kbar$ and $i_2$. For small $Z$, it is proportional to the triplet splitting $\delta\nu$.

With small $Z$, where $\delta\nu \ll \Delta\nu$, \?{at first the $i_1$ and $i_2$ come together for a resulting} $r=1 \text{ or } 0$ and this will (much more slowly than $i_1$ and $i_2$) precess about $\kbar$ (singlet-triplet grouping).

For greater $Z$, where $\Delta\nu \ll \delta\nu$, \?{$\kbar$ and $i_2$ will come together to a resulting $k_2$} (relativistic doublet) and this will then precess very slpwly about $i_1$.

TODO IMAGE p244

In between these there is a transformation, which is largely analogous to the Paschen-Back effect.And now I am where I wanted to be, namely at the inter-combination lines.