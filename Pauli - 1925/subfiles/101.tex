\letter{101}
\from{Heisenberg}
\date{October 12, 1925}
\location{Copenhagen}

Dear Pauli!

I think I owe you a letter. There is some news in quantum mechanics: the general canonical transformation reads:

\uequ{
P_k = S p_k S^{-1}\\
Q_k = S q_k S^{-1}.
}

Putting $S=\exp{1+\lambda\varphi_1 + \lambda^2\varphi_2};\dots$ yields the perturbation theory which I wrote about. Further, the general method of integration apparently goes like this: one finds a transformation function $S$ which takes
\uequ{
H(p_k,q_k) = S^{-1}H'(P_k,Q_k)S = W
}
($H'$ has the same \textit{form} of a function of $PQ$ as $H$ in $pq$) into a diagonal matrix. This problem (which I don't want to write about in detail) is equivalent with a principal axis transformation woth infinitely-many variables, i.e. a system of infinitely-many homogenous linear equations. The energy values $W_n$ are the eigenvalues of this problem and can be discretely or continuously distributed. The multiplicity of an eigenvalue gives the statistical weight.

Beside that we have also gotten some more or less specific results: in systems with axial symmetry, the momentum is an integer or half-integer multiple of $\frac{h}{2\pi}$ so that the possible states are symmetric about zero and form gapless rows (exclusing the impossible zero) whose terms differ my $\frac{h}{2\pi}$.

Aside from that, I don't believe one more word of your Zwang theory. Though it is certainly \?{true with hydrogen}, I don't believe that it has anything to do with the nucleus, but \textit{
perhaps} it is already \?{a bit moldy}, since according to quantum mechanics e.g. the Larmor theorem does not apply in general. E.g. with hydrogen even for the orbits in the normal state there is a relatively large probability to go over to free electrons via absorbtion, and for the free electrons the Larmor theorem certainly does not apply. So it is immediately clear that it does not apply in general, but it naturally cannot be known whether this viewpoint can help to understand the actual Zeeman effect. Nonetheless the Larmor theorem naturally applies in the oscillator.

From your letter to Kronig I saw with amazement with what virtuosity you play the zoology-organ: variations on an old tune. Incidentally, your model 1925-10 is not new to me, we have already calculated with it earlier (e.g. as we discussed about my work). But I am in full agreement with all of your considerations. As regards the departure of the quantum number $i$, I shed to tears for its passing, may the whole zoology follow it! If the \?{impending} obituary stems from you, many thanks! \?{A good joke in honor!}

Concerning your last two letters I must scold you, and excuse me if I \?{proceed like a Bavarian}: it is really disgusting that you are not able to stop the \?{cursing}. Your perpetual name-calling at Copenhagen and G\"ottingen is simply a shriekinh scandal. You must nevertheless give us leave, since we in any case we don't aspire to maliciously ruin physics; if you reproach us thatbwe are such great asses that we have not yet brought anything physically new to fruition, then that may be correct. But then you are a yet greater ass, since you haven't yet brought anything to fruition either... (the dots indicate a curse of approximately two minutes duration!)

No offense and many greetings,

\textsc{W. Heisenberg}

%t Fai the r and balanced skin