\letter{105}
\from{Heisenberg}
\date{November 16, 1925}
\location{G\"ottingen}

Dear Pauli! 

Thanks for your last letter. Here I am sending you the "final" version of our paper, and I would be very glad if you could briefly write me whether you still have any thoughts or other remarks. I have given all effort to make the paper more physical than it was, and I am \?{halfway satisfied with it}. But I am still rather unhappy with the whole theory, and so was glad that we are so completely on the same page as regards mathematics and physics. Here I am surrounded by those that think and feel exactly the opposite, and I don't know whether I am just too dumb to understand mathematics. G\"otting has split into two camps, the one which, like Hilbert (or also Weyl in a letter to Jordan), speaks of the great success which could be achieved by the introduction of matrix calculations in physics, the other which, like Franck, says that the matrices are \?{incomprehensible}. I am always infuriated when I hear the theory named just as matrix physics and had for a long time actually stricken the word matrix completely out of the paper and replaced it by another, e.g. "quantum theoretical quantity". (Incidentally matrix is probably one of the dumbest mathematical words that there is.) Also, I don't know whether the principal axis transform should be completely thrown out, since you have shown how one \textit{actually} integrates with hydrogen, and the rest is just formal clutter.

In recent times I have done much thinking about coupling and radiation, and I have the hope that some insight on the coupling can be gained from the new theory. The starting point of the reflections was basically that Einstein's light quantum theory was indeed entirely created from the calculation of the squared fluctuation; thus by a close analysis of the \?{grounds} that lead to the correct squared fluctuation in the new theory, all essential characteristics for the light quantum theory can be found again. However, up until now, I have had large conceptual difficulties. Yet one can already say e.g. what must emerge from Bohr's thought experiment on molecular radiation.

TODO IMAGE PAGE 256

\?{The atoms illuminated by the radiation will only be in phase \textit{in the light field}, afterwards (so from $a$ on) only the atoms in the "upper" state radiate (and indeed without phase), and so also only the ray $o$ is emitted, as expected from the light quantum theory. It is exactly the same when the incoming light (left of $a$) has a frequency different from the eigenfrequency of the atoms (only then to the right of $a$, the atoms are very seldom in the upper states)}

\?{Incidentally, there is said to be a paper by Stern and Vollmer which has everything about radiation that can be done with alternating usage of the two theories; where does the truth lie?}

I have also done some thinking about the "Zwang" and I still have the hope that it could only come in with the coupling of two (or more) electrons (and indeed just through your ban on equivalent orbits). It would interest me very much to hear whether you have meanwhile gotten a theory of relativity and the Zeeman effect in hydrogen. (The Stark effect probably works without anything extra.)

Please greet your whole institute! Some time in January I shall probably actually come to Hamburg, since I am to hold a popular lecture -- in Stade (!!) (pecuniae causa).

Many greetings to you yourself!

\textsc{W. Heisenberg}

%wyisndy thou a stranger in the alps