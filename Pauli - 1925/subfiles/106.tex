\letter{106}
\rcpt{Bohr}
\date{November 17, 1925}
\location{Hamburg}

Venerable dear Herr Professor!

You have given me great joy by sending the manuscript of your lecture, which I have read with interest and satisfaction, and by your letter, since I see from them that you still often think of me. I am in general very happy with the content of the lecture, only with two still to-be-mentioned points do I have small, unsatisfied desires.

Enclosed I send you a brief summary of my calculations on the hydrogen atom from the standpoint of the new G\"ottingen quantum theory. In addition to the Balmer formula and the Stark effect (unfortunately I have still not been successful in calculating the energy values of the relativistic fine structure) it gives as its most important result the following term manifold by removing the degeneracy with an additional central force and an axially-symmetrical force field: normal state
\uequ{
E_1 = -\frac{Rh}{1^2},\quad E_2= -\frac{Rh}{2^2},\quad E_3 = -\frac{Rh}{3^2}
}
\begin{tabular}{ccccccc}
\, & \, & \, & \, & $m=$ & $0$ & $(k=0)$\\
\, & $m=$ & $0$ & $(k=0)$ & \, &\, & \, \\
$k=m=0$ & \, & \, & \, & $m=$ & $-1\,0\,1$ & $(k=1)$ \\
\, & $m=$ & $-1\,0\,1$ & $(k=1)$ & \, &\, & \, \\
\, & \, & \, & \, & $m=$ & $-2\,-1\,0\,1\,2$ & $(k=2)$\\
\end{tabular}
generally for any (integer) $k$, $m$ runes from $-k$ to $+k$ and in the $n^\text{th}$ quantum state $k$ can assume the values $0,1,2,\dots,n-1$. There $m\frac{h}{2\pi}$ denotes the component of the momentum parallel to the force field, $k$ is a number which distinguishes the various states of the relativistic fine structure and obeys the selection rule $\Delta k = \pm 1$ when the external force field is weak / vanishes. The novelty in this term manifold is that the value $m=n$ cannot occur. This follows automatically, without additional assumptions, on the basis of the new theory. Correspondingly it follows for the Stark effect that the value $s=n$ of the Stark effectquantum number automatically does not occur, whichcorresponds to experiment.
 \textit{The difficulties with crossed fields completely dissapear.}
 
And here I would like to remark on these difficulties, which are also discussed in your lecture. I would be happy if you would append a few sentences (or, if you prefer, lengthening an existing sentence) in which it is explained what these difficultirse actually consist of. It is, I believe, not sufficiently stressed that the root of these difficulties is the (from the standpoint of the quantization rules, arbitrary) additional ban on straight-line orbits. By this I mean not the very specific exclusion of $m=0$ in axially-symmetrical fields, but rather the exclusio of $k=0$ in the Sommerfeld theory of the relativistic fine structure. This was the source of all evil and the originak sin of all considerations of the hydrogen atom resting on the quantization rules for periodic systems. The ban on $m=0$ was then only the logically-necessary consequence of this first ban and the adiabatic processes in crossed fields represent only specifically dire consequences of the ban on $k=0$ in the fine structure, which on the one hand was necessary in order to meet the requirements of experience, but on the other hand is entirely theoretically unfounded. (Incidentally, entirely disregarding the case of crossed fields, \?{one can adiabatically convert completely harmless states of the anisotropic harmonic oscillator in this straight-line orbit to the relitavistic fine structure} without passing through a degenerate system.) And yet: at the time no one was able to avoid the original sin of tge straight-line orbit, which I see as a limit to the applicability of the theory if periodic systems, since the price of its salvation is no less than the renunciation of mechanical, spatial-temporal images of the stationary states of the hydrogen atom. Since such additional restrictions don't occur in the new Heisenberg quantum mechanics with hydrogen, all of the difficulties in crossed fields also disappear. The weight of the $n^\text{th}$-quantum state becomes equal to $n^2$.

So beautiful and \?{unified} is the application of Heisenberg's quantum mechanics to the hydrogen atom, which in particular has the consequence that the normal state of the hydrogen atom should be unmagnetic -- \?{yet } don't think it is finished yet. It would indeed be a logical possibility that atomic force that you summarized as a "Zwang" only occurs in atoms with several electrons. Yet I have the definite feeling (I cannot logically prove it) that the difference between atoms with one electron and those with several electrons as regards the presence of the Zwang is exaggerated. I particularly believe that if one were to do the experiment on the deflection of H-atom rays in an inhomogenous magnetix field (to which I have so far been unable to move Stern to do) the same would be seen as with Ag-atoms, namsly $\pm 1$ magneton and \textit{not} unmagnetic atoms. In general I believe that the reality is different from the results derived from the present fundamenta,s of the Heisenberg theory for the hydrogen atom in that to every orbit term $E_n=-\frac{Rh}{n^2}$, instead of $n^2$ in truth $2n^2$ should be present, and to every energy value of the "Zwangless" Heisenberg theory in the magnetic field \?{the term-number-doubling addendum $\pm 1\times\text{Larmour freq}\times h$ is appended}.

I could go alone with the interpretation of a fundamental difference as regards the Zwang in atoms with one electron and those with several electrons if one could put forward a theory with a more unified foundation, from which on the one hand the Zwang-lessness of the hydrogen atom, and on the other hand the empirically-found anomolous Zeeman effect of the alkalines correctly followed. In the present version of Heisenberg's theory, however, this is not the case; it always leads to the Larmour theorem and I am rather certain that it can be said that it will lead to a one- instead of a two-term system. The occurence of two term systems is closely linked to the possibility of two states of the normal state of the hydrogen atom in an external force field.

Which fundamentals of the Heisenberg theory should now be modified? Its current fundamentals can be best represented by introducing in addition to the rules $pq-qp=\frac{h}{2\pi i}$ and the frequenxy conditions the energy law $H(pq) = W$ (diagonal matrix). The equations of motion are then yielded as a consequence. (It would be nice, I think, if you would base the last part of your lecture on \textit{this} representation of the theory.) Now it is indeed the simplest Ansatz, and it is also natural in view of the asymptotic correspondence with the classical theory in the limiting case of large quantum numbers, to always initially posit the classical form of $H(pq)$ for the energy function. But, in view of the Zwang and the anomolous Zeeman effect, one can also conceive of introducing (keeping the rules $pq-qp$ and the frequency condition) other functions than the classical. But of course then the problem is to do this in a natural way free of arbitrariness.

If the present form of the Heisenberg still doesn't supply a complete foundation for a theory of the structure of atoms, it has nonetheless -- totally apart from the immense theoretical progress, which I have warmly greeted and completely amazes me -- robbed the difficulties of at least some of their horror. First, as the G\"ottingers have recently shown, the rules for the intensity of the Zerman conponents could be derived from the theory.

Second, in the case of the transformation of the relativistic fine structure in the Stark effect with increasing electrcal field strength that can be treatd without a "Zwang", which was treated by Kramers and showed the deep analogy to the Paschen-Backneffect, the sum rules that I suspected\footnote{I wrote about them to Heisenberg in Copenhagen a few months ago.} for the energy values which are analogous to the magnetic sum rules, are actually given by Heisenberg theory according to my newer calculations. It can be hoped that this holds analogously for the sum rules in the Paschen-Back effect with suitable choice of the Hamiltonian function.

Now there are still some questions of principle which are often raised by the new theory. This theory is initially only cut out for such cases where all particles remain in \?{finite spaces}, it is still not suited in its present formulation for e.g. collision phenomena and the problem of simultaneously grasping coupling and interference. Above all, we have no logically unified theory which includes \textit{all} applications of the classical theory in itself in the limiting case of large quantum numbers. Perhaps the following points in the right direction of possible future progress. In the new theory, all physically-observable quantities still don't \?{actually} occur, namely it is still missing the \?{time at which transition processes occur} (like e.g. the moment of release of photo-electrons), which are indeed certainly observable in principle. It is now my firm conviction that a truly satisfactory physical theory must not only make no use of unobservable quantities, but also must connect \textit{all} observable quantities with one another. I also definitely believe that in the fundamental laws of a satisfactory physical theory, the concept of "probability" must not occur. I am prepared to pay an arbitrarily-high price for the fulfillment of this desire, but unfortunately I still don't know the price for which it is to be had.

If one now reflects on where the difficulties lie in making statements in the new theory about the time at which transitions processes occur, one soon notes that time actually doesn't enter into the new theory at all\footnote{Instead of $\exp{2\pi i \nu t}$ one could likewise write $\exp{2\pi i \nu}$ or the like and \textit{define} the time-derivative of a variable $q$ by $\frac{2\pi i}{h}(Wq-qW)$.}. The problem is then this: to establish more general concepts which encompass the actual applications which have been made from the classical space-time images. The situation is now that the concept of the "timing" of a process and specifically the "time period" resp. "\?{vibration number}" have become entirely formal. \textit{The "formal character of the frequency condition" (p. 8 of the manuscript of your lecture) is constrained by the formal character of the time and not by the formal character of the energy.}

Thus can one perhaps formulate the knowledge gained through the experiments on coupling (Geiger-Bothe, etc). \?{It floats darkly before me, like an attempt} to \textit{define} the time somehow by just the energy, by means of the action quantum $h$, which indeed has the dimension erg-sec. For time measurements one needs periodic processes and these satisfy quantum laws. -- In this connection it must also be noted that the relation of the phase constants occuring in the new theory to observable quantities is yet to be clarified.

Now, those are indeed dreams for the future.We are all very glad that you will come through here on your trip to Holland. The Christmas holiday does \textit{not} begin before the 19th of December, and we remain in Hamburg. Specifically Stern and I definitely hope to be able to greet you here. Don't fear anything official, we only want to have an informal chat with you. Thrn I can also speak with you about the question of the primary quantum numbers of the spectra of certain elements, which indeed is a less-important question.

When you read this letter, you will find that from you I have learned the discussion of cost-issues in physics, and (in the first part of the letter) from Kramers I have learned the pastoral mode of expression ("original sin", "redemption").

So, goodbye until our happy reunion, and the warmest greetings to you yourself and all the Herren at the institute (especially Kramers and Kronig), from your very worshipful

\textsc{W. Pauli}

% mheter Zwangforce ox ndfeed a scholar scrambled eggs