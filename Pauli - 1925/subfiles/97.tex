\letter{97}
\rcpt{Kramers}
\date{July 27, 1925}
\location{Hamburg}

Dear Kramers!

It was very nice that you have sent me a postcard, but it would have been much nicer still if you would actually come. I believe so strongly in the validity of the conservation laws in every case that, even when I get no news of you, I am able to conclude your continued exiatence in some form; after all, Herr Stern, who is now dealing with such questions, has already discussed the possibility that your conatituent elementary particles had all been transformed into radiation, and it would be nice if I could learn something authentic about this from you yourself.

With this letter I include the second correction to the note for the communication of the Danske Videnskabernes Selskab, since perhaps Prof. Bohr or you still wish to see it before the paper appears in press. Since it seems to me that the greatest caution is appropriate with respect \?{to all people who accept the existence of "true believers" and naturally count themselves among these}, I would like to explicitly note the following:

1. You are \textit{not} permitted to make any changes to my paper without first asking whether I am in agreement. If such changes seem desirable to you, then you must make the effort to write me; naturally I would be happy to correspond with you about it.

2. If Bohr and you are in agreement with the present version of my paper, I need no further corrections and the paper can appear in print.

3. I would like to specifically draw your attention to the footnote on p. 5, after the citation of your paper with Heisenberg, so that you don't overlook it. This is necessary because in the cited paper the following sentence is found: "The \?{procedures} build on the intepretation of the connection of the wave radiation of atoms with the stationary states which is treated in a new paper by \textit{Bohr, Kramers and Slater}, \textit{and the results, if they should prove true, might provide an interesting support} for this interpretation." (The last emphasis is mine.) Consequently, if I had not appended the footnote in question, it \?{would also be true} that the results of \textit{my} paper, if they should prove true, "might provide an interesting support for this interpretation." \textit{I must of course work against} this impression!

I hold it to be an enormous bit of luck that the interpretation of Bohr, Kramers and Slater has been so quickly contradicted by Geiger and Bothe's lovely experiment, as well \?{the recent one by} Compton. Though it is of course correct that Bohr himself, even if this experiment hadn't been done, would no longer have held to this interpretation. But many distinguished physicists (such as e.g. Ladenburg, Mie, Born) would have held to it, and this lamentable treatise by Bohr, Kramers and Slater would perhaps have become a long impediment to progress in theoretical physics! It moved in an entirely wrong direction: \textit{it is not the concept of energy that needs to be modified, bht rather the concepts of motion and force.} Just as one cannot define definite "orbits" for light quanta in cases where intereference phenomena are present, one cannot define such orbits for the electrons in atoms\footnote{In a report by Kamerlingh-Onnes on \?{superconductivity}, I have again detected with shudders in the form of the figures the footsteps of your ghost!}; and there is just as little justification in doubting the existence of ekectrons as there is justification for doubting the existence of light quanta because of the interference phenomena. It can now probably be taken as proof for any unbiased physiciat that light quanta are have just as much (and just as little) physical reality as electrons. But the classical kinematical concepts may not be applied to either.

The "community of true believers" would have neither much honor nor much success in an attempt to counteract \?{the tendency towards an analysis of the concepts of motion and force} of the present development ofthe quantum theory, because it seems to me that there is great hope \?{to even make further \textit{positive} progress in this manner}. Specifically, I have greeted Heisenberg's bold approach (of which you have probably heard in G\"ottingen) with jubilation. Certainly we are far from being able to say anything conclusive and we stand only at the very first beginnings. But what has pleased me so very much about Heisenberg's reflections is the \textit{method} of his work and the \textit{striving} from which he took on these reflections. In general, I believe that I have now come very close to Heisenberg with regard to my scientific views and that we have coinciding opinions in almoat everything, so far as this is at all poasible with two independent thinking men. With joy I have also perceived that with Bohr in Copenhagen Heisenberg has learnt some philosophical ideas and has markedly moved away from pure formalism. Hence with my whole heart I wish him success in his efforts! So now I feel less lonely than a half year ago, when I found myself rather alone (mentally and spatially) between the Scylla of the numerological Munich school and the Charybdis of the reactionary Copenhagen Putsch you were propagating with excessive zeal! Now I only hope that you don't delay the recuperation of the Copenhagen physics, which because of Bohr's strong sense of reality can't be far off, much longer.

Heisenberg wants to eventually make his return trip from England through Hamburg, and I am extremely glad for that, so that I can see him and be able to discuss with him. Hopefully \?{he is not as annihilated as you}. Yes, you think to yourself, that seems to be an unfortunate effect that befalls all physicist who would come to Hamburg. Since not only you, but also Hansen has written me a postcard, he was coming to Hamburg and would like to meet me; but then on the day in queation, I saw no trace of him. And even Ehrenfest appears to be similarly affected. \?{Perhaps Heisenberg will nonetheless succeed in becoming this spooky Herr}...

Now, many greetings to you yourself, prof. Bohr, who is hopefully doing well, to the annihilated Hansen, and the other \?{notables}. In closing, best regards to your wife, to whom I say the following: any time -- be it out loud, be it in your thoughts -- you hurl all the curses of Deuteronomy at those who by far prefer \textit{any} type of true belief over free though, then they should sing you sweet tunes of Spring, Beauty and Life!

Once again, all the best from your faithful

W. Pauli


% h nth too black too strong too black too strong