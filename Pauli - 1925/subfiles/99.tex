\letter{99}
\from{Heisenberg}
\date{September 24, 1925}
\location{Copenhagen}

Dear Pauli!

Many thanks for your letter; your suspicion about the quantum conditions is of course correct; I've already been thinking over a simple proof for the frequency condition for a few days which is so brief that I'll write it here on the card. Naturally it suffices to proof that
\nequ{
\frac{h}{2\pi i}\left(\fh\fp_\varrho - \fp_\varrho \fh\right) &= -\pddX{\fh}{\fq_\varrho}\\
\text{ and } 
\frac{h}{2\pi i}\left(\fh\fq_\varrho - \fq_\varrho \fh\right) &= \pddX{\fh}{\fp_\varrho}.
}{1}

But this formula applies not only for $\fh$, but also for every function $\fF$ which can be expanded in powers of $\fp$ and $\fq$ (even mixed terms are allowed), as one cannsee in the following way:

We assume that the formula applies for a function $\fF(\fp, \fq)$; then, we claim, it also applies for
\nequ{
\fq_\varrho \times \fF; \fp_\varrho \times f; \fq_s \times f 
\text{ and } f\times\fq_\varrho, f\times \fp_\varrho, \text{ etc}.
}{2}
Since
\uequ{
\frac{2\pi i}{h}\left\{(\fq_\varrho f)\fp_\varrho - \fp_\varrho(\fq_\varrho f)\right\} &=
\frac{2\pi i}{h}\left\{
\fq_\varrho(f\fq_\varrho - \fp_\varrho f) - (\fp_\varrho \fq_\varrho - \fq_\varrho fp_\varrho)f
\right\}\\
 &= -\fq_\varrho \pddX{f}{\fq_\varrho} - f = \pddX{}{\fq_\varrho}(\fq_\varrho f)
}
Likewise with the other parts. If $\fF$ resp. $\fh$ can now be expanded in powers of $\fp,\fq$, then obviously the above relation (1) applies for every individual term, that can be shown from (2) by induction. q.e.d.

The perturbation theory is regarded in exactly the same manner as with a single degree of freesom. However, what degeneracy means cannot quite yet be said without a decent integration method for one degree of freedom. I'm working at that now, but still without much success. Many greetings to the whoke Hamburg institute!

W. Heisenberg
