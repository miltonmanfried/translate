\letter{115}
\from{Heisenberg}
\date{January 7, 1926}
\location{Goettingen}

Dear Pauli!

Many thanks for your last letter; I've used the train-ride here from Munich to think over the famous \WTF{mean values}{Mittelwerte}; I've not been able to arrive at any concise results, but perhaps you could start something with the following: we consider the 2-dimensional hydrogen and attempt to introduce polar coordinates. Then
\uequ{
H=\inv{2m}\left(p_r^2 + \inv{r^2}\left(p_\psi^2 - \inv{4}\left(\frac{h}{2\pi}\right)^2\right)\right)
- \frac{e^2 Z}{r}.
}

You have already shown that $p_r \times r - r\times p_r = \epsilon$ ($\epsilon=\left(\frac{h}{2\pi}\right)$) (when $p_x \times x - x\times p_x = \epsilon$ etc applies).

Likewise one can now introduce the variable $\psi = \atan{\frac{y}{x}}$. (That this variable is not expandable in a Fourier series, \?{is irrelevant according to} the recent paper by Bohr and Wiener.) Then we have $p_\psi \times \psi - \psi p_\psi = \epsilon$ and $\psi$ commutes with $r$ and $p_r$. This is proved in the following manner: ($p_\psi = x p_y - y p_x$):
\uequ{
p_\psi \times \psi - \psi p_\psi = x p_y \atan{\frac{y}{x}} 
- \atan{\frac{y}{x}}x p_y - y p_x \atan{\frac{y}{x}} + \atan{\frac{y}{x}} y p_x\\
\epsilon \left(x\frac{d}{dy}\atan{\frac{y}{x}} - y\frac{d}{dx}\atan{\frac{y}{x}}\right) = \epsilon.
}

Further, one can easily show that
\uequ{
p_\psi = mr^2 \dot{\psi}.
}

Now classically the mean value of $\dot{\psi}$ simply denotes the frequency and is therefore \textit{independent} of $k$ in a Kepler orbit. Hence I would like to believe, but unfortunally cannot prove, that $\overline{\dot{\psi}}$ is also independent of $k$ in quantum mechanics. If tgis is true, then it follows immediately that $\frac{p_\psi}{me2}$ is independent of $k$; thus $\overline{\inv{r^2}}$ behaves approximately as $\frac{\text{const}}{p_\psi}$ as regards its dependence on $k$; this $p_\psi$ is however $\inv{2}$ in the $s$-term, $\frac{3}{2}$ in the $p$-term, etc. From this I would follow you further, that $\overline{\inv{r^3}} \approx \frac{\text{const}}{p_\psi\left(p_\psi^2 - \inv{4}\left(\frac{h}{2\pi}\right)^2\right)}$, and so $\infty$ for the $s$-term. Then the Goudsmit model is obviously hardly applicable. But everything depends on whether $\overline{\dot{\psi}}$ is really independent of $k$. One could even imagine the following: in the classical theory, $\overline{\dot{\psi}}$ is indeed generally independent of $k$, while for the \WTF{pendulum orbit}{Pendelbahn} it is null. So \textit{could} (??) $\overline{\dot{\psi}}$ also vanish quantum-mechanically for the $s$-orbit, but for the other orbits have the aforementioned values. But I see no way, up to the present, to decide these open questions, only the Goudsmit model also seems quite difficult to apply; however, I don't want to believe that it is totally false either.

My \WTF{???}{Preiskegelschieben} in Stade is happening in only one week, so I will probably come see you around the 20th. Farewell until then, greetings to Stern, Lenz, Minkowski and Wentzel! And further good luck!

W. Heisenberg
