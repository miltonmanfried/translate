\letter{116}
\from{Heisenberg}
\date{January 14, 1926}
\location{Goettingen}

Dear Pauli!

With this I am sending you the Born-Wiener paper and a similar paper by Lanczos for discussionnon Wednesday. So my lecture the evening of the 19th, Wednesday morning I come to Hamburg; in the evening I travel back to Goettingen. I hope to be able to do many hours of physics with you. 

Incidentally, I now believe that one can bring the proof of "$\overline{\dot{\psi}}$ independent of $k$" free of objections via the Born theory. The proof is something like the following (entirely analogous tonthe classical theory): the variable $\dot{\psi}$ is divided into one term \textit{linear} in the time, and one periodic (c.f. e.g. Born-Wiener eq (41)). If now the factor of $t$, thus $\overline{\dot{\psi}}$, \textit{only} depends on $W$, (so following Born $\dot{\psi} = t\psi_1(D)+\dots$), then $x=r\cos\psi, y=r\sin\psi$ can also be represented as matrices, in which the frequencies occurring are only functions of $W$. If however the factor of $t$ in $\psi$ depends \textit{not only} on $W$, but also on $k$, then the frequencies in $x,y$ must also depend on $k$. This is not the case, so $\overline{\dot{\psi}}$ only depends on $W$, q.e.d. One can incidentally treat the rotator by Born's method, and there the relations that hold between $\psi,x,y$ can be \WTF{reasonably ignored}{einigermaßen übersehen}. However, I've been unsuccessful in a direct calculation of $\cos\psi$ with a given $\psi$ and vice-versa; (in the \textit{oscillator} this calculation is trivial).

Yet another question is, whether the same mean value $\overline{\inv{r^2}}$ arises in the three-dimensional case as in the two-dimensional case. I've not yet gone through the calculation. Incidentally it doesn't seem totally certain that the Goudsmkt model gives false results, since the smallness of the \WTF{screening doublet}{Abschirmungsdubletts} is only measures in the Roentgen spectrum, and that is a different problem (a \textit{missing} electron).

Thus, until I see you on Wednesday! Many greEtings to everyone!

W. Heisenberg
