\letter{143}
\rcpt{Heisenberg}
\date{1926-10-19}
\location{Hamburg}
Dear Heisenberg!

Many thanks for your letter. First to the question of gas degeneracy. I feel today essentially \WTF{lukewarm}{milder} about the Fermi-Dirac statistics, and there occured to me today several arguments to discuss. There is indeed already today a distinction between crystal lattices and radiation, namely concerning the zero-point energy $\frac{h\nu}{2}$. Now it is a priori reasonable to assume that even with the ideal gas there is a zero-point energy\footnote{Have discussed extensively with Stern, who is a greater admirer of the zero-point energy, and have been partially influenced by him.}. Such an energy can however only be very articially grafted onto the Einstein-Bose theory (c.f. Schrödinger's propositions concerning this in the Berliner Berichte and Physikalische Zeitschrift), and that speaks from the outset against the Einstein-Bose theory and for the Fermi-Dirac. In order to make the thing anschaulich, I have carried out the fluctuation considerations, which are in Einstein's paper, Berlin Academy 1925, p. 8 §8, from the standpoint of Fermi-Dirac theory. In concerns the following question: one imagines a volume of gas with an opening, which is only permeable for molecules with energy between $E_s$ and $E_s + dE_s$, which connects to a \textit{very great} volume of the same gas. Collisions between molecules are not taken into account. One asks for the squared fluctuation $\fluc$ of the number $N_s$ of the \WTF{highlighted}{hervorgehobenen} molecules in the first-named volume. This can be calculated purely thermodynamically. If $Z_s$ is the number of cells in phase space (resp. number of eigenfunctions) which belong to the considered energy interval (denoted by Dirac with "$A_s$"), then one obtains with Einstein-Bose analogously to light radiation
\uequ{
\fluc = N_s + \frac{N_s^2}{Z_s},
}
but with Fermi-Dirac
\uequ{
\fluc = N_s - \frac{N_s^2}{Z_s}.
}
($N_s \leq Z_s$ is automatically satisfied.)
The term $N_s$ corresponds to the classical ideal gas (independent particles), the term $\frac{N_s^2}{Z_s}$ the interference fluctuations. At first sight, the \WTF{striking}{frappante} and incomprehensible peculiarity with Fermi-Dirac is the minus sign. If one however believes in the zero-point energy, it seems reasonable that $\fluc$ must be \textit{smaller} than classically, since then at absolute zero \textit{despite not vanishing}, we must have $N_s\delta_s^2=0$. The Einstein value of $\fluc$ can now be derived wave-theoretically under the assumption that the phases of the partial waves are disordered. (Planck's "natural radiation".) The Fermi-Dirac minus-sign thus means that the \textit{phase relations} between the $\Psi$-waves must exist. Herr Dirac should consider whether one can in this manner through appropriate postulates about the $\Psi$-field derive the above value for $\fluc$ in the \textit{three-dimensional space}. Because a reasonabke theoretical basis for the special choice among the $N!$ quantum mechanical solutions still remains an urgent need. In summary I would like to say, \textit{that I have been able to make the distinction between the thermodynamical-statistical behavior of light and matter waves \WTF{palatable}{genießbar machen} through considerations about zero-point energy}.

I just noticed that the application of the Dirac theory to gasses of atoms with spin has the consequence that a monoatomic steam with normal paramagnetic behavior at higher temperatures, must gradually cease to be paramagnetic as soon as the degeneracy from decreasing temperature sets in. The magnetic behavior of the solid and liquid alkaline metals (c.f. Soné, Philisophical Magazine) is perhaps connected with this.

Finally, as to the question about the statistics of molecules with spin, the answer must be the following. First we recall that when mixing gasses the entropies and partial pressures of the insividual constituent gasses must remain additive, because between different molecule types there is no statistical dependence. (Here the well-known paradox, already highlighted by Einstein, that the equation of state of mixtures, even in the limit of vanishing mass-difference between the two molecule-types, remain different from the equation of state of \WTF{pure}{einheitlichen} gasses\footnote{\textit{Ehrenfest} would be delighted to write a treatise woth a thousand exclamation marks in which the question is discussed, whether in a gas with two isotopes of the same mass, one should calculate with statistically dependent or independent molecules}). If one now has atoms with the spin quantum number $j$, which thus has $2j+1$ \WTF{components}{Stellungen} in the field, then it is easily seen that for \textit{any} theory of gas-degeneracy \textit{the \WTF{equation of state}{Zustandgleichung} of the gas (and the entropy) must be the same as a mixture which consists of $P$ different spin-free gas-types, each with the same number of atoms}. If then $p=f(n,T)$,$S=VF(n,T)$, ($n$= number of molecules per unit volume) are the pressure and entropy of the gas of atoms without spin, then
\uequ{
p=Pf\left(\frac{n}{P},T\right);\quad S=VPF\left(\frac{n}{P},T\right)
}
is the equation of state and entropy of a gas with atoms of \WTF{quantum weight}{Quantengewicht} $P=2j+1$. (For the above squared fluctuation $\fluc$ one obtains
\uequ{
\fluc = N_s - \inv{P}\frac{N_s^2}{Z_s}.
}
Thus special difficulties don't occur there.

Now I want to touch on the other chapter: the question of \textit{collisions}. I must however remark that the following is an \WTF{undigested dumpling}{unverdauter Knödel}. If you had not specifically encouraged me to write you about it, I would have hardly dared at this present embryonic stage of my reflections. So listen: we first consider the one-dimensional case. A point-mass runs over a \WTF{track}{barrier}, characterized by a potential energy function $V(x)$, which from a given point $x_0$ shall decrease towards zero sufficiently radidly on both sides.\WTF{Graph of normal-like distribution about $x_0$}{}. Let the maximum of $V$ be \textit{finite}. Further, let the initial energy $E$ of the point-mass be large compared to $V$, so that the unperturbed straight-line uniform motion can be used as the "zero-th" approximation, and then a successive perturbation theory can be carried out. Naturally, according to classical mechanics, when $E>|V_\text{max}|$, the point-mass always gets over the barrier. But according to Born's quantum mechanics\footnote{TODO}, even with such a large $E$, it will nevertheless still happen (already in the first approximation) many times that the point-mass is reflected back, i.e. from the barrier, the direction of its velocity reversed. And indeed according to \textit{Born} the \WTF{authoritative}{maßgebend} probability of the reflection is given by the square of the amplituden of the waves represented (for $x<<x_0$) by
\nequ{
\Y_1 = \inv{2i}\inv{k}\exp{-ikx}\int^{+\infty}_{-\infty}\exp{zik\xi}V(\xi)d\xi
}{A}

Now a rotator is considered, which is perturbed \WTF{at a point}{an einer Stelle} by a force field characterized by a potential function $V(\vartheta)$. This case is periodic and can be handled by means of the usual matrix mechanics. \iffy{343}{One could hope that thereby one could allow the radius of the rotator, and with that also the period of the rotator, to become ever larger, and with this to transfer the case accessible in the usual quantum mechanics to Born's aperiodic case.} (Kramers brought this idea up to me in Utrecht. The result is however not that which he had put forward at the time.)


TODO:IMAGE

The essential thing is now that the system is degenerate insofar as, with a given (quantized) energy $E_n$, the point-mass can rotate both to the left (quantum number $+n$) as well as to the right (quantum number $-n$). A variable $f$, constant in time, will thus have in its matrix representation in general not only diagonal elements $f_{+n,+n}$ and $f_{-n,-n}$, but also elements of the form $f_{+n,-n}$ resp. $f_{-n,+n}$. According to the Schr\"odinger recipe, \iffy{the matrix elements of any functions $F(\vartheta)$ that are real and periodic in $\vartheta$ ($\vartheta = \text{spin angle}$) for the unperturbed (force-free) rotator} are given by
\nequ{
TODO
}{1}
($F_{mn}$ and $F_{nm}$ are complex conjugates; Hermitian matrix). That the numerical value $F_{nm}$ only depends here on the difference $(n-m)$ \WTF{comes from the fact}{kommt daher} that the system is \textit{exactly} force-free. (The same applies to the three-dimensional case discussed later.)

The time-average of the perturbation energy $V(\vartheta)$ in the $\pm n$ state is in particular a Hermitian matrix of the form
\uequ{
V_{+n,+n}, & V_{+n, -n}\\
V_{-n,+n}, & V_{-n, -n},
}
where according to (1),($\widetilde{x}$ = conjugate)
\uequ{
TODO.
}
If one puts ($||$ = absolute value)
\uequ{
TODO,
}
then the "secular equation" for the \WTF{additional energy}{Zusatzenergie}
\uequ{
TODO,
}
\iffy{thus splitting the original energy values into two parts}. It further results that these two states of the \textit{perturbed} rotator correspond to \textit{standing} waves:
\uequ{
\Y = \cos{(n\vartheta - \frac{\delta}{2}))},\text{ resp. }
\Y = \sin{(n\vartheta - \frac{\delta}{2}))}
}
($\delta$ = the above phase variable; determines the position of the nodes relative to the barrier). Of course this is your \WTF{darling}{liebling} resonance phemonenon in \WTF{its purest form}{Reinkultur} (moreover, frees from the mess with the equivalent electrons; here there is only \textit{one} particle). The energy swings back and forth between the leftward- and rightward- turning oscillations. And in each of the "secular" quantized states of the perturbed rotator the particles must likewise run equally-often in the positive as the negative sense. But how is that possible? It is only possible because Born was correct in his assertion (despite all classical-mechanical preconceptions) that the particles will be reflected from the barrier from time to time, however great their kinetic energy.

And now comes the beauty: the integral, which \iffy{344 bottom}{according to Born's formula (A)} determines the probability of the reflection, \textit{is indeed none other than the matrix element $V_{+n,-n}$} (see previous page below) when one goes over to the limit of an infinitely-large radius for the rotator. This probability is (up to the leading factor $\inv{k}$) proportional to $|V_{+n,-n}|^2$, while $2|V_{+n,-n}|$ determines the energy splitting of the secular perturbation.

Naturally, this relation can be extendes to higher approximations. \textit{The essence is this: one carries out the matix-perturbation-calculation, but not for the secular perturbation, so that a the energy can be transferred to a time-constant non-diagonal matrix of the type
\uequ{
TODO.
}
Then $|E_{+n,-n}|^2$ gives the measure for the probability of a collision.}

This recipe can also be generalized to the three-dimensional case. The point-mass then moves through a central field and the potential energy $V(r)$ decreases sufficiently fast away from the center.
TODO: IMAGE
Now, any such position function that decreases sufficiently-rapidly in $x,y,z$ -- which can be represented classically in unperturbed straight-line motion as a Fouroer integral over the time -- is assigned a continuous matrix. Now one must decide what canonical variables to introduce for the \textit{unperturbed} (straight-line) motion. The $p$'s (analogous to the \WTF{action variables}{Wirkungsvariablen} $J$) must be constant in time and the energy must be a function of the $p$'s alone; the $q$'s (analogous to the angle variables $W$) must be either linear functions of the time or constant. Since the system is degenerate, they are \textit{not} uniquely determined.

\begin{enumerate}
    \item Example: The $p$'s are the usual cartesian momentum coordinates, the $q$'s are the usual cartesian coordinates.
    \item Example: The $p$'s are $E$, the spin $P$, and $Q$, the spin parallel to $Z$. The $q$'s are $t$, the perihelion $\beta$, and the node-line $\gamma$.
\end{enumerate}

\iffy{345 bottom}{These are indeed also definable for straight-line motion.}

Now comes the \WTF{murky}{dunkle} point. The $p$'s must be taken as \textit{controlled}, the $q$'s  as \textit{uncontrolled}. That means that one can only ever calculate the probabilities of certain changes in the $p$'s for given initial values of those $p$'s, \textit{averaged over all possible values of $q$}.

Cartesian coordinates correspond to Born's incident plane wave, namely to the parallel \WTF{train}{Schwarm} of electrons hitting the barrier.

TODO: IMAGE

The above \WTF{coordinates}{Koordinatenzahl} in example 2 correspond to the following electron \WTF{cloud}{Schwarm}:
TODO: image

The arrows are tangent to a certain circle (resp. sphere). We remain for now with the cartesian coordinates. Any position function $F(x,y,z)$, which falls off sufficiently strongly, is then (according to Schr\"odinger's procedure for calculating the matrices) assigned a continuous matrix
\uequ{
TODO.
}
(\WTF{Multiplication rules ordered according to Born's final formulae}{Multiplikationsregel in Ordnung gemäß Borns Endformeln}.) Then again we have
\uequ{
TODO,
}
\textit{where the \WTF{constraint of constant energy}{Nebenbedingung der Energiekonstanz}} $\sumX{x}{
p_x^2} = \sumX{x}{{p'_x}^2}$ applies, for the probably of measuring a deflection from $p_x,...$ to $p'_x,...$ (up to harmless factors containing $m,h,E_0$).

As for the higher approximations, there will probably be no essential difficulties there either. One can also start from the matrix associated with the acceleration $\frac{dp}{dt}$, and then remark that \textit{classically} the Fourier coefficient, which from this belongs to the zero-frequency,
determines the deflection. Quantum-mechanically, however, the square of the Fourier coefficient of $\frac{dp}{dt}$, which is associated with zero energy \textit{change}, determines the probability of the deflection.

For inelastic collisions it is the same in principle. Your old \WTF{ideas}{\"Uberlegungen} on the braking of $\alpha$-particles also fall into this category.

As far as the mathematics. The physics of the thing is still to a large extenr unclear to me. The first question is, why only the $p$'s, and in any case not \textit{both} the $p$'s \textit{as well as} the $q$'s, can both be written down with arbitrary precision. This is the old question that arises when the velocity-direction \textit{and} the distance of the \WTF{asymptotic path}{Bahnasymptote} from the nucleus (at least with a certain precision) are specified. About this I know nothing that I haven't already known for a long time. It is always the same thing: because of diffraction there are no arbitrarily-fine rays in the \WTF{spectrum}{Wellenoptik} of the $\Y$-field, and one may not simultaneously assign ordinary "$c$-numbers" to the "$p$-numbers" and the "$q$-numbers". One can view the world with the $p$-eye, and one can view it with the $q$-eye, but if one opens both eyes at the same time, then one \WTF{goes mad}{wird irre}.

The second question is, how to arrive at the aforementioned matrix elements which determine the collision probability. \iffy{347}{I believe that one must steer in the following direction. The historical development has brought with it, that the link between the matrix elements and the data accessible to observation takes a detour over the emitted radiation}. But I am now convinced from the \WTF{bottom of my heart}{Inbrunst meines Herzens} that \textit{the matrix elements must be linked to the (in principle) observable kinematic (perhaps statistical) data of the relevant particles in stationary states}. Entirely apart from whether and what is (electromagnetically) radiated (speed of light set to $\infty$). I also have no doubt that the key to the treatment of aperiodic motion is hidden within this. Now it is so: all \textit{diagonal} elements of the matrices (at least functions of $p$ alone or $q$ alone) can already now be interpreted kinematically. Since one can now ask the probabiloty that in a given stationary state of a system, the coordinates $q_k$ of its particles ($k=1,...,f$) lie between $q_k$ and $q_k + dq_k$. The answer is
\uequ{
|\Y(q_1,...,q_f)|^2 dq_1 ... dq_f,
}
if $\Y$ is the Schr\"odinger eigenfunction. (from a \textit{corpuscular} standpoint it therefore already makes sense that they lie in the many-dimensional space.) We must regard this probability as observable in principle, exactly as the light intensity as a function of position in standing light waves. It is then clear that the diagonal elements of the matrix of every $q$-function must be
\uequ{
F_{nn} = \int F(q_k)|\Y_n(q_k)|^2 dq_1 ... dq_f,
}
since they represent the physical "mean value of $F$ in the $\Nth{n}$ state". Here one can make a mathematical \WTF{trick}{Witz}: there is also a corresponding probability-density in the $p$-\textit{space}: One takes (formulates in one dimension for simplicity):
\uequ{
p_{ik} &= \int p \y_i(p)\widetilde{\y}_k(p) dp;\\
\frac{2\pi i}{h} q_{ik} &= -\int \y_i ...TODO...
}
($\widetilde{}$ denotes the complex conjugate variable; $\widetilde{\y}_k$ and $\y_k$ are in general only distinguished by a constant factor. Orthogonality means
\uequ{
TODO.
}
Multiplication rules and the relation $pq-qp=\frac{h}{2\pi i}\times 1$ are fulfilled.)

You see that have reveresed the rules for forming the matrix elements $p_{ik}$ and $q_{ik}$ from the eigenfunctions with respect to the usual prescription. From the matrix relation for the energy law $\frac{p^2}{2m}+V(q)=E$ one gets
\uequ{
TODO;
}
$V$ is considered as an operator, as a power series in $\frac{\partial}{\partial q}$. For the harmonic oscillator, where the Hamilton function is symmetric in $p$ and $q$, $\y$ is also a Hermitian polynomial. One can also expand $\y$ in the perturbation theory. Even with the H-atom $\y$ must be a simple function, but I still haven't yet worked it out. In any case there is also a probability that, in the $\Nth{n}$ quantum state, $p_k$ lies between $p_k$ and $p_k+dp_k$, and which is given by
\uequ{
|\y_n(p_1,...,p_f)|^2 dp_1...dp_f,
}
and so
\uequ{
TODO.
}
Since accordingly the diagonal elements of the matrixes physically follow from the kinematical \WTF{content}{Aussagen} contained in the functions $\y_k(p_1,...,p_f)$ and $\y_k(q_1,...,q_f)$, I don't believe that your fluctuation considerations can say anything essentially new beyond this and the interpretation of the unperiodic motions. Because there it is always about a time-averaged value. Dealing with collision phenomena it is likewise about time-constant elements, but non-diagonal elements of matrices in degenerate systems (whose physical interpretation is however not clear to me).

My main question is however, what the other matrix elements mean purely from point kinematics, entirely independently of electromagnetic radiation. I've tried one approach, in which it is asked: I know that at the time $t$ the particles have the position coordinate $q_0$. How large then is the \textit{probability} that at the time $t+dt$ it has the coordinate $q$? I'm however no longer certain that this is a reasonable question. \iffy{348}{I'm thinking of at least "kinematical" statistical data which determines the time-evolution of the behavior of the particle}. This data could however be of such a type that one can \textit{not} speak of particles on a certain "orbit". Here again one incidentally runs up against the fact that one cannot ask after $p$ and $q$ simultaneously.

So now you will regret having induced me to write you so extensively. I have probably given a bit too much detail. Now I already have twelve pages, and it is already 2:40 at night, a point in time where I can go to sleep with honor and a good conscience. Have you already heard that Des Courdes has suddenly died? This damned Leipzig is a perpetual source of harassment!

\iffy{I'm really looking forward to your reply.} Now for once \textit{you} can criticize and probe!

Many hearty greetings to you, Bohr, and all Copenhageners (and thlugh we haven't met, to Herrn Dirac).

Your W. Pauli