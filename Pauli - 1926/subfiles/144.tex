\letter{144}
\from{Heisenberg}
\date{October 28, 1926}
\location{Copenhagen}

Dear Pauli!

Many thanks for your long letter. The reason I am answering so late is because your letter is constantly making the rounds here and Bohr, Dirac and Hund are scuffling over it. With regards to the Dirac statistics, we are now quite in agremeement. I've understood your critique very well. But if the atoms follow your \WTF{ban}{Verbot}, \?{then the gas must as well}.

I am \textit{very} enthusiastic about your consideration on collision processes, since the physical meaning of the Born formalism is now much better understood than before. Your rotator example is indeed rather more general: everywhere in classical mechanics where \WTF{a type of motion}{Bewegungstypus} goes over discontinuously to another, QM supplies the continuous transition, which, as far as one wants to give it an \textit{anschaulich} interprtation, denotes a jump-probability. Also, the transition $H + H^+ \to H^+_2 \to He^+$, recently investigated by Hund, and so \WTF{nicely}{gern} considered by you, is indeed mathematically very similar to your rotator (Hund has found your calculactions independently of your letter, I told him that he should write to you about them). Incidentally I found Hund's reflections very beautiful. That Born's collision problem is now also the same is very pleasing. Your rotator also has many similarities with the example of two atoms in resonance, which I considered by way of the fluctuations: \?{for the instantaneous transition probability of one of the two atoms, one would perhaps like to consider $\frac{dE_1}{dt}$}. Its time-average value is however naturally zero and perhaps one can nonetheless later render plausible that in the non-periodic case to determine the jump-probability by averaging over all phases $\left(\overline{\frac{dE_1}{dt}}\right)^2$; (that would just be your $E^2_{n,-n}$).

Have you given any more thought to the meaning of your $E^2_{n,-n}$? The most interesting of your remarks however is of course the so-called "dark point". I would like to believe that your $p$-waves have just as much physical reality as the $q$-waves; only naturally not so great a practical importance. But I am very sympathetic to the in-principle equivalence between $p$ and $q$. The equation $pq-qp=hi$ thus corresponds in the wave theory to the fact that it makes no sense to speak of a monochromatic wave at a specific point in time (or a very short time-interval). If however one makes the line not too sharp, the time interval not too short, then it does havem\WTF{quite sensible eaning.}{hat sehr wohl einen Sinn} Analogously, it is meaningless to speak of the location of a corpuscule of a definite velocity. If however one takes a position and velocity that is not so exact, tthe that has a quite sensible eaning.
 It is also quite well understood that it is macroscopically meaningful to speak of a definite position and velocity of a body, to a very high approximation. (Naturally none of this is in any way new to you). In all this I have a hope for a later solution of about the following type (\?{but one shouldn't say so out loud}): the space and time are actually only statistical concepts, like temperature, pressure, etc, in a gas. I mean that spatial and temporal concepts in \textit{a single} corpuscule are meaningless and that they acquire more and more meaning the more particles are present. I often seek to go further in this direction, but have had no success so far. Your calculations have again given me much hope, since they show that Born's rather dogmatic standpoint of the probability waves is only a very possible schemata.
 
\WTF{Anyway,}{Übrigens} a practical question: couldn't your $p$-wave field be used for a \WTF{proper}{anständigen} wave formulation of the spinning electron?? In the usual method there is a difficulty in expressions which have $\sqrt{p^2_k - p^2\y}$, which because of $\sqrt{\frac{\partial^2}{\partial \y_k^2} - \frac{\partial^2}{\partial \y^2}}$ is not at all attractive.

I have also often thought about the meaning of the matrix elements. \iffy{Ich bin hier nicht ganz Ihrer Meinung}{I am here \textit{not} entirely of your opinion}. I believe that one cannot associate them with kinematic elements in \textit{one} state. I believe that one such element must always be connected with two states. But perhaps this is after all not very different from your program. Because in any \WTF{examination}{Prüfung} of the question, where the particle is located, the \WTF{allowed transitions}{Übergangmöglichkeiten} \WTF{arise}{hineinkommen} practically automatically. I also believe that there are definitions of the matrix elements independent of the electromagnetic radiation. \?{Incidentally one could probably be obtained at least implicitly by some consideration of collisions; about the collision of an $\alpha$-particle and an atom}, which we discussed in Braunschweig. But one would in any case like to have a \textit{simple} definition.

I myself have in recent times thought many times about fluctuations -- it is probably all in good order, but leads, as you yourself have said, not much further. Perhaps I shall nonetheless write a short paper on it, since it is very healthy for the Herren of the continuum theory to have something to read.

Dirac has, in the explanation of the Schr\"odinger electrical density, employed a very funny consideration. Question: what is the QM matrix of the electrical density". Definition of density: I. It is zero everywhere, where the electron is not. In equations:
\uequ{
\varrho(x_0,y_0,z_0,t)(x_0-x(t))&=0\\
\varrho(...          )(y_0-y(t))&=0\\
..........
}
further, the total charge is $e$:
\uequ{
\int\varrho(x_0,y_0,z_0,t)dx_0 dy_0 dz_0 = e.
}
The solution is (as can quite easily prove):
\uequ{
\varrho_{nm}(x_0,y_0,z_0,t_0) = e\Y_n \Y_m^*(x_0,y_0,z_0,t_0),
}
where $\Y_n$ and $\Y_m$ are Schr\"odinger's normalized functions. I am particularly pelased with this formulation; specifically this matrix formulation again gives the correct squared fluctuation (\iffy{so viel ich bis jetzt sehe}{as far as I can see}\ for the fluctuations of electric charge in a small volume element (thus the large fluctuations which correspond to the motion of point charges).

So, I don't know any more. Many thanks again for your letter. Many hearty greetings to you and the whole Hamburg institute from Bohr, Dirac, Hund and me!

W. Heisenberg