\letter{145}
\from{Heisenberg}
\date{November 4, 1926}
\location{Copenhagen}

Dear Pauli!

I send you the enclosed note on fluctuation phenomena not without without a long series of apologies: \WTF{that I want to publish on that topic}{daß ich überhaupt so ein Zeug publizieren will}, that for you and all reasonable physicists there is nothing new in it, \WTF{that even I myself don't know how to begin with it}{daß ich auch selbst nicht viel damit anzufangen weiß}, etc. I'm writing for pedagogical reasons against the Herren of the continuum theory. But after this unwelcome introduction I would like to once again write you, that I am more and more enthusiastic about the content of your last letter, the more I think about it. One should then say in general: any schema which satisfies $pq-qp=\frac{h}{2\pi i}$ is "correct" and physically reasonable. Thus one has a completely free choice in how to fulfill this equation: matrices, operators, or anything else. Incidentally your wavefunction $\chi$ in $p$-space seems to me to be the Laplace transform of the Schr\"odinger function $\y$
in the $q$-space, $\y(q)=\oint\exp{+pq}\chi(p) dp$, or conversely. But your functions in other spaces, e.g. $J$ and $w$, are naturally rather different. The problem: canonical transformations in the wave representation is indeed probably also solved with it.

Dirac has done some reflecting on the relativity theory, which in effect leads back to Klein's 5-dimensional theory: if one wants to treat space and time symmetrically, then the Hamiltonian equation
\uequ{
\left\{
p_x^2 + p_y^2 + p_z^2 - \frac{p_t^2}{c^2} - \frac{m^2 c^2}{h^2}
\right\}\Y = 0
}
is uncongenial, since the right-hand side is 0, which in QM commutes with all variables. If one now simply puts instead of 0 the new number $a$, which practically \WTF{amounts to}{läuft...heraus} the introduction of a new variable, \?{which then in turn again occurs unsymmetrically, as the time did earlier}. At the end one sets $a=0$. In Klein's \WTF{language}{Ausdrucksweise}: one introduces a $\Nth{5}$ dimension, and at the end one throws it out via averaging.

I myself have given some thought to the theory of ferromagnetism, \?{conductivity, and similarly \WTF{???}{S...ereien}}. The idea is: in order to use the Langevin theory of ferromagnetism, one must assume a large coupling force between the spinning electrons (\WTF{indeed it depends on \textit{only this}}{es drehen ja nur diese sich}). This force should be indirectly supplied, as in Helium, by resonance. I believe that one can prove in general: \WTF{the parallel position}{Parallelstellung} of the spin vectors always gives the \textit{smallest} energy. The \WTF{energy differences}{Energieunterschiede} which come into consideration are of an \textit{electrical} order of magnitude, decrease, but with increasing intervals, \textit{very rapidly}. I have the feeling (\?{without the material to even remotely find out}) that this would suffice, in principle, to give meaning to ferromagnetism. To decide the question, why most substances are not ferromagnetic, but some are, \?{if one must \WTF{stop calculating quantatively}{halt quantitativ rechnen}, one can perhaps make plausible that $Fe$, $Kr$, $Ni$ are the most favorable}. For conductivity it is similar, the \WTF{resonance migration}{Resonanzwanderschaft} of the electrons comes into question \`a la Hund.

Write me again and, when you've had enough of fluctuations, please send the manuscript to Born!

Many greetings to the whole institute,
W. Heisenberg
