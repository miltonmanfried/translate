\letter{150}
\rcpt{Schr\"odinger}
\date{December 12, 1926}
\location{Hamburg}

Dear Schr\"odinger!

Many thanks for sending me your treatises in book form, \?{I can make good use of them}. Herr Gordon, who is now in Hamburg (he has received a theoretical assistant position with Koch), told me that you are going to America on the $\Nth{18}$, and thus I wanted to write to you once more before that, not only to thank you for your gift and to wish you a happy Christnas and good travels, but also, since lately I have given much thought to the relativistic equations. I now believe with certainty that you are correct in your standpoint that these equations are meaningful and that the operator calculus must be generalized. For I have arrived at different properties of the relativistic equations and expressions for the charge density and current density, in which I have very high confidence:

1. At first sight it seems as if in the differential equation for the de Broglie field scalar $\Y$ would enter into not only the electromagnetic field strengths $F_{ik}$ (I choose the four-dimensional notation here and in the following), but also the \textit{absolute value} of the potential $\Phi_k$. But this is, thank god, only apparent. With the substitution $\Phi'_k = \Phi_k + \pddX{\lambda}{x_k}$ ($\lambda$ a function of $x_i$) which leaves the $F_{ik} = \pddX{\Phi_k}{x_i} - \pddX{\Phi_i}{x_k}$ unchanged, \?{namely multiplied by the field scalar $\Y$ only with $\exp{i\lambda}$ and all observable variables as the current density $S_k$, $\Y\widetilde{\Y}$, etc remain unchanged}.

2. I have asked myself whether, \?{in addition to the conservation of charge also there isn't a field-like formulation of the conservation law of energy and momentum}, and have found that in fact this is the case. If one inserts into the Lorentz force $\sumX{k}F_{ik}S_k$ ($S_k$ = four-current = charge + current) the expression for $S_k$ which you have given, then this can be brought into the form $\sumX{k}\pddX{T_{ik}}{x_k}$. ($T_{ik} = T_{ki}$) $T_{ik}$ for $i,k=1,2,3$ are stresses; $T_{i4}$ = energy current = $c^2\times\text{momentum density}$, $T_{44}$ = energy density; the $T_{ik}$ are differential expressions of first order and second degree in the $\Y$, which moreover contain the electromagnetic potentials in such a form that they remain invariant under the substitution given under 1. The $T_{ik}$, together with the electromagnetic energy tensor, satisfy the conservation laws in the differential form
\uequ{
\div(\text{stress}) &+ \pddt{}(\text{momentum density}) = 0\\
\div(\text{energy current}) &+ \pddt{}(\text{energy sensity}) = 0,
}
where additionally the energy current = $c^2 \times\text{momentum density}$.

For a closed system we also have
\uequ{
\int dV\times\text{charge density} &= \text{const.}\\
\text{also } \int dV\times\text{energy density} &= \text{const.}\\
\text{and } \int dV\times\text{momentum density} &= \text{const.} \text{(resp. 0)}.\\
}

Only by ignoring the relativistic corrections, i.e. if one sets the speed of light to $\infty$, will the charge density become equal to the mass density, and the current density proportional to the momentum density, otherwise one has to make do with two conservation laws.

3. If one develops in the case of a closed system (no magnetic field, electrostatic potential indpendent of time) in a time-Fourier series
\uequ{
\Y = \sumX{n}f_n \exp{2\pi i \nu_n t}, \text{$f_n$ independent of $t$},
}
then \?{the conservation laws give rise to "orthogonality relations"}, since the time-constancy of the volume integral demands the \WTF{vanishing}{Fortfallen} of the coefficients of $\exp{2\pi i(\nu_n - \nu_m)t}$ for $n \neq m$ after the integration is carried out. The "center of gravity laws" also apply for the center of mass and the center of charge:
\uequ{
\int\R\times\text{charge density}\times dV &= \int\text{current density}\times dV\\
\int\R\times\text{mass density}\times dV &= \int\text{momentum density}\times dV
}
($\R$ = spatial radius vector, mass density = $\inv{c^2}\times\text{energy density}$)

Now the main question is of course the correct generalizations of $pq-qp=\frac{h}{2\pi i}\times 1$ and the multiplication rules. That I unfortunately still do not know. I also must study Gordon's work on the Compton effect.

Naturally you will consider the above reflections as water to your continuum mill. But I am still convinced (along with many other physicists) that the quantum phenomena cannot by interpreted with the help of continuum physics alone, and the present state of the Compton effect (inclusive of your note on it, which Gordon gave me to read) only encourages me further. We will see how the thing goes. If we exert all of out best efforts to push it forward, then the correct path will be provided automatically.

Many greetings and all the best for the trip.

From your faithful W. Pauli.

\WTF{Commendations}{Empfehlungen} to your Frau Wife!
