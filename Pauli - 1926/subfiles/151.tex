\letter{151}
\from{Schr\"odinger}
\date{December 15, 1926}
\location{Z\"urich}

Dear friend!

Many hearty thanks for your \WTF{lovely}{lieben} letter from December 12, which unfortunately now in the \WTF{bustle of departure}{Drang der Abreise} I am only able to answer very incompletely.

Now you must please not be angry with me. In and of itself, it pleases me very much that our \WTF{trains of thought}{Gedankenwege} have lately been running in parallel, since it is a \WTF{prop}{Stütze} and acknowledgment for the reasonableness of mine. \?{But it nonetheless feels very \WTF{vulgar}{ordinär} to always have to write to you}: what you have written to me is already in print in an Annalen note. Incidentally, only a few days ago, on 10 December, did I deliver the little manuscript about the energy-momentum law to Vienna in transit to Munich. In this I first derive the $\Y$-wave equation and the Maxwell equations from Gordon's variation principle (extended in the obvious manner), and that then \WTF{leaves in its wake}{zieht...von selbst nach sich} the four further conservation laws in the well-known patterns.

My \WTF{outward}{äußerliche} "priority" (if it is) would incidentally in this case obviously lead back to the fact that I had Gordon's work in my hands earlier than you - in a footnote I clearly said that the genesis of my note led back to Gordon's paper. (No, forgive me, that's not right, that was in the Compton effect note; but I of course \WTF{widely and clearly}{breit und deutlich} establish Gordon's results).

You are not right in your supposition that the conservation law is water to my continuum mill. Almost: to the contrary! The "closed system of equations" \textit{does not indeed apply}! \WTF{In the sense of the equation it would be to seek a system of solutions}{Im Sinne der Gleichung wäre es, ein Lösungssystem ... zu suchen} $\Y,\y_1,\y_2,\y_3,\y_4$, which satisfy \textit{all} equations: the $\Y$-wave equation and -- by means of Gordon's expression for the four-current -- the retarded potential equations. \textit{\?{But that is not the case}}. One has to insert into the $\Y$-wave equation -- e.g. in the case of hydrogen the potential of the \textit{nucleus}. Then from this $\Y$ potentials are calculated by the Gordon formula, which act indeed on other electrons, but not on itself. At least it is thus, if one (as you woukd have it) only ever regards one eigenfunction as "excited". If one takes several, e.g. two, then one has to distinguish between the statistical part of the potentials created by $\Y$ and the \WTF{vibrational part}{Schwingungsteil}. The \textit{latter} and only the latter is likely to be added in the $\Y$-wave equation to the nucleus potential, and yield a "radiation correction". I cannot however \WTF{decide}{übersehen} whether it only provides a small correction when one regards \textit{one} eigenvibration as \WTF{vastly predominant}{weit überwiegend}, and all others as only weakly excited, or perhaps even without this assumption.

The \textit{latter} appears at first as improbable, since then the vibrating charges are of the same magnitude as the nuclear charge. \?{Therefore it also does not seem likely to me. In any case a type of "\WTF{???}{Orthogonalsein}" of the vibrating charges could lead to the eigenfunction."}

But no, it is not so\footnote{Accorsing to the energy law, it cannot be so!}. \?{Perhaps it can be understood in opposition to precisely this page, namely from my standpoint to "understand"} why in reality almost always only \textit{one} eigenfunction is ever excited. The distribution of the excitation to several seems to lead to monstrously-stronger radiation, probably with changing frequency.

It already falla back to \WTF{classical case}{die Klassik}! No no! I see already that the path ia not so straight. I see already that one cannot assume the greatest part of radiated energy is \textit{not} attributable to the \WTF{lines}{Linien}, but rather is smeared over the spectrum \WTF{als Untergrund}.

So forgive these ragged and rather harries thoughts. Overall, like you I have high hopes. \?{We will still only be ably badgered by all and from all sides}; if at the beginning we have differently-nuanced \WTF{basic ideas}{Grundeinstellung}, in the end we still come together. Specifically, since we are all nice men and are merely interested in the \WTF{subject}{Sache}, and not in whether it finally comes out as one himself or as someone else originally assumed. \?{Sollen uns Outsider immerhin wetterwendisch finden, wir wissen, daß solche Wetterwendigkeit der Wissenschaft besser taugt als Eigensinn.}

Be warmly greeted, dear friend, from your loyal Schr\"odinger.