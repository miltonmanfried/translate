\begin{paper}{1}
\newcommand{\spin}{+\frac{1}{2}\frac{h}{2\pi}}
\begin{header}
\title{On the quantum mechanics of the magnetic electron}
\author{W. Pauli}
\location{Hamburg}
\note{Received on May 3rd 1927.}
\makeheader
\end{header}

\begin{abstract}
It is shown how, \?{armed}{gest\"utzt} with the general Dirac-Jordan transformation theory, a formulation of the quantum mechanics of the magnetic electron can be achieved following the Schr\"odinger method of eigenfunctions without the use of two-valued functions by introducing alongside the position coordinates of each electron its \?{spin angular momentum}{Eigenimplusmoment} in a fixed direction as a further independent \?{variable}{Ver\"anderliche}, in order to take into account its rotational degree of freedom. In contrast to classical mechanics however this variable, totally independent of any specific type of external force fields, only assumes the values $+\frac{1}{2}\frac{h}{2\pi}$ and $-\frac{1}{2}\frac{h}{2\pi}$. The insertion of this new variable for one electron simply causes a splitting of the eigenfunction into two functions of position $\psi_\alpha$, $\alpha_\beta$ and in general for $N$ electrons into $2^N$ functions, which are to be considered as the "probability amplitudes" for a certain stationary state of the system to have not only the position coordinates of the electrons in a prescribed infinitesimal interval, but also the components of their \?{spins}{Eigenmomente} in the chosen fixed direction having values $+\spin$ for $\psi_\alpha$, $-\spin$ for $\psi_\beta$. Methods are given to specify, for a system's given Hamiltonian function, just as many simultaneous differential equations for the $\psi$-functions \?{as needed}{als ihre Anzahl betr\"agt} (so 2 resp. $2^N$). These equations are fully equivalent in their consequences with the matrix equations of Heisenberg and Jordan. Further in the case of several electrons the solution of those differential equations that satisfy the "equivalence rule", following Dirac and Heisenberg, characterized in a simple way by their symmetry properties under exchange of values of the variables of two electrons.
\end{abstract}

\section{General facts of electromagnetism in the Schr\"odinger form of quantum mechanics.}
The hypothesis first drawn by Goudsmit and Uhlenbeck for the explanation of the complex structure of the spectrum and their anomalous Zeeman effect, according to which the electron attains a spin angular momentum of magnitude $\spin$ and a magnetic moment of one magneton has, by means of matrix calculations, been integrated into quantum mechanics by Heisenberg and Jordan\footnote{\citepub{ZS. f. Phys.} \citevol{37}, \citepage{263}, \citeyear{1926}.} and made quantitatively precise. While otherwise the matrix methods are fully mathematically equivalent with the method of eigenfunctions in many-dimensional space discovered by Schr\"odinger, when trying to treat the forces and torque that would be required by the spin of the electron experienced in external fields by a corresponding method, one runs into peculiar formal difficulties. In introducing a further degree of freedom corresponding to the orientation of the \?{spin}{Eigenimpulses} of the electron in space, the empirically well-stablished fact of two possible \?{quantized}{quantenm\"a{\ss}ig} positions of this spin in an external field is expressed in the fact that one is immediately led to eigenfunctions in which the relevant rotation angles, e.g. the azimuth of the momentum about an axis fixed in space, are multi-valued, and in fact two-valued. Hence it is often suspected that this representation by means of two-valued eigenfunctions, though formally possible, is not justified by the true physical situation, and the solution of the problem is sought in another direction. So Darwin\footnote{Nature, 119, 282, 1927.} has recently tried \?{to grasp the facts summarized in the assumption of the electron momenta}{die in der Annahme des Elektronenimpulses zusammengefa{\ss}en Tatsachen...zu erfassen} without introducing an electron \?{spin degree of freedom}{Kreiselfreiheitsgraden} corresponding new dimension of the configuration space by considering the amplitudes of the de Broglie waves as directed quantities, which means the Schr\"odinger eigenfunctions are taken to be vectorial. In attempting to think this at first view apparently promising path through consistently to its end, however, one finds difficulties which are again connected to the number 2 of the electron's positions in an external field, and which I do not believe can be overcome. On the other hand, a representation of the quantum mechanical behavior of the magnetic electron according to the eigenfunction method in the case of an atom with several electrons is very desirable, since the selection which is realized in nature alone of the solutions of the quantum mechanical equations fulfilling the "equivalence rules" out of all possible solutions according to the present theory of Heisenberg\footnote{W. Heisenberg, ZS. f. Phys. 38, 411, 1926; 39, 499, 1926; 41, 239, 1927.} and Dirac\footnote{P.A.M. Dirac, Proc. Roy. Soc. 112, 661, 1926.} is carried out in the clearest way with help of the symmetry properties of the eigenfunctions under the exchange of the variable values associated with two electrons.

Here we now want to show that a quantum mechanical representation of the behavior of the magnetic electron according to the eigenfunction method is in fact possible without bringing in multi-valued functions by a suitable usage of the formulation of quantum mechanics put forward by Jordan\footnote{P. Jordan, ZS. f. Phys. 40, 808, 1927; G\"ott. Nachr. 1926, p.161.} and Dirac\footnote{P.A.M. Dirac, Proc. Roy. Soc. (A) 112, 621, 1927; cf also F. London, ZS. f. Phys. 40, 193, 1926.}, which permits making use of general transformations of the Schr\"odinger functions $\psi$. This succeeds by 


\end{paper}
