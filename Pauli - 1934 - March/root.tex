\documentclass{article}
\usepackage[utf8]{inputenc}
\renewcommand*\rmdefault{ppl}
\usepackage[utf8]{inputenc}
\usepackage{amsmath}
\usepackage{graphicx}
\usepackage{enumitem}
\usepackage{amssymb}
\usepackage{marginnote}
\newcommand{\nf}[2]{
\newcommand{#1}[1]{#2}
}
\newcommand{\nff}[2]{
\newcommand{#1}[2]{#2}
}
\newcommand{\rf}[2]{
\renewcommand{#1}[1]{#2}
}
\newcommand{\rff}[2]{
\renewcommand{#1}[2]{#2}
}

\newcommand{\nc}[2]{
  \newcommand{#1}{#2}
}
\newcommand{\rc}[2]{
  \renewcommand{#1}{#2}
}

\nff{\WTF}{#1 (\textit{#2})}

\nf{\translator}{\footnote{\textbf{Translator note:}#1}}
\nc{\sic}{{}^\text{(\textit{sic})}}

\newcommand{\nequ}[2]{
\begin{align*}
#1
\tag{#2}
\end{align*}
}

\newcommand{\uequ}[1]{
\begin{align*}
#1
\end{align*}
}

\nf{\sskip}{...\{#1\}...}
\nff{\iffy}{#2}
\nf{\?}{#1}
\nf{\tags}{#1}

\nf{\limX}{\underset{#1}{\lim}}
\newcommand{\sumXY}[2]{\underset{#1}{\overset{#2}{\sum}}}
\newcommand{\sumX}[1]{\underset{#1}{\sum}}
%\newcommand{\intXY}[2]{\int_{#1}^{#2}}
\nff{\intXY}{\underset{#1}{\overset{#2}{\int}}}

\nc{\fluc}{\overline{\delta_s^2}}

\rf{\exp}{e^{#1}}

\nc{\grad}{\operatorfont{grad}}
\rc{\div}{\operatorfont{div}}

\nf{\pddt}{\frac{\partial{#1}}{\partial t}}
\nf{\ddt}{\frac{d{#1}}{dt}}

\nf{\inv}{\frac{1}{#1}}
\nf{\Nth}{{#1}^\text{th}}
\nff{\pddX}{\frac{\partial{#1}}{\partial{#2}}}
\nf{\rot}{\operatorfont{rot}{#1}}
\nf{\spur}{\operatorfont{spur\,}{#1}}

\nc{\lap}{\Delta}
\nc{\e}{\varepsilon}
\nc{\R}{\mathfrak{r}}

\nc{\Y}{\psi}
\nc{\y}{\varphi}

\nf{\from}{From: #1}
\nf{\rcpt}{To: #1}
\rf{\date}{Date: #1}
\nf{\letter}{\section{Letter #1}}
\nf{\location}{}

\title{Pauli - March - April 1934}

\begin{document}

\letter{363}
\rcpt{Heisenberg}
\date{March 2, 1934}
\location{Zurich}
\tags{hole theory}

Dear Heisenberg!

I can easily answer our letter from the 1st: \textit{You are totally right}. Dirac has overlooked checking his \textit{Ansatz} for the field-free space (that also already attracted my attention).

One can however easily set it right. One puts
\uequ{
f=f_1 +(t-\alpha_s x_s)g
}
for the true values $f_1,g$ as well as for the Dirac ones.

In the force-free case one knows $R^a$, and thus $(f^0_1)_\text{true}$, where here and in the following the index $0$ shall indicate that it deals with the force-free case. Now one \textit{defines}
\uequ{
(f_1)_\text{true} = (f_1)_\text{Dirac} - (f^0_1)_\text{Dirac} + (f^0_1)_\text{true},
}
and defines with \textit{this} $(f_1)_\text{true}$ the separation of $R$ into $R^a$ and $R^b$.

Naturally this \textit{Ansatz} is indeed possible, but completely arbitrary. I don't believe one syllable, and I like the whole hole theory less the more I think about it. -- Monday I start the joint paper with my and Weisskopf's calculations. Please write with your address if you leave Leipzig.

Warmly,

Your W. Pauli

\letter{367}
\rcpt{Pauli and Weisskopf}
\date{April 10, 1934}
\location{Leipzig}
\tags{hole theory}

Dear Pauli and Dear Weisskopf!
\end{document}
