\documentclass{article}
\usepackage[utf8]{inputenc}
\renewcommand*\rmdefault{ppl}
\usepackage[utf8]{inputenc}
\usepackage{amsmath}
\usepackage{graphicx}
\usepackage{enumitem}
\usepackage{amssymb}
\usepackage{marginnote}
\newcommand{\nf}[2]{
\newcommand{#1}[1]{#2}
}
\newcommand{\nff}[2]{
\newcommand{#1}[2]{#2}
}
\newcommand{\rf}[2]{
\renewcommand{#1}[1]{#2}
}
\newcommand{\rff}[2]{
\renewcommand{#1}[2]{#2}
}

\newcommand{\nc}[2]{
  \newcommand{#1}{#2}
}
\newcommand{\rc}[2]{
  \renewcommand{#1}{#2}
}

\nff{\WTF}{#1 (\textit{#2})}

\nf{\translator}{\footnote{\textbf{Translator note:}#1}}
\nc{\sic}{{}^\text{(\textit{sic})}}

\newcommand{\nequ}[2]{
\begin{align*}
#1
\tag{#2}
\end{align*}
}

\newcommand{\uequ}[1]{
\begin{align*}
#1
\end{align*}
}

\nf{\sskip}{...\{#1\}...}
\nff{\iffy}{#2}
\nf{\?}{#1}
\nf{\tags}{#1}

\nf{\limX}{\underset{#1}{\lim}}
\newcommand{\sumXY}[2]{\underset{#1}{\overset{#2}{\sum}}}
\newcommand{\sumX}[1]{\underset{#1}{\sum}}
%\newcommand{\intXY}[2]{\int_{#1}^{#2}}
\nff{\intXY}{\underset{#1}{\overset{#2}{\int}}}

\nc{\fluc}{\overline{\delta_s^2}}

\rf{\exp}{e^{#1}}

\nc{\grad}{\operatorfont{grad}}
\rc{\div}{\operatorfont{div}}

\nf{\pddt}{\frac{\partial{#1}}{\partial t}}
\nf{\ddt}{\frac{d{#1}}{dt}}

\nf{\inv}{\frac{1}{#1}}
\nf{\Nth}{{#1}^\text{th}}
\nff{\pddX}{\frac{\partial{#1}}{\partial{#2}}}
\nf{\rot}{\operatorfont{rot}{#1}}
\nf{\spur}{\operatorfont{spur\,}{#1}}

\nc{\lap}{\Delta}
\nc{\e}{\varepsilon}
\nc{\R}{\mathfrak{r}}

\nff{\Elt}{\operatorfont{#1}_{#2}}

\nff{\MF}{\nc{#1}{\mathfrak{#2}}}

\nc{\Y}{\psi}
\nc{\y}{\varphi}

\nf{\from}{From: #1}
\nf{\rcpt}{To: #1}
\rf{\date}{Date: #1}
\nf{\letter}{\section{Letter #1}}
\nf{\location}{}

\title{Pauli - 1936 - May}

\begin{document}

\letter{429}
\from{Heisenberg}
\date{May 26, 1936}
\location{Leipzig}
\tags{showers}

\MF{\fU}{U}
\MF{\fE}{E}

Dear Pauli!

Since my last letter, in which there indeed wasn't much, I have stumbled into a very remarkable connection which I, according to old habit, write to you for criticism.

I begin with a typical \?{variation} of quantum mechanics from the Fermi $\beta$-decay theory. If, by tossing out unnecessary dimensions, a field function is introduced whose commutation relations give only the $\delta$-function, then instead of $\fU$ and $\fE\y = \fU/\sqrt{\hbar c}$ and $\pi=\fE/\sqrt{\hbar c}$, and further instead of the energy $\epsilon = E/\hbar c$, $\epsilon$ gets the dimensions $\text{cm}^{-1}$ and the Hamiltonian function of quantum electrodynamics becomes roughly
\uequ{
\epsilon = \int\left(
\Y^* \alpha^i \pddX{}{x_i}\Y + \frac{mc}{\hbar}\Y^*\beta\Y + 
\sqrt{\frac{e^2}{\hbar c}}\Y^* \alpha_i \Y \y_i + \inv{2}(\pi^2 + \y^2)
\right){dV}.\\
[\pi_p\, \y_{p'}]_- = \delta_{pp'}; \quad
[\Y^*_p \, \Y_{p'}] = \delta_{pp'}.
}
In such a theory, one can also set $m=0$ without difficulty \?{if interested in large energies}. A perturbation procedure always remains an expansion in $e^2/\hbar c$. That corresponds to the fact that e.g. in the theory of pair-production, the probability for the production of $n$ pairs is always less probable by a factor of $(e^2/\hbar c)^{-1}$ than that for a single pair. This applies independently of the energy of the particle.

It is entirely different in the Fermi $\beta$-theory. If, after the fashion of Uhlenbeck-Konopinski, it is assumed that
\uequ{
\epsilon = \int\left(\Y^* \alpha_i \pddX{}{x_i} \Y + \dots + 
\frac{g}{\hbar c}\Y^*_\text{proton}\Y_\text{neutron}\Y^*_\text{electron}\pddX{}{x}\Y_\text{neutrino}
\right){dV},
}
then the constant $g'=g/\hbar c$ retains the dimension $\text{cm}^3$. At large energies, where the rest masses of the particles can be ignores, this means: any perturbation procedure is an expansion in $g'/\lambda^3$, where $\lambda$ denotes the wavelengths of the particles involved. Now from this it follows: for large energies the interaction part becomes \textit{critical}; in particular, in processes where very many particles are emitted at once become no much more improbable than processes where only one or two particles ar emitted; thus from wavelengths $\sqrt[3]{\lambda}$ down, "showers" of particles are to be expected.
\sskip{More shower speculation}

\letter{454}
\rcpt{Heisenberg}
\date{November 24, 1936}
\location{Zurich}
\tags{nuclear forces,fermi theory}

Dear Heisenberg!

Thank you for your card of the 22nd. -- I don't know anything new about our model, but have something on my mind that I would like to take this opportunity to write about. First to recapitulate:

§1. \sskip{Lattice world and shower formation}

§2. Critical considerations on "Fermi's ingenious theory" of beta radioactivity.

The fact that I have \WTF{been a sucker for}{mit...hineingefallen bin} with the Born approximation in the collisions between two particles with $\delta$-type interaction (I repeat: here it leads to an interaction cross section $\text{const.} \times f^2 k^2$), has on the one hand sobered me, but on the other hand has incited me to seek further consequences. The manner thatbthe $\beta$-decay is calculated since Fermi is namely broadly analogous to my incorrect calculation for the collision cross section with $\delta$-type interaction. One can even (as Stueckelberg has also often stressed to me) represent the process as a collision process where the particle number is conserved by introducing a negative-energy neutrino in the initial state. (The \textit{infiniteness} of the \?{neutrino sea} does not seem to play an essential role this time, since the energy law indeed automatically assures that only a finite part of it can come into action.) Thus at the start \?{there is} one heavy and one light particle, they collide and at the end there is again one heavy and one light particle, which have just been given different names (the charge does not occur explicitly in the theory at all). "It is well-known" that the in the Fermi approach tje $\delta$-property of the interaction consists in the fact that the particles only act on one another when they are at the same location and, as in our collision problem, can be interpreted as a generalized central force. For this reason it seems to me that the final conclusion is rather unavoidable: \textit{the "ingenious theory" in reality leads to zero probability for $\beta$-decay. That Fermi (and his later imitators with derivatives) has gotten something other than zero out of it is due only to an unjustified application of perturbation theory ("Born approximation")}.

At the moment I don't see how anything sastsfactory can be brought against this consequence of our calculations, other than arbitrarily deciding that the perturbation \textit{shall} be assumed to be sufficient with $\beta$-decay. (We we aren't half-experimental opportunists like Fermi.)

Nevertheless even from an opportunistic standpoint it looks quite bad for the ingenious theory. Fermi's original approach is contradicted by experiment\footnote{That the correspondence with experiment cannot be significantly improved even with an arbitrary linear combinatiom of the 5 possible relativistic invariants will be shown in an upcoming paper by Fierz (which I shall come back to).} and the attempt at rescuing it by Uhlenbeck-Konopinski seems (no to mention the theoretical difficulties it involves concerning the field quantization) to likewise not match in the vicinity of the upper energy limit of the $\beta$-spectrum. I suspect that the \?{contact with the $x$-axis there must be of the 2nd and not 4th order} and that all energy limits extrapolated from Uhlenbeck-Konopinski are too high (see also a note by \textit{Lyman} in Physical Review.)

The inevitable question arises: can't a reasonable relativistically-invariant theory of $\beta$-decay that is \textit{essentially} dfiferent from Fermi's be made. So far I have been unsuccessful in finding such an approach (it must be quantized according to the exclusion principle!) - but that still wouldn't prove that one doesn't exist. E.g. the shower theory is much more general, since it only relies on a dimensional consideration (existence of a universal length). \textit{What do you think?}

§3. Less ingenious, but substantially more solid are the newer papers by the Americans on nuclear forces, which indeed move on the rather secure and infinity-free ground of non-relativistic wave mechanics. Whether the light-nucleus mass defect calculations are already good enough, I dare not say. But, the possibility that all non-electrodynamical forces between heavy particles are equal would in fact be "promising"! As regards the heavy particles, the Thomas-Fermi method is definitely nonsense! 

In Fierz's paper the problem (also treated by Weizsaecker in his [Leipzig!] habilitation paper) of the magnetic moment of the heavy particles is again tackled (in geneal)
%nicht aufs Sofa
\?{}

\include{subfiles/455}

\letter{459}
\rcpt{Heisenberg}
\date{December 13, 1936}
\location{Zurich}
\tags{nuclear theory}

\nc{\vsigma}{\vec{\sigma}}
\nc{\vx}{\vec{x}}

Dear Heisenberg!

Many thanks for your letter of the 7th. \sskip{Shower formation, fundamental unit of length, ...}
\sskip{Holiday plans}

Now two remarks on the nuclear theory:

1. Fierz found from the second approximation of the Fermi theory the formal possibility of a force whose potential is proportional to the product of the components of the two spins in the direction of the connecting line between the particles:
\uequ{
V\Y \approx \tau(r)(\vsigma^I, \vx^I - \vx^{II})(\vsigma^{II}, \vx^I - \vx^{II})\Y(x^{II}, x^I ).}

Such a force would be of a different type than the forces considered up to now. Do you believe that it coukd be of use in the theory of heavy nuclei? (NB. I haven't previously trusted the approximation methods for the heavy nuclei.)

2. Despite all skepticism against the "ingenious theory", the general idea that the "neutrino field" mediates the nuclear force could nonetheless be true. But then one would prefer greater symmetry in the Hamiltonian function with respect to proton-proton and proton-neutron forces; i.e. even for the theory of $\beta$-decay it shouldn't matter whether the particles are charged or not (so far as only the total charge is conserved). Thus if the Americans are really correct, the Hamiltonian function should also have terms of the form
\uequ{
\underset{\text{Neutron-neutron or proton-proton}}{f(\Y^* \gamma^\nu \Y)} \quad \underset{\text{Hole-electron or antineutrino-neutrino}}{(\Y^* \gamma^\nu \Y)}
}
of the same order of magnitude as the terms corresponding to the $\beta$-decay. I.e. \textit{that excited nuclei could spontaneously emit either an electron-positron pair or a pair of two neutrinos.} (This idea has often been highlighted here by \textit{Wentzel}.) Why is this pair emission never observed? If it is not just experimental chance that they are missing, \?{then the Americans have found quite a coincidence with their irrelevance of the charge of the particles in non-electromagnetic effects}. It would then be hard to see why the change in charge should \textit{always} occur half in the heavy and half in the light particles. -- But it can also be that this pair-production (unfortunately neutrinos are not seen) definitely occurs and up to now was just always interpretetes as internal conversion of $\gamma$-quanta. (The dependence of the frequency of this effect on the nuclear charge would however be different; perhaps a pair production proportional to $f^2$ coukd be recognized in it.) -- \textit{What do you think?}

So I am probably going on vacation \WTF{soon}{in Sichtbarkeitsnähe} -- my wife sends greetings and would also be very happy to see you again -- and in any case write soon.

Many greetings,

Your W. Pauli

% nicht aufs Sofa!

\letter{460}
\rcpt{Kemmer}
\date{December 15, 1936}
\location{Zurich}
\tags{nuclear theory}

Very honorable Herr Kemmer!

Thans for the letter and manuscript. So far I am in agreement with it and today have sent it to the editors at Helvetica Physica Acta.

The precise interaction cross-section formula could possibly yet be of use to me, so please send them to me \WTF{at your convenience}{gelegentlich}.

As concerns your wider work, Wentzel is now keen to obtain the manuscript of your note on the Delbr\"uck effect.

To the question as to what you should should \?{further cross out}{weiter ixen}, we shall perhaps come back to that after Christmas. Just look at the new American papers on the nuclear forces in the Physical Review of the 1st of November more closely. The possibility discussed there that all non-electromagnetic nuclear forces should be independent of the charge (totally equal for proton-neutron and proton-proton) has a certain inner reason about it. \?{It could well be that it allows reasonable problems to be calculated}.

Are you staying in London for Christmas? In any case, have a good vacation and a happy new year. Greatings from Fierz as well.

Always your W. Pauli

Bargmann has a chance of getting a position in Rostow-on-Don.% 
% niemals aufs sofa

\letter{461}
\from{Heisenberg}
\date{December 18, 1936}
\location{Leipzig}
\tags{nuclear forces, universal length, speculation}

\MF{\fr}{r}

Dear Pauli!

The present situation of field theory can probably be described thus: we have a correspondence-like theory which is just about as good and as bad as the Bohr theory in its time. In the theory at this approximation (the "semiclassical") the universal length is already \textit{correctly} taken into account if it is written as a factor to a nonlinear interaction term in the Hamiltonian function. However, the step from there to a self-contained theory is missing. The new theory will distinguish itself from the present false wave-quantum-theory in two ways. First, it will demand rather \textit{less} from the particles than the present theory: the particles of the future theory will not be smaller than the universal length, it will not be possible at all to speak of a local interaction theory with a precision going beyond the universal length (I see e.g. no grounds for the existence of a differential equation for $\Y$); therefore, it will secondly avoid the divergences of the present theory. -- The next step towards this goal can probably not be made systematically, one must simply be lucky to find the correct direction.

Now I want to write about the nuclear theory:

Some time ago I left the investigation of the action of forces of the type
\uequ{
\tau(r_{12})\frac{(\sigma_1 \fr_{12})(\sigma_2 \fr_{12})}{r_{12}^2}\left[\begin{matrix}
P_{12}\\
1
\end{matrix}\right],
}
which also occur in Weizs\"acker, \WTF{to a Pole here}{hab' ich vor einiger Zeit durch einen Polen hier untersuchen lassen}. They behave rather similarly to my old exchange forces:
\uequ{
\tau(r_{12})\inv{2}(1+\sigma_1 \sigma_2)P_{12}.
}
The deuteron can again be \textit{rigorously} separated into singlet and triplet systems. For tge singlet system, one can put
\uequ{
\frac{(\sigma_1 \fr_{12})(\sigma_2 \fr_{12})}{r_{12}^2} = -1.
}
In the triplet system, it is not generally possible to specify the value of $l$ (orbital angular momentum) for each term. There is \textit{one} sort of term for which this is possible, namely to each $j\ge 1$ (total angular momentum) one term with $l=j$. For this one can put (I don't quite know this by heart at the moment)
\uequ{
\frac{(\sigma_1 \fr_{12})(\sigma_2 \fr_{12})}{r_{12}^2} = 
\frac{j(j+2)}{(j+1)(2j+3)}.
}
Further, for every $j\ge 1$ there are two terms which are "mixed" from the values $l=j+1$ and $l=j-1$ and for which two simultaneous Schr\"odinger equations have to be solved. For $j=0$ there is only \textit{one} term (in the triplet system) with $l=1$. \?{Nothing interesting seems to be coming out of these forces (as opposed to the present ones).} I hold the additional term in the Fermi interaction, which you have discussed with Wentzel, to also be another immediate consequence of the American forces. (My recently-cited "selection rules" for the "$\rho$-spin" in $\beta$-decay apply only under this supposition); but it doesn't seem likely that the associates processes can be experimentally found. In general, the transition probability by $\gamma$-emission is \textit{much} greater. At best, if the hypothesis of isomeric nuclei is correct, which I doubt, then it could give transitions with pair-production, \WTF{most commonly without simultaneous}{wohl meist garnicht mit} $\beta$-decay. Thus I do not believe that the presently-observed pair-production can be interpreted as anything other than internal conversion. The transition probability per second for the new effect would be on the order of $1/10$ ($\text{sec}^{-1}$), that for $\gamma$ emission ia however normally $10^16$ ($\text{sec}^{-1}$). It muat already meet with many lucky circumstances in order to get past the 17 powers of ten.

\sskip{Shower formation paper by Von Bar\'onthy and Forr\'o}

I would be glad to accept the friendly invitation to come to Zurich. But it doesn't look like I shall presently be able to travel to Switzerland this time. Thus we must probably postpone our face to facs discussions; I hope to be able to go skiing in Switzerland again early in the Spring.

Best Christmas wishes to you and your wife!

Your W. Heisenberg


\letter{462}
\rcpt{Heisenberg}
\date{December 21, 1936}
\location{Zurich}
\tags{}

\nc{\vr}{\vec{r}}

Dear Heisenberg!

Many thanks for your letter. What a shame that you aren't coming this time. After Christmas I want to more closely investigate my and Weisskopf's theory in the lattice world. What you write about fields of the type
\uequ{
\tau(r_{12})\frac{(\vsigma_1\vr_{12})(\vsigma_2 r_{12})}{r_{12}^2}\left[
\begin{matrix}
P_{12}\\
1
\end{matrix}
\right],
}
has also been meanwhile worked out by Fierz here. -- Also, Wentzel has recently said exactly the same thing regarding the pair-production according to the additional terms in the Fermi interaction; that this pair production is \textit{always} negligible compared to that by the internal conversion of $\gamma$-quanta.

Your information about the experimental work of Barn\'othy and Forr\'o interested me very much. Scherrer and I would be very interested in this paper (\sskip{...}). Could you send it some time please? \WTF{There is however some time}{Es hat aber gut Zeit} until after Christmas, and perhaps by then it will already have appeared.

So if you aren't coming now, we certainly hope that you come in the new year.

Wishin you a merry Christmas and a new years toast from my wife and

W. Pauli

% nicht aufs Soda

\end{document}
