\letter{454}
\rcpt{Heisenberg}
\date{November 24, 1936}
\location{Zurich}
\tags{nuclear forces,fermi theory}

Dear Heisenberg!

Thank you for your card of the 22nd. -- I don't know anything new about our model, but have something on my mind that I would like to take this opportunity to write about. First to recapitulate:

§1. \sskip{Lattice world and shower formation}

§2. Critical considerations on "Fermi's ingenious theory" of beta radioactivity.

The fact that I have \WTF{been a sucker for}{mit...hineingefallen bin} with the Born approximation in the collisions between two particles with $\delta$-type interaction (I repeat: here it leads to an interaction cross section $\text{const.} \times f^2 k^2$), has on the one hand sobered me, but on the other hand has incited me to seek further consequences. The manner thatbthe $\beta$-decay is calculated since Fermi is namely broadly analogous to my incorrect calculation for the collision cross section with $\delta$-type interaction. One can even (as Stueckelberg has also often stressed to me) represent the process as a collision process where the particle number is conserved by introducing a negative-energy neutrino in the initial state. (The \textit{infiniteness} of the \?{neutrino sea} does not seem to play an essential role this time, since the energy law indeed automatically assures that only a finite part of it can come into action.) Thus at the start \?{there is} one heavy and one light particle, they collide and at the end there is again one heavy and one light particle, which have just been given different names (the charge does not occur explicitly in the theory at all). "It is well-known" that the in the Fermi approach tje $\delta$-property of the interaction consists in the fact that the particles only act on one another when they are at the same location and, as in our collision problem, can be interpreted as a generalized central force. For this reason it seems to me that the final conclusion is rather unavoidable: \textit{the "ingenious theory" in reality leads to zero probability for $\beta$-decay. That Fermi (and his later imitators with derivatives) has gotten something other than zero out of it is due only to an unjustified application of perturbation theory ("Born approximation")}.

At the moment I don't see how anything sastsfactory can be brought against this consequence of our calculations, other than arbitrarily deciding that the perturbation \textit{shall} be assumed to be sufficient with $\beta$-decay. (We we aren't half-experimental opportunists like Fermi.)

Nevertheless even from an opportunistic standpoint it looks quite bad for the ingenious theory. Fermi's original approach is contradicted by experiment\footnote{That the correspondence with experiment cannot be significantly improved even with an arbitrary linear combinatiom of the 5 possible relativistic invariants will be shown in an upcoming paper by Fierz (which I shall come back to).} and the attempt at rescuing it by Uhlenbeck-Konopinski seems (no to mention the theoretical difficulties it involves concerning the field quantization) to likewise not match in the vicinity of the upper energy limit of the $\beta$-spectrum. I suspect that the \?{contact with the $x$-axis there must be of the 2nd and not 4th order} and that all energy limits extrapolated from Uhlenbeck-Konopinski are too high (see also a note by \textit{Lyman} in Physical Review.)

The inevitable question arises: can't a reasonable relativistically-invariant theory of $\beta$-decay that is \textit{essentially} dfiferent from Fermi's be made. So far I have been unsuccessful in finding such an approach (it must be quantized according to the exclusion principle!) - but that still wouldn't prove that one doesn't exist. E.g. the shower theory is much more general, since it only relies on a dimensional consideration (existence of a universal length). \textit{What do you think?}

§3. Less ingenious, but substantially more solid are the newer papers by the Americans on nuclear forces, which indeed move on the rather secure and infinity-free ground of non-relativistic wave mechanics. Whether the light-nucleus mass defect calculations are already good enough, I dare not say. But, the possibility that all non-electrodynamical forces between heavy particles are equal would in fact be "promising"! As regards the heavy particles, the Thomas-Fermi method is definitely nonsense! 

In Fierz's paper the problem (also treated by Weizsaecker in his [Leipzig!] habilitation paper) of the magnetic moment of the heavy particles is again tackled (in geneal)
%nicht aufs Sofa
\?{}