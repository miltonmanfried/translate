\letter{459}
\rcpt{Heisenberg}
\date{December 13, 1936}
\location{Zurich}
\tags{nuclear theory}

\nc{\vsigma}{\vec{\sigma}}
\nc{\vx}{\vec{x}}

Dear Heisenberg!

Many thanks for your letter of the 7th. \sskip{Shower formation, fundamental unit of length, ...}
\sskip{Holiday plans}

Now two remarks on the nuclear theory:

1. Fierz found from the second approximation of the Fermi theory the formal possibility of a force whose potential is proportional to the product of the components of the two spins in the direction of the connecting line between the particles:
\uequ{
V\Y \approx \tau(r)(\vsigma^I, \vx^I - \vx^{II})(\vsigma^{II}, \vx^I - \vx^{II})\Y(x^{II}, x^I ).}

Such a force would be of a different type than the forces considered up to now. Do you believe that it coukd be of use in the theory of heavy nuclei? (NB. I haven't previously trusted the approximation methods for the heavy nuclei.)

2. Despite all skepticism against the "ingenious theory", the general idea that the "neutrino field" mediates the nuclear force could nonetheless be true. But then one would prefer greater symmetry in the Hamiltonian function with respect to proton-proton and proton-neutron forces; i.e. even for the theory of $\beta$-decay it shouldn't matter whether the particles are charged or not (so far as only the total charge is conserved). Thus if the Americans are really correct, the Hamiltonian function should also have terms of the form
\uequ{
\underset{\text{Neutron-neutron or proton-proton}}{f(\Y^* \gamma^\nu \Y)} \quad \underset{\text{Hole-electron or antineutrino-neutrino}}{(\Y^* \gamma^\nu \Y)}
}
of the same order of magnitude as the terms corresponding to the $\beta$-decay. I.e. \textit{that excited nuclei could spontaneously emit either an electron-positron pair or a pair of two neutrinos.} (This idea has often been highlighted here by \textit{Wentzel}.) Why is this pair emission never observed? If it is not just experimental chance that they are missing, \?{then the Americans have found quite a coincidence with their irrelevance of the charge of the particles in non-electromagnetic effects}. It would then be hard to see why the change in charge should \textit{always} occur half in the heavy and half in the light particles. -- But it can also be that this pair-production (unfortunately neutrinos are not seen) definitely occurs and up to now was just always interpretetes as internal conversion of $\gamma$-quanta. (The dependence of the frequency of this effect on the nuclear charge would however be different; perhaps a pair production proportional to $f^2$ coukd be recognized in it.) -- \textit{What do you think?}

So I am probably going on vacation \WTF{soon}{in Sichtbarkeitsnähe} -- my wife sends greetings and would also be very happy to see you again -- and in any case write soon.

Many greetings,

Your W. Pauli

% nicht aufs Sofa!