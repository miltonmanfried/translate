\letter{461}
\from{Heisenberg}
\date{December 18, 1936}
\location{Leipzig}
\tags{nuclear forces, universal length, speculation}

\MF{\fr}{r}

Dear Pauli!

The present situation of field theory can probably be described thus: we have a correspondence-like theory which is just about as good and as bad as the Bohr theory in its time. In the theory at this approximation (the "semiclassical") the universal length is already \textit{correctly} taken into account if it is written as a factor to a nonlinear interaction term in the Hamiltonian function. However, the step from there to a self-contained theory is missing. The new theory will distinguish itself from the present false wave-quantum-theory in two ways. First, it will demand rather \textit{less} from the particles than the present theory: the particles of the future theory will not be smaller than the universal length, it will not be possible at all to speak of a local interaction theory with a precision going beyond the universal length (I see e.g. no grounds for the existence of a differential equation for $\Y$); therefore, it will secondly avoid the divergences of the present theory. -- The next step towards this goal can probably not be made systematically, one must simply be lucky to find the correct direction.

Now I want to write about the nuclear theory:

Some time ago I left the investigation of the action of forces of the type
\uequ{
\tau(r_{12})\frac{(\sigma_1 \fr_{12})(\sigma_2 \fr_{12})}{r_{12}^2}\left[\begin{matrix}
P_{12}\\
1
\end{matrix}\right],
}
which also occur in Weizs\"acker, \WTF{to a Pole here}{hab' ich vor einiger Zeit durch einen Polen hier untersuchen lassen}. They behave rather similarly to my old exchange forces:
\uequ{
\tau(r_{12})\inv{2}(1+\sigma_1 \sigma_2)P_{12}.
}
The deuteron can again be \textit{rigorously} separated into singlet and triplet systems. For tge singlet system, one can put
\uequ{
\frac{(\sigma_1 \fr_{12})(\sigma_2 \fr_{12})}{r_{12}^2} = -1.
}
In the triplet system, it is not generally possible to specify the value of $l$ (orbital angular momentum) for each term. There is \textit{one} sort of term for which this is possible, namely to each $j\ge 1$ (total angular momentum) one term with $l=j$. For this one can put (I don't quite know this by heart at the moment)
\uequ{
\frac{(\sigma_1 \fr_{12})(\sigma_2 \fr_{12})}{r_{12}^2} = 
\frac{j(j+2)}{(j+1)(2j+3)}.
}
Further, for every $j\ge 1$ there are two terms which are "mixed" from the values $l=j+1$ and $l=j-1$ and for which two simultaneous Schr\"odinger equations have to be solved. For $j=0$ there is only \textit{one} term (in the triplet system) with $l=1$. \?{Nothing interesting seems to be coming out of these forces (as opposed to the present ones).} I hold the additional term in the Fermi interaction, which you have discussed with Wentzel, to also be another immediate consequence of the American forces. (My recently-cited "selection rules" for the "$\rho$-spin" in $\beta$-decay apply only under this supposition); but it doesn't seem likely that the associates processes can be experimentally found. In general, the transition probability by $\gamma$-emission is \textit{much} greater. At best, if the hypothesis of isomeric nuclei is correct, which I doubt, then it could give transitions with pair-production, \WTF{most commonly without simultaneous}{wohl meist garnicht mit} $\beta$-decay. Thus I do not believe that the presently-observed pair-production can be interpreted as anything other than internal conversion. The transition probability per second for the new effect would be on the order of $1/10$ ($\text{sec}^{-1}$), that for $\gamma$ emission ia however normally $10^16$ ($\text{sec}^{-1}$). It muat already meet with many lucky circumstances in order to get past the 17 powers of ten.

\sskip{Shower formation paper by Von Bar\'onthy and Forr\'o}

I would be glad to accept the friendly invitation to come to Zurich. But it doesn't look like I shall presently be able to travel to Switzerland this time. Thus we must probably postpone our face to facs discussions; I hope to be able to go skiing in Switzerland again early in the Spring.

Best Christmas wishes to you and your wife!

Your W. Heisenberg
