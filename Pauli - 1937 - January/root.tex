\documentclass{article}
\usepackage[utf8]{inputenc}
\renewcommand*\rmdefault{ppl}
\usepackage[utf8]{inputenc}
\usepackage{amsmath}
\usepackage{graphicx}
\usepackage{enumitem}
\usepackage{amssymb}
\usepackage{marginnote}
\newcommand{\nf}[2]{
\newcommand{#1}[1]{#2}
}
\newcommand{\nff}[2]{
\newcommand{#1}[2]{#2}
}
\newcommand{\rf}[2]{
\renewcommand{#1}[1]{#2}
}
\newcommand{\rff}[2]{
\renewcommand{#1}[2]{#2}
}

\newcommand{\nc}[2]{
  \newcommand{#1}{#2}
}
\newcommand{\rc}[2]{
  \renewcommand{#1}{#2}
}

\nff{\WTF}{#1 (\textit{#2})}

\nf{\translator}{\footnote{\textbf{Translator note:}#1}}
\nc{\sic}{{}^\text{(\textit{sic})}}

\newcommand{\nequ}[2]{
\begin{align*}
#1
\tag{#2}
\end{align*}
}

\newcommand{\uequ}[1]{
\begin{align*}
#1
\end{align*}
}

\nf{\sskip}{...\{#1\}...}
\nff{\iffy}{#2}
\nf{\?}{#1}
\nf{\tags}{#1}

\nf{\limX}{\underset{#1}{\lim}}
\newcommand{\sumXY}[2]{\underset{#1}{\overset{#2}{\sum}}}
\newcommand{\sumX}[1]{\underset{#1}{\sum}}
\nf{\prodX}{\underset{#1}{\prod}}
\nff{\prodXY}{\underset{#1}{\overset{#2}{\prod}}}
\nf{\intX}{\underset{#1}{\int}}
\nff{\intXY}{\underset{#1}{\overset{#2}{\int}}}

\nc{\fluc}{\overline{\delta_s^2}}

\rf{\exp}{e^{#1}}

\nc{\grad}{\operatorfont{grad}}
\rc{\div}{\operatorfont{div}}

\nf{\pddt}{\frac{\partial{#1}}{\partial t}}
\nf{\ddt}{\frac{d{#1}}{dt}}

\nf{\inv}{\frac{1}{#1}}
\nf{\Nth}{{#1}^\text{th}}
\nff{\pddX}{\frac{\partial{#1}}{\partial{#2}}}
\nf{\rot}{\operatorfont{rot}{#1}}
\nf{\spur}{\operatorfont{spur\,}{#1}}

\nc{\lap}{\Delta}
\nc{\e}{\varepsilon}
\nc{\R}{\mathfrak{r}}

\nff{\Elt}{\operatorfont{#1}_{#2}}

\nff{\MF}{\nc{#1}{\mathfrak{#2}}}

\nc{\Y}{\psi}
\nc{\y}{\varphi}

\nf{\from}{From: #1}
\nf{\rcpt}{To: #1}
\rf{\date}{Date: #1}
\nf{\letter}{\section{Letter #1}}
\nf{\location}{}
\nf{\references}{}

\title{Pauli - 1937 - January}

\begin{document}

\letter{463}
\from{Heisenberg}
\date{January 16, 1936}
\location{Leipzig}
\tags{showers,qed,interaction representation}

\nc{\Hbar}{\overline{H}}
\nc{\fg}{\mathfrak{g}}
\nc{\fr}{\mathfrak{r}}

Dear Pauli!

You have probably now received Barn\'othy's treatise, I have put off sending it for so long since I wanted to wait for a letter from Barn\'othy; but there's nothing much new in it. Incidentally, after the recent results from Anderson on the radiation of electrons (Bethe-Heitler), I would like to believe that for electrons, real shower formation is relatively rare (c.f. also Oppenheimer's note in the last issue of the Physical Review), \sskip{shower formation...}.

Now I'd like to write extenstively on our main problem, the quantization of waves. The following reflections \WTF{refer to}{beziehen sich} to the "model":

\nequ{
L = -\sumX{\nu}|\pddX{\Y}{X_\nu}|^2 - f(\Y^*\Y)^3, \quad
H = \overset{H_0}{\overbrace{\pi^*\pi + \pddX{\Y}{X_k}\pddX{\Y}{X_k}}} +
\overset{H_1}{\overbrace{f(\Y^*\Y)^3}}
}{1}
however they are not bound to this specific form (units: $c=1$, $\hbar=1$). First I want to formulate the Lorentz invariance of wave quantization rather more directly than before. Following your work with Weisskopf, I put
\nequ{
\Y = \sumX{k}\frac{-i}{\sqrt{2Vk}}(-a_k + b^*_k)\exp{i(kr)};\quad
\pi = \sum\sqrt{\frac{k}{2V}}(a_k^* + b_k)\exp{-ikr}.
}{2}
The Schr\"odinger functional should depend on the $N_k = a_k^*a_k$, $M_k = b_k^* b_k$.
\nequ{
\Phi(N_1,N_2\dots;M_1,Me2,\dots;t);\quad
i\pddX{\Phi}{t} = \Hbar.\quad (\Hbar = \int H{dV})
}{3}
Now I put
\nequ{
\Phi(N_1\dots;M_1\dots;t) = \exp{-i\sumX{k}(N_k + M_k)kt}\y(N_1\dots; M_1\dots; t),
}{4}
and accordingly
\nequ{
a_k^* = N_k^{1/2}\Delta_k^-\exp{ikt};\quad
a_k   = \Delta_k^+N_k^{1/2}\exp{-ikt}.
}{5}
Then for $\y$ the new Schr\"odinger equation applies:
\nequ{
i\pddX{\y}{t} = \Hbar_1 \y.
}{6}
Without interactions, $\y$ is of course constant in time. Further,
\nequ{
H_1 = &f\left[\sumX{k,l}\inv{\sqrt{2Vk}}\left(-\Delta^+_k N_k^{1/2}\exp{i(kr-kt)} + 
M_k^{1/2} \Delta_k^-\exp{i(kr+kt)}\right)\right.\\
&\left.\inv{\sqrt{2Vl}}\left(-N_l^{1/2}\Delta^-_l \exp{-i(lr-lt)} + 
\Delta_l^+M_l^{1/2}\exp{-i(lr+lt)}\right)
\right]^3.
}{7}
This is, as is easily seen, relativistically invariant: the $N$ and $M$ are invariant, \!{they only obtain different denominators under a Lorentz transformation} $N_{k(k)}\to N_{k'(k')}$ (where $k'=(k-\beta k)/\sqrt{1-\beta^2}$, etc); likewise the operators $\Delta$. The expressions $\sqrt{Vk}$ are also relativistically-invariant, since $V$ essentially denotes the number of stationary states per interval $dk_x dk_y dk_z$. It can then be easily found that $Vk=V'k'$. And the rest is likewise invariant, so thus also the operator $H_1$. It should be noted however that $H_1$ explicitly contains the time. Now, according to (6)
\nequ{
\y(t+{dt}) &= (1-i\Hbar_1\,{dt})\y(t) = \exp{i[\int H_1\,{dV}]dt}\\
\y(t+2{dt})&= (1-i\Hbar_1(t+{dt}){dt})(1-i\Hbar_1(t)\,{dt})\y(t)\\
 &= \exp{-i\,{dt}\int H_1(t+{dt}){dV}}\exp{-i\,{dt}\int H_1(t){dV}}\y(t)
\quad \text{etc.}
}{8}

It should again be noted that the $e$-functions might not be \?{contracted} do \textit{one} $e$-function, since the exponents, which have different $t$s, do not commute. The factors associated with later times must always stand to the left of the earlier ones. Hence, one might naturally write
\uequ{
\exp{-i{\,dt}\int H_1\,{dV}} = \prodX{V}\exp{-i\,{dt}\,H_1\,{dV}}
}
where the product refers to all space elements ${dV}$. In general equation (8) can always be written
\nequ{
\y(t_1) = \prodX{\Omega}\left(\exp{-iH_1\,{dt}\,{dx}\,{dy}\,{dz}}\right)
\y(t_0) = \prodX{\Omega}\left(\exp{-iH_1\,{d\omega}}\right)\y(t_0),
}{9}
where ${d\omega}$ denotes the four-dimensional volume element, and $\Omega$ is the four-dimensional volume enclosed between the surfaces $t=t_0$ and $t=t_1$. Equation (9) now makes the relativistic invariance of the whole scheme immediately evident. Equation (9) can also be expanded \?{so that the time point $t_1$ is measured in in the moving system ($t'=t_1$)}. Then (9) still applies and $\Omega$ denotes the volume between $t=t_0$ and $t'=t_1$.

TODO--FIGURE (x/t axis; shading $\Omega$ between $t=const$ and $t'=kx$, with $t=t_0$ and $t'=t_1$ labelled)

So far, everything is just rearranging the previous quantum theory of waves.

Now, maintaining equation (9), I would like to ammend expression (7) so that equation (9) leads to convergent results. One possibility seems to be the following: it is perhaps convenient for now, in order to get the most symmetrical formulae possible, to put $\fg=-k$, $g=k$,
\uequ{
M_k^{1/2}\Delta^-_k \exp{i(kr+kt)} = 
M_g^{1/2}\Delta_g^- \exp{-i(\fg r - gt)}.
}
Then the expression (7) can be split into individual summands which each contain a factor of the form $\exp{i\sumXY{\nu=1}{4}(k_\nu^{(1)}+k_\nu^{(2)}+\dots+k_\nu^{(6)},\fr_\nu})$. $k_\nu$, $\fr_\nu$ are the four-vectors $\fr_\nu=(x\,y\,z, it)$, $k_\nu=(k_x,k_y,k_z,it)$; in some terms there is a $\fg$ instead of a $k$. Now I replace this factor by the new one:
\nequ{
\exp{i\sumX{\nu}(k_\nu^{(1)}+\dots+k_\nu^{(6)}, \fr^\nu) 
  + \alpha f\sumXY{\nu=1}{4}(k_\nu^{(1)} + \dots + k_\nu^{(6)})^2},
}{10}
and further I put all terms with stars $(a^*,b^*)$ to the \textit{left} of those without stars.

That is essentially the old (hideous) \WTF{subtraction procedure}{Abschneideverfahren} of Born, Wataghin, etc ($\alpha$ is some number). But I claim that here it leads to a \WTF{self-consistent}{in sich geschlossenen} contradiction-free theory. Initially the relativistic invariance is no longer trivial, since it must first be proven that the new $H_1$ commutes at different \textit{spatial} points. But it seems to be that this is the case (could you check this point?). Then one must \WTF{check}{nachschauen} what results arise for the self-energy etc. For this purpose one can go back from the equations (7)+(10) to the operator $H$ and to equation (3). It is then found via a perturbation calculation that the terms increase at first, but they are all finite\footnote{I've meanwhile worked more on this. It now seems that for $k\gg\inv{\sqrt{\alpha f}}$ the perturbation calculation converges well if at the same time $k\ll\alpha^{3/2}/\sqrt{f}$. This special case can be realized when $\alpha\gg 1$. Nevertheless I want to think this over some more.}, and I have the impression that indeed $H_1$ cannot be taken as a small perturbation, \?{but that otherwise the procedure converges}. I still have not further examined the possibility (10) and would like to know what you think about it.

Personally I believe something like the following: I think the specific subtraction procedure (10) is dumb, since it is entirely arbitrary. Also I can hardly imagine that arbitrarily-many types of convergent theories can be made, and I would rather introduce the universal length through the factor of $H_1$ than through a subtraction prescription. However, it seems reasonable in the proposal that the idea of a "wavefunction at a definite position" is given up and only the idea of "particles with definite momenta" is introduced; since the momentum can always be measurd exactly. Thus I would like to provisionally claim: even the future exact theory can be represented by an operator $H_1$, which goes over into one of the type (7) for small momenta and which is connected with $\y$ by equations (6) and (9).

But first we must probably clarify the main question: whether a relativistic and convergent theory is at all possible according to this scheme. So what do you think about it?

Many warm greetings, also to your wife and to the institute.

Your W. Heisenberg

 
%nicht aufs Sofa
\letter{464}
\rcpt{Heisenberg}
\date{January 19, 1937}
\location{Zurich}
\tags{qed,interaction representation}
\references{463}

Dear Heisenberg!

First, I just want to provisionally answer your last letter (of the 16th).

\textit{1. Cosmic radiation}

\sskip{...}

\textit{2. Quantization of waves}

I find your equation (9):
\uequ{
\Y(t_1) = \prodX{\Omega}(\exp{-iH_1\,{d\omega}})\y(t_0)
}
to be quite lovely. I will write more on the factor $\exp{\alpha f\sumX{\nu=1}{4}(k_\nu^1+\dots+k_\nu^{(6)})^2}$ later; namely, I want to read Wataghin's paper again before getting into the question of the commutability of the new $H$ at different spatial points\footnote{P.S. it seems to be permitted.} and the question of convergence. It seems fishy however. For its part, I have closely read Wataghin's paper and believe that it must be rejected, since it is \textit{not gauge-invariant} in an electromagnetic field. But if Wataghin's $P_\nu$ (in the subtraction factor) is replaced by $\pi_\nu = p_\nu + \frac{a}{c}\y_\nu$, then because of the non-commutability of the $\pi$'s with one another, great difficulties arise. The Lorentz invariance was in order with Wataghin, \?{but I only want to see whether it is also included in your additional term} $f(\Y^*\Y)^3$. It seems that this is probably the case. But the convergence questions could be awkward\footnote{Does the fact that $\sumX{\nu}(k_\nu^{(1)}+\dots+k_\nu^{(6)})^2$ vanishes for parallel $k_\nu^{(1)},k_\nu^{(2)},\dots,k_\nu^{(6)}$ hurt anything?}.

I do not agre \textit{at all} with the physical intepretation that the concept "particles with definite momenta" is physically better than the concept "wavefunctions with a definite position". Since the problem with the first concept lies in the fact that the momentum of the particle is needed \textit{at a totally definite time $t$} (while a very accurate measurememt of momentum can never be carried out in an arbitrarily-short time). Only in the force-free case, where the momentum is constant in time, does this cause no problems; but as soon as there are interaction terms, the problem with the $\y(N_k,M_k,t)$ emerges exactly as with $\y(x,t)$.

Thus I would like to put off addressing your specific questions connected with the particular subtraction procedure. -- But in conclusion, a certain critique of your remark in your last letter before Christmas, whether it is certain that wavefunctions which satisfy a differential equation would be present at all in a future theory (the opposite is the case in e.g. the lattice world.) But it is hard to see how Lorentz invariance should emerge if that is given up. \?{The main difficulty lies in} bringing Lorentz invariance and the existence of a universal length into harmony.

%nicht aufs SOPA
\letter{468}
\from{Heisenberg}
\date{February 2, 1937}
\location{Leipzig}
\tags{qed,interaction representation}
\references{467}

\nc{\fG}{\mathfrak{G}}

Dear Pauli!

Many thanks for your letter. \sskip{cosmic radiation}.

Meanwhile I wanted to write you about a new formulation of the quantum theory of waves, which I take much more seriously than the recent subtraction procedure. I shall again transform the usual theory until it takes a form in which it can be expanded and changed.

The formula
\uequ{
\Y(t) = \prodX{\Omega}\exp{iH_{(1)}\,{d\omega}}\Y
t_0}


can, decomposing ${d\omega}$ into ${d\tau}{dt}$, immediately be written as:
\nequ{
\Y(t) &= \left[1+\left(i\int{d\tau}\intXY{t_0}{t}{dt_1}\,H_{(1)}\right) + 
\left(i\int{d\tau}\intXY{t_0}{t}{dt_1}\,H_{(1)}\right)
\left(i\int{d\tau}\intXY{t_0}{t_1}{dt_2}\,H_{(1)}\right) + \right.\\
&\left.
\left(i\int{d\tau}\intXY{t_0}{t}{dt_1}\,H_{(1)}\right)
\left(i\int{d\tau}\intXY{t_0}{t_1}{dt_2}\,H_{(1)}\right)
\left(i\int{d\tau}\intXY{t_0}{t_2}{dt_3}\,H_{(1)}\right) + \dots\right]\Y(t_)
.}{1}

Each summand behaves, as is easily seen, relativistically correctly. The integrations could be written in the form
\uequ{
\intX{\Omega_1}{d\omega_1}\intX{\Omega_2}{d\omega_2}\intX{\Omega_3}{d\omega_3}\dots,\\
(\Omega_1 > \Omega_2 > \Omega_3 > \dots)
}
etc, where the $>$ sign in $\Omega_1>\Omega_2$ etc means that the points in $\Omega_1$ should lie \textit{after} those in $\Omega_2$ in time.

Now for simplicity I assume that the interaction energy $H_1$ should \?{cause} only \textit{one} "elementary" transition (and its reverse). (E.g. pair production with simultaneous change in momentum of an already-present particle.) If the momentum and energy change associated with this elementary process are $\fG$ and $E$, then the associated element in $H_1$ contains the time and space factor $\exp{i(\fG\fr - Et)}$, and that associated with the reverse process contains the factor $\exp{-i(\fG\fr - Et)}$. No other dependence on $\fr$ or $t$ is present. (Equation (1) is incidentally a rather trivial transformation of the Dirac perturbation theory.)

Now if we want to calculate the interaction cross section from (1) of the transition from the state characterizef by $\Y(t_0)$ to a new state, then the function $\Y(t)$ must be \?{analyzed into various states and the components of $\Y(t)$ which belong to the desired new state must be studied}. In other words: it depends on the matrix element of the operator $\pi\exp{iH_1\,{d\omega}}$ which is associated with the transition from the initial state (i) to the final state (f). I now claim that this matrix element can in all generality be represented in the form
\uequ{
\int{d\tau}\intXY{t_0}{t}{dt}\,O_{if} = 
\exp{i\left[(\fG_i - \fG_f)\fr - (E_i - E_f)t\right]},
}
where $O_{if}$ is independent of $\fr$ and $t$ and behaves rlativistically like the matrix elements of $H_{(1)}$. This is self-evident for the spatial integration, since the momentum law applies at each individual step, and thus also in total. The assertion is not automatically correct with the time integration. Considering e.g. a process $i \to f$ which takes at least three elementary steps to complete (creation of three pairs in the above example), then there occur in the braces of (1) time-dependencies other than $\exp{-i(E_i-E_f)t}$. But, it can be seen that all these other contributions can be chalked up to the "\WTF{shaking action}{Schüttelwirkung}" (it depends on $t_0$) and with suitable assumptions (e.g. adiabatically switching on the interactions, or: large wavepackets which don't yet overlap at the time $t_0$) can be made arbitrarily small. Hence we obtain the following representation for the matrix element:
\nequ{
&\int{d\tau}\intXY{t_0}{t}{dt}\,O_{if}\exp{i\left[(\fg_i - \fg_f)\fr - (E_i-E_f)t\right]}\\
= & \sum\left(i\int{d\tau}\intXY{t_0}{t}{dt_1}H_{(1)}^{ik}\right)
\left(i\int{d\tau}\intXY{t_0}{t_1}{dt_2}H_{(1)}^{kl}\right)\dots
\left(i\int{d\tau}\intXY{t_0}{t_{n-1}}{dt_n}H_{(1)}^{rf}\right)\\
&\times \text{"shaking terms"}
}{2}
where the sum is over \textit{all} (even arbitrarily-complex) paths leading from the initial to the final state via elementary processes. (In the present wave quantization this sum always diverges, it only converges non-relativistically; but that shouldn't bother us now.)

The quantities $O_{if}$ are physically directly observable, since their squares determine the interaction cross section of the associated process. More specifically, it will give a matrix element $O_{if}$ for the transition process specified by the \WTF{elementary process}{Elementarakt}. For the elementary process I shall define
\nequ{
O_{if}^\text{El}\exp{i\left[(\fg_i-\fg_f)\fr - (E_i-E_f)t\right]} = iH_0^{if}.
}{3}

(Thus in the "1st" approximation $H_0^{if}=H_{(1)}^{if}$, in the "2nd" approximation $H_0^{if} = H_{(1)}^{if} - \sumX{l}\frac{H_{(1)}^{il} H_{(1)}^{lf}}{E_i - E_f}$, etc.)

The $H_0^{if}$ are thus likewise directly physically observable. Now I claim: for arbitrary processes $i\to f$, (2) and (3) imply
\nequ{
&\int{d\tau}\intXY{t_0}{t}{dt}\,O_{if}\exp{i\left[(\fg_i-\fg_f)\fr-(E_i-E_f)t\right]}\\
 = & \sum'\left(i\int{d\tau}\intXY{t_0}{t}{dt_1}H_{(0)}^{ik}\right)
 \left(i\int{d\tau}\intXY{t_0}{t_1}{dt_2}H_{(0)}^{kl}\right)\dots
 \left(i\int{d\tau}\intXY{t_0}{t_{n-1}}{dt_n}H_{(0)}^{rf}\right),
}{4}
where here the sum $\sum'$ is only over the (finitely many) paths which lead from the initial to the final state in the \textit{fewest} number of elementary steps. The proof is in the fact that when (3) is inserted into (4) the sum $\sum'$ is automatically expanded to the sum $\sum$ over \textit{all} paths. (At this point the matter would be more complicated if $H_{(1)}$ gave rise to several different elementary processes; hence the assumption of only one such process.)

Now comes the decisive ammendment to the present theory. Equation (4) still only contains relations between observable quantities (\?{here you recognize the old song and dance}); hence I would like to even maintain equation (4) where equations (1) through (3) fail, since they lead to divergences. Thus I would like to characterize the future theory by $H_0^{if}$, instead of by the Hamiltonian function, and regard (4) as strictly satisfied. The resulting mathematical scheme is, as is immediately seen, always relativistically invariant and convergent (since no infinite sums occur), and additionally it is much simpler than the present theory. However, I still don't know what the energy and momentum integrals look like in this theory, since there is no Hamiltonian function. It cetainly seems to me that this theory is as close to the "semi-classical" theory as is at all possible, in any case closer than the wave quantization of the present type (c.f. our correspondence before Christmas).

I am very curious about what you will think about this theory; I also have much more confidence in its physical significance than in the recent clumsy "subtraction" theory. -- I want to study the literature on cosmic radiation more and then write you.

Many warm greetings

Your W. Heisenberg

%wabjs lizzelteats
\letter{470}
\rcpt{Heisenberg}
\date{February 5, 1937}
\location{Zurich}
\tags{qed,interaction representation, perturbation theory}
\references{468}

Dear Heisenberg!

Many thanks for your letter of the 2nd, whose content has quite interested me. I am very \textit{sympathetic} to the tendency of your proposed amendments to the theory. \?{Detailed remarks follow immediately.}

1. As regards the energy-momentum law, \?{that the essence is expressed in the occurance of the famous resonance denominator $1/E_i-E_f$ in the wavefunction of the final state}, which means that only for $E_i \approx E_f$ is there a linear 

crease of the probability of the final state with the time, i.e. an observable process. \textit{For this reason}, the terms you call "shaking-action", which have a time dependence \textit{other than} $\exp{\frac{i}{\hbar}(E_i-E_f)}$, are indeed \textit{unimportant}. (The description in your letter of $O_{if}$ as being \textit{directly} observable does not seem to be exactly correct, since this only seems to apply in the case $E_i \neq E_f$. Yet it might be justified to maintain that $O_{if}$ is "in principal" observable even in the case that $E_i \neq E_f$, \?{since it is needed} in order to calculate the observable quantities of complicated processes by means of matrix multiplication.)

I am entirely certain that the future theory will contain no operators which are defined at a \textit{sharply} defined time \textit{point}. Hence even the energy and momentum law is not to be simply defined by such operators whose expectation values are defineable at every point in time and exactly the same. The formulation with the resonance denominator naturally expresses the weaker fact, that the energy is conserved using the time-average over the space-time region in which the particle is practically without interaction.

2. As concerns the \?{distinguishing} equation (4), I still have some questions. I would like to have a more general and more exact definition of the difference between "elementary" and "non-elementary" transitions. This is connected with the question of the \textit{finiteness of the expression (4)}. Since the sum $\sum'$ probably only stretches over finitely-many paths, but for \textit{each} of these paths there is a summation over the variables $k,l,\dots$ of the intermediate states, which are \textit{de facto} integrations over the particle momentum spaces. \textit{In praxi} it was probably in fact always true that these sums resp. integrals are finite (convergent) (c.f. e.g. Euler-Kockel etc). But how far can it be \textit{generally} proven resp. seen that this \textit{must} be the case?

Regarding the perturbation theory\footnote{C.f. also my Handbuch article, specifically p. 159. -- the \textit{second} term of the [...] in equation $(241_2)$ seems to be of the "shaking" type (there I have put $t_0=0$).} I would like to bring to your attention a paper by \textit{St\"uckelberg} (Annalen der Physic \textbf{21}, 367, 1934). The paper is not very well-written, but the idea behind it (which goes back to \textit{Wentzel}) seems quite reasonable; it consists in making the relativistic invariance manifest by removing space and time from the theory entirely and directly examining the coefficients of the \textit{four}-dimensional Fourier expansion of the wavefunction. That there is a special connection between the wavenumber and the frequency of the form $\nu^2-k^2=m_0^2$ then only applies exactly in the force-free case.

So, many greetings.
Akways your W. Pauli

P.S. You have probably received the copy of Fierz's paper. Are you in agreement with the contents?

%ll hsympatico
\letter{471}
\from{Heisenberg}
\date{February 14, 1937}
\location{Leipzig}
\tags{qed,interaction representation}
\references{470}

Dear Pauli!

I have done much thinking about the modification proposal that I recently wrote about to you, in connection with your questions. Some of what I wrote in my last letter has proven to be incorrect; but nevertheless by trial and error I have obtained much confidence in this new attempt. It is certainly missing some essential ideas, \?{that would be needed for it to become} a closed mathematical scheme.

It has come to light that equation (4) in my last letter unfortunately applies at best in the present theory only in special cases. \?{The problem is} that the requirement that the final state should be reached in the smallest number of steps gives a restriction on the intermediate states. For this reason the sum $\sum'$ in equation (4) cannot be extended as-is to the general sum of equation (2).

Nevertheless I believe that in a theory which contains a universal length, equation (4) may be needed without coming into contradiction with the essentialnphysical requirements. Namely, first, the equation (4) in very close agreement with the "semiclassical" theory, so that from the correspondence principle side I see no difficulties. Second, one could immediately imagine the following difficulty: one calculates
\uequ{
\Y(t_1) = \left(1+\intXY{t_0}{t_1}{dt}\,{d\tau}\,H_0+\dots\right)\Y(t_0)\\
\Y(t_2) = \left(1+\intXY{t_1}{t_2}H_0\,{dt}\,{d\tau}+\dots\right)\Y(t_1).
}
That automatically gives the same as
\uequ{
\Y(t_2) = \left(1+\intXY{t_0}{t_2}H_0\,{dt}\,{d\tau}+\dots\right)\Y(t_0),
}
precisely because some terms are left out out of the sum in equation (4). Namely, one can also possibly come to a certain final state via a detour which leads \?{over} intermediate states which are possible with energy and momentum conservation. In this case there is an additional part in the calculation $\Y(t_0)\to \Y(t_1) \to \Y(t_2)$ which is missing in the direct path $\Y(t_0)\to\Y(t_2)$. But, it seems to me that this part remains of the order of magnitude of the shaking-action, and thus doesn't disturb a theory of the type we are looking for here. Thus it seems to me that by explicitly ignoring the shaking action terms, a scheme with equation (4) is physically possible and so far contradiction-free.

Of course a connection from this requirement (4) to the size of the rest-mass and to the question of stationary states still must be found. I still don't have such a connection, only some Ansatzes, and don't know if anything will come of them. When I know more, I'll write you again.

I've received the accompanying \?{copy of a paper} from Heitler. Please send it back when you get a chance.

Many greetings!

Your W. Heisenberg

%cthhemical.

\end{document}
