\letter{463}
\from{Heisenberg}
\date{January 16, 1936}
\location{Leipzig}
\tags{showers,qed,interaction representation}

\nc{\Hbar}{\overline{H}}
\nc{\fg}{\mathfrak{g}}
\nc{\fr}{\mathfrak{r}}

Dear Pauli!

You have probably now received Barn\'othy's treatise, I have put off sending it for so long since I wanted to wait for a letter from Barn\'othy; but there's nothing much new in it. Incidentally, after the recent results from Anderson on the radiation of electrons (Bethe-Heitler), I would like to believe that for electrons, real shower formation is relatively rare (c.f. also Oppenheimer's note in the last issue of the Physical Review), \sskip{shower formation...}.

Now I'd like to write extenstively on our main problem, the quantization of waves. The following reflections \WTF{refer to}{beziehen sich} to the "model":

\nequ{
L = -\sumX{\nu}|\pddX{\Y}{X_\nu}|^2 - f(\Y^*\Y)^3, \quad
H = \overset{H_0}{\overbrace{\pi^*\pi + \pddX{\Y}{X_k}\pddX{\Y}{X_k}}} +
\overset{H_1}{\overbrace{f(\Y^*\Y)^3}}
}{1}
however they are not bound to this specific form (units: $c=1$, $\hbar=1$). First I want to formulate the Lorentz invariance of wave quantization rather more directly than before. Following your work with Weisskopf, I put
\nequ{
\Y = \sumX{k}\frac{-i}{\sqrt{2Vk}}(-a_k + b^*_k)\exp{i(kr)};\quad
\pi = \sum\sqrt{\frac{k}{2V}}(a_k^* + b_k)\exp{-ikr}.
}{2}
The Schr\"odinger functional should depend on the $N_k = a_k^*a_k$, $M_k = b_k^* b_k$.
\nequ{
\Phi(N_1,N_2\dots;M_1,Me2,\dots;t);\quad
i\pddX{\Phi}{t} = \Hbar.\quad (\Hbar = \int H{dV})
}{3}
Now I put
\nequ{
\Phi(N_1\dots;M_1\dots;t) = \exp{-i\sumX{k}(N_k + M_k)kt}\y(N_1\dots; M_1\dots; t),
}{4}
and accordingly
\nequ{
a_k^* = N_k^{1/2}\Delta_k^-\exp{ikt};\quad
a_k   = \Delta_k^+N_k^{1/2}\exp{-ikt}.
}{5}
Then for $\y$ the new Schr\"odinger equation applies:
\nequ{
i\pddX{\y}{t} = \Hbar_1 \y.
}{6}
Without interactions, $\y$ is of course constant in time. Further,
\nequ{
H_1 = &f\left[\sumX{k,l}\inv{\sqrt{2Vk}}\left(-\Delta^+_k N_k^{1/2}\exp{i(kr-kt)} + 
M_k^{1/2} \Delta_k^-\exp{i(kr+kt)}\right)\right.\\
&\left.\inv{\sqrt{2Vl}}\left(-N_l^{1/2}\Delta^-_l \exp{-i(lr-lt)} + 
\Delta_l^+M_l^{1/2}\exp{-i(lr+lt)}\right)
\right]^3.
}{7}
This is, as is easily seen, relativistically invariant: the $N$ and $M$ are invariant, \!{they only obtain different denominators under a Lorentz transformation} $N_{k(k)}\to N_{k'(k')}$ (where $k'=(k-\beta k)/\sqrt{1-\beta^2}$, etc); likewise the operators $\Delta$. The expressions $\sqrt{Vk}$ are also relativistically-invariant, since $V$ essentially denotes the number of stationary states per interval $dk_x dk_y dk_z$. It can then be easily found that $Vk=V'k'$. And the rest is likewise invariant, so thus also the operator $H_1$. It should be noted however that $H_1$ explicitly contains the time. Now, according to (6)
\nequ{
\y(t+{dt}) &= (1-i\Hbar_1\,{dt})\y(t) = \exp{i[\int H_1\,{dV}]dt}\\
\y(t+2{dt})&= (1-i\Hbar_1(t+{dt}){dt})(1-i\Hbar_1(t)\,{dt})\y(t)\\
 &= \exp{-i\,{dt}\int H_1(t+{dt}){dV}}\exp{-i\,{dt}\int H_1(t){dV}}\y(t)
\quad \text{etc.}
}{8}

It should again be noted that the $e$-functions might not be \?{contracted} do \textit{one} $e$-function, since the exponents, which have different $t$s, do not commute. The factors associated with later times must always stand to the left of the earlier ones. Hence, one might naturally write
\uequ{
\exp{-i{\,dt}\int H_1\,{dV}} = \prodX{V}\exp{-i\,{dt}\,H_1\,{dV}}
}
where the product refers to all space elements ${dV}$. In general equation (8) can always be written
\nequ{
\y(t_1) = \prodX{\Omega}\left(\exp{-iH_1\,{dt}\,{dx}\,{dy}\,{dz}}\right)
\y(t_0) = \prodX{\Omega}\left(\exp{-iH_1\,{d\omega}}\right)\y(t_0),
}{9}
where ${d\omega}$ denotes the four-dimensional volume element, and $\Omega$ is the four-dimensional volume enclosed between the surfaces $t=t_0$ and $t=t_1$. Equation (9) now makes the relativistic invariance of the whole scheme immediately evident. Equation (9) can also be expanded \?{so that the time point $t_1$ is measured in in the moving system ($t'=t_1$)}. Then (9) still applies and $\Omega$ denotes the volume between $t=t_0$ and $t'=t_1$.

TODO--FIGURE (x/t axis; shading $\Omega$ between $t=const$ and $t'=kx$, with $t=t_0$ and $t'=t_1$ labelled)

So far, everything is just rearranging the previous quantum theory of waves.

Now, maintaining equation (9), I would like to ammend expression (7) so that equation (9) leads to convergent results. One possibility seems to be the following: it is perhaps convenient for now, in order to get the most symmetrical formulae possible, to put $\fg=-k$, $g=k$,
\uequ{
M_k^{1/2}\Delta^-_k \exp{i(kr+kt)} = 
M_g^{1/2}\Delta_g^- \exp{-i(\fg r - gt)}.
}
Then the expression (7) can be split into individual summands which each contain a factor of the form $\exp{i\sumXY{\nu=1}{4}(k_\nu^{(1)}+k_\nu^{(2)}+\dots+k_\nu^{(6)},\fr_\nu})$. $k_\nu$, $\fr_\nu$ are the four-vectors $\fr_\nu=(x\,y\,z, it)$, $k_\nu=(k_x,k_y,k_z,it)$; in some terms there is a $\fg$ instead of a $k$. Now I replace this factor by the new one:
\nequ{
\exp{i\sumX{\nu}(k_\nu^{(1)}+\dots+k_\nu^{(6)}, \fr^\nu) 
  + \alpha f\sumXY{\nu=1}{4}(k_\nu^{(1)} + \dots + k_\nu^{(6)})^2},
}{10}
and further I put all terms with stars $(a^*,b^*)$ to the \textit{left} of those without stars.

That is essentially the old (hideous) \WTF{subtraction procedure}{Abschneideverfahren} of Born, Wataghin, etc ($\alpha$ is some number). But I claim that here it leads to a \WTF{self-consistent}{in sich geschlossenen} contradiction-free theory. Initially the relativistic invariance is no longer trivial, since it must first be proven that the new $H_1$ commutes at different \textit{spatial} points. But it seems to be that this is the case (could you check this point?). Then one must \WTF{check}{nachschauen} what results arise for the self-energy etc. For this purpose one can go back from the equations (7)+(10) to the operator $H$ and to equation (3). It is then found via a perturbation calculation that the terms increase at first, but they are all finite\footnote{I've meanwhile worked more on this. It now seems that for $k\gg\inv{\sqrt{\alpha f}}$ the perturbation calculation converges well if at the same time $k\ll\alpha^{3/2}/\sqrt{f}$. This special case can be realized when $\alpha\gg 1$. Nevertheless I want to think this over some more.}, and I have the impression that indeed $H_1$ cannot be taken as a small perturbation, \?{but that otherwise the procedure converges}. I still have not further examined the possibility (10) and would like to know what you think about it.

Personally I believe something like the following: I think the specific subtraction procedure (10) is dumb, since it is entirely arbitrary. Also I can hardly imagine that arbitrarily-many types of convergent theories can be made, and I would rather introduce the universal length through the factor of $H_1$ than through a subtraction prescription. However, it seems reasonable in the proposal that the idea of a "wavefunction at a definite position" is given up and only the idea of "particles with definite momenta" is introduced; since the momentum can always be measurd exactly. Thus I would like to provisionally claim: even the future exact theory can be represented by an operator $H_1$, which goes over into one of the type (7) for small momenta and which is connected with $\y$ by equations (6) and (9).

But first we must probably clarify the main question: whether a relativistic and convergent theory is at all possible according to this scheme. So what do you think about it?

Many warm greetings, also to your wife and to the institute.

Your W. Heisenberg

 
%nicht aufs Sofa