\letter{464}
\rcpt{Heisenberg}
\date{January 19, 1937}
\location{Zurich}
\tags{qed,interaction representation}
\references{463}

Dear Heisenberg!

First, I just want to provisionally answer your last letter (of the 16th).

\textit{1. Cosmic radiation}

\sskip{...}

\textit{2. Quantization of waves}

I find your equation (9):
\uequ{
\Y(t_1) = \prodX{\Omega}(\exp{-iH_1\,{d\omega}})\y(t_0)
}
to be quite lovely. I will write more on the factor $\exp{\alpha f\sumX{\nu=1}{4}(k_\nu^1+\dots+k_\nu^{(6)})^2}$ later; namely, I want to read Wataghin's paper again before getting into the question of the commutability of the new $H$ at different spatial points\footnote{P.S. it seems to be permitted.} and the question of convergence. It seems fishy however. For its part, I have closely read Wataghin's paper and believe that it must be rejected, since it is \textit{not gauge-invariant} in an electromagnetic field. But if Wataghin's $P_\nu$ (in the subtraction factor) is replaced by $\pi_\nu = p_\nu + \frac{a}{c}\y_\nu$, then because of the non-commutability of the $\pi$'s with one another, great difficulties arise. The Lorentz invariance was in order with Wataghin, \?{but I only want to see whether it is also included in your additional term} $f(\Y^*\Y)^3$. It seems that this is probably the case. But the convergence questions could be awkward\footnote{Does the fact that $\sumX{\nu}(k_\nu^{(1)}+\dots+k_\nu^{(6)})^2$ vanishes for parallel $k_\nu^{(1)},k_\nu^{(2)},\dots,k_\nu^{(6)}$ hurt anything?}.

I do not agre \textit{at all} with the physical intepretation that the concept "particles with definite momenta" is physically better than the concept "wavefunctions with a definite position". Since the problem with the first concept lies in the fact that the momentum of the particle is needed \textit{at a totally definite time $t$} (while a very accurate measurememt of momentum can never be carried out in an arbitrarily-short time). Only in the force-free case, where the momentum is constant in time, does this cause no problems; but as soon as there are interaction terms, the problem with the $\y(N_k,M_k,t)$ emerges exactly as with $\y(x,t)$.

Thus I would like to put off addressing your specific questions connected with the particular subtraction procedure. -- But in conclusion, a certain critique of your remark in your last letter before Christmas, whether it is certain that wavefunctions which satisfy a differential equation would be present at all in a future theory (the opposite is the case in e.g. the lattice world.) But it is hard to see how Lorentz invariance should emerge if that is given up. \?{The main difficulty lies in} bringing Lorentz invariance and the existence of a universal length into harmony.

%nicht aufs SOPA