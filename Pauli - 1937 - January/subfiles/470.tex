\letter{470}
\rcpt{Heisenberg}
\date{February 5, 1937}
\location{Zurich}
\tags{qed,interaction representation, perturbation theory}
\references{468}

Dear Heisenberg!

Many thanks for your letter of the 2nd, whose content has quite interested me. I am very \textit{sympathetic} to the tendency of your proposed amendments to the theory. \?{Detailed remarks follow immediately.}

1. As regards the energy-momentum law, \?{that the essence is expressed in the occurance of the famous resonance denominator $1/E_i-E_f$ in the wavefunction of the final state}, which means that only for $E_i \approx E_f$ is there a linear 

crease of the probability of the final state with the time, i.e. an observable process. \textit{For this reason}, the terms you call "shaking-action", which have a time dependence \textit{other than} $\exp{\frac{i}{\hbar}(E_i-E_f)}$, are indeed \textit{unimportant}. (The description in your letter of $O_{if}$ as being \textit{directly} observable does not seem to be exactly correct, since this only seems to apply in the case $E_i \neq E_f$. Yet it might be justified to maintain that $O_{if}$ is "in principal" observable even in the case that $E_i \neq E_f$, \?{since it is needed} in order to calculate the observable quantities of complicated processes by means of matrix multiplication.)

I am entirely certain that the future theory will contain no operators which are defined at a \textit{sharply} defined time \textit{point}. Hence even the energy and momentum law is not to be simply defined by such operators whose expectation values are defineable at every point in time and exactly the same. The formulation with the resonance denominator naturally expresses the weaker fact, that the energy is conserved using the time-average over the space-time region in which the particle is practically without interaction.

2. As concerns the \?{distinguishing} equation (4), I still have some questions. I would like to have a more general and more exact definition of the difference between "elementary" and "non-elementary" transitions. This is connected with the question of the \textit{finiteness of the expression (4)}. Since the sum $\sum'$ probably only stretches over finitely-many paths, but for \textit{each} of these paths there is a summation over the variables $k,l,\dots$ of the intermediate states, which are \textit{de facto} integrations over the particle momentum spaces. \textit{In praxi} it was probably in fact always true that these sums resp. integrals are finite (convergent) (c.f. e.g. Euler-Kockel etc). But how far can it be \textit{generally} proven resp. seen that this \textit{must} be the case?

Regarding the perturbation theory\footnote{C.f. also my Handbuch article, specifically p. 159. -- the \textit{second} term of the [...] in equation $(241_2)$ seems to be of the "shaking" type (there I have put $t_0=0$).} I would like to bring to your attention a paper by \textit{St\"uckelberg} (Annalen der Physic \textbf{21}, 367, 1934). The paper is not very well-written, but the idea behind it (which goes back to \textit{Wentzel}) seems quite reasonable; it consists in making the relativistic invariance manifest by removing space and time from the theory entirely and directly examining the coefficients of the \textit{four}-dimensional Fourier expansion of the wavefunction. That there is a special connection between the wavenumber and the frequency of the form $\nu^2-k^2=m_0^2$ then only applies exactly in the force-free case.

So, many greetings.
Akways your W. Pauli

P.S. You have probably received the copy of Fierz's paper. Are you in agreement with the contents?

%ll hsympatico