\letter{471}
\from{Heisenberg}
\date{February 14, 1937}
\location{Leipzig}
\tags{qed,interaction representation}
\references{470}

Dear Pauli!

I have done much thinking about the modification proposal that I recently wrote about to you, in connection with your questions. Some of what I wrote in my last letter has proven to be incorrect; but nevertheless by trial and error I have obtained much confidence in this new attempt. It is certainly missing some essential ideas, \?{that would be needed for it to become} a closed mathematical scheme.

It has come to light that equation (4) in my last letter unfortunately applies at best in the present theory only in special cases. \?{The problem is} that the requirement that the final state should be reached in the smallest number of steps gives a restriction on the intermediate states. For this reason the sum $\sum'$ in equation (4) cannot be extended as-is to the general sum of equation (2).

Nevertheless I believe that in a theory which contains a universal length, equation (4) may be needed without coming into contradiction with the essentialnphysical requirements. Namely, first, the equation (4) in very close agreement with the "semiclassical" theory, so that from the correspondence principle side I see no difficulties. Second, one could immediately imagine the following difficulty: one calculates
\uequ{
\Y(t_1) = \left(1+\intXY{t_0}{t_1}{dt}\,{d\tau}\,H_0+\dots\right)\Y(t_0)\\
\Y(t_2) = \left(1+\intXY{t_1}{t_2}H_0\,{dt}\,{d\tau}+\dots\right)\Y(t_1).
}
That automatically gives the same as
\uequ{
\Y(t_2) = \left(1+\intXY{t_0}{t_2}H_0\,{dt}\,{d\tau}+\dots\right)\Y(t_0),
}
precisely because some terms are left out out of the sum in equation (4). Namely, one can also possibly come to a certain final state via a detour which leads \?{over} intermediate states which are possible with energy and momentum conservation. In this case there is an additional part in the calculation $\Y(t_0)\to \Y(t_1) \to \Y(t_2)$ which is missing in the direct path $\Y(t_0)\to\Y(t_2)$. But, it seems to me that this part remains of the order of magnitude of the shaking-action, and thus doesn't disturb a theory of the type we are looking for here. Thus it seems to me that by explicitly ignoring the shaking action terms, a scheme with equation (4) is physically possible and so far contradiction-free.

Of course a connection from this requirement (4) to the size of the rest-mass and to the question of stationary states still must be found. I still don't have such a connection, only some Ansatzes, and don't know if anything will come of them. When I know more, I'll write you again.

I've received the accompanying \?{copy of a paper} from Heitler. Please send it back when you get a chance.

Many greetings!

Your W. Heisenberg

%cthhemical.