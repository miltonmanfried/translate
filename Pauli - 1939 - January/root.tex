\documentclass{article}
\usepackage[utf8]{inputenc}
\renewcommand*\rmdefault{ppl}
\usepackage[utf8]{inputenc}
\usepackage{amsmath}
\usepackage{graphicx}
\usepackage{enumitem}
\usepackage{amssymb}
\usepackage{marginnote}
\newcommand{\nf}[2]{
\newcommand{#1}[1]{#2}
}
\newcommand{\nff}[2]{
\newcommand{#1}[2]{#2}
}
\newcommand{\rf}[2]{
\renewcommand{#1}[1]{#2}
}
\newcommand{\rff}[2]{
\renewcommand{#1}[2]{#2}
}

\newcommand{\nc}[2]{
  \newcommand{#1}{#2}
}
\newcommand{\rc}[2]{
  \renewcommand{#1}{#2}
}

\nff{\WTF}{#1 (\textit{#2})}

\nf{\translator}{\footnote{\textbf{Translator note:}#1}}
\nc{\sic}{{}^\text{(\textit{sic})}}

\newcommand{\nequ}[2]{
\begin{align*}
#1
\tag{#2}
\end{align*}
}

\newcommand{\uequ}[1]{
\begin{align*}
#1
\end{align*}
}

\nf{\sskip}{...\{#1\}...}
\nff{\iffy}{#2}
\nf{\?}{#1}
\nf{\tags}{#1}

\nf{\limX}{\underset{#1}{\lim}}
\newcommand{\sumXY}[2]{\underset{#1}{\overset{#2}{\sum}}}
\newcommand{\sumX}[1]{\underset{#1}{\sum}}
\nf{\prodX}{\underset{#1}{\prod}}
\nff{\prodXY}{\underset{#1}{\overset{#2}{\prod}}}
\nf{\intX}{\underset{#1}{\int}}
\nff{\intXY}{\underset{#1}{\overset{#2}{\int}}}

\nc{\fluc}{\overline{\delta_s^2}}

\rf{\exp}{e^{#1}}

\nc{\grad}{\operatorfont{grad}}
\rc{\div}{\operatorfont{div}}

\nf{\pddt}{\frac{\partial{#1}}{\partial t}}
\nf{\ddt}{\frac{d{#1}}{dt}}

\nf{\inv}{\frac{1}{#1}}
\nf{\Nth}{{#1}^\text{th}}
\nff{\pddX}{\frac{\partial{#1}}{\partial{#2}}}
\nf{\rot}{\operatorfont{rot}{#1}}
\nf{\spur}{\operatorfont{spur\,}{#1}}

\nc{\lap}{\Delta}
\nc{\e}{\varepsilon}
\nc{\R}{\mathfrak{r}}

\nff{\Elt}{\operatorfont{#1}_{#2}}

\nff{\MF}{\nc{#1}{\mathfrak{#2}}}

\nc{\Y}{\psi}
\nc{\y}{\varphi}

\nf{\from}{From: #1}
\nf{\rcpt}{To: #1}
\rf{\date}{Date: #1}
\nf{\letter}{\section{Letter #1}}
\nf{\location}{}
\nf{\references}{}


\title{Pauli - 1939 - January}

\begin{document}

\letter{541}
\from{Kemmer}
\date{April 6, 1939}
\location{London}
\tags{beta-gymnastics spinor-calculus}

\nc{\D}{\partial}

Dear Herr Professor!

What do you think of the following formalism? There is \textit{nothing} fundamentally new, but it clarifies my ideas on many connections and is very convenient for many calculations. In the typical formulations of Bose theories, the \textit{field} side is emphasized rather than the \textit{particle} side. The following shows that this need \textit{not} be regarded as the essential nature of things.

I consider the wave equation
\nequ{
\partial_\mu \beta_\mu \Y + \kappa\Y = 0,\quad
\left(\partial_\mu = \pddX{}{x_\mu}, \kappa = \frac{mc}{\hbar}\right),
}{1}
where the $\beta$-operators satisfy the commutation relations:
\nequ{
\beta_\mu\beta_\nu\beta_\rho = \beta_\rho\beta_\nu\beta_\mu = 
\delta_{\mu\nu}\beta_\rho + \delta_{\rho\nu}\beta_{\mu}.
}{2}

If we put $\Y^+ = i\Y^*(2\beta^2_4 - 1)$, then
\nequ{
\partial_\mu\Y^u\beta_\mu - \kappa\Y^+ = 0.
}
There are no boundary conditions, though
\uequ{
\partial_\mu\partial_\rho\beta_\mu\beta_\nu\beta_\rho\Y =
-\kappa\partial_\mu\beta_\mu\beta_\nu\Y,
}
thus
\nequ{
\partial_\nu\partial_\mu\beta_\mu\Y = -\kappa\partial_\mu\beta_\mu\beta_\nu\Y,\\
\partial_\nu\Y = \partial_\mu\beta_\mu\beta_\nu\Y,
}{4}
and analogously
\nequ{
\partial_\nu\Y^+ = \partial_\mu\Y^+\beta_\nu\beta_\mu.
}{5}
It follows immediately from (1) and (4) that
\nequ{
\partial_\nu\partial_\nu\Y = \kappa^2\Y
}{4'}
and likewise
\nequ{
\partial_\nu\partial_\nu\Y^+ = {\kappa'}^2\Y^+.
}{5'}
The behavior under Lorentz transformation will later \?{fall into place}; (1) and (4) are \textit{individually} invariant. Incidentally (1) and (2) was initially given by Duffin, Physical Review, December 15th.

(1) and (3) immediately imply
\nequ{
\partial_\mu s_\mu = 0,\quad s_\mu = \Y^+\beta_\mu\Y.
}{6}
With this, $\inv{i}\Y^+\beta_4\Y$ is the particle density, and \?{we seek to assign operators $\omega$ to physical quantities such that}
\nequ{
\overline{\omega} = \inv{i}\int{dV}\Y^+\beta_4\omega\Y
}{7}
are the associated expectation values which are derived field-theoretically. (If $\omega$ does not commute with $\beta_4$, (7) \?{must be} symmetrized.) Now $\beta_4$ has no inverse, so the possibility of the representation (7) is not trivial. However, it will be proven that, for the most important quantities which have an intuitive interpretation in particle form, a representation (7) can be found.

For instance, we seek to define the energy-momentum vector as
\uequ{
\overline{p}_\mu = \inv{i}\int{dV}\Y^+\beta_4\inv{i}\D_\mu\Y,
}
resp. the energy-momentum tensor as
\nequ{
T_{\mu\nu} = \inv{2i}\left(\Y^+\beta_\nu\D_\mu\Y - \D_\mu\Y^+\beta_\nu\Y\right).
}{8}
($\hbar$ and $e$ are always left out for brevity.)

The equation
\nequ{
\pddX{T_{\mu\nu}}{x_\nu} = 0
}{9}
is easily proven, so the definition (8) is possible. Then, symmetrizing analogously to Tetrode,
\uequ{
T_{\mu\nu} = \inv{2i}\left(\Y^+\beta_\nu\delta_{\rho\mu}\D_\rho\Y - 
\D_\rho\Y\Y^+\beta_\nu\delta_{\rho\mu}\Y\right),
}
according to (2),
\uequ{
 = \inv{2i}&\left\{\Y^+\left(\beta_\nu\beta_\mu\beta_\rho + \beta_\rho\beta_\mu\beta_\nu\right)\D_\rho\Y -
\D_\rho\Y^+ \left(\beta_\rho\beta_\mu\beta_\nu + \beta_\nu\beta_\mu\beta_\rho\right)\Y\right.\\
&\left. - \delta_{\mu\nu}\Y^+\beta_\rho\D_\rho\Y + \delta_{\mu\nu}\D_\rho\Y\beta_\rho\Y\right\};
}
thus 

cause of (1)
\uequ{
 = -\frac{x}{i}\left\{\Y^+(\beta_{\nu\mu}+\beta_{\mu\nu})\Y - \delta_{\mu\nu}\Y^+\Y\right\}
 - \inv{2i}\frac{d}{dx_p}\left\{\Y^+(\beta_\nu\beta_\mu\beta_\rho - 
 \beta_\rho\beta_\mu\beta_\nu)\right\}.
}
With this, $\pddX{\theta_{\mu\nu}}{x_\nu} = 0$, where
\nequ{
\theta_{\mu\nu} = -\frac{x}{i}\left[\Y^+(\beta_\nu\beta_\mu + \beta_\mu\beta_\nu)\Y
 - \delta_{\mu\nu}\Y^+\Y\right],
}{10}
and $\theta_{\mu\nu}$ is symmetrical.

But in accordance with the definition of $\Y^+$
\nequ{
\theta_{44} = -x\Y^*\Y,
}{11}
so the energy density in this representation is positive-definite.

But, $\theta_{\mu\nu}$ is \textit{not} $\inv{2}(T_{\mu\nu} + T_{\nu\mu})$!

However, the relation $\pddX{}{x_\nu}T_{\nu\mu}=0$ can also be derived in the field-free case so that $\theta'_{\mu\nu}=\inv{2}(T_{\mu\nu}+T_{\nu\mu})$ enters as the energy-momentum tensor, but then the energy density $-T_{44}$ is not necessarily positive!

Now if the angular momentum is defined as
\uequ{
P_{ik} = \inv{i}\int{dV}(x_i\theta_{k4} - x_k\theta_{i4}),
}
it is shown with a little "$\beta$-gymnastics" that
\nequ{
P_{ik} = +\inv{i}\int\Y^+\beta_4\left(x_i\frac{\D_k}{i} - k_k\frac{\D_i}{i}\right)\Y\,{dV}
 - \int\Y^+\beta_4\left(\beta_i\beta_k - \beta_k\beta_i\right)\Y\,{dV}.
}{12}
Hence the operator $-i(\beta_i\beta_k-\beta_k\beta_i)$ (commutable with $\beta_4$) defines the \textit{spin}!

Now one has the freedom to use $\theta'_{\mu\nu}$ as well as $\theta_{\mu\nu}$ in the definition of the angular momentum. In this way another expression is found:
\uequ{
{P''}_{ik} &= \inv{i}\int\Y^+\beta_4\left(x_i\frac{\D_k}{i} - x_k\frac{\D_i}{i}\right)\Y\,{dV}
 + \inv{2}\int\Y^+\beta_i\beta_4\beta_k\Y\,{dV}\\
&\left(P'_{ik} = \frac{P_{ik} + {P''}_{ik}}{2}\right).
}

But this latter "spin" supplies no "operator" according to our definition.

Further, the Gordon \WTF{current decomposition}{Stromaufspaltung} can be carried out:
\uequ{
s_\mu = \Y^+\beta_\mu\Y &= \inv{2\kappa}\left[\D_\nu\Y^+\beta_\nu\beta_\mu\Y
 - \Y^+ \beta_\mu\beta_\nu\D_\nu\Y\right]\\
&= \inv{2\kappa}\left[\frac{d}{dx_\nu}\left\{\Y^+(\beta_\nu\beta_\mu - \beta_\mu\beta_\nu)\Y\right\}
- \Y^+\beta_\nu\beta_\mu\D_\nu\Y + \D_\nu\Y^+\beta_\mu\beta_\nu\Y\right],
}
and thus according to (4) and (5)
\uequ{
s_\mu = \inv{2\kappa}\left[\D_\mu\Y^+\Y - \Y^+\D_\mu\Y\right] +
\inv{2\kappa}\frac{d}{dx_\nu}\Y^+(\beta_\nu\beta\mu - \beta_\mu\beta_\nu).
}
Hence for the magnetic moment:
\uequ{
M_{ik} = \inv{2\kappa}\left[
\inv{i}\int\Y^+\left(x_i\frac{\D_k}{i} - x_k\frac{\D_i}{i}\right)\Y\,{dV}
- \int\Y^+(\beta_i\beta_k - \beta_k\beta_i)\Y\,{dV}\right];
}
thus the anomolous factor of $\inv{2}$ in the spin term disappears in this theory, otherwise everything is the same as with Dirac. \textit{But the magnetic moment} has no \textit{operator in the particle form!} \?{It is probably also intuitive that the magnetic moment is not as immediately connected with the particle probability density as the mechanical moment}. \?{So much for} the field-free case; if now an electromagmetic field is introduced, then it is \textit{free of contradictions} and \textit{invariant} to introduce it in (1) by $\D_\mu \to \D_\mu^- = \D_\mu - iA_\mu$ and in (3) by $\D_\mu \to \D^+\mu = \D_\mu + iA_\mu$; now instead of (4) we only then have
\nequ{
\D_\nu^-\Y = \D_\mu^-\beta_\mu\beta_\nu\Y + 
\frac{i}{2\kappa}F_{\mu\rho}(\beta_\rho\beta_\nu\beta_\mu - \delta_{\rho\nu}\beta_\mu)\Y,
}{4a}
and instead of (5)
\nequ{
\D_\nu^+\Y^+ = \D_\mu^+\Y^+\beta_\nu\beta_\mu + 
\frac{i}{2\kappa}F_{\mu\rho}(\beta_\mu\beta_\nu\beta_\rho - \beta_\mu\delta_{\nu\rho}),
}{5a}
and instead of (4') and (5')
\nequ{
\D_\nu^- \D_\nu^- \Y &= \kappa^2\Y + iF_{\mu\nu}\Y + 
\frac{i}{2\kappa}\D_\nu^-F_{\mu\rho}(\beta_\rho\beta_\nu\beta_\mu - \delta_{\rho\nu}\beta_\mu)\Y\\
\D_\nu^+ \D_\nu^+ \Y &= \kappa^2\Y - iF_{\mu\nu}\Y^+\beta_\nu\beta_\mu + 
\frac{i}{2\kappa}\D_\nu^+F_{\mu\rho}\Y^+(\beta_\mu\beta_\nu\beta_\rho - \beta_mu\delta_{\nu\rho})
}{4'a}
Thus there is a field interaction of the electric and magnetic moment as in the Dirac theory + a peculiar additional part.

Thus in the presence of fields $T_{\mu\nu}$ can be defined as before (only with $\D^+$ and $\D^-$) and $\theta_{\mu\nu}$ remains exactly the same; only $\theta'_{\mu\nu} = \inv{2}(T_{\mu\nu} + T_{\nu\mu})$ is no longer a permissable tensor, since $\pddX{\theta'_{\mu\nu}}{x_\nu} \neq F_{\mu\nu}s_\nu$, while for $T_{\mu\nu}$ and $\theta_{\mu\nu}$ the equation is true.

With this the uniqueness is again established, and we \textit{must} choose the form
\uequ{
P_{ik} = +\inv{i}\int\Y^+\beta_4\left(x_i\frac{\D_k^-}{i} + x_k\frac{\D_i^-}{i}\right)\Y\,{dV}
 - \int\Y^+\beta_4(\beta_i\beta_k - \beta_k\beta_i)\Y\,{dV}
}
for the angular momentum.

The Gordon current decomposition now becomes interesting, since it now reads
\uequ{
s_\mu &= \inv{2\kappa}[\D_\mu^+\Y^+\Y - \Y^+\D_\mu^-\Y] + 
\inv{2\kappa}\frac{d}{dx_\nu}[\Y^+(\beta_\nu\beta_\mu - \beta_\mu\beta_\nu)]\\
&+ \frac{i}{2\kappa^2}F_{\nu\rho}\Y^+(\beta_\rho\beta_\mu\beta_\nu)\Y.
}

The occurance of this last term is apparently connected with the additional term in (4'a, 5'a), \?{and} I don't see any simple interpretation for it. There is also an addendum for $\overline{M}_{ik}$:
\uequ{
M_{ik} = &\inv{2k i}\int\Y^+\left(x_i\frac{\D_k^-}{i} - x_k\frac{\D_i^-}{i}\right)\Y\,{dV}\\
 -& \inv{2k}\int\Y^+(\beta_i\beta_k - \beta_k\beta_i)\Y\,{dV}\\
 +& \inv{4\kappa^2}\int{dV}\,F_{\mu\nu}\Y^+(\beta_\nu\beta_k\beta_\mu x_i 
 - \beta_\nu\beta_i\beta_\mu x_k).
}
Now, what do you think of all this? I know the following representations of $\beta_i$: $\beta_i = \frac{\gamma_i+\gamma'_i}{2}$, where the $\gamma_i$ are Dirac matrices and $\gamma'_i$ are the same but acting on a different index. This representation is reducible (see my old Helvetica Physica Acta paper) and indeed it decomposes into a ten-rowed, a five-rowed and a trivial one-rowed irreducible representation. Inserting the ten-rowed representation into the above gives the Proca theory. The five-rowed (written linearly) \?{gives} the relativistic Schr\"odinger equation (in the latter case namely $\beta_\mu\beta_\nu\beta_\rho=0$, $\mu\neq\nu\neq\rho$) resp. the \?{reflected} theories, and indeed it is so since with a consistent definition of the reflection character \textit{before the reduction}, the Proca theory is connected with the \textit{pseudo}scalars, \?{like Møller has suggested}! (See also Belinfante, Nature from 2-3 months ago.)

It is easy to see that on second quantization the commutation relations can be written
\uequ{
[\Y_\alpha^+\,\beta_{4\alpha\beta}\Y_\gamma]_- = i\delta(x-x')(\beta_4)^2_{\beta\gamma},
}
or
\uequ{
[\Y_\alpha^+\,\beta_{4\beta\gamma}\Y_\gamma]_- = i\delta(x-x')(\beta_4)^2_{\alpha\beta},
}
and that the non-relativistic limiting case is derived from the relations
\uequ{
\beta_4\D_4\Y+\kappa\Y+\beta_k\D_k Y &= 0\\
i(\beta_4^2 - 1)\D_4\Y + \D_k\beta_k\beta_4\Y &= 0,
}
if $\beta_4$ is made diagonal. The "large components" are in two two representations naturally a vector resp. a scalar (in 3 dimensions).

This representation is essentially different from Dirac's with its Hamiltonian function, and indeed this distinction can be characterised so that Dirac starts from the $\alpha$-form and I start from the $\gamma$-form of the equation, which here cannot so immediately be transformed into one another. The $\gamma$-form has, as I've said, already been recently given by Duffin, but he has done nothing further with it, as he wrote me. But I find it a but funny that everything is so analogous, \?{and I enjoy the $\beta$-gymnastics}. The representation is also perhaps of pedagogical interest, since it gives an example of the operator calculus that can be written just as well in the tensor form, and hence can be better-understood than the Dirac equation by many. I have not been able to do anything more important in recent times, since I've been quite overloaded with classes.

Best greetings and a happy Easter to you, your wife and \textit{all the Zurichers},

Your devoted N. Kemmer

I beg you pass this letter along to Wentzel.
%thhe $eand eben
\letter{551}
\rcpt{Heisenberg}
\date{April 27, 1939}
\location{Zurich}
\tags{meson-theory}

Dear Heisenberg!

Many thanks for the letter and manuscript. I would first like to go deeper into the specific question of the \textit{scattering of mesotrons on protons}. We have been occupied with precisely this question since Bhabha was here about 10 days ago and gave us a detailed report on his classical calculations for mesotron scattering. There he has gotten rid of the (classical!) self-energy following Dirac (1938), so it is then possible to introduce an arbitrary constant for the rest mass of the proton. There he initially only took up the interactions proportional to $g_1$, but he has also -- which is \?{decisive} -- assumed that the mesotrons are neutral (\textit{real} field!). Then he naturally finds a mesotron cross section that vanishes with $M_\text{proton} \to \infty$. This is trivial in his case, since the quantum-mechanical perturbation calculation would supply the same. \{Namely: for $M\to 0$ the cross-section $\to 0$ emerges \textit{if} there are two paths from the initial state,

TODO: --(Meson)-->$p_\mu$ to $p_\mu$ --(Meson)-->

whose contributions are compensate one another by neglecting the recoil energy of the proton. This is the case for neutral mesons, since then the two intermediate states

TODO: --(Meson)-->(Neutron I)--(Meson)--> \textit{and} (Neutron II)

come into play (analogously to light scattering in electrodynamics). It is different with charged mesons; in the initial state

TODO: --(Meson+)-->($p_\mu$+) there is only \textit{one} intermediate state --(Meson+)-->(N)--(Meson+)-->

in the initial state

TODO: --(Meson-)-->($p_\mu$+) there is only the \textit{one} intermediate state (N) (without meson).

It is similar with transitions which are essentially connected with the flipping of the proton spin.\}

It is already correct that the corresponding semiclassical model for spin-protons ($g_2$) resp. charged mesons ($g_1$) permits the introduction of an arbitrary constant $K$ of dimension $\text{cm}^{-1}$ as a measure \?{moment of inertia} of the $\sigma$ resp. $\tau$ degree of freedom. But I believe that for now it can't go higher than first order in $m_\text{meson}\times c/\hbar$ (and certainly \textit{not} $M_\text{proton}\times c/\hbar$!). \textit{Since the corresponding part of the nuclear force, in particular the energy difference between the singlet- and triplet-ground states of the deuterons, must be decreased in your model just as the cross section of the mesotron-proton scattering}. (For the scattering proportional to $g_1^4$, your proposal would likewise depress all exchange forces between neutrons and protons.) The classical analogy to this energy difference is \?{just a frequency of the spin} of the heavy particles in the deuteron.

Thus it seems that only one assumption $K\approx \frac{m_\text{meson}\times c}{\hbar}$ can be discussed, which  tly altershe quantum-theoretical result 

by a factor of order 1. This factor can in turn -- at least for smaller meson energies -- be compensated by a change in the quantum-theoretical numerical value of $g_2$ with respect to the classical $g_2$ such that for mesotron energies $\ll m_\text{meson}\times c^2$ the quantum-mechanical and semiclassical values of the scattering cross-section coincide. So I ask myself whether a
\textit{in the quantum theory} the assumption $K=0$ is reasonable at all. That can't be proven, but your assumption $K\approx\frac{M_\text{proton}\times c}{\hbar}$ (with unchanged numerical value of $g_2$!) can certainly be empirically refuted by the nuclear forces.

If $K\neq 0$, I could raise the further question, \textit{why then do we not also need to introduce, for the electron in the electromagnetic field, a new \?{spin inertial-resistance} arising from the self-vector-potential $\vec{\phi}(0)$ of the electron?} I fear that would then come into conflict with experiment. Speculative types could even use such a modification of the theory to try to connect it with the empirical absence of polarization in electrons upon reflecting off of atoms as well as the empirical deviations of the $H$ fine structure from theory (see Williams, Physical Review). I would only conditionally claim: \textit{If one introduces for the proton in the meson field an additional inertial resistance $K\neq 0$ of the proton spin, then the same must be done for the electron in the electromagnetic field.}

In general, I would still object -- although I find the calculations on the mutual scattering of mesons in analogy to the "Euler-Kockel effect" beautiful and important -- that you rather over-state the importance of the dimensional argument. I believe, for the limits of applicability of the current quantum theory, it matters less whether the interaction energy contains a constant with the dimension of a length or a dimensionless constant, than rather the numerical magnitude of this coupling constant ($\frac{g_1^2}{\hbar c} \gg \frac{e^2}{\hbar c}$). Yet one still cannot know whether the divergences of the present theories are as directly connected with the values of the rest masses, as you assume.

Since it would be difficult to definitely prove anything, everyone will stick to his beliefs. The argument from Bhabha's note in Nature also seems to be incorrect, so far as it rests on the limit $M_\text{meson} \to 0$.

Now to your letter of the 23rd about the Solvay report. As much as it goes against my laziness to even write such a report, I nonetheless believe on factual grounds that I cannot reject on principle your proposal that I take over section 1. Thus I would like to make the counter-proposal that in section (1c) I \textit{don't} go into the \textit{interaction} (which should be reserved for section 2) and that we call the whole section 1 "Relativistic wave equations of \textit{force-free} particles and their quantization". I could then treat the connection of spin and statistics and (if you want) the gravitational quanta of spin 2. Further I could say something a bit less well-known about the de Broglie equations (especially \?{since} he is also writing a report on them) and stress that they neither describe partickes which are composed of two neutrinos nor photons, but rather -- if correctly interpreted -- a specific sort of mesotrons.

Regarding the work together with Bohr, I would rather propose that our manuscript be sent independently to Bohr and to Brussels. Since I have no desire to write a report which will then sit unread on Bohr's desk for an arbitrarily-long time. So please get in touch about the matter, on the one hand with the Solvay committee (Langevin) and on the other hand with Bohr. I could be ready by 7/1, but 6/1 is questionable.

Wentzel and I have attended Dirac's lectures on the subtraction-tricks in Paris. The quantum-theoretical part did not go substantially beyond an old paper of Wentzel's, was quite meager and a bit unconvincing. We could not share in his optimism in going further on this purely formal path.

Many gretings \?{from house to house}

Always your W. Pauli

% Sleachnitt
\letter{554}
\rcpt{Bhabha}
\date{May 4, 1939}
\location{Zurich}
\tags{meson-theory}

\nc{\vZ}{\vec{Z}}
\nc{\ve}{\vec{e}}
\nc{\vk}{\vec{k}}
\nc{\vs}{\vec{s}}
\nc{\va}{\vec{a}}
\nc{\vx}{\vec{x}}
\nc{\gbar}{\overline{g}}

Dear Herr Bhabha!

Meanwhile I have more closely thought over your Zurich lecture, specifically the scattering of mesons by protons. I am mostly working off of a manuscript from Heisenberg, which he tells me he has also sent to you. Meanwhile there has also been correspondence between me and Heisenberg in which some things have become clearer than they were in Heisenberg's manuscript.

\subsection*{1. Remarks on the scattering of neutral mesons by protons, interaction term $g_1$}

This is the case that you worked out in your semiclassical model with the result that in the limit $M_P \to \infty$ ($M_P$ = proton mass) the scattering vanishes. It is important to stress that for \textit{this} Hamiltonian function \textit{even the quantum theoretical perturbation theory would lead to the same result.} Namely here one has, as in electrodynamics, for the initial state

TODO: --(meson)--> ($p_\mu$) and the final state ($p_\mu$), ()--(meson)-->

\textit{two} intermediate states

TODO:

a) ()--(meson)--> ()--(neutron)-->meson

b) () (without meson)

By carrying out the perturbation calculation, it is easy to see that the contributions of the two intermediate states exactly cancel because of the opposite signs of the energy denominator in the limit $M_P \to \infty$. (For this, c.f. Electrodynamics, Dirac 1927, discussion of the term $(\vec{p}\vec{A})$.) Thus quantum-theoretically the scattering here becomes 0, exactly as in your semiclassical model in the limit $M_P \to \infty$. Hence we cannot draw any conclusions on the failure of quantum theory from your previous results.

It is different for \textit{charged} mesons. Since here for the initial state

TODO: --(meson+)-->($p_\mu$) and the final state ($p_\mu$) --(meson+)-->

there is \textit{only} one intermediate state (a); for the scattering of negatively-charged mesons there ia only the \textit{one} intermediate state (b). Hence scattering remains possible here in the limit $M_P \to \infty$.

It is similar in processes where the spin of the proton is flipped in the intermediate state interaction $g_2$.

\subsection*{2. On the scattering of neutral mesons with spin-flipping of the proton, interaction term $g_2$.}

For this case Heisenberg has now attempted to introduce a semiclassical model. The result is somewhat obscured by the factor $\eta$, which he has introduced (exclusively from laziness) in the equation (34). But it is not difficult to integrate equation (28) resp. (29) for a given $D(\vec{x})$ by consideration of the self-field for a monochromatic wave. As Heisenberg briefly informed me, the result with the \textit{Ansatz} (30) for incoming waves in the special case of $\vec{a}$ parallel to $\vec{s}_0$ is the following:\\
With
\uequ{
\vZ = \vZ_0 + \vZ_1, \quad \vs = \vs_0 + \ve\exp{-ik_0 \tau}; \quad
\ve \perp \vs_0 \quad \text{and } |\ve| \ll 1,\\
\vZ_0 = l\vs_0\int\frac{D(\vx')\exp{-\kappa r_{pp'}}}{4\pi r_{pp'}}{d^3x_{p'}};\quad
\vZ_1 = \ve l \exp{-ik_0\tau}\int\frac{D(\vx')\exp{ikr_{pp'}}}{4\pi r_{pp'}}{d^3x_{p'}}
}
the equation for $\ve$ finally follows ($\va$ from Heisenberg's equation (30) is the amplitide of the incoming wave $\vec{u}$)
\nequ{
ik_0 \ve = 2li[[\vk\va],\vs_0] + 2l(\gbar_0(-ik) - \gbar_0(\kappa)[\ve\vs_0],
}{1}
where $\gbar_0(-ik)$ and $g_0(\kappa)$ are given by
\nequ{
\gbar_0(-ik)=-\frac{2}{3}l\int D(\vx){d^3x}\,\Delta
\int\frac{D(\vx')\exp{ikr_{pp'}}}{4\pi r_{pp'}}{d^3x_{p'}}
}{1a}
\nequ{
\gbar_0(\kappa)=-\frac{2}{3}l\int D(\vx){d^3x}\,\Delta
\int\frac{D(\vx)\exp{\kappa r_{pp'}}}{4\pi r_{pp'}}{d^3x_{p'}}.
}{1b}
With the abbreviation
\uequ{
\gbar_0(-ik) - \gbar_0(\kappa) = l\Gamma(k_0),\quad (k_0^2 = \kappa^2 + k^2)
}
the interaction cross section becomes
\uequ{
Q=\frac{2}{3\pi}\frac{l^4 k^4}{k_0^2 + 4l^4|\Gamma(k_0)|^2}.
}
(1a) and (1b) give for $\Gamma(k_0)$
\nequ{
\Gamma(k_0) = +\frac{2}{3}\int D(\vx){d^3x}\int\frac{D(\vx'){d^3x'}}{4\pi r_{pp'}}
(k^2 \exp{ikr_{pp'}} + \kappa^2\exp{-\kappa r_{pp'}}).
}{2}
For very small $k_0$, $k$ is purely imaginary and $ik \approx -\kappa$ and $\Gamma(k_0)$ becomes proportional to $k_0^2$. An expansion of $\Gamma(k_0)$ in powers of $k_0^2$
\?{corresponds to} additional terms to the right side of the differential equation for $\dot{\vs} = 2l[g^\text{ext},\vs]$ proportional to $\ddot{\vs}, \ddddot{\vs}, \dots$ etc.\footnote{For this reason I would like to describe these terms as \textit{damping} terms, not "inertial" terms.}.

So far this is consistent for a spatially extended spin-distribution of the proton. If $a$ gives the "spin radius" of the proton, then $\gbar(-ik)$ and $\gbar(\kappa)$ individually go as $\approx 1/a^3$, but their difference becomes singular like $1/a$ as $a \to 0$, i.e. $\Gamma(k_r) \approx 1/a$.

I would now very much like to know from you what the Cambridge subtraction-trick of the transition to the "point-spin" gives in this case. Probably $\Gamma(k_0) = \Gamma_2 k_0^2 + \Gamma_4 k_0^4$ (\textit{without higher powers of $k_0$?}), where $\Gamma_2$ is a new \textit{arbitrary} constant (which could hardly lead back to the proton mass), while $\Gamma_4$ should be calculatable (by applying the energy-momentum-angular momentum law). \textit{Or is it otherwise?}

\subsection*{3. Applications for the spinning electron}\footnote{Regarding the classical model of the spinning electron, c.f. \textit{Kramers}, Foundations of Quantum theory, Part II, \textit{Thomas}, Phil. Mag. \textbf{3}, 1, 1927 and especially \textit{Frenkel}, Z. Phys. \textbf{37}, 243, 1926. For the \textit{Mott} polarization effect mentioned above, also \textit{F. Sauter}, Ann. Phys. \textbf{18}, 61, 1933.}

The reason I have a special interest in this question (calculation of the "damping term" in the equation of motion for $\dot{\vs}$ for "point spin" according to the subtraction method) is the application to the spinning \textit{electron} in an external electromagnetic field. Up to relativistic corrections the classical model remains the same as that discussed above, only we put $\kappa=0$ and $l=\inv{2}\frac{e}{\sqrt{\hbar c}}\frac{\hbar}{mc}$ ($m$ the electron mass).

It now seems very likely \textit{that the empirical lack of Mott's polarization effect in the scattering of $\beta$-particles on the nuclei in the surface of a metal is attributable to the action of the damping terms discussed here.} The \?{collision-times} in the relevant experiments are probably already comparable to $l/c$ and in any case it would be nice if the theory of the Mott polarization effect could be improved by taking the damping terms into account. In any case this would \textit{diminish} the polarization, as well as diminishing the scattering of mesons on protons (and indeed considerably, as soon as the frequency becomes comparable to $c/l$).

Now everything comes down to whether the theory can be carried through reasonably free of arbitratiness, without introducing new arbitrary constants (other than the rest masses).

I would now like to know what you think about this. -- We here in Zurich want to found a Non-Intervention-Committee in the conflict between you and Heisenberg on the existence or non-existence of explosions.

Warm greetings,

Always your W. Pauli

%harydthecure Manus
\letter{555}
\rcpt{Kemmer}
\date{May 5, 1939}
\location{Zurich}

Dear Herr Kemmer!

Thanks for your letter of the 2nd. -- So now everything is settled. The problem of a reasonable notation of equations for higher half-integral spin (without spinor calculus) still remains.

I have now undertaken a brief supplementary report for the Solvay Congress on "Relativistic wave equations of force-free particles and their quantization". There I'll have to deal with de Broglie. So a quick question: de Broglie's theory demands, in addition to the equation
\nequ{
\inv{2}(\gamma_\mu I' + I\gamma_\mu')\pddX{}{x^\mu}\Y + \kappa\Y = 0
}{I}
the auxilliary condition
\nequ{
\inv{2}(\gamma_4\gamma_\mu I' - I\gamma'_4\gamma_\mu')\pddX{}{x^\mu}\Y
+ \kappa(\gamma_4 I' - I\gamma'_4)\Y = 0
}{II}
\nequ{
\left\{\text{N.B. De Broglie writes instead of (I)}
\phantom{\sumXY{x=1}{3}}
\right.\\
\left.
I\,I' \pddX{}{x^4}\Y = \inv{2}\sumXY{k=1}{3}(\gamma_4\gamma_k I' + I\gamma'_4\gamma)
+ \kappa(\gamma_4 I' + I\gamma'_4)\Y = 0.\right\}
}{Ia}
Could you be so good as to briefly tell me whether the auxilliary condition (II) also follows from
\nequ{
\partial_\mu \beta_\mu \Y + \kappa\Y = 0
}{III}
when $\inv{2}(\gamma_\mu I' + \gamma'_\mu I)$ specifically is substituted for $\beta_\mu$?\footnote{I suspect: \textit{yes}. But you could tell me right away with certainty, so that I needn't plague myself with it any longer. -- How do I write (Ia) with the $\beta_\mu$?} Further: \textit{What is the fastest way to solve (III) \?{for} $\partial\Y/\partial t$}, with regard for the fact that $\beta_4$ has no reciprocal? -- Finally, could you also explicitly write out the 5-rowed and 10-rowed representations for the $\beta$? I would like to make use of them in my report. Many thanks in advance!

They've done something rather dumb in Bern: Herr A. Mercier (who we in Zurich believe understands nothing at all!) has received an Extraordinariat there. There has of course been some kind of local protection influence there.

Many greetings, 

Always your W. Pauli
% zwang
\letter{556}
\rcpt{Kemmer}
\date{May 8, 1939}
\location{Zurich}

Dear Herr Kemmer!

I have meanwhile more closely studied the de Broglie equations, and have reduced the question of the equivalence of his equations with yours \{for $\beta_\mu = \inv{2}(\gamma_\mu I' + I\gamma'_\mu$)\} to the following algebraic equation: let it be known from one quantity $\chi$ that $(\beta_\mu^2 - 1)\chi = 0$ for \textit{all} $\mu=1,2,\dots,4$ (or $\eta_\mu\chi = \chi$ with $\eta_\mu = 2\beta_\mu^2 - 1$); does it then follow that $\chi=0$? I am almost certain, but I now have much else to do and hence would like to put this question to you. If the question is to be answeres affirmatively, then de Broglie's auxilliary conditions would also follow from your basic equations.

Many greetings,

Your W. Pauli

% zig
\letter{586}
\from{Stueckelberg}
\date{December 17, 1939}
\location{Genf}

\nc{\vtint}{\int\int\int\int}
\nff{\vtintXY}{\underset{#1}{\overset{#2}{\vtint}}}
\nf{\vtintX}{\vtintXY{#1}{}}

Dear Herr Pauli!

Your letter from 12/12 has arrived in the middle of the study of the questions raised in the seminar. Your results coincide completely with mine, i.e. your $D_1$ function indeed satisfies the homogenous equation if its integral is defined by the \?{principal values}.

This result can be proven quite nicely by the Sommerfeld method: 

Let one result be defined by the four coordinates $x_i$ ($i=1,2,3,4$).  Let $x_1$ $x_2$ and $x_3$ be the three real space coordinates and $x_4 = u + it$ the \textit{complex time coordinate}. Then we first investigate which four-dimensional region $W$ (defined by a spatial volume $V$ and a complex path of integration $T$) allows the quantity
\nequ{
Q(x') = (2\pi)^{-1}\int\int\int\int{{dx''}^4}v(x'-x'')\rho(x'')
}{1}
to satisfy the inhomogenous differential equation
\nequ{
(\square - l^2)Q = -\rho.
}{2}
There $v$ is a function that depends only on $R$ ($R^2 = \sumXY{1}{4}(x'_i - x''_i)^2$), which becomes infinite as $R^{-2}$ for $R=0$ and satisfies the homogenous equation (2) for all $R$ (except $R=0$). You will find the explicit representation of this function in the accompanying Separatum (Bessel function for imaginary arguments). We decompose the spatial volume into a small volume $V'$, which includes the immediate neighborhood of $x'$, and the rest in the space $V''$. Correspondingly let the complex integration path in the $x_4=u+it$-plane be decomposed into the three parts $T=T''_l + T' + T''_r$. Of these let $T'$ be the part of this path of integration that runs through the point $x'_4 = 0+it'$ and its immediate neighborhood. $T''_l$ and $T''_r$ are the remainder of the integration path on the left resp. right of the point $x'_4$ in the $x_4$-plane (c.f. fig. 1 and fig. 2).

TODO: Fig. 1 and Fig. 2

Then the integral (1) has a finite value when:

(1) $\rho$ is finite in the whole volume $V$ and its volume integral likewise remains finite (for all times t).

(2) the integrand is singularity-free on the whole path $T$ (other than the $1/R^2$-singularity in $T'$) and vanishes sufficiently rapidly at $\infty$.

(3) this requires that $T$ crosses the $t$-axis in figure 2 \textit{only} at the position $t'$, i.e. so that $T_l$ lies entirely to the left and $T_r$ entirely to the right of the $t$-axis, and further that the singularities of the function $\rho$ (for all $\vx$) lie to the left or the right of the path $T$.

Now we define the region $W$ by the sum
\nequ{
W=VT=W'+W'',\quad \text{ with } W' = V'T'
}{3}
and accordingly in (1)
\nequ{
Q(x') = (2\pi)^{-1}\left(\underset{W'}{\int\int\int\int} + \underset{W''}{\int\int\int\int}\right).
}{4}
Since $v(R)$ satisfies the homogenous equation (2), except the point $x''=x'$ contained in $W'$, i.e. $R=0$, $\underset{W''}{\int\int\int\int}$ satisfies the homogenous equation. $\underset{W'}{\int\int\int\int}$ is the volume integral over the real vicinity $x_1 x_2 x_3 x_4$ of the \?{reference point} $x'_1 x'_2 x'_3 t'$.

For this real volume, one obtains according to Gauss's law:
\uequ{
(\square'-l^2)Q(x') &\to (2\pi)^{-1}\int\int\int{d\sumX{i}''}\rho(x'')\frac{\delta v}{\delta x''_i}\\
&\to (2\pi)^{-1}\int{d\Y}R^3\frac{\delta}{\delta R}R^{-2}\rho(x') = -\rho(x'')\\
}

(${d\sumX{i}}$ = surface element four-dimensional real space)\\
where ${d\Y}$ denotes the solid angle element in the four-dimensional Euclidean space, whose integral contributes $2\pi^2$. Thus everything is settled.

\?{Since only $\rho$ is physically on the $t$-axis (and its immediate neighborhood), and only $t''=+\infty$ and $=-\infty$ are physically meaningful limits, the following three paths satisfy the physical and mathematical conditions (I, II, III)}:

TODO: fig. 3

By Sommerfeld's method (c.f. enclosed Separatum) it can be shown that $Q^\text{I} = Q^\text{ret}$ represents the retarded and $Q^\text{II}=Q^\text{adv}$ the advanced potential, since they are defined by $\rho$ for past resp. future times $t''$.

$Q^\text{III}$ requires special examination: since the interand should contain no singularities in the immediate vicinity of the $t$-axis, $Q^\text{III}$ can be \?{arbitrarily} deformed into the paths IV or V

TODO: Fig. 4

with the relations
\nequ{
Q^\text{III} = Q^\text{IV} = Q^V = Q^\text{ret} + Q^r = Q^\text{adv} + Q^l
}{5}
($r$ and $l$ are paths which run from $-\infty$ to $+\infty$ $\parallel$ to the $t$-axis.)

Since $Q^\text{IV}$, $Q^\text{V}$, $Q^\text{ret}$, $Q^\text{adv}$ each individually satisfy the inhomogenous equation (2), it follows interestingly that $Q^l$ and $Q^r$ both satisfy the homogenous equation (2).

Specifically, for $\rho$ real in the $t$-axis, the following are real
\nequ{
Q^r - Q^l = Q^\text{adv} - Q^\text{ret} = 
(4\pi)^{-1}\vtintXY{-\infty}{+\infty}{dx^3}{dt''}D(r,t''-t')\rho(x',t'')
}{6}
\nequ{
(2i)^{-1}(Q^l + Q^r) =
(4\pi)^{-1}\vtintXY{-\infty}{+\infty}{{dx''}^3}{dt''}D_1(R)\rho(x'',t'').
}{7}
The second equalities are of a purely formal nature. $r$ and $R$ denote the positive square roots of
\nequ{
r^2 = \sumXY{1}{3}(x''_i - x_i)^2, \quad R^2 = r^2 - (t'' - t')^2.}{8}

Then $D(r,t)$ is defined by equation (8) (the last equation) of the Separatum. $D_1(R) = (2\pi i)^{-1}v(R)$, where because of the singularities on the $t$-axis, the $v(R)$ (mistakenly printed as $v(r)$) defined by equation (6) of the Separatum leads to indeterminate expressions. But since the paths are defined by the left side of equation (7), that means, as defined in your letter, that the integrals are replaced by their primary values. That applies even in the Fourier representation, so that in particular $\int{dx}\cos{x}/x$ is set equal to $0$.

Mathematically, $D$ and $D_1$ are best-defined left sides of equations (6) and (7). Since they apply for arbitrary $\rho$ (which satisfies regularity conditions), it can be said that $D$ and $D_1$ satisfy the homogenous wave equation everywhere, including $x''=x'$.

If nothing happens in the meantime, I am in Zurich for seminar on January 15, 1940. If I'm not prevented by some obstacle, I'll phone to try to postpone the seminar.

Warm greetings and Christmas and New Years wishes \?{from house to house}.

Your E.C.G. St

P.S. I'm sending a copy to Herrn Prof Sommerfeld.
%ta-defikeb$\r$ tge  stueck inside a mobile with those Zurich blues again

\end{document}
