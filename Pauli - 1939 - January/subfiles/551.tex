\letter{551}
\rcpt{Heisenberg}
\date{April 27, 1939}
\location{Zurich}
\tags{meson-theory}

Dear Heisenberg!

Many thanks for the letter and manuscript. I would first like to go deeper into the specific question of the \textit{scattering of mesotrons on protons}. We have been occupied with precisely this question since Bhabha was here about 10 days ago and gave us a detailed report on his classical calculations for mesotron scattering. There he has gotten rid of the (classical!) self-energy following Dirac (1938), so it is then possible to introduce an arbitrary constant for the rest mass of the proton. There he initially only took up the interactions proportional to $g_1$, but he has also -- which is \?{decisive} -- assumed that the mesotrons are neutral (\textit{real} field!). Then he naturally finds a mesotron cross section that vanishes with $M_\text{proton} \to \infty$. This is trivial in his case, since the quantum-mechanical perturbation calculation would supply the same. \{Namely: for $M\to 0$ the cross-section $\to 0$ emerges \textit{if} there are two paths from the initial state,

TODO: --(Meson)-->$p_\mu$ to $p_\mu$ --(Meson)-->

whose contributions are compensate one another by neglecting the recoil energy of the proton. This is the case for neutral mesons, since then the two intermediate states

TODO: --(Meson)-->(Neutron I)--(Meson)--> \textit{and} (Neutron II)

come into play (analogously to light scattering in electrodynamics). It is different with charged mesons; in the initial state

TODO: --(Meson+)-->($p_\mu$+) there is only \textit{one} intermediate state --(Meson+)-->(N)--(Meson+)-->

in the initial state

TODO: --(Meson-)-->($p_\mu$+) there is only the \textit{one} intermediate state (N) (without meson).

It is similar with transitions which are essentially connected with the flipping of the proton spin.\}

It is already correct that the corresponding semiclassical model for spin-protons ($g_2$) resp. charged mesons ($g_1$) permits the introduction of an arbitrary constant $K$ of dimension $\text{cm}^{-1}$ as a measure \?{moment of inertia} of the $\sigma$ resp. $\tau$ degree of freedom. But I believe that for now it can't go higher than first order in $m_\text{meson}\times c/\hbar$ (and certainly \textit{not} $M_\text{proton}\times c/\hbar$!). \textit{Since the corresponding part of the nuclear force, in particular the energy difference between the singlet- and triplet-ground states of the deuterons, must be decreased in your model just as the cross section of the mesotron-proton scattering}. (For the scattering proportional to $g_1^4$, your proposal would likewise depress all exchange forces between neutrons and protons.) The classical analogy to this energy difference is \?{just a frequency of the spin} of the heavy particles in the deuteron.

Thus it seems that only one assumption $K\approx \frac{m_\text{meson}\times c}{\hbar}$ can be discussed, which  tly altershe quantum-theoretical result 

by a factor of order 1. This factor can in turn -- at least for smaller meson energies -- be compensated by a change in the quantum-theoretical numerical value of $g_2$ with respect to the classical $g_2$ such that for mesotron energies $\ll m_\text{meson}\times c^2$ the quantum-mechanical and semiclassical values of the scattering cross-section coincide. So I ask myself whether a
\textit{in the quantum theory} the assumption $K=0$ is reasonable at all. That can't be proven, but your assumption $K\approx\frac{M_\text{proton}\times c}{\hbar}$ (with unchanged numerical value of $g_2$!) can certainly be empirically refuted by the nuclear forces.

If $K\neq 0$, I could raise the further question, \textit{why then do we not also need to introduce, for the electron in the electromagnetic field, a new \?{spin inertial-resistance} arising from the self-vector-potential $\vec{\phi}(0)$ of the electron?} I fear that would then come into conflict with experiment. Speculative types could even use such a modification of the theory to try to connect it with the empirical absence of polarization in electrons upon reflecting off of atoms as well as the empirical deviations of the $H$ fine structure from theory (see Williams, Physical Review). I would only conditionally claim: \textit{If one introduces for the proton in the meson field an additional inertial resistance $K\neq 0$ of the proton spin, then the same must be done for the electron in the electromagnetic field.}

In general, I would still object -- although I find the calculations on the mutual scattering of mesons in analogy to the "Euler-Kockel effect" beautiful and important -- that you rather over-state the importance of the dimensional argument. I believe, for the limits of applicability of the current quantum theory, it matters less whether the interaction energy contains a constant with the dimension of a length or a dimensionless constant, than rather the numerical magnitude of this coupling constant ($\frac{g_1^2}{\hbar c} \gg \frac{e^2}{\hbar c}$). Yet one still cannot know whether the divergences of the present theories are as directly connected with the values of the rest masses, as you assume.

Since it would be difficult to definitely prove anything, everyone will stick to his beliefs. The argument from Bhabha's note in Nature also seems to be incorrect, so far as it rests on the limit $M_\text{meson} \to 0$.

Now to your letter of the 23rd about the Solvay report. As much as it goes against my laziness to even write such a report, I nonetheless believe on factual grounds that I cannot reject on principle your proposal that I take over section 1. Thus I would like to make the counter-proposal that in section (1c) I \textit{don't} go into the \textit{interaction} (which should be reserved for section 2) and that we call the whole section 1 "Relativistic wave equations of \textit{force-free} particles and their quantization". I could then treat the connection of spin and statistics and (if you want) the gravitational quanta of spin 2. Further I could say something a bit less well-known about the de Broglie equations (especially \?{since} he is also writing a report on them) and stress that they neither describe partickes which are composed of two neutrinos nor photons, but rather -- if correctly interpreted -- a specific sort of mesotrons.

Regarding the work together with Bohr, I would rather propose that our manuscript be sent independently to Bohr and to Brussels. Since I have no desire to write a report which will then sit unread on Bohr's desk for an arbitrarily-long time. So please get in touch about the matter, on the one hand with the Solvay committee (Langevin) and on the other hand with Bohr. I could be ready by 7/1, but 6/1 is questionable.

Wentzel and I have attended Dirac's lectures on the subtraction-tricks in Paris. The quantum-theoretical part did not go substantially beyond an old paper of Wentzel's, was quite meager and a bit unconvincing. We could not share in his optimism in going further on this purely formal path.

Many gretings \?{from house to house}

Always your W. Pauli

% Sleachnitt