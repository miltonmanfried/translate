\letter{556}
\rcpt{Kemmer}
\date{May 8, 1939}
\location{Zurich}

Dear Herr Kemmer!

I have meanwhile more closely studied the de Broglie equations, and have reduced the question of the equivalence of his equations with yours \{for $\beta_\mu = \inv{2}(\gamma_\mu I' + I\gamma'_\mu$)\} to the following algebraic equation: let it be known from one quantity $\chi$ that $(\beta_\mu^2 - 1)\chi = 0$ for \textit{all} $\mu=1,2,\dots,4$ (or $\eta_\mu\chi = \chi$ with $\eta_\mu = 2\beta_\mu^2 - 1$); does it then follow that $\chi=0$? I am almost certain, but I now have much else to do and hence would like to put this question to you. If the question is to be answeres affirmatively, then de Broglie's auxilliary conditions would also follow from your basic equations.

Many greetings,

Your W. Pauli

% zig