\documentclass{article}
\usepackage[utf8]{inputenc}
\renewcommand*\rmdefault{ppl}
\usepackage[utf8]{inputenc}
\usepackage{ulem}
\usepackage{amsmath}
\usepackage{graphicx}
\usepackage{enumitem}
\usepackage{amssymb}
\usepackage{marginnote}
\newcommand{\nf}[2]{
\newcommand{#1}[1]{#2}
}
\newcommand{\nff}[2]{
\newcommand{#1}[2]{#2}
}

\newcommand{\rf}[2]{
\renewcommand{#1}[1]{#2}
}
\newcommand{\rff}[2]{
\renewcommand{#1}[2]{#2}
}

\newcommand{\nc}[2]{
  \newcommand{#1}{#2}
}
\newcommand{\rc}[2]{
  \renewcommand{#1}{#2}
}

\nff{\WTF}{#1 (\textit{#2})}

\nf{\translator}{\footnote{\textbf{Translator note:}#1}}
\nc{\sic}{{}^\text{(\textit{sic})}}

\newcommand{\nequ}[2]{
\begin{align*}
#1
\tag{#2}
\end{align*}
}

\newcommand{\uequ}[1]{
\begin{align*}
#1
\end{align*}
}

\nf{\sskip}{...\{#1\}...}
\nff{\iffy}{#2}
\nf{\?}{#1}
\nf{\tags}{#1}

\nf{\limX}{\underset{#1}{\lim}}
\newcommand{\sumXY}[2]{\underset{#1}{\overset{#2}{\sum}}}
\newcommand{\sumX}[1]{\underset{#1}{\sum}}
\nf{\prodX}{\underset{#1}{\prod}}
\nff{\prodXY}{\underset{#1}{\overset{#2}{\prod}}}
\nf{\intX}{\underset{#1}{\int}}
\nff{\intXY}{\underset{#1}{\overset{#2}{\int}}}

\nc{\fluc}{\overline{\delta_s^2}}

\rf{\exp}{e^{#1}}

\nc{\grad}{\operatorfont{grad}}
\rc{\div}{\operatorfont{div}}

\nf{\pddt}{\frac{\partial{#1}}{\partial t}}
\nf{\ddt}{\frac{d{#1}}{dt}}

\nf{\inv}{\frac{1}{#1}}
\nf{\Nth}{{#1}^\text{th}}
\nff{\pddX}{\frac{\partial{#1}}{\partial{#2}}}
\nf{\rot}{\operatorfont{rot}{#1}}
\nf{\curl}{\operatorfont{curl}{#1}}
\nf{\spur}{\operatorfont{spur\,}{#1}}

\nc{\lap}{\Delta}
\nc{\e}{\varepsilon}
\nc{\R}{\mathfrak{r}}

\nff{\Elt}{\operatorfont{#1}_{#2}}

\nff{\MF}{\nc{#1}{\mathfrak{#2}}}

\nc{\Y}{\psi}
\nc{\y}{\varphi}

\nc{\bsigma}{\boldsymbol{\sigma}}
\nc{\bM}{\mathbf{M}}
\nc{\bx}{\mathbf{x}}
\nc{\hx}{\chi} % wannabe italic x

\nf{\from}{From: #1}
\nf{\rcpt}{To: #1}
\rf{\date}{Date: #1}
\nf{\letter}{\section{Letter #1}}
\nf{\location}{}
\nf{\references}{}

\title{Pauli - 1940}

\begin{document}

\letter{594}
\rcpt{Weisskopf}
\date{March 10, 1940}
\location{Princeton}

Dear Herr Weisskopf!

We have been very happy to hear from you both again, and I would like to take the occasion of the semester break, when I am lazy and have time, to write you again. First, on Physics:

Early last year, Dirac had really extensively studied the manuscript of your paper (incidentally: I still have no preprint) and wrote the following laconic remarks (June 29, 1939):

"It is interesting to have this discussion of the infinities, though it does not seem to help in showing how to eliminate them. -- I have been examining the $e^4$ terms in my theory (one-electron theory, no holes) and find that the $\int\nu{d\nu}$ infinity can be easily eliminated, but the logarithmic infinity is very complicated and I have not found any way of removing it. This is what you predicted last March. However, I still believe that there must be some mathematical trick which will eliminate it."

Since then he is completely stuck and has not gotten any further. I have critically answered Dirac's last sentence, that it would seem improper if one could make a convergent quantum electrodynamics for arbitrary values of charge and rest mass, since then it would probably be helpless to be able to understand the latter. For this reason I have always expected that Dirac would somehow get stuck with his \?{artificial subtraction formalism}. But it is quite interesting that he did \textit{not} get stuck with the \textit{strongest} singularities, but rather with the logarithmic one (which first appears in the one-electron theory at the $e^4$ approximation, in the positron theory at the $e^2$ approximation). Probably the terms which spring from $G(\xi) \approx 1/\xi^4$ in the Bose theory make no difficulties for him either.

Independent of the Cambridge subtraction-arts, I would contradict the remark in your letter that even the force-free field theory of Bose particles could contain contradictions. In this there is still no electromagnetic field production (quasi $e=0$) and hence also no self-energy, hence indeed the function $G(\xi)$ only comes into play in the absence (internal or external) of electromagnetic fields. However I do admit that it is very probable in a future correct theory the concept of "\textit{force-free}" particles is no longer possible. That it is logically possible at all to regard particles independent of their interactions (idealization of a "force-free" case) might however be characteristic for the present theories, which then leads to divergences if interaction energies are added \textit{after the fact}.

I might have some bad luck with the publication of my Solvay report (it pleases me that it has reached you), i.e. I might have to wait a long time for it. On the one hand the report should be published by the Solvay Institute, so that \?{it might not be possible} to publish parts of it elsewhere (e.g. in the Review of Modern Physics, \?{where Tait would have already been!}). On the other hand the young Belgian (an entirely hopeful youth of physics named G\'eh\'niau; I have wickedly translated his name as "Hellish"), who shall do the French translations, is never done with them since he is constantly being called into military service in the Belgian army.

In the comparison of the Solvay report with Fierz's paper believe that you have done the latter an injustice, since his paper (and the one following it by both of us in the Proceedings of the Royal Society; preprints have been sent out to you) treats an essentially more difficult problem than the Solvay report. This is namely not occupied with the proof that a relativistic theory for particles with spin > 1, which satisfies all physical requirements, is actually possible. I have believed for a very long time that this, at least in an external electromagnetic field, would \textit{not} be the case -- until the quite difficult birth of the paper in the Royal Proceedings (?{to which Fierz has performed great service}). (N.B. Dirac's remark in his paper of 1936, that one might simply replace $p_\mu$ by $p_\mu + e/c\varphi_\mu$ in \textit{all} force-free equations, was false -- indeed, already for spin 1!)

It is incidentally to be expected \textit{that your function $G(\xi)$ becomes more strongly singular, the higher the particle spin is} (\?{perhaps you'll look into that sometime}!). The particles with higher spin contain (in the rest system) also have magnetic quadropole-, ..., etc. (and successively higher) moments. Also the "broadening effect" enters into the electromagnetic field already for spin > 1 in the commutation relations (as soon as $e\neq 0$), which rather complicates the matter.

I have recently written a letter on the latter point to Herrn Schiff in Berkeley (do you know him personally?). He has sent me 3 manuscripts and a letter. From the contents of his papers I could learn nothing new, but I had the impression that he has really understood the situation (also the last paper by Fierz and me), and for that reason have written him what he should be able to investigate in my opinion. Fierz and I don't want to work on it any longer if we don't get any new ideas.

The \textit{punctum saliens} are naturally to remedy the singularities of the present theories and the values of the rest masses. In nature \textit{all} possible spin values probably exist, the higher only very energetically unstable (thus short-lived). That seems more likely to me than there only being spin 0 (?), $1/2$ and 1. So far I am in agreement with Heitler's ideas; on the other hand his Ansatzes are essentially non-relativistically, which I hold to be entirely wrong, since then the number of logical possibilities are much too great.

Fierz has recently written a letter to Bethe, following my advice, because of a physical question. I would also like to know what he himself now still actually holds to his hasty communication on the meson theory from last Summer (especially after his face-to-face discussion with Heisenberg in August).

While the politicians figure out how to best successfully pursue the war, academic life still continues -- at least provisionally -- somewhat normally. (So please keep your fingers crossed, that the Swiss stay out of the war!) Wentzel has now been taken as a citizen into the city of Zurich, and with that the matter is probably settled, and that would be a further formality. I have put in a naturalization request. We will see.

As for local gossip, it reported that Fierz is betrothed (with the daugther of a lady-doctor from Zurich)\footnote{I have only seen her twice. She seems to be very intelligent, only is rather shy. Otherwise she loves music, she plays violin in the Zurich Tonhalle Orchestra.} and will be officially married (in \?{Grossmuenster}!) on March 29th. My wife and I are officially invited and I am already very nervous at the possibility of the necessity to have to put on a top hat (!). \?{But it is redeemed by the fact} that there shall be a great midday feast with much wine and subsequent dancing (which shall certainly last until evening).

Incidentally, Fierz is abandoning me in the Summer semester, he is going to Basel, where he has found a better position with a professorship and prospect for an extraordinariat soon. But we hope to see him often in our seminar. I have offered Herrn Jauch the assistant position (have you met him yet?), who is now returning from Minneapolis with a fresh Dr to his Swiss homeland -- with an American wife and absolutely no money. I fear that it will be a dangerous experiment! -- Many greetings to you and your wife, also from my wife, who shall write again.

Always,

Your W. Pauli

Please report soon on the new $\epsilon$ in the family. Incidentally Dirac's expecting one at about the same time!

P.S. I am always interested in hearing from our friend Max. Please give him my greetings. I only fear that even his relationship with science \?{is like that with your wife}: namely, friendly, platonic!

%Fmore iierz and ote sonr B ost Sanvay
\letter{598}
\rcpt{Fierz}
\date{July 3, 1940}
\location{Zurich}
\tags{majorana higher spin particles}

Dear Herr Fierz!

Unfortunately yeaterday I forgot to discuss a certain physical question with you, one which has already occupied me for some time, namely: How is \textit{Majorana's} paper, Nuovo Cimento, 1932, "Anno IX, N. 10" related to \textit{our} theory for particles with arbitrary spin? The essential thing with Majorana is the usage of \textit{infinitely-many} eigenfunctions
\uequ{
\psi_{j,m} \quad -j \leq m \leq +j, \quad j=0,1,2,\dots\,\,\textit{or} \,\, j =1/2, 3/2, \dots
}

This idea seems good and interesting, and his (relativistically-invariant) system of equations is considerably simpler than ours. But it is essential that with Majorana, only in the rest system \?{is} a particle with spin $s$ described by $2s+1$ non-vanishing functions, in any other reference system it is dscribed by $\infty$-many eigenfunctions.

The problem there is that there are plane waves which correspond to an imaginary rest-mass, i.e. particles which always move with superluminal velocities according to $E/c = \pm \sqrt{p^2 - p_0^2}$ ($p_0>0$ arbitrary, $|p|>p_0$).

It might be very difficult to get away from these crazy solutions \?{with Majorana's theory}. On the other hand, with it \textit{the particle density is always positive definite} (the case of even spins can also be quantized with the exclusion principle!) -- and the solutions which can be transformed to rest always have positive energy (with integer \textit{and half-integer} $j$). For \textit{this} solution, the dependence of the rest-mass on the spin is $m_s = \text{const.}/(s+1/2)$.

The question would be: for infinitely-many eigenfunctions (i.e. \?{infinite graded} representations of the Lorentz group; incidentally, for these my proof from the Solvay report does \textit{not} apply!) are there also such systems of equations (resp. auxilliary conditions) where pathological solutions with $(\nu^2/c^2)-k^2<0$ are excluded? The latter roll should play a similar complicating role there as the requirement of energy resp. charge density plays in our theory.

I would be very interested in hearing your opinion on this some time. In any case, the case of a representation of the Lorentz group with infinitely-many rows still seems to not have been sufficiently investigated. If you, as it seems from your last letter, are looking for a problem, this would be very interesting one! So read the Majorana paper some time! (Unfortunately, I only have a preprint, and I need it myself. The Italian language is very awkward for me!)

Warmest greetings \?{to everyone},

As always, your W. Pauli

%Jack's bicycle is music
\letter{599}
\rcpt{Fierz}
\date{July 17, 1940}
\location{Zurich}

Dear Herr Fierz!

Thanks for your letter of the 12th, whose contents are incidentally well-known to me. The learned express the matter like so: the representations of the Lorentz group of finite rank are all equivalent to those given in van der Waerden's book. They are \textit{not} unitary ($\sum\varphi_r^*\varphi_r$ is not an invariant), since there $b_r$ are anti-Hermitian; thus there are no unitary representations of the Lorentz group of finite rank. On the other hand there are apparently unitary representations of the Lorentz group ($\sum\varphi_r^*\varphi_r$ invariant; $a_r$ and $b_r$ Hermitian) of infinite rank, one of which is Majorana's. -- Your further suspicion, that each $\infty$-rank unitary representation of the three-dimensional spin groups \?{decays in the well-known finite ones}, is likewise correct. (They are in the almost unreadable paper by \textit{Wigner}, in Annales of Mathematics 1939, and it is proven there.)

However, I would like to tie on to the conclusion of your last letter: I see no physical \textit{a priori} reason, why the quantities $A_k$, $B_k$ -- which indeed have no direct physical meaning and are only mathematically convenient because of their commutativity -- must be Hermitian. For this reason I see nothing pathological in the Majorana equation $\sum A_k^2 = \approx 3/4$ either. The only pathologies I see in the Majorana equation are the solutions with imaginary mass.

The question remains open: \textit{are there other $\infty$-rowed representations of the Lorentz group with non-Hermitian $A_k$, $B_k$ where such solutions don't exist?} (For Hermitian $A$, $B$ it is certain that the representation \?{decomposes in the well-known finite manner}, which corresponds to tensors and spinors.)

Lately I have been working with Jauch on the electron scattering experiment. Up to now it seems that they can be best explained by non-Coulomb forces when the range $r_0$ of the forces has the considerable value of $10^{-12}\text{cm}$. (This comes from the fact that $kr_0$, with $k=\text{momentum}/\hbar$, must be of order $Z/137$.) However I am still concerned that the usual fine structure would \?{perturb too strongly}. But we want to \?{work on that} further. -- Jauch \?{was working very quickly} and made many errors; his ideas and his understanding are however good. Thus I am trying to assert my pedagogical influence upon him.

The Park Hotel in Vitznau where you live was once Hilbert's regular hotel. Perhaps the older hotel personnel still remember him and could tell you some anecdotes. (I myself know a story that played out there.) I assume that the normal Alpine weather \?{is dominating} there and that you are bored.

My departure within a finite time is not ruled out, but everything is still undetermined.

With many greetings (to your wife as well, and also from mine),

Your W. Pauli

%  neither cray nor fish
\letter{600}
\rcpt{Fierz}
\date{September 3, 1940}
\location{Princeton}

Dear Herr Fierz!

After we departed from Genf on July 31st, we indeed arrived safwly on the 24th of August in the USA, and Neumann took us in a car to Princeton. I won't begin this letter with a description of the trip, since \?{otherwise this letter will have no end}.

I would however like to report to you on my calculations those Majorana equations (on which we began corresponding in the Summer). These led me rather too far into pure mathematics; since I don't believe that these equations have any possibilities for actual physical application, I will not pursue the matter further.

\subsection*{§1.}

\nc{\ba}{\mathbf{a}}
\nc{\bb}{\mathbf{b}}
\nc{\bg}{\mathbf{g}}
\nc{\bq}{\mathbf{q}}
\nc{\bP}{\mathbf{P}}
\nc{\bp}{\mathbf{p}}
\nc{\br}{\mathbf{r}}
\nc{\bgamma}{\mathbf{\gamma}}

My first remark was that the equations (8) (I cite the equations from Majorana's paper in the following with M) for the matrices $\ba$, $\bb$ have still more general solutions than those given in the paper. Even holding to the constraint $Z \equiv (\ba\bb) = 0$ (M10), \?{I have worked out that the invariant}
\uequ{
J \equiv \bb^2 - \ba^2
}
has:

1. A continuous spectrum $J\ge 0$, $j=0,1,2,\dots$ integral.

2. A discrete spectrum with eigenvalues $J=1-k^2$, $k=0, \inv{2}, 1, \frac{3}{2}, \dots$. There $k$ is \?{respectively the minimal value of $j$ (which is simultaneously integral or half-integral with $k$)}: $j\geq k$.

Majorana considered only the special numerical value $J=3/4$, which e.g. distinguished by the fact that it has a continuous as well as a discrete spectrum, and integer as well as half-integer $j$ are possible.

Majorana's $J=3/4$ case is further distinguished by the fact that the equations (M13) for the $\gamma_0, \bg$ are compatible with the requirement
\uequ{
[J,\gamma_i] = 0. (i=1,\dots,4).
}

$J=3/4$ necessarily follows from this condition. -- On the other hand, this condition -- which does \textit{not} follow from (M13) -- is a luxury, which is hardly physically justifiable. Indeed all eigenvalues of $J$ could occur in the theory and the $\gamma_i$ could contain elements that are non-diagonal with respect to $J$ (in contrast to $\ba$, $\bb$, which are always diagonal with respect to $J$).

\subsection*{§2.} This becomes much more intuitive with the following considerations, which are more analytic than algebraic, but which have the deficiency that they only relate to the case of \textit{integer} $J$. The fact that all spin values occur in the equations suggests introducing an \?{\textit{actual}} $\bq$-spac

(different from the $x$-space!) so that the spin matrices $\ba$ are defined by $a_1 = (-i)\left(q_2 \pddX{}{q_3} - q_3 \pddX{}{q_2}\right)\dots$. For the $\bb$ one can likewise put $\bb = (-i)\left(q_0\frac{\partial}{\partial\bq + \bq\pddX{}{q_0}}\right)$. It is to be noted that the $\ba$, $\bb$ commute with the squared length $R^2 = \bq^2 - q_0^2$. For this reason, it seems natural \?{to relate} the eigenfunctions to the hyperboloid
\uequ{
q_0^2 - \bq^2 = +1 \quad \text{ or }\quad
q_0^2 - \bq^2 = -1
}
on which the Lorentz group induces in a well-known manner the motion group of the Bolyai-Lobatschewski geometry. I then denote by $iP_0, i\bP$ the projection of $\left(-\pddX{}{q_0}, \pddX{}{\bq}\right)$ onto this hyperboloid \{N.B. in contrast to $ip=\pddX{}{x}$ in $x$-space; the $P_0, \bP$ do not commute because of the projection\}\footnote{I had forgotten: in ordee that $P_k$ be Hermitian with respect to the volume element of the hyperboloid, \?{one must still add $3/2 q_k$ to the vector $\partial/\partial q_k$}.}, so the following operator relations apply (with $J=\bb^2 - \ba^2$)
\uequ{
[J,q_k] = -2iP_k; \quad i[J, P_k] = 2q_k(J-3/4) - 5iP_k\quad (k=0,1,2,3),
}
and for $\gamma_k=\text{const.}\{7q_k + 2(Jq_k + q_k J)\}$
\uequ{
[J,\gamma_k] = -8(ip_k+q_k)(J-3/4)\times\text{const.}
}

The numerical coefficients are chosen so that the terms proportional to $P_k$ (without the factor $J-3/4$) fall away. This is to some extent only a \?{little trick} to \?{dispense with} the matrix elementa of $\gamma_k$, which are not diagonal with respect to $J$. \textit{The matrix elements that are diagonal with respect to $J$ are} (up to an arbitrary constant factor) \textit{the same as those of $q_k$}. The Majorana equations are thus (for integer $j$, $m$) equivalent to the following
\uequ{
J\Psi(q,x) \equiv (\bb^2 - \ba^2)\Psi(q,x) = \frac{3}{4}\Psi(q,x)\\
(\sum\gamma_k p_k + mc)\Psi(q,x) = 0 \quad (q_0^2 - \bq^2 \pm 1) \quad (\gamma_k \text{from above}).
}
It is important that they are only compatible with one another for the numerical value $3/4$.

\{N.B. For $q_0 = ch\chi$, $q_3 = sh_\chi \cos\vartheta$, $q_{1,2} = sh_\chi \sin\begin{cases}\cos\varphi\\ \sin\varphi\end{cases}$,
\uequ{
J\equiv -\inv{sh^2\chi}\pddX{}{\chi}\left(sh^2\chi\pddX{}{\chi}\right) - \inv{sh^2\chi}\left\{
\inv{\sin\xi}\pddX{}{\vartheta}\left(\sin\vartheta\pddX{}{\vartheta}\right)r
\frac{1\partial}{\sin^2\vartheta\partial\varphi}.\right.
}
In order to solve the equation $J\Psi = A\Psi$, one puts $\Psi=f_j(\chi)P_{j,m}(\vartheta,\varphi)$. ($A$ eigenvalue of $J$; $P_{j,m}$ spherical function), with
\uequ{
\inv{sh^2\chi}\pddX{}{\chi}\left(sh^2\chi\pddX{f_i}{\chi}\right)
 - \frac{j(j+1)}{sh^2\chi}f_j(\chi) + Af_j = 0.
}

This equation leads to hypergeometric \?{} in the variables $ch^2\chi$ or $tgh^2\chi$ and supplies unitary representations of the Lorentz group\footnote{C.f. Bargmann, Zeitschrift für Physik. -- Case of the continuous hydrogen spectrum.} with $Z=0$ (M19). The transition to Majorana's matrix elements goes without further ado. The functions in the 3-dimensional space $q_0^2 - \bq^2 = \pm 1$ are equivalent to those which depend on the parameters $J$, $j$, $m$ with $-j \le m \le +j$ and the above-specified range of $J$, $j$.\}

Now I return to the critical remarks at the end of §1. One can likewise put
\uequ{
\left\{\sumX{k}(C_1 P_k + C_2 q_k)p_k - mc\right\}\Psi = 0
}
($C_1$, $C_2$ are arbitrary numerical coefficients) without presuming any specific eigenvalue of $J$, since the requirement $(J,\gamma_k)$ is not justifiable. Then of course $p_0^2-\bp^2<0$ can assume all real positive and negative numerical values.

In general there always states with $p_0^2 - \bp^2 < 0$, if $(p_0 - \br/c)$
\uequ{
\{-\gamma_0 p_0 + (\bgamma, \bp) - mc\}\Psi = 0
}
with some Hermitian $\gamma_0, \bgamma$. \{\textit{According to Dirac}, $(-\gamma_0 p_0 + i(\bgamma, \bp) + mc)\Psi = +i\pddX{}{x_0}$, $\bp = 0$; $p_0 = -i\pddX{}{\bx}$, with $\gamma_0, \bgamma$ Hermitian!\}

It seems to me, that by excluding unphysical states with $p_0^2 - \bp^2 < 0$, our [old] equations (possibly several resp. infinitely-many systems, which belong to various $m$-values) must emerge.

I want to investigate the special case $p_0^2 - \bp^2 = 0$ further.

What do you think? It is mathematically interesting (from a mathematical standpoint I am still rather unsatisfied, since I cannot get the case with half-integer $j,m$), but not physically.

Many greetings to you and all the colleagues

Your old W. Pauli

Regards to your wife and your parents (from my wife as well).
 % n assume athe ical co Fistwhistle
\letter{603}
\from{Klein}
\date{October 1940}
\location{Stockholm}

Dear Pauli!

I think that it will not be unwelcome if I again write to you on physics after so long a time. In this time our ideas have probably in many respects gone in similar directions. Indeed we both live in (up to now) remarkably favorable lands, while the countries of most of our scientific friends have been directly affected by the war. We are very isolated here. But we have some connection with Denmark. In the Summmer Bohr was here for a few days, and we expect him again next month. Despite the circumstances, he was as full of energy as ever, and in the institute as well work is going \?{swimmingly}, especially on the fission problem. Have you heard anything from the French physicists? I would like to know how things are going for them, especially Auger and Perrin. Frau Meitner, who now lives here, recently heard from Gorter that things are going relatively well for the Dutch, including Kramers. I don't know whether you have heard that Gordon died the past winter after a serious illness.

From your papers with Fierz on the relativistic wave equations that you have been busy with the generalization of the Dirac electron theory. I would like to write some about this problem. In an earlier paper (Vetenskaps Academy Archive 1936) I have described a procedure for the generalization of the Dirac equation, which appears with some improvements in the following. Let $q_{\mu\nu}$ be the unit operator, whose commutation relations $[\,]=$ characterize an infinitesimal rotation in the six-dimensional space. We put
\nequ{
\Gamma_k = iq_{k6}, \quad k=1,2,3,4,5
}{1}
and consider the wave equation
\nequ{
\sumXY{1}{4} = \Gamma_k p_k \psi = \mu\psi,
}{2}
where $p_1, p_2, p_3, p_4, = iE/c$ denote the energy-momentum values and $\mu$ denotes a matrix to be further characterized later, which commutes with the quantities $q_{kl}$, $k,l=1,2,3,4$. Now let $Q$ be a matrix which describes a finite rotation in the four-dimensional subspace $x_1,x_2,x_3,x_4,=ict$ of our six-dimensional helper-space. Then
\nequ{
\sum\Gamma_k p_k = Q^{-1}\sum\Gamma_k p'_k Q, \quad Q^{-1}\mu Q = \mu,
}{3}
where $p'_k$ denote the corresponding transformed values of the energy-momentum four-vector. Now let $\Gamma$ be a Hermitian matrix which commutes with $\Gamma_4$ and anti-commutes with $\Gamma_1$, $\Gamma_2$, $\Gamma_3$, $\Gamma_5$, and whose square is 1. Then we put
\nequ{
\alpha_0 = \Gamma\Gamma_4, \quad
\alpha_k = i\Gamma\Gamma_k, \quad k=1,2,3,5.
}{4}

Since we can choose the $\Gamma_k$ \?{which follow the exchange relations between the $q_{kl}$} to be a Hermitian matrix, \?{then the corresponding applies for $\alpha_0$, $\alpha_k$ only if the quantity $\Gamma$ exists}, which always seems to be the case.

Now, as is easily seen, we have $Q^*=\Gamma Q^{-1}\Gamma$, where $Q^*$ denotes the complex-conjugate matrix to $Q$. From this follows
\nequ{
\sumXY{0}{3}\alpha_k p_k = i\Gamma \sumXY{1}{4}\Gamma_k p_k 
= i\Gamma Q^{-1}\Gamma\sumXY{1}{4}\Gamma\Gamma_k p'_k Q
= Q^*\sumXY{0}{3}\alpha p'_k Q,
}{5}
where $p_0=ip_4=-iE/c$ denotes the momentum component conjugate to $x_0=ct$. Further, we put $i\Gamma\mu=\lambda$, and choose $\mu$ so that $\lambda$ is a Hermitian matrix. The exchangability of $\mu$ with $Q$ implies $Q^*\lambda Q$, and the wave equation can be written
\nequ{
\left\{\sumXY{0}{3}\alpha_k p_k - \lambda\right\}\psi = 0,
}{6}
and for the complex conjugate quantity $\psi^*$:
\nequ{
\psi^*\left\{\sumXY{0}{3}\alpha_k p_k - \lambda\right\} = 0,
}{7}
where now $x_1,x_2,x_3,t,\psi$ and $\psi^*$ now transform in the following manner under a Lorentz transformation
\nequ{
\psi'=Q\psi,\quad \psi'^* = \psi^*Q^*.
}{8}

If $\gamma_1, \gamma_2, \gamma_3, \gamma_4, \gamma_5=\gamma_1\gamma_2\gamma_3\gamma_4$ denote a system of Dirac matrices, then it is well-known that we can fulfill the commutation relations in $q_{\mu\nu}$ by the following specific \?{definitions}:
\nequ{
\Gamma_k = 1/2\gamma_k, \quad
q_{kl} = [\Gamma_k, \Gamma_l],\,\text{ where $\Gamma$ is still $\gamma_4$. }
}{9}

If $\gamma'_k$, $\gamma''_k, \dots$ are a series of independent (commutable) systems of Dirac matrices, then from this it follows that
\nequ{
\Gamma_k = 1/2(\gamma'_k + \gamma''_k + \dots),\quad
q_{kl} = q'_{kl} + q''_{kl} + \dots, \quad
\Gamma = \gamma_4'\gamma_4''
}{10}
also fulfill all requirements. From this, one can then obtain all possible irreducible representations. One easily gets a general (but apparently not the most general) irreducible representation with Weyl's method starting from a four-component Dirac function. However, I still don't know what the general $\Gamma$ looks like in this representation. Apparently for the spin matrices the following apply:
\nequ{
S_1 = -i\hbar q_{23}, \quad S_2 = -i\hbar q_{31}, \quad S_3 = -i\hbar q_{12}.
}{11}
For density $\varrho$ and current density $J$ we have
\nequ{
\varrho = \psi^* \alpha_0 \psi, \quad J = C\psi^* \alpha_k \psi,
}{12}
which satisfy the continuity equation because of the wave equation. Specific representations, which correspond to integer spin, are obtained by assuming that $\psi$ should be an irreducible tensor in the six-dimensional helper-space. e.g. $\psi = \text{vector}$ gives the scalar wave equation $\psi=S_{klm}$, antisymmetric in all indices, where the dual tensor $\widetilde{S} = iS$ gives a ten-component representation which leads to Proca's resp. Maxwell's equations.

The quantization, where we assume an external electromagnetic field $(V,A)$, can be carried out in the following manner. We start from the Hamiltonian function
\nequ{
H=\int {dr}\psi^*\left\{c(\alpha.\Pi) + \alpha_0 eV - C\lambda\right\}\psi,
}{13}
where $\alpha=(\alpha_1,\alpha_2,\alpha_3)$, $\Pi=p-(e/c)A$. Further, we put
\nequ{
\chi = \alpha_0 \psi, \quad \chi^* = \psi^* \alpha_0
}{14}
and assume for the time being that $\alpha_0$ does not have the eigenvalue 0. Then the commutation relations read
\nequ{
&[\psi_r^*(r'), \chi_s(r)] = \mp \delta_{rs}\delta(r-r'), \quad 
&[\psi_r(r), \psi_s(r')] = 0\\
&[\psi_r(r'), \chi_s^*(r)] = \delta_{rs}\delta(r-r'), \quad
&[\psi_r^*(r), \psi_s^*(r')] = 0,
}{15}
where the upper sign applies for symmetric, the lower for antisymmetric quantization with the corresponding modified definition of $[]$. But here it seems, as I have also suggested, that a restriction dominates that does not coincide with any earlier results.

Now from
\nequ{
[\xi, H] = -i\hbar \pddX{\xi}{t}
}{16}
follow the wave equations
\nequ{
\left\{-\alpha_0\frac{E-eV}{c} + \alpha\Pi\right\}\psi &= \lambda\psi\\
\psi^*\left\{-\alpha_0\frac{E-eV}{c} + \alpha\Pi\right\} &= \psi^*\lambda.
}{17}

In the case wher $\alpha_0$ has the eigenvalue 0, we will assume that $\alpha_0$ is a diagonal matrix whose first $m$ elements vanish, while the remaining $n-m$ elements are nonzero. If for brevity we write
\nequ{
\eta = c(\alpha\Pi) + \alpha_0 e\psi - c\lambda,
}{18}
then the wave equations for $\psi$ run
\nequ{
\alpha_0 E \psi = \eta\psi
}{19}
and thus
\nequ{
(\eta\psi)_k = 0, \quad k=1,2,\dots, m,
}{20}
so that 
\nequ{
H=\int {dr}\sumXY{m+1}{n} \psi_k^*(\eta\psi)_k.
}{21}

Further, only the components $\chi_{m+1},\dots,\chi_n$ of $\chi$ are nonzero. Of the quantum conditions (15), we only retain those which relate to $\psi_{m+1}, \psi_{m+2},\dots, \psi_n$,, and we use the $m$ equations (20) as the \textit{defining equations} for $\psi_1,\dots,\psi_m$. (For specific $\lambda$ \?{the latter does not hold}, but it does in general. E.g. the timeless equations in the Maxwell case would be $\rot{A} - H = 0$ and $\div{E} = 0$. But if a finite rest-mass is assumed, then we would have $V= \text{const.}\div{E}$.) Since in the expression (21) for $H$ lacks the components $\psi_1^*, \psi_2^*, \dots, \psi_m^*$, by commutation with $\chi_k$ one gets exactly the $n-m$ missing equations of (19).

[As regards the possibility of antisymmetric quantization, it seems to be constrained to the case where all eigenvalues of $\alpha_0$ are positive. Namely, in the other case, if we instead of $r$ go over to a discrete \?{state variable} $n$, one would have relations of the form $\psi_s^*(n)\psi_s(n) + \psi_s(n)\psi_s^*(n) = $ a negative real number, which is apparently impossible. In the original Dirac equation one has $\alpha_0=1$, so that the antisymmetric quantization is without difficulties here. In the scalar wave equation $\alpha_0$ has the eigenvalues $0,1,-1$. Here the antisymmetric quantization would also be excluded. (You see that I henceforth share your standpoint with respect to the Hermiticity.) I suspect, but without proof, that $\alpha_0$ always has positive eigenvalues for half-integer spin.]

In coming days I will study mkre closely the literature on the subject, which I only know poorly, and then try to ready the above for publication. In my aforementioned paper (where dubious speculations also occur), the quantity $\Gamma$ is missing, because I incorrectly came to \?{not-generally-real} expresions for $\varrho$ and $J$. After I had introduced it as above and put it forward it in Paris (early 1939), I heard from de Broglie and P\'etiau that the latter had already (Thesis, 1936) used this quantity in a fully correct manner in a paper on the case spin = 1, 0.

Now I want to close this long letter briefly with a warm greeting, also from my wife and to your wife.

Your old O. Klein.

% in  thelizing Rub twigs on yr armpits u pansy, eat pine needles
\letter{613}
\rcpt{Weisskopf}
\date{December 6, 1940}
\location{Princeton}

Dear Herr Weisskopf!

Thanks for your letter. I don't know what will happen in April, perhaps I shall be called somewhere else and hence it would be better if you came at the end of January. (You shouldn't take the article on nuclear forces too seriously -- \?{why say at the moment, "but say, you are so beautiful"? And even less delivery dates}.) Also: please, think it over again.

Two days ago I saw Rabi at that remarkable museum in Philadelphia called the Benjamin Franklin Institute, and about that not much more can be said than that it is over 100 years old. I had a very informative conversation with him about the MIT problem, national defense, physics; but it is preferable to discuss this with you face to face. However, I would like to report to you one bit of wickedness from Rabi (please spread!): He said: "Bohr is the Stalin of physics and Einstein isthe Trotsky!" (The former alludes to certain despotic characteristics in Bohr, the latter naturally to Einstein's stance toward modern physics.)

Now the spiral orbit in the Coulomb field: these were, at least for attraction, extensively studied by Sommerfeld (c.f. the paper Annalen der Physic 51, \textbf{1}, 1916, specifically p. 49, §3; further the earlier editions of Atombau und Spektrallinien.) The \?{limit momentum} $p_0 = \mathbf{Z}e^2/c$ plays a great role here; namely, he has long tried to quantize the angular momentum according to $p_\varphi = p_0 + kh/2\pi$ (instead of $p_\varphi = kh/2\pi$), which naturally leads to nonsense. \?{Perhaps you can check this with Sommerfeld}. -- \?{Did} you know that there is a certain analog to the spiral orbit in the Dirac theory when $Ze^2/\hbar c \gg 1$?

Then, instead of the $K$-shell, there is a continuous eigenvalue spectrum! \?{Check out the formulae some time}!

With the positron theory, it seems a bit as if I am to jump into cold water. But when you're in it, it's even quite pleasant. I have just jumped into the concept. (Incidentally Rabi has also strongly advocated for it!) When you come at the end of January, I hope to already know much about it and it would be a great help to me to be able to be able to discuss it with you extenstively and continuously! (Eisenbud has held a very good presentation on the work of Wentzel.) By the way: The Ithacians haven't sent me any check. I had understood from you that they would also do so (not only Rochester).

So, hopefully we will see eachother at the end of January. Many greetings to you and your wife from Franca and your old

W. Pauli

I will be at the Philadelphia meeting. Max Delbr\"uck wrote that he also would resp. should come.

% spirakHaemster
\letter{615}
\rcpt{Weisskopf}
\date{December 30, 1940}
\location{Princeton}

Dear Herr Weisskopf!

Thanks for your letter to Philadelphia. In it there is only one interesting situation, on a discrepancy of the electromagnetic theory for mesons of spin 1 (knock on and radiation) with the new measurements on showers created by mesons. Marshak will report to you on 
hat. I am glad that Corben is coming to Princeton for a longer time in January, I hope to learn details on the theory of cosmic rays from him.

My physics work is not going well at all, for now I am rather \?{immersed} in my methods for the positron theory for large coupling ($e^2/\hbar c \gg 1$); there are no consistent approximations. Wentzel's method doesn't work either, since (I suspect anyway) his inequality $g^2 \gg 1/\mu l$ is never fulfilled for photons ($\mu=0$). I still want to try it with the W.K.B. method, but it seems to me that the separation of the various oscillators will not work. Perhaps it will help to talk the situation through with you from the ground up.

After the provisional information from Wigner and Ladenburg, you should have no trouble getting money for travel if you come for three weeks at the end of January, since at least the end of your stay here falls again in the official Princeton semester. Later I write in more detail. Thus it would be very nice if you would come at the end of January. Then you could also hear not-entirely-trivial things in my Monday lecture on the Duffin numbers and possibly on the Majorana equations.

I would still like to abuse you by sending you the longer manuscript by Guido Beck, which a (personally unknown to me) Hungarian named Havas had sent from Lyon sent (along with a revealing accompanying letter) to Zurich and which then from there went on to me here. I firmly believe that everything is essentially false, but perhaps you could sift out where the error(s) are more quickly than me. That might be possible now; earlier brief notes were incomprehensible. I also would like to hear your opinion about whether the paper is suitable for publication (I hardly believe it is). -- How the poor devil can be helped is of course a much more difficult question.

From Klein I got a longer letter in October out of Stockholm (also via Zurich) with Physics (regarding: Duffin numbers and higher spin; however he hasn't gotten further than I have). He wrote that Bohr was in Stockholm in the Summer, and is expected again "in the next month" (= November) and \?{Lise Meitner as well.} Meanwhile Lauritsen is visiting here from Copenhagen and has reported only good things about Bohr and his institute. Bohr will remain there out of responsibility towards Denmark.

The Ithacians have still not paid, but Gibbs told me in Philadelphia (without having to ask him) that it would now soon come -- there has been some kind of bureaucratic sloppiness there.

I would like to speak at length about Wentzel's paper with you. I have told Marshak my specific thougts against its result that the self-energy of $N$ particles at an infinite distance should not be equal to $N$ times the self-energy of an isolated particle. Something must be wrong in his approximation.

It have been very interested in what Landau has written in the Physical Review about the possibility of interpreting the nuclear forces as electromagnetic -- though it is so far only a \?{glimmer} and still no theory. But it would be much nicer than the \?{problematic} Yukawa theory.

If only my physics work was again going better, then the whole M.I.T. thing wouldn't matter. So come at the end of January. But where do you want to stay? In the graduate college? -- Perhaps later I'll have still other worries, at the moment thank god not.

Now I would like to add still something more about Max Delbr\"uck. I still feel a strong friendly bond with him \?{as is often cultivated with two problematic natures}. Also he seems to be on the road, to have stabilized and have rather decided to remain in this country. -- But it has emerged that \?{there are roads in connection with politics where, despite my best efforts, I cannot follow him}. You have so far probably hardly guessed that the question as to whether it is desirable that Germany should lose the war would be a very difficult question, which necessitates long thinking or even perpetual brooding. But in the depths of a Prussian mind that still doesn't seem to be the case (despite opposition to the Nazi ideology and undiminished sympathy for Jews). There is a further "deep" problem \?{festering}, whether a German domination wouldn't be very interesting, if at the same time it's nevertheless not too likely to come about. There were moments which had been difficult for me, but I wanted to have patience with him, and \?{I still feel sympathetic towards him}. Additionally I would cite in Max's favor that \?{as a backround for these considerations a certain anxiety for his impending Americanization plays a part}, which is exacerbated by one specific circsumstance (about which I still cannot yet speak).

Now, that has become a rather long letter. All the best in the new year, also from Franca and also to Ellen.

Your W. Pauli


 %tpino y forntolerance

\end{document}
