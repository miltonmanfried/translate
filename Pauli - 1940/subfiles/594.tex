\letter{594}
\rcpt{Weisskopf}
\date{March 10, 1940}
\location{Princeton}

Dear Herr Weisskopf!

We have been very happy to hear from you both again, and I would like to take the occasion of the semester break, when I am lazy and have time, to write you again. First, on Physics:

Early last year, Dirac had really extensively studied the manuscript of your paper (incidentally: I still have no preprint) and wrote the following laconic remarks (June 29, 1939):

"It is interesting to have this discussion of the infinities, though it does not seem to help in showing how to eliminate them. -- I have been examining the $e^4$ terms in my theory (one-electron theory, no holes) and find that the $\int\nu{d\nu}$ infinity can be easily eliminated, but the logarithmic infinity is very complicated and I have not found any way of removing it. This is what you predicted last March. However, I still believe that there must be some mathematical trick which will eliminate it."

Since then he is completely stuck and has not gotten any further. I have critically answered Dirac's last sentence, that it would seem improper if one could make a convergent quantum electrodynamics for arbitrary values of charge and rest mass, since then it would probably be helpless to be able to understand the latter. For this reason I have always expected that Dirac would somehow get stuck with his \?{artificial subtraction formalism}. But it is quite interesting that he did \textit{not} get stuck with the \textit{strongest} singularities, but rather with the logarithmic one (which first appears in the one-electron theory at the $e^4$ approximation, in the positron theory at the $e^2$ approximation). Probably the terms which spring from $G(\xi) \approx 1/\xi^4$ in the Bose theory make no difficulties for him either.

Independent of the Cambridge subtraction-arts, I would contradict the remark in your letter that even the force-free field theory of Bose particles could contain contradictions. In this there is still no electromagnetic field production (quasi $e=0$) and hence also no self-energy, hence indeed the function $G(\xi)$ only comes into play in the absence (internal or external) of electromagnetic fields. However I do admit that it is very probable in a future correct theory the concept of "\textit{force-free}" particles is no longer possible. That it is logically possible at all to regard particles independent of their interactions (idealization of a "force-free" case) might however be characteristic for the present theories, which then leads to divergences if interaction energies are added \textit{after the fact}.

I might have some bad luck with the publication of my Solvay report (it pleases me that it has reached you), i.e. I might have to wait a long time for it. On the one hand the report should be published by the Solvay Institute, so that \?{it might not be possible} to publish parts of it elsewhere (e.g. in the Review of Modern Physics, \?{where Tait would have already been!}). On the other hand the young Belgian (an entirely hopeful youth of physics named G\'eh\'niau; I have wickedly translated his name as "Hellish"), who shall do the French translations, is never done with them since he is constantly being called into military service in the Belgian army.

In the comparison of the Solvay report with Fierz's paper believe that you have done the latter an injustice, since his paper (and the one following it by both of us in the Proceedings of the Royal Society; preprints have been sent out to you) treats an essentially more difficult problem than the Solvay report. This is namely not occupied with the proof that a relativistic theory for particles with spin > 1, which satisfies all physical requirements, is actually possible. I have believed for a very long time that this, at least in an external electromagnetic field, would \textit{not} be the case -- until the quite difficult birth of the paper in the Royal Proceedings (?{to which Fierz has performed great service}). (N.B. Dirac's remark in his paper of 1936, that one might simply replace $p_\mu$ by $p_\mu + e/c\varphi_\mu$ in \textit{all} force-free equations, was false -- indeed, already for spin 1!)

It is incidentally to be expected \textit{that your function $G(\xi)$ becomes more strongly singular, the higher the particle spin is} (\?{perhaps you'll look into that sometime}!). The particles with higher spin contain (in the rest system) also have magnetic quadropole-, ..., etc. (and successively higher) moments. Also the "broadening effect" enters into the electromagnetic field already for spin > 1 in the commutation relations (as soon as $e\neq 0$), which rather complicates the matter.

I have recently written a letter on the latter point to Herrn Schiff in Berkeley (do you know him personally?). He has sent me 3 manuscripts and a letter. From the contents of his papers I could learn nothing new, but I had the impression that he has really understood the situation (also the last paper by Fierz and me), and for that reason have written him what he should be able to investigate in my opinion. Fierz and I don't want to work on it any longer if we don't get any new ideas.

The \textit{punctum saliens} are naturally to remedy the singularities of the present theories and the values of the rest masses. In nature \textit{all} possible spin values probably exist, the higher only very energetically unstable (thus short-lived). That seems more likely to me than there only being spin 0 (?), $1/2$ and 1. So far I am in agreement with Heitler's ideas; on the other hand his Ansatzes are essentially non-relativistically, which I hold to be entirely wrong, since then the number of logical possibilities are much too great.

Fierz has recently written a letter to Bethe, following my advice, because of a physical question. I would also like to know what he himself now still actually holds to his hasty communication on the meson theory from last Summer (especially after his face-to-face discussion with Heisenberg in August).

While the politicians figure out how to best successfully pursue the war, academic life still continues -- at least provisionally -- somewhat normally. (So please keep your fingers crossed, that the Swiss stay out of the war!) Wentzel has now been taken as a citizen into the city of Zurich, and with that the matter is probably settled, and that would be a further formality. I have put in a naturalization request. We will see.

As for local gossip, it reported that Fierz is betrothed (with the daugther of a lady-doctor from Zurich)\footnote{I have only seen her twice. She seems to be very intelligent, only is rather shy. Otherwise she loves music, she plays violin in the Zurich Tonhalle Orchestra.} and will be officially married (in \?{Grossmuenster}!) on March 29th. My wife and I are officially invited and I am already very nervous at the possibility of the necessity to have to put on a top hat (!). \?{But it is redeemed by the fact} that there shall be a great midday feast with much wine and subsequent dancing (which shall certainly last until evening).

Incidentally, Fierz is abandoning me in the Summer semester, he is going to Basel, where he has found a better position with a professorship and prospect for an extraordinariat soon. But we hope to see him often in our seminar. I have offered Herrn Jauch the assistant position (have you met him yet?), who is now returning from Minneapolis with a fresh Dr to his Swiss homeland -- with an American wife and absolutely no money. I fear that it will be a dangerous experiment! -- Many greetings to you and your wife, also from my wife, who shall write again.

Always,

Your W. Pauli

Please report soon on the new $\epsilon$ in the family. Incidentally Dirac's expecting one at about the same time!

P.S. I am always interested in hearing from our friend Max. Please give him my greetings. I only fear that even his relationship with science \?{is like that with your wife}: namely, friendly, platonic!

%Fmore iierz and ote sonr B ost Sanvay