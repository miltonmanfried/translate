\letter{598}
\rcpt{Fierz}
\date{July 3, 1940}
\location{Zurich}
\tags{majorana higher spin particles}

Dear Herr Fierz!

Unfortunately yeaterday I forgot to discuss a certain physical question with you, one which has already occupied me for some time, namely: How is \textit{Majorana's} paper, Nuovo Cimento, 1932, "Anno IX, N. 10" related to \textit{our} theory for particles with arbitrary spin? The essential thing with Majorana is the usage of \textit{infinitely-many} eigenfunctions
\uequ{
\psi_{j,m} \quad -j \leq m \leq +j, \quad j=0,1,2,\dots\,\,\textit{or} \,\, j =1/2, 3/2, \dots
}

This idea seems good and interesting, and his (relativistically-invariant) system of equations is considerably simpler than ours. But it is essential that with Majorana, only in the rest system \?{is} a particle with spin $s$ described by $2s+1$ non-vanishing functions, in any other reference system it is dscribed by $\infty$-many eigenfunctions.

The problem there is that there are plane waves which correspond to an imaginary rest-mass, i.e. particles which always move with superluminal velocities according to $E/c = \pm \sqrt{p^2 - p_0^2}$ ($p_0>0$ arbitrary, $|p|>p_0$).

It might be very difficult to get away from these crazy solutions \?{with Majorana's theory}. On the other hand, with it \textit{the particle density is always positive definite} (the case of even spins can also be quantized with the exclusion principle!) -- and the solutions which can be transformed to rest always have positive energy (with integer \textit{and half-integer} $j$). For \textit{this} solution, the dependence of the rest-mass on the spin is $m_s = \text{const.}/(s+1/2)$.

The question would be: for infinitely-many eigenfunctions (i.e. \?{infinite graded} representations of the Lorentz group; incidentally, for these my proof from the Solvay report does \textit{not} apply!) are there also such systems of equations (resp. auxilliary conditions) where pathological solutions with $(\nu^2/c^2)-k^2<0$ are excluded? The latter roll should play a similar complicating role there as the requirement of energy resp. charge density plays in our theory.

I would be very interested in hearing your opinion on this some time. In any case, the case of a representation of the Lorentz group with infinitely-many rows still seems to not have been sufficiently investigated. If you, as it seems from your last letter, are looking for a problem, this would be very interesting one! So read the Majorana paper some time! (Unfortunately, I only have a preprint, and I need it myself. The Italian language is very awkward for me!)

Warmest greetings \?{to everyone},

As always, your W. Pauli

%Jack's bicycle is music