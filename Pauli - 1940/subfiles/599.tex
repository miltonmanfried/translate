\letter{599}
\rcpt{Fierz}
\date{July 17, 1940}
\location{Zurich}

Dear Herr Fierz!

Thanks for your letter of the 12th, whose contents are incidentally well-known to me. The learned express the matter like so: the representations of the Lorentz group of finite rank are all equivalent to those given in van der Waerden's book. They are \textit{not} unitary ($\sum\varphi_r^*\varphi_r$ is not an invariant), since there $b_r$ are anti-Hermitian; thus there are no unitary representations of the Lorentz group of finite rank. On the other hand there are apparently unitary representations of the Lorentz group ($\sum\varphi_r^*\varphi_r$ invariant; $a_r$ and $b_r$ Hermitian) of infinite rank, one of which is Majorana's. -- Your further suspicion, that each $\infty$-rank unitary representation of the three-dimensional spin groups \?{decays in the well-known finite ones}, is likewise correct. (They are in the almost unreadable paper by \textit{Wigner}, in Annales of Mathematics 1939, and it is proven there.)

However, I would like to tie on to the conclusion of your last letter: I see no physical \textit{a priori} reason, why the quantities $A_k$, $B_k$ -- which indeed have no direct physical meaning and are only mathematically convenient because of their commutativity -- must be Hermitian. For this reason I see nothing pathological in the Majorana equation $\sum A_k^2 = \approx 3/4$ either. The only pathologies I see in the Majorana equation are the solutions with imaginary mass.

The question remains open: \textit{are there other $\infty$-rowed representations of the Lorentz group with non-Hermitian $A_k$, $B_k$ where such solutions don't exist?} (For Hermitian $A$, $B$ it is certain that the representation \?{decomposes in the well-known finite manner}, which corresponds to tensors and spinors.)

Lately I have been working with Jauch on the electron scattering experiment. Up to now it seems that they can be best explained by non-Coulomb forces when the range $r_0$ of the forces has the considerable value of $10^{-12}\text{cm}$. (This comes from the fact that $kr_0$, with $k=\text{momentum}/\hbar$, must be of order $Z/137$.) However I am still concerned that the usual fine structure would \?{perturb too strongly}. But we want to \?{work on that} further. -- Jauch \?{was working very quickly} and made many errors; his ideas and his understanding are however good. Thus I am trying to assert my pedagogical influence upon him.

The Park Hotel in Vitznau where you live was once Hilbert's regular hotel. Perhaps the older hotel personnel still remember him and could tell you some anecdotes. (I myself know a story that played out there.) I assume that the normal Alpine weather \?{is dominating} there and that you are bored.

My departure within a finite time is not ruled out, but everything is still undetermined.

With many greetings (to your wife as well, and also from mine),

Your W. Pauli

%  neither cray nor fish