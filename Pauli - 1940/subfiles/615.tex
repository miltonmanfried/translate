\letter{615}
\rcpt{Weisskopf}
\date{December 30, 1940}
\location{Princeton}

Dear Herr Weisskopf!

Thanks for your letter to Philadelphia. In it there is only one interesting situation, on a discrepancy of the electromagnetic theory for mesons of spin 1 (knock on and radiation) with the new measurements on showers created by mesons. Marshak will report to you on 
hat. I am glad that Corben is coming to Princeton for a longer time in January, I hope to learn details on the theory of cosmic rays from him.

My physics work is not going well at all, for now I am rather \?{immersed} in my methods for the positron theory for large coupling ($e^2/\hbar c \gg 1$); there are no consistent approximations. Wentzel's method doesn't work either, since (I suspect anyway) his inequality $g^2 \gg 1/\mu l$ is never fulfilled for photons ($\mu=0$). I still want to try it with the W.K.B. method, but it seems to me that the separation of the various oscillators will not work. Perhaps it will help to talk the situation through with you from the ground up.

After the provisional information from Wigner and Ladenburg, you should have no trouble getting money for travel if you come for three weeks at the end of January, since at least the end of your stay here falls again in the official Princeton semester. Later I write in more detail. Thus it would be very nice if you would come at the end of January. Then you could also hear not-entirely-trivial things in my Monday lecture on the Duffin numbers and possibly on the Majorana equations.

I would still like to abuse you by sending you the longer manuscript by Guido Beck, which a (personally unknown to me) Hungarian named Havas had sent from Lyon sent (along with a revealing accompanying letter) to Zurich and which then from there went on to me here. I firmly believe that everything is essentially false, but perhaps you could sift out where the error(s) are more quickly than me. That might be possible now; earlier brief notes were incomprehensible. I also would like to hear your opinion about whether the paper is suitable for publication (I hardly believe it is). -- How the poor devil can be helped is of course a much more difficult question.

From Klein I got a longer letter in October out of Stockholm (also via Zurich) with Physics (regarding: Duffin numbers and higher spin; however he hasn't gotten further than I have). He wrote that Bohr was in Stockholm in the Summer, and is expected again "in the next month" (= November) and \?{Lise Meitner as well.} Meanwhile Lauritsen is visiting here from Copenhagen and has reported only good things about Bohr and his institute. Bohr will remain there out of responsibility towards Denmark.

The Ithacians have still not paid, but Gibbs told me in Philadelphia (without having to ask him) that it would now soon come -- there has been some kind of bureaucratic sloppiness there.

I would like to speak at length about Wentzel's paper with you. I have told Marshak my specific thougts against its result that the self-energy of $N$ particles at an infinite distance should not be equal to $N$ times the self-energy of an isolated particle. Something must be wrong in his approximation.

It have been very interested in what Landau has written in the Physical Review about the possibility of interpreting the nuclear forces as electromagnetic -- though it is so far only a \?{glimmer} and still no theory. But it would be much nicer than the \?{problematic} Yukawa theory.

If only my physics work was again going better, then the whole M.I.T. thing wouldn't matter. So come at the end of January. But where do you want to stay? In the graduate college? -- Perhaps later I'll have still other worries, at the moment thank god not.

Now I would like to add still something more about Max Delbr\"uck. I still feel a strong friendly bond with him \?{as is often cultivated with two problematic natures}. Also he seems to be on the road, to have stabilized and have rather decided to remain in this country. -- But it has emerged that \?{there are roads in connection with politics where, despite my best efforts, I cannot follow him}. You have so far probably hardly guessed that the question as to whether it is desirable that Germany should lose the war would be a very difficult question, which necessitates long thinking or even perpetual brooding. But in the depths of a Prussian mind that still doesn't seem to be the case (despite opposition to the Nazi ideology and undiminished sympathy for Jews). There is a further "deep" problem \?{festering}, whether a German domination wouldn't be very interesting, if at the same time it's nevertheless not too likely to come about. There were moments which had been difficult for me, but I wanted to have patience with him, and \?{I still feel sympathetic towards him}. Additionally I would cite in Max's favor that \?{as a backround for these considerations a certain anxiety for his impending Americanization plays a part}, which is exacerbated by one specific circsumstance (about which I still cannot yet speak).

Now, that has become a rather long letter. All the best in the new year, also from Franca and also to Ellen.

Your W. Pauli


 %tpino y forntolerance