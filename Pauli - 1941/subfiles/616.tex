\letter{616}
\rcpt{Jauch}
\date{January 1, 1941}
\location{Princeton}

Dear Herr Jauch!

I don't believe I have answered your letter of October 20. -- As things lie now, I think it is as good as excluded that I can return as long as the war lasts. No shipping company would take me on as a passenger over the Atlantic in the America $\to$ Europe direction, and in a trip over Japan-Siberia I am certain to get stuck in Germany. I will probably have to wait here until the war is over. -- This makes itself felt here namely in the experimental physics is being very strongly reduced in favor of defense-work (German "Wehrwissenschaft") and will be again reduced further.

It will probably be the job of the foreigners to once more lead science. I don't believe that I can do anything at the moment to find you a job here. You have also never indicated that you wanted to come over here \?{when} I was in Zurich. But try it out yourself if you want.

I will perhaps sooner or later need my lecture notes and would beg you to send them to me. They are all in my desk in the institute in Zurich in the bottom right drawer. I think it best to send them registered (perhaps as a parcel), but \textit{not} air mail. Up to now everything you've sent has arrived, if slowly (often after 6-8 weeks).

We have come to an arrangement where a friend of my wife in Zurich shall send you 29 Franks, which hopefully has happened. Please write an \?{invoice} if you need more.

I would like to have your paper on electron scattering (manuscript or corrections) here and if possible the experimental work from the Zurich institute as well. The experimental results are often doubted here, especially by Crane, who has found a much smaller deviation from theory than Bosshard. I hope the experiment will be performed here with van de Graff generators at higher voltages. Please also send me my (or your) notes on the errors that Rose has made. (1. Chose the wrong particular solution. 2. Application of an inadequate asymptotic formula for the solution). He wanted to know more precisely.

An interesting experimental-paper by Kikuchi on the polarization of electrons upon reflection has come out.

Has anything else come of your calculations with Wentzel's paper? Incidentally I doubt its result that the self-energy of $N$ particles at an infinite distance should be different from $N$ times the self-energy of an isolated particle.

Here, the most interest with the meson is directed towards whether it has spin $1/2$ or spin $1$. Against the former there is the fact that the interpretation of the spontaneous decay if the meson then becomes very artifical. On the other hand, the latest result on the mesons created by showers is that the electromagnetic theory of charged mesons with spin $1$ (not the Yukawa theory, I only mean the interaction with the electromagnetic field \`a la Proca) cannot \?{be correct} for high meson-energy. However, that may be due to a failure in the perturbation theory.

I found the remark by Landau in the Physical Review on the possibility of interpreting the nuclear forces as electromagnetic (though non-Coulomb) between protons and (negative) mesons to be very interesting. However, it still lacks any real theory.

Here there is an interesting experiment underway on proton-proton scattering with 8 million volt protons, which promises to explain the range of nuclear forces. But there are still no results.

I have nothing new to report of me myself at the moment. Many greetings to the whole institute, especially to professor Scherrer as well as to professor Wentzel. Greetings to Fierz and to Stueckelberg. \?{Despite the circumstances}, a happy new year to all! Are you celebrating much? Are the lecture hours reduced? Is there a \?{black out}?

Your W. Pauli

%fr on electrind  stoat