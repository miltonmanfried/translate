\letter{620}
\rcpt{Klein}
\date{January 14, 1941}
\location{Princeton}

Dear Klein!

Since the end of August 1940 I have again been, following an invitation, a visiting professor here (as was also the case in the Winter of 35/36). How long I remain here depends on the war. -- So your letter reached me rather late, but it pleased me all the more when it arrived.

I know nothing of Auger, Perrin, de Broglie and his scholars, \?{Langevin was unwell}, but now rather better again. -- Rosenfeld is the successor of Uhlenbeck in Utrecht; I have a letter from him out of Utrecht from November, things are going well for him there. -- Please greet \?{the} Lise Meitner from me, and especially Bohr if you see him. I still haven't written him anything from here, but we have all heard much about him from young Lauritsen (who recently arrived in the U.S.A.), and actually only good. On the other hand we were disappointed that Hevesy, who was expected, has not come.

Now to the physics. I have been occupying myself with the equations
\nequ{
\sumXY{1}{4}\Gamma_k p_k \psi = \mu\psi
}{1}
for the case $\Gamma_k = \inv{2}(\gamma'_k + \gamma''_k + \dots)$, $q_{kl} = [\Gamma_k, \Gamma_l]$, $\Gamma = \gamma'_4\gamma''_4\dots$; $\mu = \text{const. I}$. This matrix can also be decomposed into irreducible component parts and characterized by algebraic relations, which are a generalization of the P\'etiau-Duffin-Kemmer relations. \{Mine is this: for $\beta_k = \inv{2}(\gamma'_k + \gamma''_k)$, $\beta_l\beta_k\beta_m + \beta_m\beta_k\beta_l + \beta_m\beta_k\beta_l = \delta_{lk}\beta_m + \delta_{kn}\beta_l$.\} -- How the 6-dimensional space helps, I could not understand.

I have however come to the result that the (naturally Lorentz-invariant) equations (1) -- except in the case that no more than two Dirac matrices are added, which corresponds to spin $1/2$ (Dirac) or spin 0 and 1 -- are not physically usable in this form. They have the following characteristics:

(1) \?{There are no} 2nd order wave equations, and it can be seen that the equations describe several particles with (in general) different values of the rest-mass.

(2) In the case of integral spins (even number of $\gamma_k$ added) the energy (not only the energy density, but also the total energy integrated over the volume) is not positive definite. More precisely: if it is positive for one type of particle, it is always negative for the other.

(3) For half-integral spin $> 1/2$ (odd number of $\gamma_k$, greater than or equal to 3), the same applies in the $c$-number theory for the total charge $\int\varrho{dV}$, i.e. the eigenvalues of $\alpha_0$ are also then always negative (contrary to your suspicion).

This is exactly the type of equation, which I had discussed and discarded in the introduction of the paper with Fierz (Proceedings of the Royal Society). They are only more nicely written with the $\Gamma_k$, but not nicer in content. It is also useless to reduce the $\Gamma_k$.

It is naturally seen from that how strongly the spin values $0$, $1/2$ and $1$ are distinguished from the higher ones. In fact, along with Fierz I have long attempted to prove that the higher spin values are impossible in a relativistic theory, but eventually found the opposite. \textit{What is apparently possible (e.g. according to Dirac 1936) is to supplement the equations (1) by further equations which then exclude the particles of lower spins with the wrong sign of energy resp. charge, and only the particles with the highest spin value remain.}

However I have not succeeded in nicely writing these supplementary equations without spinor- or tensor calculus with hypercomplex numbers. \?{I believe one cannot manage with the system of $\Gamma\, \Gamma_k$.} -- It would be very nice if progress could be made there. \?{Perhaps try some time} to rewrite the equations for spin 2, which are explicitly given in my paper with Fierz, with suitable questions. I would be particularly happy if the poor world could at least be rid of the plague of the spinor calculus, so that these are eliminated by means of the introduction of suitable hypercomplex numbers.

Here there has been a long discussion about whether the mesons have spin 0, $1/2$ or 1. There is now some evidence against a spin-value of 1, which only leads back to the electromagnetic interactions of the mesons. Whether the mesons found in cosmic rays have anything to do with nuclear forces again seems to be quite doubtful. And even the whole Yukawa theory of nuclear forces is not very convincing. On grounds of his last publication, I believe that Møller has fully cracked up and that Bohr has passed much too little onto him.

Here in Princeton experiments are underway on proton-proton scattering at energies of about 8 MeV. Then one will at least know something reliable about the range of the proton-proton forces; Rossi and the very active group of physicists in Chicago proceed with the experiment on cosmic rays. Their spontaneous decay is still very murky (what is actually created?).

My wife greets you, and I do as well. Has the number $N$ of your children meanwhile remained constant?

My trip to Russia in the year 37 was incidentally very revealing.

All the best to you and your family in the new year!

Your old W. Pauli


% ncu litGroPehas rolik,l ieaxßmodule