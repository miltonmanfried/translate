\letter{621}
\rcpt{Klein}
\date{January 28, 1941}
\location{Princeton}

Dear Klein!

I would like to add an appendix to my last letter concerning your equations
\nequ{
\sumXY{1}{4}\Gamma_k p_k \psi = \mu\psi,
}{1}
where I (with you) now specifically insert
\uequ{
\Gamma_k = \inv{2}\left(\gamma_k^{(1)} + \gamma_k^{(2)} + \dots \gamma_k^{(N)}\right),
\quad \Gamma = \gamma_4^{(1)}\gamma_4^{(2)}\dots\gamma_4^{(n)}.
}
$\gamma^{(n)}$ Dirac matrices with $\gamma_i^{(n)}\gamma_k^{(m)} = \gamma_k^{(m)}\gamma_i^{(n)}$ for $m \neq n$. In my last letter I spoke of auxilliary conditions which must be appended to (1) in order to give the theory a reasonable physical meaning, i.e. so that

(a) a second-order wave equation $(\square \mu^2)\psi = 0$ applies for $\psi$ and

(b) with integral spin the total energy, with half-integral spin (in the $c$-number theory) the total charge is positive definite.

I have meanwhile recalled that these auxilliary conditions have already been explicitly formulated in the literature and indeed by \textit{Belinfante}, Physica \textbf{6}, 870, 1939; specifically pages 890 and 891; equation (1) is his equation (19), the auxilliary conditions his equation (20). These are simply of the following type: let $\psi \equiv \psi(x_k; r_1, r_2, \dots, r_N)$ where $r_n=1,\dots,4$ correspond to the matrix $\gamma^{(n)}$. Then

(1) \textit{$\psi$ should be symmetrical in $r_1 \dots r_N$.}

(2) $\sumXY{k=1}{4}\gamma_k^{(n)}p_k\psi = \hx\psi$.

(1) implies: if (2) is correct for arbitrary $(n)$, then it is correct for \textit{all} $n=1,\dots,N$ and $\mu=1/2 N\hx$.

It can be shown that the wavefunctions satisfying the conditions (1) and (2) actually only describe \textit{one} particle with spin $N/2$ and rest mass $\hbar\hx/c$.

At the moment, the Belinfante formulation for particles with higher spin \{in the force-free case - in external electromagnetic fields the auxilliary conditions cause problems, which \textit{Fierz} and I removed by special tricks (Proceedings of the Royal Society (A)\textbf{173}, 211, 1939) \} seems to be the most satisfactory, since it is a natural generalization of Dirac's theory and does \textit{not} use the spinor calculus.

The fact that the auxilliary conditions are superfluous with $N=2$ (spin 1) is due to the following: multiply the equation
\uequ{
\inv{2}\sumX{k}(\gamma_k + \gamma'_k)p_k\psi = \hx\psi
}
by $\inv{2}\sumX{k}(\gamma_l - \gamma'_l)p_l$ on the left. Then you easily see that the left side vanishes identically and it follows that (assuming $\hx \neq 0$)
\uequ{
\sumX{k}(\gamma_l - \gamma'_l)p_l\psi = 0,\quad\text{thus}\\
\sumX{l}\gamma_k p_k\psi = \hx\psi \quad \text{ and } \quad
\sumX{k}\gamma'_k p_k \psi = \hx \psi\,\,\text{individually.}
}
The skew part of $\psi$ corresponds to a particle with spin 0, the symmetric part a particle with spin 1.

Again many greetings \?{to everyone},

Your W. Pauli
% he literMy runic tunic