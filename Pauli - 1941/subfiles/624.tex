\letter{624}
\rcpt{Bethe}
\date{March 17, 1941}
\location{Princeton}

Dear Herr Bethe!

Today I have just two remarks on physics to write. The first concerns the "academic" self-energy problem, which we have discussed. \?{It is to be noted that by virtue of the equation ${d^2u}/{d^2r} + k^2u = \text{const.}rD$ itself, when in the Fourier expansion of $rD$ the waves with $k>k_0$ do not occur,the $u$ with $k>k_0$ are \textit{nevertheless} perturbed, since the Fourier expansion of $u(k)$ contains for every such $k$ \textit{all} wavelengths $>2p/k_0$}. Thus I now suspect that for your earlier result $E/\hbar c \approx \inv{r_0}\sqrt{\gamma}$, $\left(\gamma = \frac{e^2}{mc^2r_0}\right)$ for $\gamma \gg 1$ is \textit{not} connected to the positivity of $D$ rather applies in general. Thus I hope to hear from you on the problem, I would agree with a publication of my method, perhaps you could still simplify the transition to transversal waves.

The second remark concerns the meson theory. I have read with some interest the paper by Schwinger-Rarita and the note by Oppenheimer in the issue of the Physical Review from March 1st. After that it seems that the "case d" of Kemmer's classification (Proceedings of the Royal Society 1939) of the possible interactions ($\hat{\psi}$, pseudoscalar, describes a meson with spin 0,
\uequ{
\text{Interaction } f \sumXY{k=1}{4}\psi^*_N\beta\gamma_5\gamma_k\psi_p\pddX{\hat{\psi}}{x_k} 
 + \text{complex conjugate};
}
$\beta = \gamma_4$; $\gamma_5 = \gamma_1\gamma_2\gamma_3\gamma_4$) in your article published in the Physical Review. (Only \textit{charged} mesons are needed, no neutral one). At the time, Kemmer had discarded this case on grounds of a very sloppy treatment of the spin-dipole interaction $\left(\frac{3(\sigma^Ix)(\sigma^{II}x)}{r^2} - \sigma^I\sigma^{II}\right)f(r)$. The assertion of the Oppenheimer school, that this case the singlet-triplet energy difference in the deuteron (with respect to neutron-proton scattering) \textit{and even the sign of the deuteron quadropole-moment} comes out correctly, is plausible to me. Naturally one must cut off the $r$-dependence exactly as you have done in your published paper, but the sign and the spin-dependence of the forces are taken seriously. \textit{Do you have specific mathematical objections against this claim of Oppenheimed and his scholars?}

Now, don't misunderstand me: even this Hamiltonian and its natural constant $f$ or $g$ is not to be taken seriously. First, according to Stueckelberg-Patry the higher approximations of the perturbation calculation in $f$ or $g$ are not small, second even here the objection remains that a too-strong meson scattering emerges from the theory (or am I wrong in this point?). But perhaps it suffices, following Bhabha to introduce higher charge states of the proton, in order to diminish the latter. The whole thing is rather barmy. But nothing seems to speak against spin 0. (The decay of the meson into electron + photon seriously contradicts the lack of secondary ionization associated with the photon; now a Russian has also again proved the same. But then it doesn't go right with spin $1/2$.)

Perhaps the thing can be turned around so: \textit{the negative result of your paper concerning the sign of the deuteron quadropole moment is a further argument against spin 1 and for spin 0 mesons.\footnote{Heisenberg's objection against your calculation was wrong.}}

Your other alternative, that the cosmic ray mesons have nothing at all to do with nuclear forces \comment{!!!} would only be acceptable if the \textit{creation} of mesons was explainable purely electromagnetically (pair production). It is well-known that the opposite is true. Incidentally, last week Johnson and a student \?{presented a paper} here. They found a non-electromagnetic component of meson scattering (with Wilson images), which Johnson holds to be certain.

Many greetings \?{from door to door}

Your W. Pauli

% swagger