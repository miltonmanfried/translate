\letter{628}
\rcpt{Fierz}
\date{March 29, 1941}
\location{Princeton}

%\nc{\vsigma}{\vec{\sigma}}
\nc{\vx}{\vec{x}}
\nc{\lblI}{\text{I}}
\nc{\lblII}{\text{II}}

Dear Herr Fierz!

Many thanks for your letter of March 7th, which arrived relatively quickly. I have some news to report on meson theory, but first a few words on the unitary representations of thr Lorentz group. Wigner uses terminology in his paper that I first had to clarify. Dirac's equations (for spin $1/2$ as well as higher spins) in fact give rise to unitary representations of the \textit{inhomogenous} (including translations $x'_k = x_k + a_k$; the eigenfunctions in momentum space are hence multiplied by $\exp{ip_ka_k}$) Lorentz group, if the \textit{spin + momentum space}, not the spin space alone, is interpreted as the representation space. E.g. in the usual Dirac equations $\sumXY{\varrho=1}{4}\varphi^*_\varrho(\vp)\varphi_\varrho(\vp)d^3p$ is invariant, which is interpreted as a definite quadratic form in the $(\varrho, \vp)$-space\footnote{In principle the $(r,x)$-case works just as well, the $x_0$-dependence is then given implicitly by the wave equation.}. The component $p_0$ always be considered as eliminated by virtue of the Dirac equation \?{and its sign can be fixed}. Then by a Lorentz transformation each function $\varphi_\varrho(\vp)$ is assigned another $\varphi'_\varrho(\vp')$. \{With infinitesimal Lorentz transformations, the \?{standpoint} is similar as in my Handbuch article, page 264, equation (83), without electromagnetic field (here $c$-number theory).\}

It is analogous for higher spins, the total charge supplies, for a given sign of $p_0$, a \textit{definite} (as you yourself have said) form in a $(\xi; \vp)$-space. For $\xi = 1, \dots,2s+1$ (i.e. $\xi=1,2$ with spin $1/2$, sign of $p_0=+\sqrt{\pi^2+x^2}$ fixed) with spin $s$ one thus gets an \textit{irreducible, unitary representation} of the \textit{inhomogenous} Lorentz group. In this manner one gets \textit{all} such representations which for the case $p_0^2 - \vp^2 > 0$ (as Wigner has shown). For $p_0^2 - \vp^2 = 0$ and $p_0^2 - \vp < 0$ (the latter case is "unphysical"), there are still others. I also needed a long time to understand that, because of the mix-up of the $(\xi, \vp)$-space with the $\xi$-space. -- Wigner's other remark about Majorana and its relation to Dirac is wrong. Wigner has not understood Majorana's paper, as he has admitted to me. The Majorana equations with a $\xi$-space of $\infty$-many directions give rise in the $(\xi; \vp)$-space to a \textit{reducible}\footnote{That means it is possible to \?{adjoin} further Lorentz-invariant equations.} unitary representation of the inhomogenous Lorentz group (which includes all cases $p_0^2 - \vp^2 > 0$, $p_0^2 - \vp^2 = 0$ and $p_0^2 - \vp^2 < 0$, but in addition to the Dirac cases others as well).

If one wants to go from the inhomogenous Lorentz group to the homogenous group and reduce the representations with respect to the latter, the eigenvalues of $J=\sumX{i<k}m_{ik}$ and $C=m_{12}m_{31} + \dots$ \?{must be adjoined}. Now after some effort Bargmann has shown the following: if all $p_k$ are not identically zero, in the representations of the homogenous Lorentz group for $C=0$, $J$ is always $>1$, i.e. the representations $C=0$, $0<J<1$ are then \textit{not} contained in those of the Lorentz group (that is generally so, it does \textit{not} lie on the light cone!). The latter are only representations of the \textit{in}homogenous Lorentz group insofar as all $p_k \equiv 0$, i.e. \?{the unit matrix can be taken for the (commutative) tranalations}. Bargmann has further shown (via a nontrivial trick) that $C=0$, $0 \le J < 1$ exists even for \textit{finite} Lorentz transformations (he had also doubted that for a very long time), \?{in the epsilontics however nothing happens}; incidentally one can even explicitly specify the matrices for finite Lorentz transformations. The point $J=1$ is not a singularity in these matrices themselves, only in the \?{connection} between the representations of the inhomogenous and homogenous group is $J=1$ singular. With that, this problem is probably substantially clarified.

Now to the meson theory. You have totally missed the mark with your remark that an addition of $\gamma^i\gamma^kF_{ik}\psi$ in the Dirac theory, or a similar addition $\mu F_{ik}A_k$ for spin 
, could make the scattering small. Oh no, that \textit{increases} all radiation effects (emission or Compton effect) of an order of magnitude (or even more) in $E/\mu c^2$ for light quanta woth large energies. That was more precisely worked out for the elasti   (in the center of mass system scattering of mesons and electrons (M\"oller approximation; "knock on" effect) by Corben and Schwinger (Physical Review 58, 953, 1940, issue from December 1st); for light emission it was just \?{crossed out} by two students in Berkeley. With spin $1/2$, that can already be seen on dimensional grounds, but it is also so with spin 1: the electromagnetic effects are all \textit{at least}, for magnetic moment 1, $e\hbar/2\mu c$ for both spins (i.e. spin $1/2$ Dirac equation without additional term, spin 1 Proca equation without additional term). In the latter case (spin 1) it has been shown that even this \textit{minimum} already gives a too-large burst-production by photons emitted by mesons. The 2 papers by Christy and Kusaka where this is shown have appeared in the Physical Review of March 1st, which also contains a little note by Oppenheimer, where (among other things) the possibility is discussed that the nuclear forces are explained by mesons of spin 0.

On this I would now like to say a bit more than is explicitly stated in this note. The long-circulated claim, that the empirical spin-dependence of the nuclear forces in the deuteron could only be explained by spin 1, is false. \textit{With spin 0}, we say cautiously, \textit{it doesn't go any worse}. In order to see this, consider the interaction energy
\uequ{
U = \text{const.} \sumXY{k=1}{4}\sumX{a=1,2,3}
  \Psi^*(x)i\beta\gamma^5\gamma^k\tau^{(a)}\Psi(x)\pddX{u^{(a)}(x)}{x_k}
}
where $\tau_1, \tau_2, \tau_3$ are the Hermitian isotopic spin matrices, $\beta=\gamma^4$, $\gamma^5=\gamma^1\gamma^2\gamma^3\gamma^4$, $\gamma^k$ are the Dirac matrices, $u^{(1)}$, $u^{(2)}$, $u^{)}$ are the three real pseudoscalar fields. The latter describe two charged and one neutral meson with spin 0 \comment{!!}. Such a theory without neutral mesons was treated by Kemmer (Proceedings of the Royal Society, 1938), the theory discussed here is "symmetrical". In the non-relativistic approximation approximation for heavy particles that becomes
\uequ{
U = \text{const.} \sumX{a=1,2,3}\Psi^* \vsigma \tau^{(a)}\Psi \pddX{u^{(a)}}{\vx}.
}

For the interaction of two heavy particles $\lblI$ and $\lblII$, with the abbreviation
\uequ{
\vx = \vx^\lblI - \vx^\lblII,\quad r = |\vx|,\quad
S_{12} \equiv \frac{3(\vsigma^\lblI \vx)(\vsigma^\lblII \vx)}{r^2} 
  - (\vsigma^\lblI \cdot \vsigma^\lblII),
}
one obtains the result
\uequ{
V(r) &= + |\text{const.}|^2 \sumXY{a=1}{3}(\tau^{(a)\lblI}\tau^{(a)\lblII})\cdot\left(
\vsigma^\lblI\pddX{}{\vx}\right)\left(\vsigma^\lblII\pddX{}{\vx}\right)\frac{\exp{-kr}}{r}\\
     &= + |\text{const.}|^2 \sumXY{a=1}{3}(\tau^{(a)\lblI}\tau^{(a)\lblII})\left\{
\exp{-kr}\left(\inv{r^3} + \frac{k}{r^2} + \frac{k^2}{3r}\right)S_{12} + 
\frac{k^2}{3r}\exp{-kr}(\vsigma^\lblI\cdot \vsigma^\lblII)\right\}.
}
For the states $({}^3S_1 + {}^3D_1)$ and ${}^1S_0$ of the deuteron this becomes
\uequ{
V(r) = -|\text{const.}|^2 \left\{\exp{-kr}\left(\frac{3}{r^3} + \frac{3k}{r^2} + \frac{k^2}{r}\right)
S_{12} + \exp{-kr}\frac{k^2}{r}\right\}.
}

For the singlet, $S_{12} = 0$ and the "tensor force" $S_{12}$ (\?{one says here}) distinguishes the singlet and the triplet. Naturally there must be a cutoff for small $r$, the $r$ dependence of tbis function is not to be taken seriously. The triplet state \?{correctly becomes the lower} and if one is liberal with the mangling of the $r$-dependent functions, everything can be brought into qualitative agreement. And to crown it all, even the sign of the electrical quadropole moment of the deuteron comes out correctly!

Now all of that is not to be taken all too seriously; the adopted Hamiltonian function seems suspiciously similar to the Konopinski-Uhlenbeck theory, with its arbitrary gradients, the higher approximations not small according to Stueckelberg-Patry, the manglings are considerable, etc. etc. Only: it is no worse than the spin-1 theory, and all difficulties with the electromagnetic radiation interaction fall away.

Now comes the point that will interest you: the scattering of charged mesons by protons (and neutrons) in this spin-0 theory is exactly as large as in the spin-1 theory, and so already for energies $\approx \mu c^2$ too large with respect to experiment. Hence I have thought of carrying Bhabha's higher charge states (higher spin is not needed!) to the spin-0 theory. For this reason your reflections on the influence on the nuclear forces of such excited photon states with all possible charges are again of interest for me. I also don't believe that Weisskopf's objection against your reflections can be very substantial (at the moment I don't know what Weisskopf, who is again in Rochester, thinks about this). Wigner very much agreed with your reflections in principle when I spoke with him about them. The question is however a \textit{quantitative} one: how high can one make the excitation energy of the new proton states (\?{primarily} charges $-e$ and $+Ze$) without disturbing the nuclear forces? And will the meson scattering be sufficiently reduced? I could not reproduce the quantitative results from your letter. Could the thing not be somehow \?{shifted around}? Naturally repulsion forces could be introduced for the $+Ze$ and $-e$ protons, but that would be rather artificial. I hope to hear more from you about this.

Incidentally: the discrepancy of the theoretical meson scattering with experiment is already present at smaller energies $E \approx \mu c^2$, where the damping forces could play no essential role. As concerns the latter, I have gathered from some brief remarks in a letter from Bhabha in Bangalore \?{that he had might have done something similar to your classical $\tau$-field theory}.

As concerns the journals here, they all arive here (German, Dutch, Englosh, etc), only the Helvetica Physica Acta does not. There must be some special stupidity from Birkh\"auser behind it. So please kindly send us any preprints of the papers from the Helvetica Physica Acta \?{that come out}. -- Unfortunately it is not entirely accurate that America is the promised land of physics, \?{since the experimental physics of this country have shown the strong influence of the war}.

Many greetings to Dr. $C + A = F$, I want to write to him next. Greetings as well to all colleagues (Wentzel, Scherrer, Stueckelberg). Please give your parents many regards from me, amd warm greetings to you yourself from my wife and me.

Your W. Pauli  

%p stao tk o a black sun bloody stool