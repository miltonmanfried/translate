\letter{631}
\rcpt{Bethe}
\date{April 4, 1941}
\location{Princeton}

\nc{\bp}{\mathbf{p}}
\nc{\bA}{\mathbf{A}}

Dear Herr Bethe!

Many thanks for your letter. We are here over Easter and look forward to a visit from you and your wife in Princeton. But please don't make it all too short, so that we can talk shop at leisure. Easter Sunday (the 13th) suits us particularly well. Please write more precisely on the time of your coming. I am also very happy about the other part of your letter, and am very much for publishing a table together. (Though it is true that you have done the whole calculation, but the method is still mine, so I need not have a bad conscience if my name is also on it.)

Yukawa's D case would then be interesting when in this special case the self-energy can be specified for arbitrary values of charge (not only for large charge). (However, this is still only \textit{academic}, since the term $(\bp, \bA)$ is omitted in the Hamiltonian.) -- If the factor $2/3$ applies \textit{generally} ()
r arbitrary $e$
 for transversal waves in centrally-symmetrical $D(x)$, I am certain that we could find a \textit{simpler} proof for it on Easter Sunday. (There are various useful formulae for getting from $\bA_\text{tr}$ from $\bA$.)
 
I have put aside the problem of the positron theory for strong coupling for the time being, but will probably take it up again in Ann Arbor. Incidentally it could well be that the \textit{academic problem} is a good preparation for the more difficult problem, and for that reason as well I am for publication. Please bring your calculations with you on Easter.

Now some remarks on the meson theory, which I have meanwhile regarded more closely, and about which I now also know more details than in my last letter. (Oppenheimer also wrote to me and to Corben on it.) I am also completely certain that one can get by with the 'symmetrical pseudoscalar theory' with no stronger requirements on it than you had used in your 1940-paper. \{The interaction Hamiltonian of this theory is 
$\sumXY{k=1}{4}\sumXY{a=1}{3}\Psi^*i\beta\gamma^5\gamma^k\tau^{(a)}\Psi\pddX{u^{(a)}}{x^k}$ 
with $\gamma^k$-Dirac matrices, $\beta=\gamma^4$, $\gamma^1\gamma^2\gamma^3\gamma^4=\gamma^5$, $\tau^{(0)}$ isotopic spin, $\Psi$ the wave function of the heavy particle, $u^{(a)}$ three pseudoscalar fields for 2 charged mesons and 1 neutral meson.\}

One need not necessarrily cut off every part of the potential energy at the same $r_0$. The sign of the deuteron quadropole moment then comes out \textit{correctly}. Thus I believe that the failure of your 1940 paper is only due to the hypothesis of a spin value of 1, and that the \textit{Bethe difficulties} of the meson theory don't exist when one assumes spin 0.

The deeper difficulties of this theory (non-convergence of the perturbation theory; too-large scattering of mesons, $\beta$-decay, etc) however still exist undiminished. I also hope to be able to talk that over with you on Easter. I also know something about the results of Millikan's measurements in Bangalore, which strongly point to primary protons in cosmic rays.

Newer gossip: Oppenheimer is expecting a baby. I put my hope in a new generation, which already at the birth of the 20th century said "njm, njm, it is not important!"

My wife and I expect you \textit{both} for Easter and send warm greetings. (You could then also admire our new car.)

Your W. Pauli

% simon the beaver