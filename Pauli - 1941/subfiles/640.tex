\letter{640}
\rcpt{Fierz}
\date{September 30, 1941}
\location{Princeton}

Dear Herr Fierz!

My wife and I had a very nice time in California. There I took a very active part in the calculations on nuclear forces with strong meson coupling. The result was very negative, as I had foreseen in my last letter. Then we \?{crossed out} the quantization of the spin $3/2$ theory in the presence of external electromagnetic fields. I found the notation of our theory of Schwinger and Rarita, which avoids spinor calculus, very nice.

We stopped for 4 days in Chicago and there I heard some lectures on cosmic rays. Must new experimental material has just come back from an expedition from Mt Evans and another from South America. But the evaluation will take some time yet. There is again a meeting in Chicago in mid-November which I should like to visit, and then I will be able to write to you more about it. With a three-day car ride from Chicago to Princeton we have then \?{concluded} our summer travels.

Luckily, the pre-print of your paper on the "classical theory of the scattering of charged mesons" arrived on the last day of my stay there, and as a consequence Oppenheimer has read it and wrote you a letter (a copy of this letter has just reached me). I would like to append some remarks to that here. It seems to me (you see, I still act as if you were still my assistant) that you should publish a second part of your paper, in which the following is laid out:

\subsection*{1. \?{The inertial terms of the charge degree of freedom are conserved}}

By "interial term" I understand terms in the equations for $\dot{\tau}$ which are proportional to $[\tau\times\ddot{\tau}]_\epsilon = \frac{d}{{dt}}[\tau\times\dot{\tau}]$. I have sent Bhabha a long airmail letter from Zurich in which I protested against the omission of this term. I was so successful that the corresponding terms for the spin $\sigma$ in the paper by Bhabha and Corben (Proceedings of the Royal Society \textbf{178}, 273-314, 1941; just out) with an undetermined multiplicative constant $K$ have again \?{come to light}. I hold that subtracting this term away is totally unjustified and an very much in favor of a return to the standpoint of \textit{Heisenberg's} paper (Zeitschrift für Physik, 1939) which first considered this term and correctly discussed its influence on the scattering. I.e. in your paper one should introduce a \?{form factor function} $K(x)$ (with $K(x){d^3x}=1$) into $(1, 2)$ of your paper and replace $(\Delta Z)_{x=0}$ by $\int\Delta ZK(x){d^3 x}$. Then it is no longer subtracted, neither in the expression for the charge nor in the expression for the energy. The cut-off length $a$ (proton radius
 is then defined by 
\uequ{
\inv{a} = \int\int\inv{|x-x'|}K(x)K(x'){d^3x}{d^3x'}.
}

Only Heisenberg's specific choice of the value of $a$ is inconvenient. (N.B. the 'symmetrical' theory, which \textit{also} introduces neutral mesons and \?{is analogous to spin coupling} gives somewhat simpler results than your theory with only charged mesons. But of course the latter is meaningful as well.)

\subsection*{2. The physical interpretation of free \?{precession motion}.}

I'm talking about the case you discussed on page 266, where the external fields vanish. By taking into consideration the inertial term, the equation for the precession frequency $\lambda$ becomes
\uequ{
\lambda = \left[\frac{g^2\lambda^2}{\mu^2a} + g^2(\mu^2 - \lambda^2)\right]^{3/2}(2\tau_\xi),
\quad\tau_\xi = \text{const.}
}
(You were missing the first term in the bracket.)

(N.B. In the 'symmetrical' theory there is still a subtractive term, which always practically compensates the \textit{second} term.) $\tau_\xi$ is the integration constant. If you calculate (without subtraction) the energy $E$ and the total charge number $N$ as a function of $\tau_\xi$, then $\lambda = \partial E/\partial N$. In the classical theory $N$ is continuous, but the translation according to the correspondence principle demands the restriction if $N$ to whole numbers. This interpretation of $\lambda$ is due to Oppenheimer and Schwinger. \textit{$E$ as a function of $N$ now directly gives the higher charge states of the proton according to your classical theory.} This generally tends to give similar results to the quantum theory, if the coupling is strong.

Since now "case I" leads to certain statements about the mass of the excited proton states, it seems to me that those \?{extra introductions} into the Hamiltonian function according to case II is senseless. So if you would conclude your new paper with some \?{justly} critical remarks on case II, it would be good.

I know well that this letter is rather \?{outrageously bossy}. But perhaps that is quite well.

So, actually write a second part of your paper, and warm greetings to you and your family from us both!

As always,

Your W. Pauli

%h_ bra  yceckse spim davis's husband's hat