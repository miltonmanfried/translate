\letter{661}
\rcpt{Wentzel}
\date{July 20, 1942}
\location{Lake Clear Junction}

Dear Gregor!

Your registered letter from January 16 with the manuscript reached me on May 20th (!), while I was holding a guest lecture at Purdue University in Lafayette (Indiana). Next to my office is Schwinger's, who has almost exclusively gone over to war physics, but nonetheless had sufficient time to talk shop. Shortly after this letter your preprint also arrived, and we were very happy at your \?{correction}, which we can now simply cite without constantly repeating the whole thing. Further, your letter of April 9th reached me on the 8th of July (!) (this record of slowness is previously unheard of with air mail; was your letter perhaps held up by the censor because of your report on Hg? However, please write nonetheless as soon as you learn anything new about Heisenberg. He has many friends here, especially at Purdue University, \?{where he was last time}.) I am still happier yet that you have meanwhile \?{accepted} the additivity of self-energy. Perhaps you can write a report on it. Now there is no longer any substantial difference in opinion; you will certainly soon accept that the "lattice space" is only a technical impediment and that everything is much simpler to calculate with the introduction of a finite size (length $a$) for the heavy particles instead of the lattice space. So why the complication?

Your contribution, having first introduced the "strong coupling", remains undiminished, and I gladly and wholly agree with your remarsk concerning this (\?{"Notice" etc to F. Klein}). (I find even the idea of your old work better than that of the later work with the infinitely-many constants, which are probably only to be regarded as a product of despair.)

I would now like to go rather deeper into the meson theory with integral spin with strong coupling. First, as regards Schwinger's paper, it discusses the 'charged scalar theory' (which also underlies your work) with finite extent for the 'nucleon' instead of the lattice space. I have read his manuscript, which is still missing only one or two pages, but I doubt that he will have it ready in the foreseeable future. Then there is my paper with Dancoff, which is in print and shall appear in the August issue of the Physical Review. Although it is restricted to the case of the "one-source problem" (the nuclear forces are still regarded as external; on this, see below), it has become quite long. We write the Hamiltonian of the "symmetrical pseudoscalar theory" in the form
\uequ{
H = \sumXY{\alpha=1}{3}\inv{2}\int \left\{
\pi_\alpha^2 + (\Delta\varphi_\alpha)^2 + \hx^2\varphi_\alpha^2\right\}{dV} -
\int\frac{g\sqrt{4\pi}}{k\sqrt{2}}K(x)\tau_\alpha \bsigma \pddX{\varphi_\alpha}{\bx}{dV}
}
$\pi_\alpha - \dot{\varphi}_\alpha$, $\varphi_\alpha$ real, $\mu$ = meson mass, $\hx = \mu c/\hbar$, $H=\text{energy}/\hbar c$, $g$ dimensionless.
\uequ{
\int K(x){dV} &= 1;\,\, K(x) = K(r), r =|x|, \text{spherically symmetrical};\\
\inv{a} &= \int\int K(x)\inv{|\bx-\bx'|}K(x'){dV}{dV'}.
}

In contrast to Oppenheimer and Schwinger, the isotopic spin $\tau$ and the usual spin $\bsigma$ are treated correctly as matrices, \?{while before these were taken as classical unit vectors}\footnote{They have only exactly calculated the "neutral pseudoscalar theory" and the "charged scakar theory.} (which in this theory does not read to quantitatively-correct results). In the interesting case $\hx a \ll 1$ the condition for strong coupling is $g^2 \gg (\hx a)^2$ and we find for this case the \?{excitation energy} $\Delta A$ of the 'nucleon' with spin $j$ and charge $n+1/2$ ($n$ and $j$ half-integral)
\uequ{
\frac{\Delta}{\mu c^2} = \frac{3}{2}\frac{\hx a}{g^2}\left[j(j+1) - \frac{3}{4}\right],\quad
\textit{$n$ is constrained according to $-j \leq n \leq j$}
}
(otherwise the \?{isobars} don't exist!), but in this region $\Delta E$ independent of $n$. (The latter is constrained by the isotropy of the Hamiltonian in the isotopic-spin-space $\tau$; in the 'charged pseudoscalar theory' (no 'neutrettos'), which we have also treated,
\uequ{
\frac{\Delta E}{\mu c^2} = \frac{3}{2}\frac{\hx a}{g^2}\left[2j(j+1) - n^2 - \frac{5}{4}\right]
}
with equal restrictions for $n$ as above.) For $j$ so large that $\frac{\Delta E}{\mu c^2} > 1$, there are under \?{certain} conditions meta-stable states, which according to a remark by Serber \?{could be interpreted} as intermediate products in meson production. The scattering cross section of mesons (with momentum $p$ and energy $E$ greater than $\mu c^2 + \text{the smallest} \Delta E$) become
\uequ{
{dq} &= \left(\frac{cp}{E}\right)^4 a^2 \frac{3}{4}(1+\cos^2\theta){d\Omega}\quad &\text{in the symmetric theory;}\\
& & \text{$\theta = $ scattering angle.}\\
{dq} &= \left(\frac{cp}{E}\right)^4 a^2 \frac{3}{4}(1+3\cos^2\theta){d\Omega} & \text{in the "charged" theory \textit{for} $p < \hbar/\alpha$}
}
The \textit{magnetic moment} gives rise to a great difficulty. Since namely in the strong coupling the nucleon emerges in the stationary states with charge 0 or 1 with equal probability $1/2$, the total magnetic moment of the neutron should be equal and opposite to that of the proton, which contradicts experience (Bloch-Alvarez experiment). \?{Only perturbation terms in the (small) parameter} $(\hx a)^3/g^2$ could change that. We currently know of no way out of this difficulty and it can be considered as an argument against strong coupling. You will find details in our paper, which contains the whole formalism (which is perhaps more important than the specific Hamiltonian function) \textit{in extenso}.

Now I come to the \textit{nuclear forces}, about which Dancoff and Serber are preapring a paper. It shall discuss the typical cases of the charged scalar theory and the neutral pseudoscalar theory, and there is hope that it shall soon be sent to the presses (although Serber has now gone over to war physics as well). In all strong coupling theories with integral meson spin (on pair-production see below), the \textit{coincidence between the interaction energy of the 'nucleon' and the excitation energy of the isobars} is decisive for this problem. In \?{regions} of heavy particles where the first is small with respect to the second, the forces are of the same type as calculated with perturbation theory with weaker coupling (only differing by a numerical factor which is equal to $1/2$ in the scalar theory and $1/9$ in the pseudo-scalar theory). If on the other hand the \?{region} of the heavy particles is so small that the interaction energy of the heavy particles is large with respect to the excitation energy of the isobars, the forces lose the exchange character (no more saturation) and there is also no longer a quadropole moment of the deuteron. Thus the latter case is in contradiction with experiment. (You have read the brief abstract on this by Serber.)

In the charged scalar theory, e.g. with the Hamiltonian
\uequ{
H=\int{dV}&\left\{\pi^*\pi + \Delta\varphi^*\Delta\varphi + \hx^2\varphi^*\varphi
- g\sqrt{4\pi}[(\pi_+^I K_I + \tau_+^{II} K_{II})\varphi\right.\\
&\left.(\tau_-^I K_I + \tau_-^{II}K_{II})\varphi^*]\right\}
}
for the nuclear forces
\uequ{
\int\varphi K_I {dV} = q_I \exp{-i\theta_I},&\quad
\int\varphi K_{II} {dV} = q_{II} \exp{-i\theta_{II}}\\
I = {dV}\,{dV'}\,K_I(x)K_{II}(x')\frac{\exp{-kr}}{r};&\quad
J = \int{dV}\,{dV'}\,K_I(x)K_{II}(x')\frac{\exp{-kr}}{r},\quad (r=|\bx-\bx'|)
}
(for \?{point sources} $J=\frac{\exp{-kR}}{R}$, $R$= distance of the heavy particles)\\
we arrive at the reduced problem (for $\hx a \ll 1$, $J<I$)
\uequ{
E=\frac{2\hx}{g^2}(p_{\theta_I}^2 + p_{\theta_{II}}^2)
- \inv{2}g^2[I + J\cos{(\theta_I - \theta_{II})}].
}
$(2\hx/g^2)(p_{\theta_I}^2 + p_{\theta_{II}}^2)$: Isobar excitation energy; I: self-energy) and wity $\theta_I = \theta + \psi$, $\theta_{II} = \theta - \psi$; $p_{\theta_I} = \inv{2}(p_\theta + p_\psi)$, $p_{\theta_{II}} = \inv{2}(p_\theta - p_\psi)$, $p_\theta = \text{total charge} - 1$,
\uequ{
E = \frac{\hx}{g^2}p_{\theta_I}^2 - \inv{2}g^2 I + E';\quad
E' = \frac{\hx}{g^2}p_\psi^2 - \inv{2}g^2 J\cos{2\psi}.
}
\{Eigenfunctions \?{unique} in $\theta$ and in $\psi$; $p_\theta + p_\psi$ odd;
\uequ{
u(\psi + \pi) = u(\psi - \pi) =(-1)^{p_\theta + 1}u(\psi)\}.
}

You likewise see that the \?{pendulum problem} with the Hamiltonian function $E'$ for $g^2 J \ll\frac{\hx}{g^2}$ leads to weakly-perturbed rotation with the zeroth-approximation eigenfunction $\cos{m\psi}$ and $\sin{m\psi}$ (where $p_\psi^2 = m^2$), while for $g^2 J \gg \frac{\hx}{g^2}$ there are oscillations about the equilibrium points $\psi=0$ and $\psi=\pi$ with frequency $\omega=2\sqrt{\hx J}$ (note additional zero-point energy, which is equal to $1/2\,\hbar\omega$.) (N.B.: Here I have neglected the dependence of the isobar energy on the distance, which can be taken into account if desired.)

A corresponding "freezing of the angle" with regard to the spin happens in the analogous case of the pseudoscalar theory and destroys the usual type of nuclear forces. -- This result from Serber for the scalar theory is naturally implicitly contained in the formulae of your old work, but there it is not clearly discussed, since there the term with $p_\psi^2$ is not \WTF{properly related}{put in the correct connection} with the term $\approx\cos{2\psi}$. -- From this it must be concluded that in reality the excitation energy of the isobars must be large with respect to theinteraction energy of the heavy particles, and indeed even for values of $R$ which are of the order of magnitude of the range of the nuclear forces. (As Schwinger has shown in a note in the Physical Review, a lower bound for this range is obtained from the empirical value of the deuteron quadropole moment.) \?{Further, with strong coupling, as soon as $\hx a\ll 1$, the term $\approx 1/R^3$ (which however loses its validity for $R < a$) gives rise with correct values of the nuclear forces to a too-small range.}

For this reason Schwinger has attempted to again revive the Rosenfeld-Møller patent mixture in a new form. First, he has pointed to the possibility that the rest masses $\hx_0$ resp. $\hx$ of the spin 0 resp. spin 1 be taken to be different, while $f=g_0/\hx_0\sqrt{2} = g_0/\hx_1\sqrt{2}$ (for the above definition of $g_0$) are according to Møller-Rosenfeld \textit{equal} for the two particles (ad hoc assumption). Then the usual perturbation theory yields for the force-potential inthe symmetrical theory with point sources:
\uequ{
V(R) = \left(\inv{3}\sumX{\alpha}\tau_I^\alpha\tau_{II}^\alpha\right)\left\{
\bsigma_I \bsigma_{II} J_0(R) + S_{III}J_1(R)\right\}
}
with
\uequ{
S_{III} &
 \frac{3}{R^3}(\bsigma_I\bx)(\bsigma_{II}\bx) - (\bsigma_I\bsigma_{II}),\quad
 \bx = \bx_I - \bx_{II}, R = |\bx|;\\
J_0(R) &= f^2\left[\hx_0^2\frac{\exp{-\hx_0 R}}{R} +2\hx_I^2 \frac{\exp{-\hx_1R}}{R}\right]\\
J_1(R) &= f^2\left\{\frac{3}{R^3}[(1+\hx_0 R)\exp{-\hx_0 R} 
 - (1+\hx_1 R)\exp{-\hx_1 R}] + \inv{R}[\hx_0^2\exp{-\hx_0 R} - \hx_1^2\exp{-\hx_1 R}]
\right\}.}

For small $R$, $J_0(R)$ as well as $J_1(R)$ have only the singularity $\approx 1/R$. While for $\hx_0 = \hx_1$ (with resting heavy particles resp. in non-relativistic approximation for the heavy particles) no quadropole moment arises for the deuteron, the correct sign is obtainex for $\hx_1 > \hx_0$, i.e. the spin-1 meson will be unstable (that is nice, since \textit{stable} spin-1 mesons are not seen experimentally). -- \textit{But of course now} the strong-coupling formalism ($f^2 \gg a^2$, $\hx a \ll 1$) \textit{must be applied to the mixture}. (The perturbation theory is only applicable for $f^2 \hx^2 \ll (\hx a)^3$, which is practically never fulfilled for $\hx a \ll 1$. On the inapplicability of the perturbation theory, see Stueckelberg-Patry.) \?{So much for Schwinger.}

I have shown that - disregarding a factor of $1/9$ -- the formula for $V(R)$ remain applicable even with strong coupling for those $R$ where $V(R)$ is small with respect to the excitation energy of the isobars. A very crude and provisional estimate (I still haven't worked out this isobar excitation energy for the mixture) seems to give that the \?{agreement in this case is right on the edge}, i.e. here the motion of the heavy particles must be taken into account. It seems worthwhile to be able to work through this problem in detail. (Schwinger himself has no time, but the Japanese Kusaka, who is now in Princeton, can work it out according to Schwinger's ideas.) I would like to hear your opinion on this, whether the patent mixture seems so artificial to you (because of its ad hoc assumptions) that your physical intuition bristles against such a theory. Here the majority have this attitude, Oppenheimer in particular is entirely against the mixture. On the other hand, I think it is possible that this mixturw theory could yet be reasonable (but only with \textit{strong} coupling, if at all). One could try to \?{retroactively justify} the provisional arbitrary assumptions by a limit $a \to 0$ and a transition to a theory that is also relativistic for the heavy particles. But it cannot be denied that the mixture theory contains, if not infinitely-many arbitrary constants, at least too many for my taste.

Now to the pair theory. It seems essential to take into account the spin-dependence of the interaction, and that means -- as you also remarked in your letter -- that the "exact" solvability was enforced at the expense of an essential characteristic of the interaction, hence \textit{the "exactly"-solvable cases are physically uninteresting.} Unfortunately I don't know any specifics from the Oppenheimer and Nelson paper, but rather only the result that for strong coupling the nuclear forces for $R>a$ become zero (our $a$ is your $A^{-1}$). This seems to coincide in essence with your proposal that cut-off length $\approx$ range of nuclear force. (N.B. I have meanwhile also considered the method of complex integration for averaging the energy, without having calculated anything with it.) Oppenheimer has now also gone over very strongly to the war physics, but as far as I know, Nelson is occupied with preparing the paper for completion.

From Jauch, first came a letter from Portugal, then from New York, thus he has arrived OK. (I have still not seen him.) I advises him to \?{get } to contact with
 Nelson, perhaps he can soon write you more on the subject. \?{Principally} I would like to say the following: the cut-off length (resp. the proton radius) $a$ is a misunderstood part of the theory, which is not deduced from the meson field. It is possible that the range of the nuclear forces is of the same order of magnitude as $a$, but that means that the nuclear forces would not be essentially constrained by the meson resp. pair particles. Then one could just as well assume that the heavy particles are hard spheres of radius $a$ as \textit{a priori} introducing some other force-\textit{Ansatz} between them with range $a$. Then why are the pairs even needed at all? I would say: \?{only a theory, where the $a$ is small with respect to the range of the nuclear forces, explains this at all from the field of intermediate particles}. The main problem is to decide whether a theory of the latter type is possible. If not, we must fall back to the former possibility and the new new problem would be to grasp the nature of $a$ and the value of the meson mass, which so far appeared rather hopeless. I still have the hope that $a$ is small with rspect to the range of the nuclear forces.

In recent days I have been occupied with the treatise by Dirac in the March issue of the Proceedings of the Royal Society, in which he proposes a peculiar formalism to eliminate the divergences in quantum electrodynamics. However, I would like to postpone the discussion on it until later, since Dirac and I are currently corresponding on an essential point. -- \?{This is roughly what I now know about physics, whose dimunition in the whole world I foresee to with meloncholy.} I am still in regular contact with Bhabha (airmail over South Africa-South America.)

How is Fierz? It's been a while since I've heard from him. Give him my pleasant greetings, perhaps this letter on the meson will interest him as well and you could show it to him, and reverse the procedure of previous years. Experiments on the production of mesons by Schein, Auger, Regener Jr continue, also experiments by Rossi on the lifetime of mesons. Who will succeed Hagenbach? Stahel?

Regarding my trip to Zurich I would like to saya that I really want to do it when technically possible. Your rapid conclusion of the one leg of your trip over the Atlantic to the way back shows me however that -- at least on April 5 1942 -- what I used to call "provincial ignorance" continues to exist unabated in Zurich. I have already writtem extensively to Rohn im March on the difference between the "regulatory limits" for the two \?{directions} (and others). \?{He} kindly answered my letter in April and promised to try to supply me with travel documents. Since then I have unfortunately heard nothing more of these travel documents, and since I am always awaiting further information in this regard, much time has been wasted. I have taken note of your assumption that Rohn's influence would be essentially greater in the organization of my trip back to Zurich than in my \?{application for citizenship}. If my power in tye E.T.H. has meanwhile really grown that much, it would make me very happy. Then I'll come back -- although in Ascona, i.e. in Canton Tessin, for reasons of Swiss military security I wouldn't live in a summer house like you or like I do here, but rather this Swiss security would only be guaranteed if I were to \?{live} outside the canton of Zurich in a hotel or a pension. So perhaps in September I'll swim across the Atlantic after overcoming the regulatory limits with the help of Rohn. I anxiously await the things to come, and meanwhile enjoy my vacation in a summer house. Many greetings to Scherrer, Weisskopf has reported to me by his sentimental letters. Has he recieved our postcard where we ordered him a stratosphere-aircraft for foreigners?

Warm greetings to you,and until I see or write you again

Your Wolfgang


% that eezerot  meby Dancoff  tformhe.  Kriegsphysik