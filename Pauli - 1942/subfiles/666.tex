\letter{666}
\rcpt{Wentzel}
\date{October 16, 1942}
\location{Princeton}

\nc{\be}{\mathbf{e}}
\nc{\bS}{\mathbf{S}}
\nc{\bL}{\mathbf{L}}

Dear Gregor!

Your letter of the 14th of September arrived here on the 14th of this month. Many thanks for interesting information about Rohn and Heisenberg. Please pass on my greetings when he comes to Zurich in December. \?{I am very interested in whether he also still pursues the "Dogmatik"}. Since certain "regulatory limits" also apply for the content of correspondence, I would be satisfied to express the hope that, in a not-too-distant future, and free of regulatory limits of all kinds, we may see eachother again -- and then go on to the more detailed scientific part of this letter.

You have unfortunately misunderstood something essential in my last report on the meson theory with stronger coupling as regards the magnetic moment. This theory is not intended to explain the difference of $1.79$ and $1.93$ nuclear magnetons, but "rather the entire difference of $\textbf{2}.79$ and $1.93$ magnetons. According to theory, the \textit{last} empirical numbers should be equally large, and that seems to be a clearer clue to the fact that the theory must be false. This can also be expressed by saying that the coefficient $\alpha$, introduced by Heitler, Fro\"olich, Kemmer \{Proceedings of the Royal Society \textbf{166}, 1938; c.f. particularly p. 175, equation (64)\} and which is proportional to $f^2$ in the weak coupling theory, \textit{becomes in the stronger coupling case equal to $1/2$ so that $\mu_N + \mu_P = 0$ results}. This follows from the fact that the expectation value $\bsigma$ is \textit{zero} in the stationary states, while the expectation value of $\tau_3\bsigma$ is equal and opposite for protons and neutrons. Hence the expectation value of the contribution of the bare nucleus
\uequ{
\bM' = \frac{e\hbar}{m_P c}\inv{2}(1+\tau_3)\bsigma
}
to the magnetic moment for the proton and the neutron in this theory \?{also cancel out}. That the same applies for the \?{portion} of the meson field is not surprising. The latter is again proportional to $1/a$ as in the perturbation theory.

In the deuteron, it is again analogous: the whole magnetic moment of the ground state should consist in the\?{orbital angular momentum} that arises from the admixture of the ${}^3D$-eigenfunction, which is certainly too strong.

Do you honestly think that the \textit{entire} difference between $|\mu_N|$ and $|\mu_P|$ can be explained by \?{electromagnetic perturbations}? I don't think so and for this reason have been thinking about the possibility of another theory.

But before that, more detail on the two-nucleon problem in the mixture theory with \textit{strong} coupling. The Hamiltonian function is (in units $\hbar=c=1$; dimensions of $f$ are cm; $i=1,2,3$; $\alpha=1,2,3$)
\nequ{
H&=\inv{2}\sumX{i,\alpha}\int\{(\pi_i^\alpha)^2 + (\grad\Y_i^\alpha)^2 + \hx_1^2(\Y_i^\alpha)^2
 + (\pi^\alpha)^2 + (\grad\y_i^\alpha)^2 + \hx_0^2(\y^\alpha)^2\}{dV}\\
&- f\sqrt{4\pi}\sumX{\alpha}\int{dV}\{\tau_\alpha^I K^I(x)
\left[(\sigma^I \curl\y)+(\sigma^I\grad\y)\right]\\
&+ \tau_\alpha^{II}K^{II}(x)\left[(\sigma^{II}\curl\y) + (\sigma^{II}\grad\y)\right]\}.
}{1}

Here $\y_i^\alpha$ is the vector field, $\y$ the pseudo-scalar field, $\pi_i^\alpha$, $\pi$ their time derivarives; $\tau$, $\sigma$, $K$ are the isotopic spin, spin and source function of nucleon I resp. II, the latter normalized according to $\int K(x){dV}$\footnote{The $g_1$-interaction of the longitudinal vector mesons is only omitted for simplicity.}.

The condition for strong coupling, under which the following applies, is in physically interesting cases $\hx_0 a \ll 1$ and $\hx_1 a \ll 1$.

Here the radius of the nucleon is defined by
\uequ{
1/a = \int\int K(x)\inv{|x-x'|}K(x'){dV}\,{dV'}.
}

In the one-nucleon problem, the excitation energy of the isobars is $1/3$ that of the pseudoscalar theory, namely
\uequ{
E=\frac{a}{4f^2}\left[s(s+1) - \frac{3}{4}\right],
}
where $s$ denotes the (half-integral) spin of the excited nucleon.

Just as in my paper with Dancoff, the description of the field of the bound meson follows with the help of a moving orthogonal axis $\be_\alpha$ ($\alpha=1,2,3$) which satisfies the relation $(\be_\alpha \be_\beta)=\delta_{\alpha\beta}$, and the angular momentum $\bS$ which fulfills the well-known commutation relations
\uequ{
i[S_1, S_2] = -S_3,\dots; [e_{\alpha i}, e_{\beta k}] = 0\\
i[S_1, e_{\alpha 1}] = 0,\quad i[S_1, e{\alpha 2}] = i[e_{\alpha 1}, S_2] = -e_{\alpha 3}.
}
$\be_\alpha$ can be expressed by three Eulerian angles, and $S_k$ by the corresponding differential operators. But one could also introduce the half-integral quantum numbers $m$, $n$, $s$ defined by the eigenvalues $S_3=m$, $(\bS e^{(3)})=n$, $S^2 = s(s+1)$ where $-s \leq m \leq s$, and then all $e_{\alpha i}$, $S_k$ can be represented by matrix elements with respect to these quantum numbers. For those \?{submatrices} the $\be_\alpha$ which are diagonal with respect to $s$ give in the special case $s=\inv{2}$
\uequ{
\left(\inv{2}|\be_\alpha|\inv{2}\right) = \inv{3}\tau\bsigma,
}
while the full matrices of the $\be_\alpha$ commute with one another. The meaning of the quantum number $n$ is generally the charge number $\inv{2}$.

For two nucleons, all of these exist for both sources I and II. The problem of the usual wave mechanics, to which this \textit{Ansatz} leads under the given assumptions in sufficient approximation, is characterized by the Hamiltonian function
\nequ{
H = \inv{M}\left(p_R^2 + \frac{\bL^2}{R^2}\right)
&+ \sumX{\alpha}\inv{3}(\be_\alpha^I + \be_\alpha^{II})J_0(k) 
 + \sumX{\alpha}\inv{3}\left[\frac{3}{r^2}(\be_\alpha^I\bx) 
 - (\be_\alpha^I\be_\alpha^{II})\right]J_1(k)\\
&+ \frac{a}{4f^2}\left[S^I(S^I + 1) + S^{II}(S^{II} + 1) - \frac{3}{2}\right].
}{II}
Here $\bL$ is the orbital angular momentum of the relative motion; $\bx = \bx^I - \bx^{II}$, $r = |\bx|$, $M = \text{proton mass} = \text{neutron mass}$.
The wavefunction must be antisymmetrical when $\be_\alpha^I,\bx^I$ is swapped with $\be_\alpha^I,\bx^{II}$. As in my last letter, following Schwinger, we have
\uequ{
J_0(r) &= f^2\left[\hx_0^2 \frac{\exp{-\hx_0 r}}{r} + 2\hx_1^2 \frac{\exp{-\hx_1}}{r}\right]\\
J_1(r) &= f^2\left\{\frac{3}{r^3}\left[(1 + \hx_0 r)\exp{-\hx_0 r}
 - (1 + \hx_1 r)\exp{-\hx_1 r}\right] + \inv{r}\left(
 \hx_0^2\exp{-\hx_0 r} - \hx_1^2\exp{-\hx_1 r}\right)\right\}.
}

Kusaka has already zealously tackled this problem, which gives rise to quite amusing "term zoology", and after the previous results it is very likely that by a suitable choice of constants, all empirical data on protons, neutrons and deuterons other than the magnetic moment can be explained. It is then necessary that the excitation energy of the next isobar be large with respect to the binding energy of the ground-state deuteron. It is about $\Delta E \approx 20 MeV$ for $\hx_0 a \approx 0.1$ In this case it is seen that the excited states of the nucleons give only a small perturbation to the ground state of the deuteron and hence can be considered as a perturbation with respect to the matrix elements of $\be^I$ and $\be^{II}$ (which are not diagonal with respect to $s^I$ or $s^{II}$), \WTF{in which it must be expanded}{nach der entwickelt werden darf}.

That there are no further problems with the deuteron (if we disregard the magnetic moment for now) still doesn't mean that such a theory is possible. Of course it seems very likely that for the higher nuclei, if an interaction like (II) is posited for every pair of nucleons, Fierz's objections (Helvetica Physica Acta, \textbf{14}, 105, 1941) are again revived, although the excited states are not here "\?{God-given}". Some have repulsive forces, some have attractive forces between one another, but nevertheless the result is similar. Otherwise, the Dancoff-Serber "Freezing of the angle", with loss of the exchange-character of the forces, has a very close qualitative connection to Fierz's reflections. For this reason I would very much like to hear Fierz's opinion on the qualitative consequences of a theory of higher nuclei based on (II).

Now the other possibility. I came to it through a comparison with the classical theories ($\tau_\alpha$ and $\bsigma$ are classical unit vectors in this) of Fierz (Helvetica Physica Acta \textbf{14}, 257, 1941) and Bhabha (Proceedings of the Royal Society \textbf{178}, 314, 1941). The recently-introduced constant $K$ (\?{spin inertia}) is connected with the radius $a$ following from the model of the spatially-extended source according to the equation
\nequ{
K = \frac{2f^2}{3a}
}{1}
\{the quantity $\beta$ in Bhabha, equation (73) l.c. is $1/a$; I always use units $h = c = 1$ \} and it is of course the same constant $K$ or $a$ which on the one hand determines the excitation energy of the isobars (free precession motion) and the scattering cross section of the meson. In the aforementioned theories however this $a$ does not at all mean the size of a source, but rather is an arbitrary characteristic constant of a point source. The singular part of the field of this point source is subtracted off. \?{Specifically, if $K=0$, and so $1/a=0$, is a simple logical possibility, the above inequality $\hx a \ll 1$ is no longer important, one nevertheless has a point source}. I have then asked myself how this distinguished case can be translated into quantum mechanics, and found that the "Wentzel-Dirac $\lambda$-process is particularly suited as technical aid for this, \?{since it causes all of the $1/a$ terms to vanish}. Jauch is working on this now any everything is going smoothly. As long as $(f\hx_0)^2 \ll 1$ and $(f\hx_1)^2 \ll 1$, there are no stable isobars in this theory and the perturbation theory in $f^2$ is allowed in the two-nucleon theory as long as $r^2 \gg f^2$ (I've recently written in more detail to Stueckelberg; the theory is convergent as long as the nucleons can be considered as resting). What do you think of this theory?

Although I have fought rather violently against this possibility $\hx=0$ in my last letter to Fierz (about a year ago), I have now remorsefully come back to it and now \?{favor it}. Heitler and Peng (Proceedings of the Cambridge Philosophical Society, \textbf{37}, 291, 1941) have developed a method to take account of the damping in the quantum mechanical calculation of the scattering cross section and find (similar to the aforementioned papers of Bhabha and Fierz) for the scattering cross section of the pseudoscalar mesons with the Hamiltonian function (I) used here
\uequ{
q = 4\pi\frac{3f^4p^4}{E^2 + (3f^2p^3)^2}
}
with $p$ = momentum, $E$ = energy of the incoming meson of units with $\hbar = c = 1$. (The nucleon is assumed to be resting.) I had earlier believed that this formula was contradicted by the experimental meson cross section, since for slow mesons it gives a greater cross section than experiment, but Heitler and Bhabha dispute this, since the dependence of the scattering cross section on the energy is very badly defined at just these small energies\footnote{P.S. Have discussed this with Rossi. Experiments unsatisfactory!}. I have not yet worked out the influence of the difference in the two masses $\hx_0$ and $\hx_1$ on the scattering. (In the derivation of the formula for $q$, it was assumed that $\hx_0 = \hx_1$.)

The value for the magnetic moment of the proton and the neutron is which arises from this theory is still open. The diverging term $1/a$ in the formula of Heitler, Frohlich and Kemmer (1938) for this moment will of course be made finite by the $\lambda$-process. I want to work on this with Jauch; it is rather more complicated, since the recoil of the nucleons in the intermediate states must be taken into account.\footnote{This is done. Results not very convincing!}

The $\lambda$-process still seems to show that the case $K=0$ ($1/a=0$) is physically distinguished. On the other hand, it leads to difficulties when two particles come closer to one another than $\lambda$ and that is also the reason why it fails in the hole theory, where there are always many pairs in the vicinity of a charged particle. On Dirac's idea of negative energy photons, another time.

I have stilk had little contact with the American physicists, apart from the foreigners here. Dancoff has still not completed his paper with Serber on nuclear forces \{P.S. Meanwhile appeared! \} and Nelson has still not been able to hear anything about the pair theory -- not to mention Schwinger. Thus meanwhile I have been trying, \?{at least in writing, to play the role of Bacchus on several islands of Naxos, one of which is Switzerland}. -- How is Fierz? Is the position in Basel improved by Hagenbach's resignation? Does he already have "Private-docentitus"?

Greetings to Scherrer and the institute, actually he could write me again. What "does he find now"? Have you heard 

ything from or about Bohr?

Best greetings to you from

Wolfgang.%turbat no l'rz l attewang