\letter{667}
\rcpt{Wentzel}
\date{November 12, 1942}
\location{Princeton}

\nc{\br}{\mathbf{r}}
\nc{\bX}{\mathbf{X}}

Dear Gregor!

I hasten to answer your letter of October 2nd on account of the possibility of an interruption of the postal connection to Switzerland, which I would regret very much. It is a shame that you could not also treat the case $A<0$ (but for \textit{this} case the Fermi approximation might not be satisfactory). It namely seems to me that the pseudoscalar theories \textit{and} the mixture theories are more analogous to the case $A<0$. The "reduced" Hamiltonian function can be described as follows:

Three mutually-orthogonal unit vectors $\be^\alpha$ are introduced $(\be^\alpha\,\be^\beta)=\delta_{\alpha\beta}$ which can be described by Eulerian angles (in my paper called $a,b,c$, here $\vartheta,\varphi,\psi$);

\uequ{
\be^1 &= (\cos\vartheta \cos\varphi \cos\psi - \sin\varphi \sin\psi, 
  \cos\vartheta \sin\varphi \cos\psi + \cos\varphi \sin\psi, -\sin\vartheta \cos\varphi)\\
\be^2 &= (-\cos\vartheta \cos\varphi \sin\psi - \sin\varphi \cos\psi,
  -\cos\vartheta \sin\varphi \sin\psi + \cos\varphi \cos\psi, + \sin\vartheta \sin\psi)\\
\be^3 &= (\sin\vartheta\cos\varphi, \sin\vartheta \sin\varphi, \cos\vartheta).
}
The eigenvalue of $p_\psi$ determines the charge quantum number $n$ (= charge $- 1/2$), those of $p_\varphi$ the components of the \?{internal angular momentum} in the fixed spatial $x_3$-direction. The total square of the angular momentum $\uwave{S}^2$ (with eigenvalues $s(s+1)$, where $s$ is half-integral, can be written
\uequ{
\uwave{S}&=p_\vartheta^\dagger + (p_\psi - \cos\vartheta p_\varphi)^2/\sin^2\vartheta + p_\varphi^2\\
&=p_\vartheta^\dagger p_\vartheta + (\cos\vartheta p_\psi - p_\varphi)^2/\sin^2\vartheta + p_\psi^2\\
p_\vartheta = -i\pddX{}{\vartheta},&\quad p_\varphi = -i\pddX{}{\varphi}, \quad
p_\psi = -i\pddX{}{\psi},\\
p_\vartheta^\dagger &= \inv{\sin\vartheta} p_\vartheta \sin\vartheta.
}

For several particles, each of these quantities ($\vartheta$, $\varphi$, $\psi$, $\be^\alpha$, $S$) must have the number of the index identifying the particle appended to it.

The Hamiltonian function for two particles ia given by ($\hbar=c=1$; $[f]=\text{cm}$, usual units, \textit{not} Heaviside; $\bx = \bx^I - \bx^{II}$, $\br = |\bX|$)
\uequ{
H = H_\text{Translation} &+ \frac{a}{4f^2}\left[\uwave{S}^2 - \frac{3}{4}\right]_{I+II}\\
& + \sumXY{\alpha=1}{3}\inv{3}J_0(\br)(\be_\alpha^I\be_\alpha^{II})
 + \inv{3}J_1(\br)\left[\frac{3}{r^2}(\be_\alpha^I\bx)(\be_\alpha^{II}) - \be_\alpha^I\be_\alpha^{II}\right].
}
\{This factor $\left[\frac{a}{4f^2}\right]$ is correct for the patent mixture; for the pseudoscalar theory, it is three times greater.\}

$J_0(r)$, $J_1(r)$ are specified in my last letter and are \textit{both positive}, for the mixture as well as for the pseudoscalar theory. The "neutral" theory is obtained instead of the symmetrical theory if instead of $\sumXY{\alpha=1}{3}$ only $\be^{(3)}$ is used and $p_\psi$ is replaced by $1/2$. The interaction energy is a minimum when $\be_\alpha^I = \be_\alpha^{II}$ and one of the three $\be_\alpha$ 


chosen to be parallel to $\bx$. (N.B. the corresponsing expression for the interaction in the corresponding theory with \textit{weak} coupling is given when $\be_\alpha$ is replaced by $\tau_\alpha \bsigma$). Because of the preference for the \textit{anti}-parallel \?{configuration} of $\be_\alpha^I$ and $\be_\alpha^{II}$, the Fermi approximation might not be satisfactory. But it seems certain that no saturation can be expected and (as Serber has claimed) in this case as well there is a problem. For $\hx a \approx 1$ the excitation energy of the next isobar cannot be made greater than about 20MeV, in order to remain in harmony with the correct magnitude of the nuclear binding energy of the deuteron.

Not another remark on the (here ignored) $r$-dependence of the excitation of the isobars. Last winter Dancoff and I did more precist calculations on this. If $\chi_I$, $\chi_{II}$ are the Yukawa potentials of the two (extended) sources, then $\int\chi_I\chi_{II}{dV}$ enters into the scalar theory \?{in conflict with} $\int\chi_I^2{dV}$ and the correction is considerable when $\hx r \approx 1$. But in pseudoscalar theories $\int\pddX{\chi_I}{x}\pddX{\chi_{II}}{x}{dV}$ enters in conflict with $\int\left(\pddX{\chi_I}{x}\right)^2{dV}$ and the correction is completely negligible for $r\gg a$.

If you could land a final "knockout-blow" against the strong coupling theory via a continuation of the considerations of the type in your letter on higher nuclei, I would be very happy. I am rather convinced of the correctness of the \textit{other} path (which I described in my last letter).

Your news on my professorship has interested me, because I received no information that this remained open to me \textit{until the end of the \?{selection period}}, but rather only received a telegram from Rohn that my leave was again extendes through the end of March 1943. Is your information authentic?  It is Rohn's practice to inform me of as little as possible. \?{Perhaps something could be done there.}

If only the post continues running and this letter arrives. Everything is going well (I hope it does in Switzerland as well!). Greetings to all the physicists.

As always,

Wolfgang

% Besteros