\letter{699}
\rcpt{Jauch}
\date{August 22, 1944}
\location{Saranac Lake}

Dear Herr Jauch!

After reading the last letter from Lopes more closely, a great idea occured to me against the correctness of the operators he derived to take account of recoil. Namely, this operator is \textit{not} Hermitian! (That means that for his integral kernel $F$, $F(Q, Q') \neq F(Q', Q)$ because of the form of the denominator.) This seems to be mathematically nonsensical, since Hamiltonian operatora must always be Hermitian. Unfortunately it is assumed that this is not due to a simple arithmetical error, but rather to an inconsistency in the perturbation theory beivng utilized.


I recall that a similar difficulty was discussed by Hulme (Proceedings of the Royal Society) when he generalized the original Møller method to inelastic collisions (with energy loss by radiation). (The difficulty does not occur with elastic collisions.) I would be very happy if you would write me with your opinion on this matter as soon as possible. \?{The whole calculation was there (and also your own from last year), but I don't have it here}. Hence at the moment I don't know \textit{how} to set the thing in order (merely symmetrizing the kernel $F$ with respect to $Q$ and $Q'$ is perhaps too simple).

Please tell Hu that I don't care that he signed the letter from Lopes, so long as he doesn't give his own contribution to this paper. 

Many greetings,

W. Pauli

%1 gambit