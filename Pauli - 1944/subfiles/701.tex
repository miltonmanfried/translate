\letter{701}
\rcpt{Jauch}
\date{August 29, 1944}
\location{Saranac Lake}

Dear Herr Jauch!

Your letter of the 26th has brought us one essential step further with your result that the correct integral operations emerge from Lopes's by simple Hermitization. Then the indeterminancy that you found seems to me to be harmless and trivial, as I hope to make clear in the following considerations. Assume that the form $\Hbar = H_0 + f^2 V + \dots$ is already established.

Apparently it remains unchanged under the unitary transformation $\overline{\Hbar} = W\Hbar W^{-1}$ with $W W^\dagger = 1$, if $W$ has the form $W = I + f^2 W_2 + W_2 + W_2^\dagger = 0, \dots$. That is all usual wave mechanics; in our case $W$ is only connected to states where no mesons are present. $W$ can be chosen to be convenient for integrating the Schr\"odinger equation, e.g. $W=I$ (i.e. $W_2=0$). (Your P.S. seems incorrect to me, since to be consistent one should put $S=\exp{fT_1 + f^1T_2 + \dots}$ and not $\exp{fT_1}$.) It is well-known that the physical results of wave mechanics ar unchanged under unitary transformations. Just here it is \textit{not} required to bring a given operator into diagonal form!

I have now carried through the idea laid out in my letter to Lopes, \?{the question of the regularity of integral operators leads back to their behavior for large $\mathbf{p}$ (more precisely: large $\mathbf{k} = \mathbf{p} - \mathbf{p}'$ at fixed $\mathbf{p}+\mathbf{p}'$) -- which I based on the now-Hermitized operators}. There I got a great surprise: the behavior of the integrands for large $\mathbf{k}$ is of the \textit{same} order whether the recoil is taken into account or not. (Namely, const. for the vector or pseudoscalar theory, which according to the "$(n-3)$-rule" mentioned in my letter to Lopes corresponds to a $1/r^3$ singularity.) Please verify this important point with Lopes. I had always tacitly assumed \?{that the recoil acts like a cutoff on the integrands} fin momentum space) for $p \approx M$ (proton mass). That seems to be entirely incorrect. In this case it is very doubtful whether taking waccount of the recoil can at all improve the relativistic calculation of Hulth\'en. What do you think? -- Then you would be correct that one can initially calculate the Schwinger mixture non-relativistically with recoil, since in the other cases the singularities remain.

Meanwhile I received a letter from Hu in which he claimed that when eliminating the "small terms" still further additional terms occur (also ignoring recoil) which aren't in Rosenfeld-Møller and Hulth\'en. What do you think of that? It seems rather fishy to me, but is nevertheless perhaps true. I would propose to put off discussing this question until we can discuss it together. Please tell Hu that I have received his letter and propose that he should wait in order to make further calculations (like the quadropole moment, etcon it.



My return to Princeton will fall exactly in your ten-day vacation. On which day of the week is the seminar in Princeton? Is it now held regularly?

I always read the book review in the Sunday number of the New York Times and had also read the review of Huxley's new book. How the book really is, I can hardly tell from it. Many thanks for the parcel. My wife and I thank you for your efforts in the csse of Hans Baer. Hopefully he passes his exams.

Many greetings,

Your W. Pauli

%ookZoftig