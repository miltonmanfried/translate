\letter{708}
\rcpt{Wentzel}
\date{November 27, 1944}
\location{Princeton}

Dear Gregor!

It has been more than two years since the post from here to Switzerland was interrupted. I had written you at least two letters \?{which were not delivered}; in one, the importance of the condition $V(r)>0$ for the stability of the higher nuclei was stressed. Meanwhile the Helvetica Physica Acta has come again, with the numerous papers by you and Fierz. The last, which I have read, is your paper on the deuteron problem II. I was especially interested by the paper by Coester, for which I had already been prepared by your letter to Heitler of April 17, 1944. I am no longer so sure that the strong-coupling-theory is to be rejected. The present situation seems to be that \textit{all} theories have great difficulties. The magnetic moment is wrong in the strong coupling theory as well as in the $\lambda$-limiting-process theory. Whether your way out can lead to a sufficiently large effect in the former, I don't know. In the latter, $1/a=0$ (no spin inertia, although point sources). But that has the consequence that the magnetic moment of the proton is smaller than 1, and that of the nucleus has the wrong sign, as Jauch has shown. The solution put forward in Jauch's paper of additional terms in the Hamiltonian function seems too arbitrary to be acceptable. Rather, it is my impression that the empirical magnetic moment for a medium coupling is approximately $a_\mu \approx 1$.

As regards the charge/mass ratio for heavy nuclei on the basis of the strong coupling theory, Kusaka and I have laid the foundations for Fierz's old estimate, according to which $\epsilon$ must be $\approx 100\text{MeV}$, so that the Uranium nucleus must be stable with respect to $\beta$-radioactivity. But I admit that the case in the strong coupling theory is more \?{propituous}, since with heavy nuclei the interaction energy is \?{partly averaged out}. In Costler's calculations, however, the interesting case $P^2/2Ma \approx 1$ and $g^2/4\pi^2 \approx 0.1$ (according to me and Kusaka, this value is important for getting the binding energy of the deuteron) falls exactly between his approximations. So I wouldn't rule out getting something satisfactory out of his calculations. Incidentally, with strong coupling the neutral pseudoscalar theory leads to an energetically-favorable arrangement of the nuclei in a plane with the spins perpendicular to this plane and parallel to one another, and energy proportional to the nucleon number. That does not naturally emerge from the Thomas-Fermi approximation. In the symmetrical pseudoscalar theory however this problem does not seem to enter.

At the moment I am most interested in the anisotropy of the neutron-proton scattering for higher energies. There is an important experiment by Amaldi and collaborators (Naturwissenschaften \textbf{30} 582, 1942), according to which a pronounced forward scattering appears at energies from 12 to 14 MeV. If that is true, it contradicts at least all symmetrical weak coupling theories. The anisotropy of the N-P scattering is essentially determined by the potential of the P-term. Repulsion forces in the P-terms give back-scattering and vice-versa. That is why I was interested by the remark in your paper according 

 which in the strong coupling theory the P-terms, at least at small distances, have an attractive potential. Do you know anything more quantitative on the anisotropy of the N-P scattering with strong coupling? I have put this question to Lopes, a Brazillian doctor, but \?{he has only just started.} I almost believe that you will already know the result when he is ready. -- Hulth\'en has worked much on this question on the basis of the weak coupling theory, but has put only forward very artificial solutions.
 
That is everything that I know at the moment. In my last paper, which appeared in 1943, I have shown that the $\lambda$-limiting-process is a special case of a general class of models, which also encompasses extended sources. This class is characterized by the fact that the weight function $G(k)$ need not necessarily be positive-definite. But as far as I know, relativistic invariance has only been proven for the $\lambda$-process. Do you know anything about whether there is a \textit{relativistically-invariant} cutting-off formalism in quantum theory in which the spin inertia, i.e. $1/a$ can assume an arbitrary positive value? I have probed this some, but found nothing.

Jauch is interested in pair theories with spin-dependent 

upling, but I have the feeling that these theories are too complicated to be true.

I no longer believe in Dirac's theory of photons with negative energies. Eliezer (a Hindu and a student of Dirac, now in Ceylon) has shown that the self-energy becomes infinite in this theory as well in the $e^4$-approximation. In addition, it gives incorrect values for the scattering of many photons of small frequency. (The same also applies for the theories of Stueckelberg and Heitler.) Thus at the moment it seems that there are no contradiction-free relativistically-invariant cutoff-prescriptions for field quantization. (The transition to static fields is included with 'contradiction-free', for which the old problem of the emission of photons of small frequencies is a good 'test-case'.)

Greetings to Fierz. A letter from him to me \?{had taken a year} and arrived in September 1943. It seems that a crucial sign in his calculations is wrong. When I find time, to go through it again in more detail, I will write him. -- I have seen the papers of Scherrer, Bradt and Heine. It really looks as if a new positive particle exists. Stronger sources of artificial radioactivity must be made with the cyclotron and then the attempt repeated with it. That is still not yet possible here. Is the cyclotron running 

 Zurich? What is the present state of this problem?
 
It is nice that you were happy with my official letter to Rohn. At the moment I write this letter, the technical travelling situation has not changed much for me. I look forward to the coming developments with great interest. Who is now the rector at E.T.H.? And who is head of department IX? When will Rohn retire?

Many greetings to Scherrer\footnote{I found his long letter from Summer of 1942 very nice; it was no longer possible to answer him.g and the same to the whole physical institute of r } E.T.H.

Warm greetings,

from Wolfgang.

P.S. On December 8 I am going to a celebration in Pittsburgh for Stern's Nobel prize. Bloch and Rabi will probably also be there, hopefully it will be amusing.%  swamp=