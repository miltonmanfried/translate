\documentclass{article}
\usepackage[utf8]{inputenc}
\renewcommand*\rmdefault{ppl}
\usepackage[utf8]{inputenc}
\usepackage{amsmath}
\usepackage{graphicx}
\usepackage{enumitem}
\usepackage{amssymb}
\usepackage{marginnote}
\newcommand{\nf}[2]{
\newcommand{#1}[1]{#2}
}
\newcommand{\nff}[2]{
\newcommand{#1}[2]{#2}
}
\newcommand{\rf}[2]{
\renewcommand{#1}[1]{#2}
}
\newcommand{\rff}[2]{
\renewcommand{#1}[2]{#2}
}

\newcommand{\nc}[2]{
  \newcommand{#1}{#2}
}
\newcommand{\rc}[2]{
  \renewcommand{#1}{#2}
}

\nff{\WTF}{#1 (\textit{#2})}

\nf{\translator}{\footnote{\textbf{Translator note:}#1}}
\nc{\sic}{{}^\text{(\textit{sic})}}

\newcommand{\nequ}[2]{
\begin{align*}
#1
\tag{#2}
\end{align*}
}

\newcommand{\uequ}[1]{
\begin{align*}
#1
\end{align*}
}

\nf{\sskip}{...\{#1\}...}
\nff{\iffy}{#2}
\nf{\?}{#1}
\nf{\tags}{#1}

\nf{\limX}{\underset{#1}{\lim}}
\newcommand{\sumXY}[2]{\underset{#1}{\overset{#2}{\sum}}}
\newcommand{\sumX}[1]{\underset{#1}{\sum}}
%\newcommand{\intXY}[2]{\int_{#1}^{#2}}
\nff{\intXY}{\underset{#1}{\overset{#2}{\int}}}

\nc{\fluc}{\overline{\delta_s^2}}

\rf{\exp}{e^{#1}}

\nc{\grad}{\operatorfont{grad}}
\rc{\div}{\operatorfont{div}}

\nf{\pddt}{\frac{\partial{#1}}{\partial t}}
\nf{\ddt}{\frac{d{#1}}{dt}}

\nf{\inv}{\frac{1}{#1}}
\nf{\Nth}{{#1}^\text{th}}
\nff{\pddX}{\frac{\partial{#1}}{\partial{#2}}}
\nf{\rot}{\operatorfont{rot}{#1}}
\nf{\spur}{\operatorfont{spur\,}{#1}}

\nc{\lap}{\Delta}
\nc{\e}{\varepsilon}
\nc{\R}{\mathfrak{r}}

\nc{\Y}{\psi}
\nc{\y}{\varphi}

\nf{\from}{From: #1}
\nf{\rcpt}{To: #1}
\rf{\date}{Date: #1}
\nf{\letter}{\section{Letter #1}}
\nf{\location}{}

\title{Pauli - 1948 - July}

\begin{document}

\letter{TODO -962}
\rcpt{Jordan}
\date{July 13, 1948}
\location{Zurich}
\tags{affine field theory,classical relativity}

Dear Herr Jordan!

I would like to answer your letter of 6/21, since I have been rather busy with the so-called affine field theory.

\letter{993}
\rcpt{Pais}
\date{December 26, 1948}
\location{Zurich}
\tags{meson theory, scalar particles}

Dear Pais!

Many thanks for your letter from December 9. -- Your results for the four-current-density for the spin-0 particles can be very useful for us. However, I am not happy about the subtraction formalisms. The essence is that, even \textit{after} "renornalization" of charge and mass, the integrals (e.g. for the magnetic moment of the electron) cannot be written so that for every summand of the \textit{integral} in 4-dimensional momentum space is Lorentz invariant \textit{as well as} giving convergence of the integral. In the calculations one must either (through various variable transforms in various summands) destroy the invariance (Weisskopf) or introduce additional assumptions in order to achieve convergence (e.g. an association of the \WTF{smearing}{Verschmieren} of the $D$-functions to those of the $D_1$-functions). The Schwinger integral representation is e.g. equivalent to an integration over a small, continuous spectrum of rest masses, where only \textit{after} the integration over the four-dimensional momentum space does one again go over to the limit of $\delta$-functions for the mass-spectrum. \?{However that only helps because one selects the \textit{same} spectrum in various terms with the $D$ and $D_1$ functions.}

Incidentally I would be happy if you could write me the \textit{total} integrands in four-dimensional momentum space (Here in Zurich, Villars, Jost and Schafroth also use the 4-dimensional momentum space) where you get the Lamb shift (and possibly also the magnetic moment). -- What is "only" a question of evaluating? Either a theory is unique or it needs additional assumptions in order to make it unique. These additional assumptions should however not be \?{hidden behind mathematical elegance}, but rather they should be physically grounded. For that however I still see no possibility. (N.B. I understand very well how you rewrote the denominator $(k_\mu p_\mu)(k_\varrho p_\varrho')$ -- \WTF{probably everyone knowsvthe center of gravity system}{das center of gravity system kennt wohl jeder} -- but the essential thing is that the integral is ambiguous, since it cannot be written to be simultaneously convergent and invariant in the four-dimensional momentum space.) -- The closer work with the subtraction formalism leads me more and more to the view that it is indeed quantitatively correct, but that the formal foundation is unsatisfactory.

That is probably also the reason why Schwinger is never finished with his paper. And what is Dyson doing? (I have obtained the paper by Weisskopf and French).

Many greetings to you, as well as to Uhlenbeck and Oppenheimer. It now looks as if everything is going well with the appointment in Zurich (Staub and Heitler), so that I shall come to Princeton in the Fall as planned, which I'm already looking forward to. Will you still be there then?

Now the "Einstein-Festschrift". As you perhaps already know, I have written a historical article about "Einstein's works on quantum theory" for the Einstein volume of the "library of living philosophers". With that I actually believe to have fulfilled my obligation with respect to Einstein. Also I feel the kind of close personal contact as I have woth Bohr, since with me the perhistory in both cases is totally different. I cannot write the type of personal introduction for Einstein like I did for the Bohr issue. However I believe that \textit{Otto Stern}, who is on your list anyway, would be the right man for this purpose. He knew Einstein very well from their time together in Prague (while the only time I was only ever in the same place as Einstein for long was during the last war).

\?{Thus I leaveyou with this}: if I have something suitable for the Einstein-Fest issue by April 1st, I will send it to you -- if not, then nothing can be done.

Always yours,

W. Pauli

\textit{All cheers for the new year!}

\end{document}
