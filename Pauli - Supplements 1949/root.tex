\documentclass{article}
\usepackage[utf8]{inputenc}
\renewcommand*\rmdefault{ppl}
\usepackage[utf8]{inputenc}
\usepackage{amsmath}
\usepackage{graphicx}
\usepackage{enumitem}
\usepackage{amssymb}
\usepackage{marginnote}
\newcommand{\nf}[2]{
\newcommand{#1}[1]{#2}
}
\newcommand{\nff}[2]{
\newcommand{#1}[2]{#2}
}
\newcommand{\rf}[2]{
\renewcommand{#1}[1]{#2}
}
\newcommand{\rff}[2]{
\renewcommand{#1}[2]{#2}
}

\newcommand{\nc}[2]{
  \newcommand{#1}{#2}
}
\newcommand{\rc}[2]{
  \renewcommand{#1}{#2}
}

\nff{\WTF}{#1 (\textit{#2})}

\nf{\translator}{\footnote{\textbf{Translator note:}#1}}
\nc{\sic}{{}^\text{(\textit{sic})}}

\newcommand{\nequ}[2]{
\begin{align*}
#1
\tag{#2}
\end{align*}
}

\newcommand{\uequ}[1]{
\begin{align*}
#1
\end{align*}
}

\nf{\sskip}{...\{#1\}...}
\nff{\iffy}{#2}
\nf{\?}{#1}
\nf{\tags}{#1}

\nf{\limX}{\underset{#1}{\lim}}
\newcommand{\sumXY}[2]{\underset{#1}{\overset{#2}{\sum}}}
\newcommand{\sumX}[1]{\underset{#1}{\sum}}
%\newcommand{\intXY}[2]{\int_{#1}^{#2}}
\nff{\intXY}{\underset{#1}{\overset{#2}{\int}}}

\nc{\fluc}{\overline{\delta_s^2}}

\rf{\exp}{e^{#1}}

\nc{\grad}{\operatorfont{grad}}
\rc{\div}{\operatorfont{div}}

\nf{\pddt}{\frac{\partial{#1}}{\partial t}}
\nf{\ddt}{\frac{d{#1}}{dt}}

\nf{\inv}{\frac{1}{#1}}
\nf{\Nth}{{#1}^\text{th}}
\nff{\pddX}{\frac{\partial{#1}}{\partial{#2}}}
\nf{\rot}{\operatorfont{rot}{#1}}
\nf{\spur}{\operatorfont{spur\,}{#1}}

\nc{\lap}{\Delta}
\nc{\e}{\varepsilon}
\nc{\R}{\mathfrak{r}}

\nc{\Y}{\psi}
\nc{\y}{\varphi}

\nf{\from}{From: #1}
\nf{\rcpt}{To: #1}
\rf{\date}{Date: #1}
\nf{\letter}{\section{Letter #1}}
\nf{\location}{}

\title{Pauli - Supplements}

\begin{document}

\letter{251c}
\rcpt{Rosenfeld}
\date{September 30, 1930}
\location{Zurich}
\tags{self-energy}

Dear Herr Rosenfeld!

Many thanks for your card. Now I would like to ask you about a generalization of the results of your paper on the gravitational energy of a light quantum, or induce you to answer the question by calculation. (If you have the answer, you should add something about it in the corrections.)

You specifically assume that in the initial state the momenta of the light quanta are exactly known, thus the positions are \WTF{not defined at all}{beliebig unbestimmt}. It is for me of the greatest interest to know whether the result that the perturbation energy $=\infty$ remains even in the most general initial state where $N_r$ is \textit{non}-diagonal, i.e. characterized by the most general (complex) eigenfunctions $\varphi_r(N_r)$, which only satisfy the conditions
\uequ{
\sumXY{N_r=0}{\infty}{|\varphi_r(N_r)|^2} = 1
\quad
\sumX{r}{\sumX{N_r}{N_r}{|\varphi_r(N_r)|^2}} = N
\quad
(\varphi_r(N_r) = 0 \text{ for } N_r > N).
}
We could however restrict ourselves to the case that \textit{one} light quantum is present ($N=1$), so that we always have
\uequ{
\varphi_r(N_r) = 0 \quad \text{ for } N_r > 1;\\
|\varphi_r(N_r = 0)|^2 + |\varphi_r(N_r = 1)|^2 = 1;\\
\sumXY{r}{\infty}{|\varphi_r(N_r = 1)|^2} = 1
}
(for short I write $\varphi_r(1_r)$ for $\varphi_r(N_r = 1)$).

Up to the middle of page 6 of your manuscript everything then \?{remains}, only the form of the diagonal elements belonging to the state is different. \textit{I would like to know what (for $N=1$) the general formula for the perturbation energy looks like}; in the case of $N=0$ (\textit{no} light quanta) nothing will change. Of interest is the finiteness or non-finiteness of the difference in the perturbation energies for $N=1$ and $N=0$. -- Naturally your previous calculation corresponds to the special case $\varphi(1_n) = \delta{rn}$, where $r$ is a specific index, and in this special case you have shown that the difference under discussion is infinite. In the general case the diagonal part of \?{any operator $D_{rsmn}$ acting on only the 4 eigenvibrations $r,s,m,n$} is naturally given by
\uequ{
\sumX{N_r,N_s,N_m,N_n}{\varphi_r^*(N_r)\dots\varphi_m^*(N_m)D_{rsmn}\left\{
\varphi_r(N_r)\varphi_s(N_s)\varphi_n(N_n)\varphi_m(N_m)\right\}},
}
where the $A,B$ act in the known fashion on the $N_r$. The essential distinction with respect yo your calculation occurs in the transition from equation (22) to equation (24). -- If one has the general \?{end formula}, one can e.g. insert the well-known Gauss distribution for the $\varphi_r(1_r)$ (written one-dimensionally and without polarization for simplicity):
\uequ{
\varphi_r(1_r) = \text{const}\times \exp{-\frac{(k_r - \overline{k})^2}{(2\Delta k)^2}} + 2 p i \overline{x}(k_r - \overline{k}).
}

Even in the general case, the gravitational self-energy of the light quantum and the electromagnetic self-energy of the electron will be analogous. I would like to have an answer from you \textit{as soon as possible}, whether the self-energy is also infinite in the general case, or whether this only happens in the special cases $\Delta k=0$ and $\Delta k=\infty$. (I rather suspect the first.)

Many greetings to you, Bohr, Klein, Landau, Gamow, etc.

From your W. Pauli

\letter{271a}
\rcpt{Rosenfeld}
\date{April 12, 1931}
\location{Zurich}
\tags{}

Dear Herr Rosenfeld!

Yesterday I wrote Bohr a long letter, among other things about your calculation concerning the self-energy of the oscillator. But since in this lettwr I hadn't made one point sufficiently clear, I want to briefly add something here. Gamow has written me that he has now proven that the questionable self-energy is actually infinitely-large. Now it occurs to me that it could perhaps be more clear, if one considers the non-stationary solution in which at the time $t=0$ no light quanta are present and the oscillator is in the ground state. One then has to consider at a later time $t>0$ the expectation values (a) of the oscillator energy for a given number of light quanta and (b) of the total radiation energy for a given state of the oscillator.
\uequ{
&(a) \quad \sumX{n}{n|\varphi(N_1,\dots,N_r,\dots,n,t)|^2}\\
&(b) \quad \sumX{r}{\sumX{N_r}{\left(\sum{N_r h\nu_r\right)}
|\varphi(N_1,\dots,N_r,\dots,n,t)|^2}.
}

In the letter to Bohr I have already opined that (a) must be infinitely great. Now I rather believe that

(b) must become infinitely great

(a) however does not, and indeed it might suffice to take $n$ approximately equal to 1 and set all $N_r$ to zero except for one, which is equal to 1.

The probability of the presence of a light quantum of very short wavelength might not decrease sufficiently strongly with $\nu_r$, so that the $\sum$ will probability diverge. Naturally these are all states whose total energy is different from that of the initial state (\?{\textit{no} vanishing resonance denominators}).

Of course physically it means the same, when either

(1) no \textit{stationary} solutions with finite energy exist, or

(2) the expectation value of the energy after an arbitrarily-brief time $t$ is already infinitely-great. (When the oscillator $t=0$ is in the ground state and with \textit{no} photon.)

Please send me your calculations \textit{as soon as possible}, as well as your results concerning the case under consideration here of the non-stationary solutions.

With many greetings to you, Bohr and the others.

Your W. Pauli

\end{document}
