\begin{letter}{1}

\begin{header}
\from{Weyl}
\to{Pauli}
\date{1919/05/10}
\location{Zurich}

\makeheader

\end{header}

Very honorable Herr Pauli!

It brings be extraordinary joy to be able to greet you as a coworker. How you have somehow managed, at such a young age to set yourself up in possession of all the means of knowledge, and to gain the freeness in ideas that is essential to make the relativity theory your own, is nearly incomprehensible to me.

I have recently received proofs of the third edition of my book I have briefly, \?{after which I wrote to Herr Dr Thirring}{nachdem ich Herrn Dr Thirring schrieb} (they were only provisionally set; when it will be printed and published is still undetermined). I sent the proofs to Herr Thirring, which contain what I believe to be the final version of the "pure infinitesimal geometry", and the conclusion of the book, which treats its physical meaning (electricity and gravitation). Unfortunately I can only spare one copy; but I assume that you could easily get ahold of the proofs from Herr Dr Thirring if you want to see them.

In it I believe, the question as to what quantities are to be introduced in an arbitrary action principle as the current and energy-momentum is clearly answered. I have naturally in no way overlooked the invariants you specified; indeed one first runs into them \?{when starting with the Einstein theory}{wenn man von der Einsteinschen Theorie herkommt}. To the four you mentioned I add a fifth, $R_{iklm}R^{ilmk}$. Can every \?{rank-2 quadratic invariant}{quadratisch aufgebaute Invariante vom Gewichte-2} formed from the curvature components $R$ be \?{put together from linear combinations}{linear...zusammensetzen} of these 5? I have not explicitly proven it; but the question can certainly be easily decided, with some expenditure of calculations. \WTF{A.a.O} I also directly consider the invariant $R^2$; I find there that one inevitably arrives at the cosmological term. You will find the remarks about the normalization $R=\text{const}$ in harmony with those in yours. \?{Setting up the mechanical equations for a material particle has given me many headaches}{Viel Kopfzerbrechen hat mir die Aufstellung der mechanischen Gleichungen für ein materielles Teilchen bereitet}. To get clarity there, I first had to put aside the investigation of the action principle $R_{iklm}^{iklm}$ and instead of this I have attempted on the one hand to draw totally general valid consequences of my theory, on the other hand to discuss that in the easiest way and \?{present the connection to the $R^2$ that is immediately apparent in Maxwell-Einstein}{den Anschluß an Maxwell-Einstein unmitte lbar darbietende $R^2$ vorgenommen}. So there I touch on your results in part. In the paper in Annalen der Physik I have also formulated the electron problem for just this simplest case, without being able to solve it (I have not yet worked out the elimination like you did it: to a single 3rd-order equation — even if I reduce the problem from 4th- to 3rd-order by quadrature).

Only publish your results in the form that seems appropriate to you; a collision in one point or another is in the end not a disaster. I still haven't yet received any proofs of the Annalen paper; who knows when it will be published! But it contains beyond that what I started in the last chapters of the book, only with the formulation of the electron problem.

To the question of the \?{asymmetry}{Ungleichwertigkeit} of positive and negative electricity, I say this: if the invariant entering into the action variable is \?{formed rationally}{rational...gebildet} from the $g_{ik}$, $\varphi_i$ and their derivatives, then the charge of a body is determined independently of any mass unit, incl the sign, so far as the world tube swept out by the body has a distinguished past $\to$ future sense. The asymmetry in electricities would also lead to an asymmetry past and future, which of course is not expressed in the "\?{laws of nature}{Weltgesetzen}". But how this problem is to be solved, we currently have no idea. I hold that is in no way excluded that in matter there is something fundamentally different than mere "knots" of gravi-electromagnetic fields. You must not think me such a dogmatist that I would believe now the Philosopher's Stone has finally been found. — If one also takes action invariants into consideration that are not \?{rationally formed}{.}, but rather something with a square root, then the asymmetry of the electricities also naturally emerges independently from past and future.

I unfortunately have to more \?{prints}{Separata} of my paper in the Berliner Berichten (they are out of print); per your request I have set aside the one published in the Zeitschrift für Mathematik, though I would now prefer to put it aside in favor of the improved version in the 3rd edition of Space-Time-Matter.

It would make me very happy if the road of your studies should sometime take you to Zurich. If you want to come to Switzerland now, it would not be, I think, too difficult \?{to provide you with the travel documents}{Ihnen von hier aus die Einreisebewilligung zu verschaffen}.

With best wishes,

H. Weyl

\end{letter}