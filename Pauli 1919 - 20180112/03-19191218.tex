\begin{letter}{3}
\begin{header}
\from{Pauli}
\to{Land\'e}
\date{1919/12/18}
\location{Munich}

\makeheader

\end{header}

Very honorable Herr Doktor Land\'e!

On behalf of Herrn Geheimrat Sommerfeld I have studied your paper on the reflections on the interactions of circular orbits in their quantization by means of the adiabatic hypothesis in more detail. I have found that the goal can be reached much more simply if, one doesnt take the kinetic energy as a function of $a$ from the equilibrium condition likeyou do, but rather introduces the angular momentum and instead write the equilibrium condition in the Hamiltonian in the Lagrangian form
\nequ{1}{
H_k(p_k, a_k)= \frac{p_k^2}{2ma_k^2} + U_k\\
\pX{}\pY{a_k}\left(H_k + \sum\limits_{j\neq k}V_{jk}\right)=0.
}
Your condition
\nequ{3}{
\D{W_k} + \pX{}\pY{a_k}\left(\sum\limits_{j\neq k}V_{jk}\D{a_k}\right) = 0
}
then says nothing more than the adiabatic invariance of the angular momentum. Namely
\uequ{
\D{W_k} = \D{H_k} = \pX{H_k}\pY{a_k}\D{a_k} + \pX{H_k}\pY{p_k}\D{p_k},
}
so as a consequence of (1)
\uequ{
\D{p_k} = 0.
}

Hence it requires longer investigation as to whether your equation (3) is integrable. On the contrary one is led immediately to Sommerfeld's rule which is also justified from the standpoint of the adiabatic hypothesis here: the \WTF{Ringradien} from the equilibirium condition (1) are used as angular momenta $p_k$ and are set equal to the unperturned values, i.e. equal to $n_k\frac{h}{2\pi}$.

With best greetings yours faithfully,

Pauli
\end{letter}