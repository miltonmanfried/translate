\begin{letter}{4}
\begin{header}
\from{Born}
\to{Pauli}
\date{1919/12/23}
\location{Frankfurt}


\makeheader

\end{header}

Very honorable Herr Pauli!

My answer do Enwald, which has probably been shown to you, will probably already have shown you that I have very much enjoyed your contribution. I have already applied \?{your method}{die von Ihnen gebrauchte Umsummierung} for the case of \?{surface energy}{Oberflächenenergie} (capillarity constant) myself, but did not notice that it can even be found for the tension components. With your permission, I will include this consideration in the encyclopedia article; but do you not want to publish something brief somewhere>

I have read your paper in the new issue of the Proceedings of the German Physical Society on the Weyl theory with great interest; however I am momentarily not totally \?{up to speed with}{bin...in...zu Hause} the relativistic formulae, since I haven't thought about it in a month, but I understood the idea of your observations. I was especially interested in your remarks at the end, that you hold that applying continuum theory inside the electron to be meaningless, since there it is dealing with in-principle unobservable things. I have pursued exactly this idea for a long time, but up to now without any positive results, namely, that the way out of all quantum difficulties must be sought totally \?{theoretical}{prinzipiellen} points: one cannot carry over the concept of space and time as a 4-dimensional continuum from the world of macroscopic experience, this apparently demands another type of \WTF{numeric manifold} as an adequate basis. But. how that would be made, I have no inkling. Though I am not old, I am too old and weighted down to have something like that occur to me. That is your task; from all I have heard of you, you are made for such problems.

If however you want to wander around the lowlands of the usual physics for the present, I recommend a problem to you that was taken up in a paper by Stern and me that just came out, which I am sending you simultaneously. One should show that the surface energy of the cubic surfaces of NaCl crystals (in vacuum at absolute zero) is smaller than that of any other surface. The problem requires, I believe, a rather deep study of \?{the potentials at half-lattice points}{der Potentiale von Halbgittern}; there one could perhaps take advantage of Ewald's method. I must however confess that I have always found Madelung's procedure for calculating lattice potentials to be more convenient and more intuitive.

I would be very glad if you could visit us here sometime or even one day work with us. \?{Even if all connections are closer here like in Munich}{Wenn auch alle Verhältnisse hier enger sind wie in München}, we still have some specialties of different types than they do.

With warm greetings to Sommerfeld, Ewald and all the other colleagues.

Very faithfully yours

M. Born

\end{letter}