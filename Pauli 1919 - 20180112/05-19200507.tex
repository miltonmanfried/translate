\begin{letter}{5}
\begin{header}
\from{Pauli}
\to{Oppenheim}
\date{1920/05/07}
\location{Munich}
\note{Carbon copy}

\makeheader

\end{header}

Very honorable Herr Professor!

Herr Geheimrat Sommerfeld has conveyed to me your desire to learn about the article on relativity theory from the Physics volume of the Encyclopedia, and so I share its status with you:

I. Critical discussions on the foundations of the special theory of relativity.

II. Mathematical aids (tensor calculus, variation laws).

III. Further extensions of the special theory of relativity (applications to electrodynamics, mechanics, etc).

IV. Foundations of the general theory of relativity. (Equivalence hypothesis, generality of the behavior of measuring sticks any clocks, motion of point masses and light rays, influence of gravitational fields in material processes.)

V. Einstein's field equations of gravity and their consequences.

VI. Cosmology.

VII. Theories on the nature of the electric elementary particles. (Poincaré, Mie, Weyl, Einstein.)

A collision with Herrn Kottler's article, with discussion of the general foundations of the theory of relativity can hardly be avoided, on the other hand an agreement can be made on the following points: a) I had the intention to explain in rather more detail the attempt by Poincaré et al to arrange gravitation under the special theory of relativity. But since according to your information Herr Kottler will treat it anyway, I will be brief. b) I would like to ask whether Herr Kottler will also mention the Nordstrom theory. It would suffice if in the two articles there was about an hour. I would like to coordinate with Herrn Kottler this.

c) Also as regards Einstein's cosmological observations, a division of labor can perhaps be reached.

It would perhaps be good if the same notation could be used in both articles and I would like to make the following proposals there:

1) Give the line element $\D{s}^2$ in the general theory of relativity 3 positive and 1 negative (and not, like many authors do, 1 negative and 3 positive) signs.

2) Denote the coordinates by the Hessenberg notation with $x^1,\dots,x^4$ (instead of with $x_1,\dots,x_4$)

3) Instead of $\left\{{}^{rs}_{i}\right\}$ and $\left[{}^{rs}_{i}\right]$ always write just $\Gamma^i_{rs}$ and $\Gamma_{i,rs}$ and always call these quantities the geodetic components (of the associated coordinate system). 

I would ask you to share these proposals with Herrn Kottler, it would probably be best if I can then \?{come to an agreement}{mich...ins Einvernehmen setze} with him directly.


\end{letter}