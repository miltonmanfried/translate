\begin{letter}{6}
\begin{header}
\from{Pauli}
\to{Land\'e}
\date{1920/05/12}
\location{Munich}

\makeheader

\end{header}

Dear Dr Land\'e!

With the approval of Geheimrat Sommerfeld I would be free to share with you my views on the inertia of potential energy that you have postulated. I start from the fact that the equations of motion must read in every case

\nequ{1}{
\dX{}\dY{t}\frac{m_0 \vec{r}}{\sqrt{1-\frac{v^2}{c^2}}} = 
e\left(\vec{E} + \frac{1}{c} [\vec{v}\, \vec{H}]\right).
}
Any inertia of the potential energy must then be caused by a deviation of the electromagnetic field from the Coulomb field. First a magnetic field is present and second instead of the Coulomb, calculating with the \textit{retarded} (four-)potential. The calculation (which incidentally Herr \textit{Fokker} has carried out in detail, to be published shortly in the Archive Néerlandaise) shows that the matter is not at all as simple as you assume. There is indeed a term on the right side with $\ddot{\vec{r}}$, but this does not have the same form as the term that is given by your Ansatz. As far as I see, that is connected with the circumstance that at no moment do the electrons move in parallel, as is the case with the \textit{additional} velocities of the electrons with a \WTF{Gesamttranslation des Ringes}. If an \textit{external} force acts on the whole ring, the calculation with the retarded potentials must naturally lead to the inertia of potential energy, at least ignoring additional terms of higher order. — Here the principle difficulty is still present, that one never knows how far classical mechanics is correct. If one were to calculate totally consistently according to the classical scheme, then one would naturally arrive at radiation in the static orbits. My view is then, briefly summarized, the following: the theory of doublets is to be improved by considering the retardation. Any inertia of the potential energy \textit{is likewise caused by the retardation} and would already be taken into account with it. The calculation shows however that your simpler Ansatz is incorrect and has to be replaced by much more complicated one.

I again thank you for sending your offprint, and remain

Yours very faithfully

Pauli
\end{letter}