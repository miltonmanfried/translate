\begin{letter}{7}
\begin{header}
\from{de Sitter}
\to{Pauli}
\date{1920/05/25}
\location{Arosa}

\makeheader

\end{header}

Very honorable Herr Doktor!

Many thanks for your interesting papers. Pre-prints of my papers are being sent to you from Leyden. Allow me to draw attention to two points — though of secondary interest — which are not in my printed papers, or are only raised in footnotes.

1. The $\lambda$-term introduced by Einstein now seems to be generally called the "cosmological term". That is wrong. It has nothing to do with cosmology. True, Einstein's treatment of cosmological considerations where he introduced it started with a mention of a cosmological problem; but, as I have shown, that problem is \textit{not} solved by $\lambda$, not even touched upon. If $\lambda$ is to be given a name, call it the \textit{inertial flux}, or inertial coefficient, since it serves to make the inertia relative. Inertia has nothing more or less to do with cosmology than e.g. gravitation, or light or magnetism.

2. In the same treatment Einstein has assumed a finite space of constant positive curvature for the three-dimensional physical world. There are two of these: the elliptic (simple-elliptical) and the spherical (double-elliptical). In almost all papers the second is discussed, yet the \textit{first}, the \textit{simple-elliptical}, is the simpler and more natural. The spherical space is only a totally unnecessary doubling of it.

Our usual Euclidian geometry is the limiting case of the elliptical, not the spherical case. Since a line has a point at infinity, a plane a line, two lines cannot cross at more than \textit{one} point (at infinity or not), and finally the paradox that in the elliptical space a plane is \textit{one}-sided exists in our Euclidian geometry too: if we go along a branch of a hyperbola, we come back to the other branch, and indeed on the other side of the asymptote (or any other line) without having to cross the asymptote. If one wants to come back to the same side of the asymptote, then the trip must be made twice.

Faithfully yours,

W. de Sitter

\end{letter}