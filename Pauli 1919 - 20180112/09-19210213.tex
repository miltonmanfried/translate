\begin{letter}{9}
\begin{header}
\from{Schr\"odinger}
\to{Pauli}
\date{1921/02/13}
\location{Stuttgart}

\makeheader

\end{header}

Dear Herr Pauli!

Also for your \?{suggestion}{Anregungen}, I thank you very much. The difficulties that you mention of course \textit{exist}. Yet the following \WTF{Verse} can be made of them.

The non-existence of the circular orbit $n=1, n'=0$ \?{would line up with the fact that}{wäre in Parallele zu setzen damit} (according to Land\'e, Zeitschrift f\"ur Physik \textbf{2}, 87, 1920) with alkaline \textit{ions} the elliptical, \textit{not} the circular shell is built out. Perhaps the outermost system falls out to the elliptic state because of this elbow room. In any case \textit{if} my hypothesis is correct at all, then, \WTF{wegen der Stetigkeit des Anschlusses von $\frac{3}{2}S$ an $\frac{5}{2}S$ \dots}, the normal orbit must be my lowest orbit $n=1$, $n'=1$, or eventually even $n=1$, $n'=2$ and no other. At least with the alkali. With the alkaline earth metals the normal orbit is $\frac{3}{5}S$ (\?{singlet lines}{Einfachlinien}). Here a "jump" \WTF{gegen} $\frac{5}{2}S$ etc seems to actually be present (if I understand \textit{Dunz} correctly). \?{It can perhaps touch on the fact that}{Er kann vielleicht daher rühren, daß} at $\frac{3}{2}S$ the two electrons run in a "\WTF{Biellipsenverein}" (I mean both symmetric to the nucleus in the type of orbit I worked out), at $\frac{5}{2}S\dots$ on the other hand only \textit{one} of them is lifted out. For the alkaline earth triplets I would let \textit{one} electron lie in the circular orbit ($n-1,n'=0$) — indeed there is no longer complete freedom — the \textit{other} I would let run through an orbit of the new type. Basically, triplet terms and singlet terms would be distinguished by the distinguishable behavior of the \textit{second} electron and thereby very naturally explains the occurrence of \textit{two} \WTF{Seriensysteme} rather than a single one.
Naturally that all needs calculation, which unfortunate;y because of the ignored perturbations still can only yield very qualitative results. I have not even worked out the alkaline earth doublets yet. In the absence of my wife, who had to accompany my mother to Vienna, I am presently running "my own kitchen" and that takes up rather much time.

I must have a look at the situation with the Roentgen doublet. Land\'e's \WTF{Ellipsenvereine}, which probably come into play there for the $L$-shell, \?{have the irksome property that the high (cubic-)symmetry only sets in secularly through the relativistic perihelion precession}{haben das lästige, daß sich die hohe Symmetrie erst säkular durch die relativistische Perihelwanderung einstellt}. They correspond - somewhat - to spherical shells that \textit{pulsate} and simultaneously make \?{flattening vibrations}{Abplattungsschwingungen} wherein the pulsation frequency and flattening frequency are slightly different, so that "beats" occur; e.g. the strongest flattening or the spherical form happens together now at the stage woth the highest, now at the stage with the lowest volume. (The flattening takes place in quick succession on the three coordinate directions.) Replacing this object by uniformly pulsating spherical shells is even less permissible than the shell with circular orbits. \WTF{Oder ist das schon gehupft wie gesprungen?}

Warmest greetings, faithfully yours,

Schr\"odinger

\end{letter}