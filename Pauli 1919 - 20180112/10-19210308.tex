\begin{letter}{10}
\begin{header}
\from{F. Klein}
\to{Pauli}
\date{1921/03/08}
\location{G\"ottingen}

\makeheader

\end{header}

Dear Herr Pauli!

Your parcel has arrived here correctly and I thank you and Sommerfeld for it. and declare myself in agreement with this Modus procedendi. So I have written directly to Teubner that he is to begin with the essay, also gave him the desired instructions on mailing the proofs. Hopefully that also has \?{helpful}{fördernden} influence on Kottler's paper.

My own papers have no speculations on natural philosophy, but rather only the flow of the path of mathematical ideas, which appeared to me with Laue e.g. often as very tortuous and inscrutable. I like to mention that Einstein wrote me in this manner on my third note: he felt himself lucky, like a child who has been given chocolate bar by his mother (Einstein is always so charming in his personal correspondence, in total contrast to the \?{foolish reports put out to honor him}{törichten Reklamatum, das ihm zu Ehren in Bewegung gesetzt wird}). Another item that I have particularly attempted to make clear in my lecture issues is the historical facts of the matter. It is true that Poincare's first note in the Comptes Rendus 140 preceded Einstein and additionally (in Rendiconti di Palmero) first showed that Lorentz had a \textit{group} of transformations. From then on a contrast, that alone makes it comprehensible that Poincare, in his 1911 G\"ottingen lecture "sur la nouvelle mécanique" does not mention the name Einstein at all. I would hold that it is important that such and similar facts come our of your article. There is still enough left over for Einstein.

You could otherwise leave my issue so long that the entire work of the encyclopedia article is taken up by it. \?{Perhaps it is even more important than it was with me to consult the Dutchman on certain questions}{Vielleicht sollten noch stärker, als es bei mir geschehen ist, betreffs der Einzelfragen die Holländer herangezogen werden}. I have long been hindered by the unaccustomed language, since I only have the original, written in Dutch, at hand.

Another much simpler matter. After Batemann had drawn attention to the fact that the Maxwell equations \?{go over into themselves}{in sich übergehen} with his $G_{15}$, it is clear from E Noether's laws that it gives 15 divergence relations for the named equations. This had been meanwhile explicitly worked out by one of my students, Dr Bessel-Hagen, who found some relations that were unknown in the contemporary literature. But, before I take it to the Mathematischen Annalen, I would like to first have the physics part \?{checked}{kontrolliert}. Dr Bessel-Hagen has just traveled to Berlin for the holiday (Address Berlin W, Kurfürstendamm 200), and I advised him to use his personal connections to prepare the Planck section. Perhaps it would also be useful if he made personal contact with you. I would also like to inquire whether you have anything to say on this question, \?{i.e.}{ev.} whether I should request Bessel-Hagen to contact you.

Unfortunately I myself cannot go any further into these things now. I have to prepare the second volume of my collected works and accordingly I am deep in the theory of algebraic equations.

Best greetings, faithfully yours,

Klein

\end{letter}