\begin{letter}{11}
\begin{header}
\from{F. Klein}
\to{Pauli}
\date{1921/04/28}
\location{G\"ottingen}

\makeheader

\end{header}

Dear Herr Pauli!

Generally I would ask you to check whether the presentation is sufficiently clearly-written and orderly for the reader. Naturally you know these things from personal experience in Sommerfeld's seminar, which cannot be assumed of the reader and perhaps has given rise to occasional vagueness of expressions (which does not disturb daily intercourse). 


Best greetings, yours faithfully,

Klein

...In general I would say, one should now present the thing partially from the standpoint of the new theory and only over say afterwards how complicated, e.g. unsymmetric the things came out earlier. The other way around, where the reader always expects to just work from the incomplete version through to the improved one, made 
 book so unbearable. Your presentation is already much more \?{to my taste}{in meinem Sinne}...

\end{letter}