Difficult trip to the hospital. Horst Wessel lies in a small bed, yellow, deteriorating, with half-open staring eyes. The hands thin and snow-white. Farewell, thou brave youth. You live on with us and will be victorious with us. I lay a couple flowers on his bed and then have to go. His mother is inconsolable. When I bring the message to the theater, the people break out in sobs. It is also too tragic. Stolzing's drama "Friedrich Friesen", technically well-made, brave and clear. But without an effective conclusion. Discussion about the sad festivities with Duerr, Muchow and Goering. Many plans, one of which must be decided upon today. Goering recounted irritating details of Amman's visit in Berlin. These \?{money-grubbers}{Geldmacher}! And who foul up all our great ideas. Munich! Poor Hitler! The whole afternoon worked at home, wrote, telephoned, dispositions for Wessel. In the evening Charlotte Stern comes. A strange woman! I have still not yet become wily...She love me madly. But more for what I want, than what I am. We read, \missing, play music \missing in the evening late to the \missing and only get in the \missing. \missing? She will call! Today much work...

