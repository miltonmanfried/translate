\header{November 2nd 1932 (Kaiserhof)}

The S.A. man Harwick\footnote{Correction: Harwik. Thr Catholic church refused him a Christian funeral. See "The 'Martyr'. History of the SA man Richars Harwik", in \textit{Vossische Zeitung} from 4.11.1932 and "\WTF{Evening of celebration in the Sportspalast}{Prominentenabend im Sportspalast}", in: \textit{Vossische Zeitung} from 3.11.1932. How already between June and August 1932 the political violence \?{boiled over}{überschulgen sich} even before the Reichstag election on 6.11.1932.} was buried. He had been shot down by red murderers. His children cried heart-rendingly at the grave. It is frightful, to have to see that. One reads mountains of newspapers daily. The opposition opinion is not reported. Germany is stuck in a spiritual anarchy, which would overflow into total chaos if a firm hand is not taken here sooner or later. In the evening, the Sportspalast is overflowing. The fuhrer spoke about the election in the Reich's capital. He got undescribable ovations from the party membership. His \WTF{will to fight}{Kampfesmut} and the consistency of his \?{stand}{Haltung} demand everywhere mkre and more respect. Gradually the Volk are beginning to understand him and are again following him the widest circles. In the evening after the assembly in the Kaiserhof the fuhrer is in a better mood. He is almost convinced that even if we lose votes on a large scale, this election will nevertheless be a great psychological success for us. The workers of the Berlin \?{Transport Union}{Verkehrsgesellschaft} have joined the strike. We have even put out a party slogan for the strike. The whole press grumbles magnificently at us. They call it Bolshevism; and thus there is nothing at all left for us. If we had withdrawn from this strike, which revolves around the most primitive right to life of the streetcar workers, then we would have thereby shaken our position with the working Volk. Here, before the election, we have yet again a great opportunity to publicly show that our anti-reactionary course us really \?{meant from the bottom of our hearts}{von innen heraus gemeint und gewollen ist}, that the NSDAP is in fact about a new type of political action and a conscious turning away from the bourgeois methods. The strike is having devestating consequences. No streetcars are running in Berlin, nor any subway lines. The public is demonstrating an admirable solidarity with the strikers. The red press has taken to every method of agitation against us. Though the Sozis stammer in their newspapers that we are only appearing to take part in the strike and in reality are employing the strikebreakers, but that does nothing for them. The generak strike is a frightful weapon. Machine guns and bayonets are useless against it; it is all the more contemptible when workers' organizations stab them in the back. That is the case with the Social Democratic unions, \WTF{which torpedo the struggle in the most contemptible manner}{die auf das schnödeste den begonnenen Kampf torpedieren}. But the workers wenr over the heads of their leaders and \?{did their own negotiating}{handeln auf eigne Faust}. The BVG-strike expands hour by hour. The unions are frantically trying to cut them off from the means to sabotage. We find ourselves in a not-at-all enviable situation. Many bourgois circles will be put off by our participation in the strike. But that is not the decisive thing. These circles can be very easily won back later; but once the workers are lost, they are lost forever. The lack of money has become a chronic illness in this electoral struggle. We are missing the most primitive prerequisites for carrying out a proper electoral struggle. The reactionary, bourgeois papers have found a welcomed trap in the strike. They agitate against us in the most unheard-of ways. Even many of our old party comrades have become mad; but nevertheless we must hold out and remain firm. If we back down now as we are here and there advised to do, then we will have lost everything. I speak a couple of times in the evening at the health resorts, in Tempelhof, Mariendorf and Sudende. The meetings are all overflowing. If the vote goes how the mood is, then we have nothing to fear. The consequences arising from the BVG-strike put us in new situations daily. Until deep into the night we must keep hard and strong resolve. The striking workers have proceded to active terror against the strikebreakers. In the city the tracks were torn up, \?{isolated streetcars covered in stone bombardments}{vereinzelt fahrende Straßenwagen mit Steinbombardments zugedeckt}. There has already been a heap of wounded. The greater part of the public is in solidarity with the strikers, the lesser part is nervous and intimidated. In Berlin there reigns a bleak mood of resignation. Any day an explosion could break out. Overnight in Schoeneberg an SA man, who was on the picket line, was shot by the Schupo. That is the politics of reaction. The unions are stabbing the workers in the back. If the strike is lost, then it is thanks to them ans their fat-cat bosses. A happier experience: overnight a group of S.A. men \WTF{set out in street-workers' uniforms}{macht sich eine Gruppe von SA Männern in der Kluft von Straßenarbeitern auf den Weg}, blocked off a part of Chaussee-strasse, put up construction signs and barriers, and with pickaxes and shovels started to dig up the tracks. The police assumed in good faith that they were actual street workers. Another part of the \WTF{troop}{Sturm} in question was on the sidewalk and carried the comedy further, the non-working SA men abusing the working ones as strikebreakers with horrid insults. In the interst of peace and order the police had no alternative but to protect the workers from the strikers. The next morning people found with astonishment that all streetcar traffic was shut down on Chaussee-strasse; \WTF{and the police wondered}{und die Polizei wundert sich einigermaßen} that they had taken the workers into their warm protection against the strikers.

%staming meln