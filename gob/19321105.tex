\header{November 5th, 1932 (Kaiserhof)}

Final push. Desperate drive by the party against defeat. The press campaign has been \WTF{put down}{niedergewettert} with simple means. We have succeeded in raising ten thousand Marks at the last minute, which we \WTF{used}{hineinpfeffern} for propaganda on Sunday afternoon. What could be done has been done. Now destiny shall decide. The directorate of the BVG, \WTF{Social Democrats}{sprich Sozialdemokratie}m let a totally mendacious, hypocritical poster be put up. On this poster the strikers are insulted in the most vulgar manner. We couldn't do anything else about this, since Berlin \WTF{bulletin union}{Plakatgesellschaft} refused to put out our reply. There is nothing left for us to do but to put a discussion group at each bulletin board, who can properly explain the poster to the reader. Because of the brevity of time, this counter-action was not very effective. In the working-class neighborhoods, the posters for our people were torn down; but their effect is self-evidently not totally absent. Our position has thus somewhat worsened; but I trust in the healthy common sense of the Berlin Volk, and thus with that the electoral struggle is, thank god, finished. I have made the last stroke of the pen for this election. It has been the most difficult, but, I hope also the most glorious for us. What we hold, that is firmly and unwaveringly for us. For the last number of "Angriff" before the election, the government \?{imposed regulations on us}{nötigt uns...eine Zwangsauflage auf}. We \?{found ourselves}{geraten} in an extraordinarily precarious situation. This number is also to be the last \WTF{reading material}{Lesekost} for the party comrades and followers before the election. Before we ourselves offer our hands to alienate our own party comrades, we will reach for desperate means. The whole printing of the issue of "Angriff" subject to this regulation will be festively dumped into the canal this evening. Pick up and read! In the evening, I took a  \?{drive around to get a feel for}{Orientierungsfahrt} the city. Everything remains peaceful, but a dull, oppressive atmosphere hangs over Berlin. I remained all night in Hedemannstrasse, to make myself available at any time if needed. Very late, I spoke with the fuehrer on the telephone. We are all glad that this electoral battle is finished.

% Berlin violence in reverse