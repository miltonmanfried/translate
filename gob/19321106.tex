\header{November 6th 1932 (Kaiserhof)}

Against all expectations, the participation in this election was very strong. They find instead a Berlin under completely changed circumstances. Transport is closed, and the whole populace is flooding through the streets. The day passes in enormous tension. In the evening we sat at home with some guests and awaited the results. They were not so bad as the pessimists feared, but there is still revolting \WTF{puking around}{Herumwürgen} on the radio. Every new piece of news brings a new defeat. We have lost 34 mandates. Even the Zentrum has recorded some losses, the Nationalists added a few, Social Democrats lost some. The electral participation has returned. The KPD has strongly increased; that was to be expected. A government of reactionaries is always the \?{midwife}{Schrittmacherin} of bolshevism. We have suffered a setback. The reason: the masses' understanding of the 13th of August still hasn't advanced far-enough, and the unscrupulous utilization of our \?{diplomatic contact}{Fühlungnahme} with the Zentrum in German-National propaganda. We are not to blame for either circumstance, so there is no need to reproach ourselves. We now stand before a difficult struggle that will demand many sacrifices. The main thing is that we keep the party. The organization must be consolidated and the mood must be lifted. A series of errors and deficiencies which have crept their way in which demand rectification. It must not be overlooked that hardly ten percent of the people stand behind 
the government. They cannot hold out with that. Some change must come soon. I lay out our standpoint in an essay with the theme "Chancellor without Volk". It comes out very sharply against the government. \?{I have it ready}{Ich bin gleich damir bei der Hand} to prevent the depressive mood in the party from assuming too-large a scope. It is admirable how strongly and cheerfully the party leadership has behaved. There can be no question of fatigue and \WTF{queasiness}{Flaumacherei}. We have gotten through other crises, we will also tackle this crisis. As a consequence of the electoral defeat, the prospects for a victorious outcome of the BVG strike have strongly diminished. The SPD has betrayed them. \WTF{As the cat cannot resist the mouse, so can Marxism not resist a stab in the back}{Wie die Katze nicht vom Mausen läßt, so läßt der Marxismus nicht vom Dolchstoß}. Though the red fatcats are triumphant today, they won't be laughing for long. What is unpopular today will become popular tomorrow. We must only remain firm and steadfast, not give in, and insist on our rights.

% hilfendehand