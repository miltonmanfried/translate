\header{November 12th 1932 (Kaiserhof)}

I proposed an editorial change in "Angriff"; this has become necessary in order to bring our \?{battle paper}{Kampfblatt} to its old heights. On November 20th the truce is to end; then we will attack again. It is also important that we get a morning paper in Berlin. With only an evening paper we cannot prevail against the hostile power of the big press. Our press always remains our problem child; above all it will be hard to maneuver the National Socialist papers through the difficult political situation. In "Angriff" we could only with effort and with much care and foresight hold to a convoluted tactical line. That is very difficult for our people. That means we have to be careful, above all just after the electoral struggle has ended and we find ourselves in a heated atmosphere. My hope, that the work would die down somewhat with the electoral battle, has been dashed. Because of the defeat we must get right back in the saddle again. Not one chance to catch up on lost sleep. The fuehrer keeps entirely away from Berlin. The Wilhelmstrasse can wait on him for a long while. That is a good thing. We must not become soft, and not \?{begin the tug-of-war}{anfangen, Tau zu ziehen} again 

 before August 13th. The government has slid into a heavy conflict with the states. How much porcelain will be slammed together there! We must stay entirely out of it! If it should come to negotiations, then our slogan will remain unflagging and unchang: Hitler must become the reich chancellor! Without that, there is nothing. It will come to pass if we don't give up. Only watch outvthat Strasser doesn't \?{make a side-deal}{Seitensprünge macht}. In my mind, I am already in the next electoral battle. That will bring unheard-of difficulties. God grant that we don't have to go through it.%