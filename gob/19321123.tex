The Berlin Jewish press writes that there was a serious brawl between the underbosses of the parties at the Kaiserhof, and Hitler had gone to the theater from resentment and bitterness. In the meantime, Frick and I came to loggerheads: however, the conflict finally died down and when Hitler returned, I sat at a desk and drew up an answer to Williamstrasse. In reality we all sat separately-peacefully concealed in the State Opera and heard the "Master Singer". As I came to the Kaiserhof, the fuehrer started right in on the dictation of his answer to Wilhelmstrasse. At four o'clock in the afternoon he was ready with it. We talked it through one more time. It is a classic document of the precision of his thought. In the first part, he repeats with unassailable logic the impracticality of the task given him, in the second part he makes the proposal to solve the crisis within three days, if he is given a free hand. The whole letter is a masterstroke of political strategy. With a view to the oft-repeated press campaign, which is supposed to have driven a wedge between the fuehrer and us, we all together give a very strong statement of solidarity in which we solemnly reject ever responding to these stupid lies again in the future. This explanation works like a charm in the public. The press is in feverish tension. The greatest \WTF{Tatarennachrichten} are spread by means of the \WTF{Druckschw�rze}. The fuehrer had a discussion with General Schleicher. The situation has not changed in the least. From the side of one part of the cabinet a \?{shot across the bough will be fired}{St�rungsfeuer gesendet}. The Conti-Bureau spreads a thoroughly bent representation of the whole situation. In the evening, we all sit at our house, and seek by chatting and music to unwind from the heavy demands of the day.

