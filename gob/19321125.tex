There are still a few aftershocks in the press; and then the daily work resumes again. The fuehrer has had a meeting with some other party leaders. But that is totally pointless. Wilhelmstrasse had yet again won; but for how long? Some in the party take the position that we must now take power in Prussia. We all very sharply object. We would be put in the most awkward situation, \?{and in the end the government could stick a government commissar in our faces}{und am Ende gar k�nnte die Regierung uns einen Regierungskommissar vor die Nase setzen}. The consequences would be unthinkable. The fuehrer of course also shared this opinion. In no case can we now rush to a decision. The important thing is that we all get out of Berlin to no longer be under the influence of the nervous atmosphere of this city. I speak in the evening in the tennis hall before the magistrates. I take the opportunity to lay our the whole situation. This time the people understand us. The party stands firm and unshaken. It is claimed that Papen will return, to head up an emergency government. But this attempt could only be undertaken for a very short time.