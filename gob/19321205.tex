In a little conference I try to bring the party bureaucrats back into line. There I made the effort, substantively and going into detail, to clarify the development and the instantaneous situation. In the long run, I hope to reach the goal. I wrote a very aggressive essay against General Schleicher. What is most detrimental to him in the eyes of the public at the moment is that the whole Jewish press praises him to the skies. That always speaks against a man in the eyes of the public. In Kaiserhof we held a substantive discussion with the fuehrer. It was about our posture towards the Schleicher cabinet. Strasser adopts the standpoint that Schleicjer must be tolerated. The fuehrer had the sharpest clash with him. Strasser as in old times paints the situation of the party in black-on-black. But even if that were so, one must capitulate before the resignation of the masses. By chance we also discovered the true reason for Strasser's political sabotage: on Sunday evening he had a conversation with General Schleicher, in the course of which the general offered him the post of vice-chancellor. Strasser has not only not ruled out this offer, but rather has communicated his decision to put up his own Strasser lost in an upcoming new election. That is an even worse betrayal to the fuehrer and to the party. That was not totally unexpected to me, I have never believed anything else. We now only wait for the moment where he carries out his betrayal openly. In a serious nervous crisis, one only proves himself with action; if he then fails, that only proves that he was not destined for greatness. In decisive questions, it always comes down more to character than to reason. Strasser tries to pull over everyone present at the \WTF{Fuehrertagung} over to his side. Everyone stands so firmly at the fuehrer's side however that it was out of the question. To top it all off, he brought over to the fuehrer Schleicher's threat: if we don't accept his cabinet, he would go straight to a new dissolution of the Reichstag. Yet again we formulate our conditions, under which there is a possibility of giving him a \WTF{Lauffrist}: Amnesty, social betterment, emergency law and freedom of demonstrations, and a provisional adjournment of the Reichstag. Caucus meeting: the fuehrer speaks very sharply about the search for compromise. There can be no talk of indulgence. It is not about his person, but rather about the honor and the prestige of the party. He who now practices betrayal only proves that he has not understood the greatness of our movement. Strasser's face visibly \?{hardens}{versteinert sich}. The caucus itself is naturally unanimous for consistent progress of the struggle. 
For now, as far as possible a dissolution of the Reichstag should be avoided, since we currently have no good possibilities for advancement. For a long time we formulate the conditions that should be put to the Schleicher cabinet. Goering and Frick bring them over. In the evening we are at home with a big artist clique, looking for a little relaxation from the heavy burdens on our souls these days. The music lifts the heart above the everyday and again proves the nature of the eternal value of our work.

