Yesterday: Annoyances at the bureau. They are intriguing against me in my own ranks. Try to get me to arrange phony bills of exchange. The cows! Everything is clean. But these vulgarities! Talked over the popular education paper with Dr Ziegler. We met Wednesday in Lippe. Funeral at 1:30. Theater. Minister Loerzer spoke. Good. No parents! Then the procession through whispering rain. Touching picture. I marched the two and a half hours with them. Through a neverending mass of people. The whole SA, SS and HJ, a hundred thousand marching. At 4:30 to Friedhof. Immeasurable masses of people. Burial. Loerzer speaks briefly. Then Jahn, Schirach and I. Great touchingness. And all the ten thousand filed past the grave of a 16-year-old Hitler Youth. This proud, glorious party. I drove quickly to the clinic. Magda is not well and entirely dejected. She cries, the poorest little thing! I comfort her. Nerves exhausted. I again have this nonsensical anxiety. Professor Fromhold gives us courage. But the danger is still that the blood clot dissolves. That would be very bad. I pray and shudder. Hitler calls and comforts me. God comrade! To the Lustgarten. 100000 marching. Colorful picture in the dark and fog. \WTF{Kameradenlied}. Ernst and Schirach speak. Then I. Sharp accusations against the Jews. The masses go wild. That is a slogan. Fanaticism. Then a raucous departure. I went to the clinic. Magda well again. Up and out. I am good to her. She has earned it. And then home. Sad and beaten down...lectures. Press still dwelling on the Hitler/Papen conversation. How anxious Schleicher must still be. This frightful torment and anxiety. Today fever again, 37.5. Despair. Hilter calls. He is very worried. Trip Monday from Lippe with me to Berlin and will visit Magda. I am so thankful to him. Today I speak to two groups.

Unabsehbare Men� schenmassen. Begr�bnis. Loerzer spricht kurz. Dann Jahn, Schi� rach und ich. Gro�e Ergriffenheit. Und all die Zehntausende defi�lieren am Grabe eines 16j�hrigen Hitlerjungen vorbei. Diese stolze, herrliche Partei. Ich fahre schnell zur Klinik. Magda ist nicht wohl und ganz mutlos. Sie weint, die �rmste ! Ich tr�ste sie. Nerven abge�spannt. Ich habe wieder diese irrsinnige Angst. Professor Fromhold gibt uns Mut. Dazu Gefahr noch, das[!] Blutgerinsel sich l�st. Das w�re sehr schlimm. Ich bete und zittere. Hitler ruft an und tr�stet mich . Guter Kamerad ! Zum Lustgarten . 100 000 aufmarschiert . Fa� belhaftes Bild im Dunkel und Nebel. Kameradenlied. Ernst und Schirach sprechen. Dann ich. Scharfe Anklage gegen die Juden. Die Massen rasen. Das ist eine Parole. Fanatismus. Und dann klinge