Kaiserhof.

We decide to call the entire German Volk to a "Day of the awakening nation" on March 4th. In the evening the fuehrer will speak from Koenigsberg to the whole Reich. In an as-yet unprecedented concentration of all propagandistic and agitation possibilities, the electoral struggle shall find its unique highpoint. With it we shall pull the last holdouts over to our side. Our propaganda will not only be recognized by the Germans, but also by the international press, as exemplary and unprecedented. We have acquired in past electoral struggles such extensive knowledge in this area that, by virtue of our better routine, we have been able to triumph over all opposition without difficulty. In any case, they are so cowed that they hardly make a sound. Now we show them what can be done with the apparatus of the state if one understands how to use it. The presses thunder and spit millions of copies of our election material out from their iron mouths. A fantastic song of political power and action. In the evening we unwind with the fuehrer at the Linden Opera, and hear Wagner's "Liebesverbot" for the first time. It already contains very much of the later Wagnerian elements. The approach though was still primitive, but on the whole the music is boldly and masterfully executed. At home the fuehrer recounted the Kapp Putsch and all the other failed undertakings which he was always somehow involved in. He was always an activist, and if he couldn't undertake his own action, he had as a matter of course to be involved in others' actions. Could listen to the fuehrer for hours.

