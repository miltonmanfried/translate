Kaiserhof

High in the sky, the sun sits and beams Spring warmth on this wonderful Germany. It is again a joy to work and to create. A trial of the work possibilities in my new house on Wilhelmplatz ends very unsatisfactorily. But first masons and cleaners must be sent into this room; they are to knock the stucco from the wall, tear down the heavy, moldy, musty plush curtains, so that some sun can again come through the window. One cannot work in this dimness. I need clarity, cleanliness and pure, bright lines around me. Half-light is repugnant to me. And just as the there must be a cleaning out in the rooms must be cleaned out, there must also be one among the people. Those of yesterday cannot be the trailblazers of tomorrow. Midday I was with the fuehrer. The Reich President has just signed a decree according to which the black-white-red and the crooked-cross flakes are to be elevated to the flags of the Reich. What an unimaginable triumph! Our ostracized, mocked and ridiculed flag goes up as a symbol of the whole Reich. It s the flag of the German revolution! In the Lustgarten there are a hundred and fifty thousand workers marching. It is an intoxicating feeling to sleak before this immeasurable crowd. In the evening dimness the whole castle glowed in red light. We drove through endless masses of people to the exhibition hall. \WTF{The streets reach for below the knee} and across the Charlottenburg Chaussee, overflowing with people. I introduce the fuehrer's speech with a detailed report. He spoke once more about the communal votes taking place everywhere tomorrow. He is honored by the masses with spontaneous ovations. Now all of the voting business is finally at an end. At home, I forged the future plans for my new ministry together with the fuehrer. I become somewhat timid when I think that I am only a bit over 35 years old and am now loaded with such a great burden of responsibility. I am grateful to the fuehrer that he has bestowed upon me such a measure of trust.
