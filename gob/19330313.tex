Kaiserhof

I am ready with my work in the Berlin Gau. Today the Reich president has signed the decree appointing me. Immediately the deluge of mail and visits starts. There isn't much time to dawdle with the well-wishers. I gather my coworkers once more in the Gau and speak to them. Express my thanks for having shared the struggle and worries with me, shake everyone's hand and promise them all that I will never abandon them. May the great work continue to thrive! In the ministry everything is already being rebuilt. It will still be some time yet before I am ready and have established myself everywhere. The bureaucracy attempts to make difficulties, but that is no use. These bureaucrats would most like it if I only came by for an occasional visit; \?{but I will soon be finished with them}{aber ich werde schon mitcihnen fertig werden}. Since I am confronted on all sides by difficulties in the construction and even in setting up my own room, without further ado I take a craftsman from the SA and have him knock off plaster and \WTF{paint}{Holzverkleidung}: ancient newspapers and files which have been vegetating on the shelves since \WTF{Anno Tobak}, are carried down the steps with claps of thunder. Only clouds of dust bear witness to the vanished bureaucratic splendor. As for the worthy Herren whom I shall now promote to the sky appeared the next morning, they were deeply shaken. One of them clasped his hands together over his head, and only mumbles, horrified: "Herr Minister. do you also know that you could be put in jail for that?" Now shove off, my good elder! And if it still hasn't gotten around to you, then you are hereby solemnly notified that a revolution is being made in Germany, and that this revolution does not stop short of action. Dr Lippert is named Commissar of Berlin.
