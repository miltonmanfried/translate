Kaiserhof

Saturday: I develop my program before the radio directors and get their agreement. A part of the leadership still must go. They are left over from the old days and are not suited for the new work of reconstruction. My essay against the atrocity agitation appears in the "Sunday Express" and works well. It makes things somewhat easier in England. At night I drive to Munich and from there to Berchtesgaden, where the fuehrer has summoned me. He has carefully surveyed the whole situation up in the isolation of the mountaintop and has now come to a decision. We will only deal with the foreign agitation when we get a handle on the originators, or at least the beneficiaries -- namely the Jews living in Germany, who have remained unmolested up to now. So, we must proceed to a large, strenuous boycott of all Jewish businesses in Germany. Perhaps then the foreign Jews will think twice, when their German racial comrades are in the hot seat. Party Comrade Schleicher is named as the leader of the action. I immediately write a call for a boycott and give out a brief statement for the press, which has already worked like a charm. In the evening I drive back to Berlin again. Now the course is clear. The fuehrer aways stands as a star above us. We have him alone to thank for Germany rising up again.