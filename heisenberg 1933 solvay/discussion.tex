\documentclass[a4paper,11pt]{article}
\usepackage{amsmath}
\usepackage{amsfonts}
\usepackage{titling}
\usepackage{graphicx}
\usepackage[utf8]{inputenc}
%\newcommand{\location}[1]{}
%\newcommand{\publication}[1]{}
\newcommand{\textbb}[1]{\textbf{#1}}
\newcommand{\WTF}[1]{\textbf{???}\textit{#1}\textbf{???}}
\newcommand{\?}[2]{#1\footnote{\textsc{Translator note}: #2}}
\newcommand{\nequ}[2]{\begin{align*}\tag{#1}#2\end{align*}}
\newcommand{\uequ}[1]{\begin{align*}#1\end{align*}}
\newcommand{\unit}[1]{\text{#1}}
\providecommand{\operatorfont}[1]{\texttt{#1}}
%\newcommand{\operatorfont}[1]{}
\newcommand{\grad}{\operatorfont{grad}}
\renewcommand{\div}{\operatorfont{div}}
\newcommand{\curl}{\operatorfont{curl}}
\newcommand{\rot}{\,\operatorfont{rot}\,}
\renewcommand{\exp}[1]{e^{#1}}
\newcommand{\pXpY}[2]{\frac{\partial #1}{\partial #2}}
\newcommand{\ppXpYY}[2]{\frac{\partial^2 #1}{\partial {#2}^2}}
\newcommand{\dXdY}[2]{\frac{d{#1}}{{d{#2}}}}
\newcommand{\ddXdYY}[2]{\frac{d^2{#1}}{d{#2}^2}}
\newcommand{\mf}[1]{\mathfrak{#1}}
\newcommand{\Nth}[1]{{#1}^\text{th}}

\newcommand{\citeauthor}[1]{\textsc{#1}}
\newcommand{\citetitle}[1]{\textit{#1}}
\newcommand{\citepub}[1]{#1}
\newcommand{\citevol}[1]{\textbf{#1}}
\newcommand{\citepage}[1]{#1}
\newcommand{\citedate}[1]{#1}
\newcommand{\citeyear}[1]{#1}

\newcommand{\El}[1]{\text{#1}}
\newcommand{\mnEl}[3]{{}^{#1}_{#2}{\El{#3}}}

\newcommand{\publication}[1]{%
    \gdef\puB{#1}}
\newcommand{\puB}{}
\renewcommand{\maketitlehooka}{%
    \par\noindent \puB}


\newcommand{\location}[1]{%
    \gdef\loB{#1}}
\newcommand{\loB}{}
\renewcommand{\maketitlehooka}{%
    \par\noindent \loB}

%\newcommand{\dX}[2]{\frac{d#1}{{#2}}}
%\newcommand{\dY}[1]{{d#1}}
%\newcommand{\pX}[1]{\frac{\partial{#1}}}
%\newcommand{\pY}[1]{{\partial{#1}}}

% \newcommand{\original}[1]{}
\newenvironment{translation}[0]{
}

\newenvironment{original}{
\renewcommand{\footnote}[1]{\small{(Footnote: ##1)}}
(\textit{Begin original text}:
}{
\textit{-- end original text})
}

\newcommand{\who}[1]{\textsc{#1}}

\begin{document}

\title{Discussion of Heisenberg's report.}

\who{Pauli}. — The difficulty arising from the existence of the continuous spectrum of $\beta$ rays consists, as is known, in the fact that the mean life span of nuclei which emit these rays, like those of nuclei of radioactive bodies which result in it, have well-determined values. From this one necessarily concludes that the state, and thus the energy and the mass of the nucleus which remains after the expulsion of the $\beta$ particle are also well determined. I do not dwell upon the efforts that would be needed to escape this conclusion, but I believe, in agreement with the opinion of Bohr, that one will always run into insurmountable difficulties in explaining the experimental facts.

\?{Along these lines}{Dans cet ordee d'idées}, two interpretations of the experiments present themselves. That defended by Bohr assumes that the laws of conservation of energy and momentum are imperfect in nuclear processes where the light particles play an essential role.

\end{document}