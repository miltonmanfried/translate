No more letters, only a diary, undertaken in total, threefold seclusion. Threefold: since outside the "White Guards" have Munich surrounded, and inside at most — but even at the very most a tenth of the populace — since Munich is not an industrial city! — the remaining hundred thousand as if in chains, and this tenth on its part, "Red Guards" and class-conscious proletarians, is the absolutely spineless and naive tool of a tiny handful of outside adventurers, who are feuding with one another and whose passionate and bohemian nature is by necessity being soaked from hour to hour with the more robust criminal element. That is completely unexaggerated: the total naivety is the condition off the soul which can increasingly be observed in all parties and sections of the populace. Munich swallows its tragicomic destiny passively, even the apparently reigning proletariat is totally passive, it can be shoved to and fro. This passivity is the singular genuinely Bavarian ingredient to this revolution, which is being played by non-Bavarians and childishly imitate foreign names and foreign institution.

The first part of my light prophesying: "From Landauer, Levien, from Levien, Epp", came true on Sunday, and a hair later (if only the Munich citizenry had marrow in their bones instead of some type of beer-like jelly) this anyhow awkward middle-position of the development would remain spared through the bravery of a little piece of the city. So it happened. That the Räterepublik did not feel itself secure, wary of the bourgeois to the right, threatened by the communists on the left, who only "had scorn and derision for the monstrum of the Räterepublik" ("Communication of the \WTF{Minutes}{Vollzugrats} of Workers' and Soldiers' Councils" from April 16th; this sheet is handed out for free and is our only newspaper) — the increasing insecurity of the governing parties emerges from the mass of flyers, which they throw from automobiles onto the streets, which they distribute and have posted everywhere. Entreaties for unity within the proletariat, announcements that this unity has been achieved by giving the communists a chairmanship on the Zentralrat, and above all advertisements, ever more advertisements for the Red Army. In one of these flyers it says "According to reluable reports Noske has threatened to march on Bavaria with his notorious paid murderers...the reaction is also marching in Munich, and the most bloodthirsty among them, the students, have already pronounced a death penalty on the honorable deputies of the proletariat ... so, comrades and brothers, join the Red Army in masses!" But, for the time being, the masses remain out, even though it pays very well. Now, each Red Guardist gets his 19M daily; for that, the current "honorable deputies of the proletariat" have opened the bourgeoisie's safes in the banks. But I anticipated the invasion (and collapse) of the nameless leader, I am still with the "old" government, the Zentralrat. On midday Sunday all of Munich was astonished. The regiments had just individually and by name stood behind the Zentralrat, with highest ideals, since they had not yet received 19M, but as specified by the Red Guardists, "only" 14M — and now it says in a notice, signed laconically/mystically by "The Munich Garrison", the Zentralrat has been dispersed, the garrison "offers" their opinion on the Hoffmann government, food trains ready for Munich. General astonishment. What had happened, no one knew, but yet everyone could notice that the "power" of the government, of the much-beloved Zentralrat, had shattered somewhere. What conclusions do the Munich citizens draw from such a thing? — Who in Munich will draw conclusions on a holy Sunday? That would be Prussian \?{busy-ness}{Geschäftigkeit}! But in the evening, as everyone here was still happily walking around at the victory gate, there began a clatter in the inner city. First gunshots, then ever longer machinegun strafing, then a powerful clash of the music of both, mixed with some hand grenades, then after about an hour, three powerful frightening tremors and right after what the French would undoubtedly call un silence tragique , which the people of Munich only greeted as a welcomed break in the nightly disturbances.Then we slept sweetly and awoke on Monday under Levien's protection, who, without his name coming up, now actually ruled as the leader of the local Munich Spartakists. "Proletarians! Soldiers! Fighters! Victory! Victory! Victory! stormed the station! ...the first day of the glorious struggle of the class-conscious proletariat of Munich! ...Let the Noskes, Epps, Schneppenhorsts come! We will greet them!" So says a jubilant communist flyer \WTF{in ellenlanger Geschwollenheit}. But in truth there had been no glorious struggle. Those rebelling against Munich Russian Republic were a little clump of Republican defense troops, among them had been the most famous hated station commandant Aschenbrenner. Probably they had, in childlike sunny optimism, counted on the protection of the reasonable circles. They remained alone, they succumbed to the superior numbers and their mortars. And now naturally Levien's hour arrived. The communists had stormed the station, saved the noble Russian-Bavarian state system: they pushed aside the sleepy Zentralrat government and erected the longed-for total dictatorship of the proletariat. In the aforementioned ersatz news-sheet from the 16th of April there was also the boldfaced declaration from Landauer: "I recognize and welcome the changeover. The old Zentralrat no longer exists, I put my power at the disposal of the Aktionsausschuss, wherever it is needed." But this power is of no use, since it has been curbed.

One thing must be conceded to the new government: it gives the city an overall warlike aura, they understand it, to "impress" the populace, yes they understand it, the uniformity, which has almost already become boring, of the images of the revolution, known for months now in many German cities, with new powerful colors to help them to their somewhat drunkenly swaying feet. That there will be strikes, it self-evidently goes without mention. But we also have announcements on the walls, that citizens "under penalty of death" must give up their weapons within 12 hours. And we have, in ever increasing numbers, armed civil guards; class-conscious, guns on their backs — occasionally even women. And the military! Infantrymen and sailors marching together, or more accurately, wandering into one another. 



