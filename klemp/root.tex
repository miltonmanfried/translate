\documentclass{article}
\usepackage[utf8]{inputenc}
\newcommand{\nf}[2]{
\newcommand{#1}[1]{#2}
}
\newcommand{\nff}[2]{
\newcommand{#1}[2]{#2}
}
\newcommand{\rf}[2]{
\renewcommand{#1}[1]{#2}
}
\newcommand{\rff}[2]{
\renewcommand{#1}[2]{#2}
}

\newcommand{\nc}[2]{
  \newcommand{#1}{#2}
}
\newcommand{\rc}[2]{
  \renewcommand{#1}{#2}
}

\nff{\WTF}{#1 (\textit{#2})}
\nf{\translator}{\footnote{\textbf{Translator note:}#1}}
\nff{\iffy}{#2}
\nf{\?}{#1}

\nc{\skipped}{\{...\}}
\nf{\sskip}{\{...[#1]...\}}

\rf{\date}{#1}

\title{klemp}

\begin{document}

\date{November 9, 1919}
Sunday, midnignt.

Yesterday and now - anniversary of the revolution - there was fear of unrest and the Abwehr was prepared. Naturally nothing came of it. But yesterday the train workers struck, no newspapers appeared, and the post was even more disordered than usual.

\date{November 13, 1919}
Thursday, after supper.

Monday morning Pontius was here again, this time a generalized confession about his unsatisfying marriage and his plight - streams of tears. A lousy, ungrounded, if not Russian, then ne ertheless strongly Russified man. At this moment he is again sleeping here - on the divan, which I myself would very much have liked to use. On Monday the result of his confession was that I almost improvised my lecture. But it went quite well. Afterwords I learned from Kaeser that Voßler had informed the students about the final rejection of the \?{call to Berlin}...as I came from the dentist in the late afternoon, I met Vossler in a snowstorm on Odeonsplatz. He seemed to have the need to comfort me, since he forcefully invited me into the nearby Cafe Parace and told me there that Lerch was to become an Extraordinairus - pro forma he had nominated Spitzer in the second place - thatbhe however had forcefully recommended me for Vienna and would otherwiwe champion me everywhere. We chatted then quite nicely, and for me very instructively about Italian literature. He recommended me \?{Gothein}: Loyola and the Counter-Reformation. But how desperate things are for me was clarified for me yesterday evening in the Technical Committee on New Philology, where again as a non-Ordinairus I sat next to Kraus. [...] I am very, very sickened and cheerless.

\date{March 14, 1920}

Yesterday the \textit{\WTF{Right coup}{Rechtsumsturz}} in Berlin has pushed me once again, after a long time, from the private to the general. But how deadened, how indifferent am I now. My inclination towards the right has meanwhile suffered much from the \WTF{Arco affair}{Arcorummels} and the incessant antisemitism. I saw the current \WTF{coup group}{Staatsstreichleute} \?{willingly put to the wall}, I was in truth not passionate for the \WTF{perjured}{meineidige} Reichswehr, certainly not for the immature Radau students - but not for the "legitimate" Ebert-government either, and for the radical left even less. They are all revolting to me. Where are the democratic, the German, the human ideas? I am a neutral spectator. The one committing the rape is always the one I hate, thus for now the "Right Spartacism", as beautifully coined by the "Reich government". In conversations I maneuver carefully, and \?{am likely to meet a swastika student, as Weißberger}.

Excessive work has almost killed me this week. \skipped

Counterbalance to all of the work is (1) the best possible food, which our remarkably \WTF{glittering}{glänzenden} finances allow. Opulent breakfasts and coffee breaks, and evenings regularly go to the old Pinakothek in splendor, for 12-14M wothout any \WTF{tip}{Markenabgabe}. Sometimes E even prepares us something delicious at home in the little oven\skipped. We only go to the poor Raab pension for lunch. There we are usually alone. Occasional society: a swastika student, an older, quieter, mkr fanatical right-wing ex-officer, and a younger \textit{studiosus}, with very many ideals, plans, and artistic-political-philosophical hopes. (2) Create all types of social gatherings and amusements and distractions. We however feel it is coming to an end, exploit the last time, we also have all kinds of obligations.
\skipped

At Monday evening at Rudi's, who by wrangling has found a flat on Schellingstr and proudly and graciously hosted us \skipped. Frau Rudi, a Siebenb\"urgerin, probably with Balkan blood, is a \WTF{fool}{Dummerchen}; he, a chemist, assistant to the Technical highschool, officer and \WTF{chopped-up body student}{zerhauener Corpsstudent}, had a \WTF{keen}{enge} intelligence, Bavarian rudeness, and in all and everything not of the spiritual characteristics psychological and other opinions that I love; but he is a capable, open, honorable, pleasing, quiet, in his field certainly a very practical man, and I appreciate all of that, and it gives me the greatest joy to be able to deal with \WTF{contrary}{gegensätzlichen} men. R told me without shame and without \WTF{adornment}{Brillieren} thatbhis father had startes as a wandering journeyman cooper and later had become a timber merchant. R is a convinced antisemite, royalist, etc. I am however certain that in the field he was no \WTF{martinet}{Leuteschinder}, and that he would not take part in a pogrom. Rudi, Ritter, Kaeser, Weißberger, Pontius, Hamecher - I \WTF{get on with}{komme mit...aus} all these people, and \WTF{assess}{übersehe} them all in all their \WTF{limitations}{Begrenztheiten} and differences. On Tuesday evening Kaeser was here. He stands nearest to my heart and my spirit. On Wednesday Weißberger. Nervous, bitterly upset about antisemitism, about the \WTF{unappreciated profession}{ungeliebten Beruf} - he is a chemist, and is attracted by philosophy and literature - \?{spiritually at home in everything}, doubts everything and himself, total Jew, total opposite of Rudi. On Thursday evening I went to Vossler's alone, his wife in Italy. Also Lerch there without partner. Vossler read aloud from \textit{Analysen zu Leopardi'schen} poetry (\?{while each of us had an exsmplar of poetry in hand -- the purest privatism!}). I knew Vossler's \WTF{humanity}{Menschlichkeiten}; but here it would again become clear to me that he is a true and great genius. \?{With him, one cannot even say whether science, philosophy, or art.} All in one - Vosslerish. He read a few lines of the deepest psychology, simultaneously general linguistics and aesthetics. I was overwhelmed by many things.
On Friday afternoon we absolutely had to go to \textit{Muncker's} dreadfully boring tea, and had as always the greatest revulsion to it. Finally I went there alone, Eva excused with sickness, left already before 7, was around 7:45 with the so-sick Eva in the old Pinakothek, and with E at 8:45 in the \WTF{Rathau cinema}{Rathauslichtspielen}. I had been the first guest of the lady Muncker - her husband had a faculty meeting. She warmly received me, as always; feared by all as toxic, the little woman has only ever proven friendship to us both. We were also very secretly invited, before the other guests to a musical tee on 3/21. Later a few ladies and some Herr arrived, there was speaking about the plight of the time, and I soon vanished.

On Saturday morning, I slept in and ran a few errands. In the afternoon we went with Kaeser to Isarthal. With the streetcar after the Thalkirchen, over the wooden bridges, beneath the \WTF{???}{zool}.\skipped. Conversation above all about the Berlin putsch. Kaeser \?{bitterly democratic} against the Spartacists of the right. At 6 back, ate at the AP, then Hamechers came for coffee with us. There I heard Kaeser's democratic bitterness with USP colorinb yet again. Besides that there was discussion of stamps, which give H lots of money, on which he almost lives. It was getting late. They were on alert in the barracks. The rooms are \WTF{bright}{hell}, coming and going of automobiles, lit the lantern in front of our rather dark bedroom. However it remained calm overnight. In the city was a sign, signed by government, party, and \textit{Reichswehr} (Moehl), \?{that one not take part in the Berlin Putsch}. Today at the table an old officer explained that Moehl's name was added to the declaration without his knowledge, and that it could also \WTF{run into}{verlaufen} others. I also believe that the Reichswehr \WTF{fell}{fällt um}. There is again a very large commotion over there, black-white-red flags (not black-red-gold) are hung out, troops in Stahlhelms with weapons, etc, just retured (on Sunday) from an exercise, soldiers crowded at every window. In the city many \WTF{civil guards}{Einwohnerwehr}, \WTF{who are said to be on leave}{die der Ruhe an sich dienen soll} and probably still don't know in whose service they will avoid bloodshed by going home, and a moseeable telegram from the government is posted, they are "to avoid bloodshed", to go to \textit{Dresden}, and \iffy{bleibe die wahre Regierung}{\textit{remain} the true government}; at the same time a wordy appeal of the Democrats here for the existing order. What a torturous tragicomedy, that 5-8,000 soldiers could overturn the whole German reich, that the government has to rely upon fickle paid troops, that the immature student body is a power and which perhaps correspondingly means that \WTF{over all it must be said}{man sich bei allem immer sagen muß}: it will hurt us abroad, it will bring us only \WTF{more}{recht} hunger through \WTF{inflation}{Valutasturz}!

\sskip{Discussion about Catholicism, Jesuits, ends and means.}

\textit{Marta} has sent us a precious package of food. Her frequent gifts are very touching. Spiritually we have little to do with one another, we are foreign. Aside from gifts, I don't \WTF{deal with}{mache} her at all. It is purely faithfulness to blood, and comfortable sisterhood, which she gives to us. \?{It presses all the more on my conscience, that I tried for an instant had tried to push provisions with inflated price on her. Nothing came of it, but she remains to me an embarassing living sin.}

The large package overwhelms me. War-letters from my last days on the front stare me in
the face and \WTF{leave me ???}{ließen mich nicht lokker}.

\skipped

\date{March 20, 1920}
Sunday towards evening.

We have thoroughly enjoyed "Herrin der Welt" and at 1:30 saw the final part, which is now naturally very kitschy. It amused me how exactly the ingredients of the 18th century, melodrama and high tragedy, came into play. [...] Overall we have thoroughly enjoyed the series.

The whole week in dull political unrest, in blunt excitement. It didn't lead to any extremities: the general strike was settled, and a bourgeois, but nonetheless constitutional and not-extreme ministry was formed. But a bookprinter wage strike followed, and one learns only \WTF{scanty details}{Spärliches} from posters, and these scanty details are always somehow partisan and contradictory. It seems that at many places in the Reich blood is flowing and there is \WTF{chaos}{chaotisch durcheinanderzugehen}: Putsches of the right and the left, local soviet republics, local \WTF{advances of the Reichswehr}{Reichswehrvorstöße}. One day, swarms of troops here; yesterday and now many have moved to the north - it is said that Nuremberg and Hof have gone Bolshevik. Everything is rumor, tension everywhere. We have a \WTF{stage army}: \WTF{the same troops march everywhere as an "army"}{dieselbe Truppe zieht überall als "Heer" auf}. It is said that a new general strike is imminent. Specifics from these days: one time I met Mathias Meyer on guard at the entrance of the university with Stahlhelm and uniform and loaded gun. \WTF{He looked eerie with his shotgun}{Ihm schien die eigene Flinte unheimlich}, and he complained bitterly about the unclear situation; he didn't know who he was working for, the constitution or the military dictatorship. Everytging is just disoriented and uncertain.
Rothenbuecher told how, on the day that \WTF{the troops were called in}{Truppeneinzugs}, there was a kind of parade in front of his window on Kaiseeplatz. The first thing that he heard was a rousing little speech: "\?{My lords, the Herrn Major von Wedel has uncomfortably noticed that the belts are worn under the coats}!"
Once I spoke briefly to Dr Ritter. He (in the Civil Guard) regretted with childish calm that it hadn't come to battle. \?{The ringleader would been so beautifully shot}! Once I at Hamecher's I met a \WTF{staid Palatiner}{biederen Pfälzer Bürger}, who said in the most beautiful dialect: he had always been against the revolution, but if it was already done, one had to do it right and right at the start all conservatives shoukd be guillotined! \iffy{wasn't vossler a lefty? Vossler steht sehr naiv rechts}{Vossler is a very naive right-winger}. In the window of the barracks a swastika, lately the frightful antisemitic flyers are handed out in front of the barracks, black-white-red flags and decorations on the helmets are typical (although the Reich colors are black-red-gold and any other colors are now just symbols of opposition). \?{I'm not sympathetic with any of them}; but if I must choose, better the soviet republic than the Herren Lieutenants and antisemites.
\skipped

\date{March 21, 1920}
Sunday after eating.

Yesterday morning I searched the library of the Technical Highschool \?{to see} whether I could muster something for my "technical French" in Dresden. Basically nothing here. \skipped The highschool here is closed and guarded by the Civil Guard; they must legitimate themselves. I then went to the German bank, \?{to announce the withdrawal of my account for Tuesday}. The city was peaceful and alive. Posters and telegrams replace the newspaper.s Over the course of the afternoon the bookprinter's strike was ended (they now get 5M an hour, which must run up to 1000M!), in the evening we had newspaper again after a week - \?{but were in reality not as intersting as before}. Everywhere turmoil, bloodshed; Kapp is finished, but Communists now struggly victoriously at different spots. It seems there was much bloodshed
in Leipzig. \textit{Leo Sternberg}, who these days after many years has sought to connect with me, writes of the death of a brother-in-law, who fell as a marine captain in the struggles in Kiel. Sternberg's letter is otherwise disagreeable: \WTF{???}{autoreneitel, reklamehaschend} - and politically on Kapp's side. A Jewish district court judge in Rudesheim on Kapp's side! As a Jew, judge, Rheinlander! Three impossibilities.
The T\"urkenkaserne is making progress. Today I saw a little black-white-red flag flown with the swastika in the white stripe.
Whether we will be able to travel on Thursday is questionable. A new general strike and stoppage of traffic from the coal shortage is threatened, and bloody chaos seems to reign in Saxony. One must always say "seems", one indeed never knows, and all reports lie.

In the evening we were summoned to "a very simple tea" with Meyer. The Catholic philosophy is always a bit \WTF{fussy}{ethepetete}, and so \?{it went off this time very socially ans actually rather depressingly elegant}. \sskip{expensive stuff for a simple tea, guest list, Lerch, Pauli, a young Doctor and his wife...} There was immediateky political talk, the doctor spoke out hard against the strike, which \?{brought} his babies in the clinic for the essential milk, Meyerhof defended him, even Eva was a bit on his side. I attack rather violently: what joy would it have brought Hans Meyerhof, had he heard me standing up for the arming of the workers! But I did not need to expose myself much, since Lerch behaved much more radically, and the most bitting and red was Aster. He is said to be known for this \WTF{party-position}{Parteistellung}. In the end it didn't lead to as bad a clash as the time with Borcherdt and Fray Joachimsen about Arco. We and Lerch went last. At 12 at home Eva brewed yet more coffee. We went to sleep only at 2:30, and I am today very \WTF{smashed}{entzwei}.

In the morning I brought Lerch the newspaper \WTF{as promised}{versprochenermaßen}, which Meyer had been given to me at Meyer's yesterday; \sskip{shop talk}.

I then went through the sweltering and verdant Hofgarten to make a farewell visit to Kraus, but didn't meet him; he lives in the same \?{Eckhaus Liebig-Reitmorstr.} as Munckers (and Borcherdts) and we will
wander by once more in the afternoon for music and farwell tea with Muncker.

The T\"ukenkaserne with German flag and swastikas! And inside a grammophone playing the Marseillaise!

\date{March 23, 1920}
Tuesday, after eating. Professor's room in the newspaper hall. 20:00.
\sskip{Shop talk..}
In the evening Lerch, Fraulein Richart and Vossler came to our pigsty. He sat between Fraulien Dr Richart and Frau Lerch on the \WTF{tilted}{windschiefe} sofa. He stayed there, boyish and happy, until around midnight over tea and cookies. He spoke very animatedly about the contents of an Italian manuscript from the 1800s, which he had just found at the state library, the burlesque life story of a castrato in Russia and Munich. A bit like \textit{Lettres Persanes}, it seemed to me, but \WTF{experienced}{erlebt}. Then also much politics was discussed, Fraulien Richart on the right, Vossler again on the left. He is convinced that the loss of the was is good for Germany, \?{compared to what a victory would have brought us}. "We would have become the \WTF{Polizerknechte}{policemen} of the world. And Ludendorff and Wilhelm as victors!" He was optimistic for the future for a change. \WTF{Lately believes in world Bolshevism...moody wothout a plan - like me as well}{Glaubt dabei neuerdings an Weltbolschewismus...Stimmingsmensch ohne Überblick - wie ich auch}. Brief farewell, more humorous than moving. I told him at the doorway that I consideres myseld as his apostle in Dresden. 
He is still one of the \WTF{most unpretentious}{einfachsten}, most open, kindest, and not to mention most brilliant people that I know. \iffy{He's the benefactor of those?}{My great Munich experience, my destiny, \WTF{even my benefactor}{auch mein Wohltäter}}. I have been, despite the woe over Lerch's position here, \WTF{sleeping well}{gut gebettet} and have been \WTF{warmly looked-after}{warm gefördert}. \?{If I am now gradually \WTF{being separated}{loslöse} from this university and its beautiful rooms (delivery of books, etc), then my heart is nevertheless a little \WTF{cramped}{beengt}}. But I depart happily from Munich and Bavaria, and Dresden means secure ground under my feet for the first time, in a time where it is shaking for everyone else. It's a shame that Fathee is not alive, a shame that I cannot be thankful to a personal god, it's a shame that I cannot convince myself that it is thanks to my abilities, it's a shame that I can no longer be youthfully disinterested in mortality, but rather am no longer at all free from thoughts of death.

I am a bundle of nerves and have consequently made Eva's life in latter days very difficult. And it would still be sufficiently-difficult for her anyway: the eternal tooth affair, endless \WTF{coffee-cakes}{Kaffeekochen}, heating, washing - thus \WTF{patching}{Ausbessern} and baking. And all that now culminates in the most desolate preparations for the move! 9 suitcases, \WTF{???}{Wetsch}...she is justifiably sore, and I cannot help her, in part because I am too clumsy, in part because it will take me until the last moment to \WTF{take account of my lectures}{mich...die Collegien in Anspruch zu nehmen}.
\sskip{Lectures}

\date{March 24, 1920}
Wednesday

The chaos intensifiea, the \WTF{break-up}{Aufbruch} draws nearer. Eva just ate breakfast at the Stephanie, since we don't have any wood for the oven, but has yet to return - \WTF{and I await the feet which shall carry the piano}{und ich warte auf die Füße derwr, die den Flügel hinausschleppen werden}. Afterwards E. must go to Wetsch, I to the bank, the \WTF{???}{Markenerledigung}, to the dentist.\skipped{Last lecture. Students stomping. Then a farewell party.}

The newspapers leave hope that our travel will be, if a peaceful, then at least \WTF{restful}{erschlaffte}. It seems that so much blood has flowed in the Reich that it will be enough for a little while. And the last spasms of the battles had passed, until we were outside Bavaria. Of course yesterday Dr. Ritter warned that a Bavarian \WTF{rail lockout}{Bahnsperre} was immediately imminent.

\date{March 25, 1920}
Thursday afternoon

Yesterday morning desperate \?{\WTF{run on marks}{Laufen um Marken-} and \textit{official notice of departure}{polizeiliche Abmeldung}}. Afternoon an hour and a half alone interpreting \skipped. After class I muat give many attestations and to some students, tests. Then the nervous was really behind me and at home I helped roll up the carpet and nailed a box. Today to the dentist, then to the university, hand over the key to the seminar, inform of my new address, to the library, also farewells from Daisy once more.

Then after eating still a large visiting-tour: Lerch, Amorettis, a quick moment with Frau Rudi. Now all that is left for us in Munich is just the coffee, on which E just worked, the preparatioms for the station, supper - and at 8:35 we want to drive to \textit{Landshut}, to a fully new chapter of life. I don't have any big feelings in the stomach, I \?{took a little drive} without overwhelming happiness or anxiety.\?{At the root is naturally however the sensation of all things being secure}. \sskip{Leaving behind the hardships of late youth...} I come to Dresden very aged.

\date{March 28, 1920}
Palm Sunday, morning towards 9
Bamberg, Hotel Baumann am Bahnhof

We've never yet had a trip that would so disappoint us: hunger and immeasurable \WTF{inflation}{Teuerung} follow us. And yet it is interesting: for the first time we learn whatvthe Russian-Austrian situations mean. In Munich we still \WTF{had it easy}{waren sanft eingebettet}, and only complained about the price hikes in the Old Pinakothek, and didn't know how good we had it. Now on the other hand...

On Thursday (3/25) I \WTF{strapped myself into}{spannte...vor} a hand wagon with the little Karl Sch\"afer, and together we pulled my luggage to the station. On the way I met Dr. Ritter and didn't embarass myself at all. The drive to Landshut went well. We found a place, and the mixed-goods train gradually, gradually came into Landshut around midnight. There were railwaymen in the Coup\'e. They told us that one gets to the station early and cannot find a place. We thought it was an exageration. So we went to the city... In front of the only open guest house was a crowd of people arguing with a porter, who explained mercifully that one could perhaps stay the night on chairs. Then we turned back. We inferred from hints that Landshut was the center point of \iffy{Schieber}{black-market} buyers and sellers, \?{who book their hotel rooms in advance}. \?{The shelves were over-abundant, and naturally went not to the commune but to the "black"}...a 4th-class waiting room was open, overflowing, and stank. At first we sat outside. Then in the main hall. I drapes my coat over Eva and went to the waiting room. After some time she also showed up there. The air had gotten a bit better. One sleeps sitting. Some lay on tables or on newspapers on the ground -- I have seen this picture happen so often in the last years. A Galician Jew, in European clothes but with a little cap and \WTF{distinctive type}{unverkennbaren Typs}, small, blonde, bespectacled, \WTF{stretched}{reckte sich} unspeakable comically; every part, and particularly the in-turned feet, \WTF{advertises the whole to}{bekannten sich einzeln zum} the god of our fathers. Incidentally a brabe act between all the swastikas. At 4, so-called coffee was served. I took a second-class billet to Regensburg, \WTF{checked in}{gab...auf} the suitcase \WTF{just after}{gleich nach} Probstzella. At 5:40 a passenger train left, which arrived at Regensburg around 8. In the Coup\'e I met the little Jew again; he spoke a remarkably precious all-to-perfect German. "Pardon me, could you perhaps tell me when W shall be in R; perhaps you know the local geography better than me." So he seemed to be a resident of Rechenburg.

\date{March 26, 1920}
Thus on Friday Morning we were in Regensburg. We \WTF{treated ourselves}{gönnten uns} to a \WTF{day room}{Tagzimmer} in the same Hotel National where we had already stayed once during the war. The room was cold and musty and the super demanded 14M towards evening for it, but settled for 8 when I explained that we were not going to sleep in the beds. Incidentally, that was a lie; since aftef supper E laid down for an hour and I also slept on the sofa under bedclothes. Still, Regensburg was only a \iffy{Vorschmack}{preview} of the coming privation. Here we got to know the misery of the time from opposite sides. While all over the land is in deepest misery, in R one \WTF{enjoys plenty}{schwelgt man...markenfrei}: In Sternbrau, on \WTF{meatless}{fleischlosen} Friday, 5 \WTF{grilled meats}{Fleischgerichte} between 3.50 and 5.50 on a food cart, \WTF{Rohr noodles}{Röhrnudeln} for a mark, \WTF{fat-roasted}{fettgeröstete} potatoes for 80pf. In the cafe, the \WTF{whitest yeast-cake}{weißeste Hefenkuchen} unrationed for 1M. We ate and ate and felt the indecency of this engulfing of provisions, although up to now we only has Munich as a point of comparison. \WTF{what have we first said before in Nuremberg and here}{Was haben wir erst in Nurnberg und hker gesagt}! \?{I would be for the Raterepublik} if they \?{were to} force the farmers to supply the community through billeting and martial law.

We have done more in Regenburg than just eating and sleeping, however. We had the luck to find the Cathedral open this time, which was closed before. (I don't know the year we travelled through.) From the outside, it \WTF{does little for me}{gibt er mir wenig}, with its dull gray regularity, the two openwork spiers left and right on the high roof. But within it is wonderfully beautiful in a glorious \WTF{imageless}{bildlosen} simplicity. Only architecture, no painting. But glowing stained-glass windows. \sskip{more Cathedral}.
We stuffed ourselves for the last time, I was horrified by the enormous price at the hotel \?{for the first time}, I tortured myself with the luggage, which my heart couldn't take, and at 6:15 we went on to Nuremberg.

At first it was comfortable in the train (\?{3 Kl?}), but already long before Nuremberg some \WTF{hoarders}{Hamsterer} boarded, little people with rucksacks, and from their \WTF{crowding} and their tales one could already guess at the misery of Nuremberg. Before the inrush of hoarders a civil patrol of three men arrived, from the \WTF{Usury office}{Wucheramt}. The pure \WTF{mockery}{Hohn}. \?{The small downtrodden privates are pursued, and are by and larged shoved and shoved}. An \WTF{upright}{biederer} man told of the privation. \?{He also told of the Erlanger students, who would have opened fire on the people in N, and who then the Erlanger landlords had denied housing}. Now that is settled; I read in the Frankischen Courier a threatening note from the students, which threatened Erlangen with a boycott \WTF{if it happened again}{Wiederholungsfall}. I've since heard more talk of the students. They are said to have been drunk and have opened fire, where the Reichswehr \WTF{refused}{weigerte}. My sympathies are this time definitely not with the Swastika people.
I want to anticipate the price and lack of food of Nuremberg. We had to pay 19M fore night in Bama berg we farm right under the roof, 5M for a piece of dry fish without any fat in a little restaurant at an uncovered table. We were hungry and depressed, although I said to myself that my finances could withstand the shock.

It strongly affects us, how we still travel peacefully, while a few hundred kilometers to the west civil war rages, how we fatalistically accomodate ourselves to the insane prices. It is actually a shame to be so blunted against all these hair-raising fantastic situations. But how else is one to live? And one must be thankful of his fate, to live through such historical greatness. This trip, despite hunger and money problems, is uncommonly interesting...a worker tells us in the coup\'e: in Nuremberg, the brain of someone shot in the head lay in a square for two days before it was swept away. The worker said naturally: a "\WTF{big-head}{Großkopfeter}" had fired the first shot, at the troops, in order to provoke them against the people. And Dr Ritter and Frl Richart swear to the strengthening of patriotic feelings among the students and Reichswehr!
We went to town, and it was again militarily closed, as in June when I went to mother's funeral. But this time I had an overview of the entire \?{\WTF{situation}{Anlage}}. \sskip{church}
All prices in the store windows are very much higher than in Munich. After 4 a coffee break in the militarily-occupied station, at 5:30 in a very sparsely-occupied train \?{which only drove 4th class}, to Bamberg. But on another train the people were crowded and stood on the connections between platforms. That drove through the Hamster area. At 8:00 we arrived here. Thus we were in Nuremberg from Friday evening until Saturday afternoon, and Saturday evening (3/27) we arrived here and became terribly dissapointed. First we ran through the whole long \?{\WTF{road}}, then back to the station, we asked at different hotels. Exorbitant prices. Finally very close to the station we quartered here, for 14M. For supper we obtained a tiny piece of sausage with salad, a tiny Kas, a cup of coffee and cake. That cost about 16M. The beds had nice linens, but no cover on the upper blanket. Eva constructed a full bed from two sheets, and there we lay close tother \WTF{at least clean}{wenigstens reinlich}. Now no trains go on Sunday, so we have no choice but to remain. Now at 10:30 we have a breakfast behind us, which could have been worse - one cannpt be choosy here - and now we want to have a look around the place. It may sound miserable, that I'm constantly reporting on food and prices; but everyone is now occuiped with these just as in the war. 
On the trip I read the very interesting Pere de famille.

\date{March 30, 1920}
Probstzella, \?{Hotel Meiniger Coutryard}, towards 8.
\sskip{Palm Sunday was great.}
In the evening...as we wandered through the city, a police officer recommended two inns: The Red Ox and Das Faesschen. The Red Ox looked proletarian and had only marinated herring to eat, \?{we hadn't even visited Das Faesschen}. Now it was suddenly before us, and fed us afternoon and evening, not at all expensive and very good. Further, we found not far from the town hall a little cafe (Schreiber) with the loveliest coffee and cakes. We have been ther three times, before and after wandering, and after the theater. Always a great bustle, the whole family seemed to be working (and rejoicing in our giant bill). And finally there was fabulous morning weather, in the afternoon really already too warm for the coats.

In the morning we \WTF{explored our surroundings}{machten wir einen Oriwnterungsweg}, in the afternoon, at the close of our long walk, we had been through the entire city. We rested \WTF{during}{von} the strenuous walk in Cafe Schreiber and in das Faesschen, and \WTF{dashed}{jagten} to the theater and arrived just in time. An old house without a facade with a \WTF{broad, dark tiled roof}{breitem dunklem Ziegeldach}, reminiscent of the old theater in Leipzig, \WTF{austere in the facades of a square}{schmucklos in der Häuserfront eines Platzes gelegen}. Pitiful interior. The small \WTF{floor}{Parkett}, the two tiers over it could together hold 400 people, in the halls one is crowsed, a little room, which also seems to be a rehearsal room with a piano, serves as a foyer, \WTF{there was not much of a cloakroom, since liability was declined and there were also no cloakroom lady in sight}{Garderobe wurde kaum abgegeben, da Haftpflicht abgelehnt war und auch keine Gardenrobenfrau sichtbar war}. All the more touching and admirable, \WTF{???}{wie Gutes in diesem 'Musenstall' geboten wurde}. The day closes for us in the most pleasant mood. We didn't regret the 60M it cost us. It was the most beautiful, and up to now only pleasant day of our trip.

Yesterday, on Monday, the strain and disappointment returned. In the morning we made a hurried shopping trip in Bamberg, went twice to Schreiber for coffee, and departed (with a half-hour delay) at 12:45 in a very full express train. We have tickets up to Leipzig (47M, that ticket!), our luggage awaited us here in Probstzella. We stood in the aisle, E sat on the \WTF{purse}{Korbtasche}. Hilly terrain, nothing special. I must \WTF{admit}{übrigens nachholen} that the Regensburg-Nuremberg leg of the trip had its charms: \?{river and rocky shore}. The conductor came and explained that the train did not stop in P, only before and after. So we got off one station early in Rothenkirchen, \WTF{around two}{etwa um zwei}. An area which strongly recalled Driburg in all elements. We went along the \WTF{country road}{Landstrasse} between the track and the hills. In the guest house, a particularly horrific scene: the landlady, morbidly obese with heartsick bulging eyes behind her glasses, complained in an educated and \WTF{beautiful Saxon}{schon sächselnden} German almost desperately. Worry, distress, economic collapse - provisions are seriously lacking. The embittered, careworn tone of the few people we had dealings with stuck out to us. A teacher in the train told methat in this area - \?{high plateau ofvthe Thuringia Waides} - is dominated by the most extreme want. So the landlady in Rothenkirchen put us out. But we would probably get something right away in Pressig at the station. Thus back again. Sat for a while on the \WTF{ledge}{Waldrand}. Then salvation \WTF{beckons}{versucht} at the initially-overlooked hotel. The landlord complained that he had to take care of overnight guests. When we ordered a real coffee (which wasn't available in Rothenkirchen, even at the grocer), he marched in with Kitzbraten as well. We ate and drank our fill for 15M. But
my nerves were at an end. When I lost Eva before the incoming train, there was a scene. And then I arrived here totally broken. Incidentally, I left my pipe in Pressig. Slow train. Towards 9 in got to Probstzella. \WTF{We were referred to}{Man hatte uns im...genannt} a train to Meininger Hof, and we took that here. A shabby slate house (\?{the slate goes under the walls}), with sleepy maid, but with anxious amd again careworn landlady. We gave 16M for the room, \WTF{which was nothing less than magnificent}{nichts weniger als prunkvolle beste}. But at least we slept in, and again the most beautiful Spring weather. Up here, of course, cooler.

Early today Eva felt ill, probably with a cold. And my \WTF{mental and physical fatigue}{Abspannung} is also not gone. But we want to explore. I would have so enjoyed rest and \WTF{staying in}{ein Zuhause}. Where am I to find it? It is already the same whether we further "enjoy ourselves" or not.

\textit{Afternoon, 2:30}. We had a long rest, \WTF{??? and that did tremendous good}{und das tut ungeheuer not}. In the morning I wrote in the diary until 11. After breakfast (thinnest corn-coffee: Saxony!) we took a \WTF{walk}{Spazierschlich} around the place. At 12 a good lunch (Saxon: one can cook!), then E had an hour of sleep, and I took notes here in the guestroom on Pere de Famille, which I had read on the way, at first very interesting, then ever more repellent. Now a cup of coffee, then a walk.

\textit{Evening, 9:30}. We took the loveliest walk a few kilometers southwards. A wide well-maintained road, in the narrow long \WTF{???}{Waldhügelthal} by the railway line over a creek (Loquitz - \?{from loquare}?), \{utterly Kipsdorf, gently inclined . After a \iffy{reichliche}{long} kilometer, a frontier post: "Kingdom of Bavaria". Three quarters of an hour later, a surprisingly picturesque view: a \WTF{peaked forested hill}{spitziger Waldhügel} diagonally in front of us, a tiny castle with spires upon it: Lauenstein. We returned home on foot the same way we came. The forest was dark, there moon was quite full. Yet again the view of the castle, which now stood in dark silhouette. When we emerged from he \WTF{narrow valley}{Thalenge} around 8, Probstzella glittered with many electrical lamps, mounted in front of the dark forest. The dining room here was very full, but emptied soon. The best supper. Eva unfortunately was over-tired. One of us always breaks down, and oftentimes both of us break down. 
Morning probably to Jena, where I would like to seek out Schuktz-Gora.
The suddenness of the Saxon dialect here is very strange.

\date{April 1, 1920}
Morning towards 8:30.
Jena, Hotel Victoria across from the university.

Thia trip goes in cycles: one good, one bad day. Yesterday again a rather bad one. For the first time, unfriendly weather, foggy and cold. Otherwise no rain, and now, actually already yesterday evening again milder Spring. Very bad sniffles, which started the night before last and now causes much discomfort, even now.

In Probstzella we took another little morning walk in the northerly direction around the place. \?{From there one gets a  wider view, since it extends into a side-valley}. But gloomy by the slate-houses. Still the narrow valley dominates everything there. The one difference with respect to Kipsdorf is all the slate. Whole heaps with dark shards, then again splintering shimmering slabs heaped wide over one another.

We ate, paid a very high bill to the very attentive landlady of the Meininger Hof (\?{these travelling days come consistently to about 70M}) and drove at \?{1:19} in a slow train to Saalfeld. There I sat in a waiting room until 3:30, while E \?{went around a little}. The fast train, which was of course delayed, was very full, we stood in front of a constantly-used closet. We went very quickly through all kinds of \?{states} - \?{a mighty yellow \WTF{castle}{Schloßkasten} perched high over the area caught my eye in Rudolstadt}: got here at 5:30. We looked for a hitel and found very handsome accomodations here. We found a cafe "Hanfried"; from there I asked, telephoning to the university, for Schulz-Gora's address. He's still not in the address book. \?{We had been im the university before, and \WTF{found no soul}{keine Seele gefunden}}. Entirely new, a bit of \?{Wertheim style}. Inner halls, paintings, pomp, wreaths in front of a great \WTF{list of the fallen}{Gefallenen-Tafel}; outside a cannon from the war. I visited Schulz-Gora, who lived near the cafe, Johannisplatz, met up with E again after 7; we went to an elegant hotel restaurant to eat, elegant but with dirty paper tablecloths, good but not much food, very, very expensive. Then in a good bourgeois music-cafe - loveliest little buttercakes, very, very expensive. One can get anything at all here, and all with more culture and better-prepared than in Bavaria - but such inhumane prices. \?{The pinnacle however meant my hotel bill}: the breakfast, though with sausage, butter, real coffee, tea, 17M, the night y todo with 15\% \WTF{added for}{Ablösung} tip 35.60M. I paid with a kind of dulled fatalism. My reserves are rushing away. But from today forward, I am finally a full professor in Dresden. In the evening we walked a little through the rather dark streets. I dragged myself around wearily. I felt very sore, and it always depresses me that Eva \?{doesn't externally seem to care about it}. I know-- externally, but it is still always makes me sad. We have different natures.

Schultz-Gora greeted me himself in slippers and a jacket without collar, which he afterwards took off. We apologized to eachother about our \?{gruffness}. He lives in a \WTF{furnished}{möbliert} flat, after he lived half a year in a hotel. It his room bed, toiletries, boots, books, Bohemian. The greatest part of his belongings are held by the French in Strasbourg. Hoepfner is over there, discovering his French heart; he himself came here in H's place (who incidentally has not received the Strasbourg ordinariat). A haggard man, dry gray-blue eyes, strong Berlin speaking \?{-voice}, extremely kind to me. First we talked politics and all went well. We is frightfully pessimistic: everything is corrupt, we're heading into the ground and France with us. Then \WTF{personalisms}{Persönliches}: he is 59 and married -- his wife \WTF{had gone out}{war ausgegangen}. I told him that I had heard him \?{lecture} in Berlin, I \?{reported on} Kaeser, who he gladly recalled. He said that he had begun as a senior teacher, only habilitated at 34, Berlin, Koenigsberg, Strasbourg. (From Kaeser I knew that he was very all-German). Then there were awkward minutes, where I was squirming. He bitterly attacked Vossler (and Lerch). Vossler was no philologist, was corrupting the youth, \?{Follows thoughts without exactitude}. I remained diplomatic, emphasised how \?{attached} I was to Vossler \?{as a person}, how I considered myself however inwardly free and not like Lerch, with whom I was on very friendly terms, \?{who is} Vossler's "megaphone". "If you take literary history seriously as in your Montesquieu, you will also take linguistics seriously." \sskip{More criticism}.

\date{April 2, 1920}
Good Friday after lunch, 2:00.
Leipzig, Reichelstr. 16 in our old room.

At 10:00 in the morning yesterday we embarked on our pilgrimage in Jena. Warm, sunny weather, \WTF{Budding and blooming trees}{Knospen- und Blütenbäumen} - this week we have literally seen the waxing of Spring. From our hotel room one sees a house lying on a higher hill. A student house \?{on the Landgrafenberg}. Up there we went, first through the city, then in the countryside, steeply over a some hundreds of steps. As we came into the city around noon, I tried in vain to \WTF{intrude on Gelzer}{bei Gelzer einzudringen}, who lives in a lovely villa in the North (Botzstr.). \sskip{...} Not only powerful church with towers, many university institutes, pretty but unsophisticated old houses. Much culture and life in a small city. Hardly 5000 residents. The importance of Jena, however, we are told by a friendly man in the station, shall not much longer lie with the conservative and antisemitic university, \?{which gives the place with fraternity houses a literary character ("Schiller lived here", "here ...., etc)}, but rather with Zeiss and industry.
We ate very badly in a dive behind the university, went yet again to the Cafe Hinfried, went to the main bridge, and then slowly to the station. Coffee in the waiting room, fast train (general admission) from 4-5 to Corbetha, then a slow train to Leipzig, which however arrived here well before seven.

In a week, we were in: Landshut, Regensburg, Nuremberg, Bamberg, Rothenkirchen, Probstzella, and Jena. Bamberg, Probstzella and Jena were the \WTF{most enriching}{eigentlichen Bereicherungen}. A bit expensive in terms of money and strain, but nevertheless lovely and actually intersting exactly through the strain and cost, which introduced us to the misery of the time. We always had good, most radiant Spring weather.

Here there was a pleasant arrival: similarly the luggage was in the right place, to get a cab of course cost 10M, as always the inordinate expensiveness caused me serious concern. Frau Streller took us in joyfully, I'm writing in the old writing-table place. Flowers and a letter from Scherner were here, we should also come to eat with them. We drove to Connewitz. The old hearty welcome, and abundant hospitality. Affection, food, wine, liquor, music. Hans Scherner has become fat, has done his abiturium, and has remained a warmhearted child. We left after 1, I carried the bedsheets which the Scherners lent us in a suitcase. Through moonlight and Spring the whole way back. To bed around 3.

Today had breakfast in Merkur.

\end{document}
