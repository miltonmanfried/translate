Monday evening

Today was totally lost for my Corneille, since I was so over-tired from the many debauched activities yesterday. It was the presidential election. In the morning we drove there and back by car, since Eva's foot is again so completely bad, with Johannes Koehler's parents, to whose marriage we were invited. \WTF{Flat from being the Station Inspector of the Friedrichstadt, the same Waltherstrasse.} Good-natured petit-bourgeoisie people; the wedding, which I dreaded, will be survivable. On the way back to our house, \?{Hindenburg was elected}{...Haus gegen�ber Hindenburg gew�hlt}. In the evening with the Blumenfelds for dinner, then with them over to Dember's. There until 1 o'clock, 7 men up in a little smoky room sitting before an overpowering radio loudspeaker, listening to the election results. As it progressed, it soon became certain that Hindenburg would remain a few thousand votes short of an absolute majority. We had reckoned on the possibility of an outright Hitler victory. We have heard much lately of the agitation and lies the National Socialists work with. Too bad I no longer have any time or desire to keep up with my diaries. The faces (a baker's wife is Meissen: "I voted for Adolf, he has promised to reduce the unused flats, then the official who has been living for so long in the apartment we want will have to move out!") recall almost exactly what Feuchtwanger told about the Beer Hall Putsch in Munich in 24, of the "True Germans", by Cajetan Lechner in "Success". In these last weeks I have read aloud from this book with enormous interest. It is the best of the Feuchtwanger things. Kuske, our supplier of butter, chicken, etc, said that he usually read Huetter's astrology newspapers, and that Hindenburg's stars are very good.