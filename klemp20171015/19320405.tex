Morning and later

It doesn't take much to accept: that I finished Corneille on March 31st, yesterday \WTF{couvertiert}, today \?{dispatched}{absandte} - actual work October 9th 1931 - March 31st 1932, preparation one year - is deep down more important to me the telegram Felix RIP April 2nd.

This death has affected me purely egoistically: in a year two left, the eldest and the youngest remain. As an excuse for my apathy: F and I were always very estranged, we were hardly ever friendly or enemies. But in any case, so much common experience. I cannot feel any longer. Only anxiety for Eva and myself. And the occasional apathy. I now understand the apathy of older people when relatives die.  So unimportant whether sooner or later! And: \WTF{who has it behind him and knows where he is and where he is sleeping}. 
I spoke with Georg on the telephone, for which this must be very bitter; I will drive to the funeral - the last time I was in Berlin was for the same reason. The repetition will be ghastly. I'm always calculating: Felix born October 9th 1866 - will I also be 65 years old, or merely 59 like Berthold, or even less? Or will I go blind like mother, but earlier? And then always: actually that is all unimportant. Eva is so weary of life, so shrunken, \WTF{she would be perfectly happy to have a reason that she herself would be finished}. And everything good is behind us and there is no more good ahead of us.

The last weeks and months, always the same, all the same pressure. Very rarely, E's violent fit of crying; even behind her happy hours the deep melancholy, the feeling of her emptiness. On the outside, all to great tenderness for the car Mucius, who I also love very much, already because he is so nice to Eva, and comforts her. Work at home - and otherwise nothing and always the feeling of this nothingness. In the year we have had it, the harmonium has not been played fifteen minutes. Not much more for the gramophone.

\missing

Now, the final hope: the house. But, two question marks. Will it help? And this is the worst question mark. The other: Will it be done, and when? I keep up the payments, despite the halving of income. There are now new negotiations for \?{Zwischencredit}{interim credit}. I have promised Blauert 2\% if he can provide it. \WTF{I should get 2000M of the Iduna at 8\%}, and perhaps buy some land in Doelzschen with it. Everything momentarily hangs in the balance, but it still is not entirely hopeless, just today \WTF{the Iduna has presented itself}.

My body aches from \?{too little motion}{zu geringen Bewegung} and constant mental work. I a very fat, very sleepy, the throat and heart issues have stopped. I don't believe that I have many years ahead of me.

I read all of the diary notes up until the Corneille was finished — and now I have no desire to write. I now play with the idea more often than before of a book of recollections, \WTF{but the present impassively slips away from me}. I must anaesthetize myself in work or reading (unfortunately without notes) and forget. Also, the wine plays a big role.

After Leipzig, E von Jan was called to Becker's position. Now. my road has really reached its end. \WTF{On V_3 I obtained a shock critique from Teubner.} Much praise, much spite (so, Neubert).

\missing

On March 19th was Johannes Koehler's feared wedding, it went rather cooly at first, then very nice and tolerable. Religious ceremony in the Sophienkirche (priest from Kirchbach), ate in Gotha City; the very unsophisticated parents, a Volksschul teacher, one married couple, friends of the bride, landowner from Leipzig area (\textit{no} parents of the bride, since she did not marry according to her social position; her father is a manufacturer and a reserve officer, her father-in-law served 12 years at a \WTF{supply post}, better he was a railway worker! Then what else was there. The Koehler father incidentally did not speak without education and obviously on the left, socialistic. But the son says grace before eating!) Ome of K's study friends: von Mannteuffel, whose mother (my dinner partner), stupidly/good naturedly speaking about bygone glory: estates and high posts in the army. On the other side, the bride's friends: two daughters of my architect-colleague Schneegan, one of them a \WTF{Heilgymnast} like the bride. \WTF{Everything by the book.} The young couple in a bridal \textit{carriage}, a little \?{flower basket}{Blumenstreuerin}, a \?{torchlight reception}{Blitzlicht-aufnahme} of the guests (and earlier at the church entrance).
I gave a toast, it had to be done. Lasted all in all from 1 o'clock until around 11 o'clock in the evening!

On April 1st, the Thiemes' anniversary, we had them, their Trude, the younger Koehlers and Wieghardts over here. Meanwhile it seems certain that Thieme will be \WTF{broken down} on May 1st. Great misère.

\missing

\?{For once}{Einmal in all der Zeit} a talking film: \textit{Hauptmann von Koepenick}, in many points, above all Adalbert, the same casting as the one we saw in Berlin. Very interesting to compare the very good film with the very good \?{piece}{Stück}. He created something different from the piece, and creates through images and motion — the march of Koepenick!, in the town hall! — what was lost from the words. The language as such sounded good. But useless! Why these few (necessarily few!) boring sentences? Here one cannot compete with the stage. Silent film is art for its own sake, talking film is a poor copy of the theater.

\missing

I have read much these weeks. The "Erfolg" by Feuchtwanger. (Once it mocks the saying of the old field service manual: Action, better wrong than none at all! Hindenburg invoked exactly this saying — with justice! — defending the state of exception in a campaign speech!)
Marlitt "Secret of the old Housekeeper". Recommended and lent by Gusti W; most interesting, much better than its reputation.
\textit{Traven: "Cottonpicker"}. I worked on Traven some more. \missing
