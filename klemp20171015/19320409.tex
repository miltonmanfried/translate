Noon Saturday

The most bitter was: Georg said to me that Berthold and Felix had died in exactly the same manner, a hardening of the coronary arteries. That is \textit{my} end, I have felt it coming on for years, and lately, stronger than ever; it hit me earlier than the other two. And I cannot >overcome the horror.

Georg very old and worn-out, Marta looking frightful, yellow, slim, with oversized eyes, shapeless ankles. Grete younger than ever.
\WTF{Finally, the pastor, horrible, the closing door over the sinking casket.} Betty cried more bitterly than I thought her capable of. As I reached for her hand in the gruesome Defilé, she drew me to her and I had to kiss her. \missing

I left the station in the terrible April weather, \WTF{fighting the cylinder}, first (to kill time) to Wertheim, a little through the \WTF{book district}; then a coffee break at Aschinger on Buelowstrasse. Then to Georg. He was alone — his family in Badenweiler. You could see the scars from his auto accident on his face, you could see his age. Hans Kl and his young wife came. Georg's car in Badenweiler. Anny Kl's car arrived. My brother Berthold must have left behind quite a fortune! The widow kept her large flat, car and chauffeur. \missing
In the car now his elder son with \WTF{beflowered - umflorter} schoolboy cap, a robust youth.
In the crematorium rather many people, even though everything was unofficial. Sober, at least no tactless pastor's speech (without the lord Jesus). No sight of Marta's children. Lotte Sussmann, thin, wretched, medical student \WTH{in the 9th Sem.}, wants to become a psychiatrist. Father Sussmann, gray, mummified. Grete, young. Frankes unchanged.
Afterwards, tension between Marta and Grete, \WTF{whom I am to join}. I finally went with Grete, with whom the old, good-natured, dumb, sober Trude Scherk was. She said: "Betty will console herself. In these cases \WTF{it - der Zuruckgebliebenen}  always comes down to the financial situation". And Marta whispered to me in her excited manner: Lilly is engaged to a man from Montevideo. Grete told of the familial feud, of \WTF{Zimmerherren}, of Walter's extortion, etc. It was extremely revolting. That is all so horribly alien, distant and repugnant to me. 
I had to drink coffee with Grete. Finally I went past old memories in a tram car over Wilmersdorf and Schoenberg to Anhalter station and arrived back in Dresden at 10 o'clock in tatters. 
Other than funerals I don't want to go to Berlin any longer. It is my necropolis. The parents in Weissensee, Berthold in Dahlem, Felix in the Berlinerstrasse in Wilm., Eva's mother in — I don't even know the names anymore.

On the way I read \textit{Stephan Zweig Fouché}.
I have begun the \textit{Corneille anthology}.

\missing