Sunday morning

Always, with tiniest variations, the same situation internally and with the family. Especially hopeless, only that one can almost get used to hopelessness. I.e. every day hopeless minutes and neutral hours.  \WTF{I always see death, Eva's and my unhappiness always depresses me.}
Numbing work. I deal with further serious recensions: Lerch/Molière, Beyer/Pater, Tolzien/Rivière, Jaeckel/Wagner, Goerres-Jahrbook. Since I took up the campaign in April, there are now \textit{nine}. On top of that, a column on Guzmán "Eagle and Snake" — not entirely trivial. On top of that, the constant difficult Corneille corrections. Naturally also readings. 
At home I am now reading aloud from: Fontane: Before the Storm. (I read the Guzmán-Rest to myself). For now, next and last recension Carré's Fontenelle.
I offered Teubner a little book. "The new image of France". He does not want \WTF{to bind himself}. Perhaps it is good that way. Then \WTF{I would have to get to} the 18th century in a couple weeks. Finally, the literary history is my own work. and god knows how far I will get with it.

Did I mention the visit from Comrade Stoessl on May 7th? My training comrade from the 7th field regiment in Munich, 1915. 19 years old at the time. Now in his own car as a sales representative for his father's shop in Ansbach. At the time he had been a guest at our house, a good petit-bourgeois youth. Now — how remote. \textit{We} however were memorable to him, "we speak of you often, Herr Professor".

On May 16th we again looked into purchasing the offered house. Villa with garden, "almost" passable for us — if it could be renovated...in short, not suitable. The people in it were interesting. The husband, older than me, a publishing rep from Velhagen & Klasing. Bought the house two years ago, now has to leave. He, his wife, his daughter — unsophisticated, peaceful, obviously very careworn. On the writing desk: the crooked-cross. For these careworn, impoverished people it is nothing more than \WTF{Kreuz-Hoffnung}.
I myself brought the Guzman article to the DrNN and, as promised, visited the published Wollf.. Very Jewish, very vain, very shifty. Similar to Walzel (complaining about Walzel). Tried hard to be very friendly with me. I can't get into it too far. The husband reported with particular effort and grief how he, a medical autodidact, struggled against the quacks and "should have been made" a Medical Doctor many times over. He is however vain about 1000 other things and is totally convinced of his importance. Has just received an audience with Mussolini and was gushing over him. "He" did not ask whether one is a Jew, which is different than with Hitler, he keeps order and doesn't bother anybody who is not systematical;y in the way! In short: a German Mussolini, if only we had one! That is now the German democratic feeling!