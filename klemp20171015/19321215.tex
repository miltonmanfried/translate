\missing

I have discovered a lending library across the Reka and now I no longer need to purchase books. I read aloud for many hours. Very good (bought): Emil Ludwig, Lincoln, very good: Upton Sinclair: "They call me carpenter". Very weak: two crime novels by Wallace: "Lonba der Spieler" and "Der Hexer". That takes place formerly held by the cinema.

The third Spain study was completed yesterday after a week of work. It is good. But, I no longer have the certainty that it will be printed. Now I want to at last return to the \textit{Image of France}. — The lectures drag on before unspeakably few students, I get nothing out of them, they are superficial and still take me two full days, often three, also wears me out extraordinarily.

Yesterday, as has already happened often, a secret discussion about the rector election with the "Munckel commission" at the Hotel Monopol. I need to think about what suffering I struggled with Max and Foerster a few years ago. They are both, I \WTF{am now among the very old} and my \WTF{interest in suffering} is also dead.

Jules Romain sent me a novel with his signature and I wrote him. He answered me in broken German and asked me about my book ("an exemplar") "on the present situation of French literature", he holds me "in high regard". I sent him V.

Vossler, back from South America, wrote me warmly and sickly. It was the first letter since Lugano; I gave a detailed answer and he gave no further replies. (cf my Spanish Study III - oui, oui et puis non, non.) I had \textit{not} congratulated him on his 60th birthday. Vale Erasme.

A postcard to Grete bore the sender \textit{Bukofzer}. The remarkable nature of childhood memories. A very fat butcher's wife in Bromberg had that name and it made a great impression on me since she died from — mysteriously picturesque word — galloping pneumonia. The name and an unclear portrait with big Galician Jewish eyes stuck with me. \missing

Recently it crossed my mind that for several of the early years of marriage, on the newspaper calendar pages, every day was marked with sexual delight. What an awareness of the value and length of my life at that time. And now every line in my diary must be wrung out of me. And I always think back on my "Paper soldiers", just as how they were never glued together nor put into the planned battles or reviews, only lying unused in the cigar box and finally vanished, so it will be with my many diaries, and with everything else unwritten within me.

We eat well, we drink much wine, and we have never been more unrestrainedly satisfied.

Yesterday at the rector meeting, it was said that 80\% of our students are followers of Hitler.