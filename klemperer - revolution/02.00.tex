\title{Politics and Bohemianism}
\author{From our A.B. comrade}

\dateline{Munich, start of February [1919]}

\contents{The Munich Council. -- The \?{Consummate Bavarians}{Urbayern} Eisner, Muehsam and Levien. -- The political bohemianism. -- The communists-Good with two kinds of love. -- The effects abroad. -- Eisner's prospects for the future.}

It is now with Munich politics as it was with Munich art; one asks: Where are the people from Munich or the Bavarians? In art, one runs up against East-Prussian, Wuerttembergish, all possible names -- and yet it was still "Munich" art. And now in politics? It is really unnecessary to shove aside the Galician Minister-President and to doubt the Germanness of his name. Indeed he himself has confessed yo being a "Preiss" and now even a Berliner.