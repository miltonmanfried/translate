\header{Sunday afternoon, 14/XII Leipzig. The Munich days.}

Up until the last moment in Munich I was \?{over-busy and over-rushed}{überfüllt und überhetzt}. I now want to \?{group together}{in Gruppen zusammengefaßt} the keywords that I have written down and \?{work out their significance}{das Bedeutende ausführen}. There is really so much of significance here that I have taken it upon myself to put this work ahead of the Astrée and the preparations for class.

I have already written extensively on the \?{miserable housing situation}{Wohnungsmisere} \?{in M}{in M}. In addition to that was a trip to the academic housing office. There I met Muncker, who \?{sadly}{wehmütig} relinquished a room, and complained first to him about my situation, and then to a lovely \?{helpful}{beratenden} woman, probably a student. She advised a furnished room, which would come cheaper than a pension (Frau Hauser told me that full room and board could be found for 7M) if one ate lunch in the "student kitchen", a new facility, for 60Pf. The concept of "appropriate to one's station" has apparently now disappeared, c.f. the lecturer Pauli, who appeared in the faculty meeting in uniform and then wore a \?{civillian coat}{Civilmantel} over it; he was dismissed, but he had no \?{civilian clothes}{Civilanzug}! If something turns up, the woman, and likewise the homeowners' bank, will write to Hans, whom I have authorized, and who has also heard much from the Bohemian and merchants' circles. It is temporary, it is not beneath my station, the times are exceptional: these must be my consolations. The academic housing woman complained that the Italians or even the French would march into Germany, since Spartakus was becoming ever more powerdul and giving rise to chaos -- \?{and then there will be even more of a housing shortage in M}{und dann werde es in M erst recht Wohnungsnot geben}. I reassured her on this point. \?{With Spartakus alone would we be prepared if it comes to the barricades...OR...We would be finished with Spartakus alone, and the same if it comes to barricades}{Wir würden mit Spartacus allein fertig werden, und wenn es auch auf die Barrikaden gehe}. How this war has inserted itself into everything, absolutely everything!

\textit{The university}. When I went to see Vossler on Tuesday afternoon, the \?{Romance literature department}{Romanisten} had put together their "shopping list" and entrusted modern literary history to me. I selected the \?{recently-closed lecture and lab}{kurz entschlossen Vorlesung und Übung} for 17th and 18th century France. How I am to start that on the 1st of January I still do not know, but I am going about the preparations -- until then Astrée, as soon as these notes here are done -- and in February \textit{tant bien que mal} to the lectern! \WTF{It must and shall be done, and I must and want to give it due recognition}{Es muß und wird gehen, und ich muß und will Ehre einlegen}. In doing so, I shall learn all sorts of things, above all confidence in my teaching ability. V told me about a general faculty meeting which took place at 5:30 Friday the 13th of December in the small assembly hall. \?{I took my invitation for it to the university}{Ich holte mir eine Karte dazu} and went quite proudly and \WTF{???}{fadennaß} through the driving rain. Only I looked for the small assembly hall at the wrong end of the first floor, though I had been habilitated there. In the large hall I was among almost only unfamiliar faces; individual people introduced themselves to me, even the names sounded foreign. Indeed I am myself a foreigner there. That must change now. I knew Dr Janentzki \?{from Muncker}{noch von Muncker her}, Dr Lerch, who has become a surly opposition-type -- "I left at 6:45, I protested against the fact that account was not taken of restaurants' dinner time!", incidentally he looks well-fed, had overcome the tragicomic death of his poor Sonja, and now only complains of the 5 children of his landlords, who disturb the peace of his furnished room -- Vossler, Muncker and Bäumcker; the last two sat at the special table of the rector and dean, while Vossler sat with us common folk at the big long table, on which lay green plates, white \?{napkins}{Bogen} and \?{utencils}{Bleistifte} as in the family meeting of Antonie Buddenbrook. \?{Lerch pointed Professor Jordan and Lector Simon out to me at a glass table}{An einem Fenstertisch zeigte mir Lerch den Professor Jordan und den Lector Simon}, there were 6 or 7 people sitting on a bench with no table \?{?}{und Blatt}, in all there may have been 40 men present. Only one fast struck me as almost genial, fat, sensuous, with graying curls, smiling at me next to Vossler; I introduced myself, he said that he knew me already -- \WTF{from Muncker}{von Muncker her}: Bernecker, the Slavist. Time went on and people came and left, the caretaker Röder slipped around with decent ceremonious noiselessness and all kinds of papers and lists in which one could \?{sign up}{sich einzeichnete} for the holiday course were passed around; the old Crusius came late, modestly dressed, but emaciated and stooping, \?{shaking as after a stroke}{schleifen gehend wie nach einem Schlaganfall}. The meeting itself came off as rather childish to me; it emerged very rapidly that \?{those who convened the meeting}{die Einberufenden} were completely at odds as to the actual \?{points of to be dicussed}{Verhandlungspunk}.