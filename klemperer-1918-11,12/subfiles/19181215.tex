\header{15/XII}

The question of renumeration was raised. Should the government pay, or the students, should the Ordinarius or the docents or should everyone be compensated, and how does this relate to the wage question, and what does the distinction between "\?{remedial}{Wiederholungs-} and \?{Catch-up}{Fortführungs} courses"?? A ministerial director had given information to an Ordinarius about that, but it was neither clear nor binding -- so everything is just \?{gossip in the wind}{ein in der Luft schwebendes Geschwätz}. The rector Bäumcker, who spoke with delight of the participation of his students at a "colloquium", distinguished himself via smooth phrasemongering. "You mean '\?{lab}{Übungen}', which we have indeed always held" said Vossler very dryly, and then went very dryly into the money question. A dean or similar \textit{pezzo grosso} who was unfamiliar to me practiced his speechcraft by constructing tricky cases \WTF{in which anyone with too few semesters could be pressed into exams}{in denen jemand mit unvorschriftsmäßig geringdn Semestern sich etwa zum Examen drängen könne}. Prof Rehm was upset that in Munich the exercises were mostly public. In total a bubbling mess without clarity, without attention to imprecise language, \WTF{without dignity and depth}{ohne Würde und Tiefe}. My first faculty meeting made me one illusion poorer. Besides that the powerlessness of the docents became completely clear to me. And as Pauli \?{departed}{aufbrach} the college in uniform (faute de mieux) at 6:30, I followed him out of the meeting. So then I also still found a place to sit on the train..\?{here in Leipzig I found on the form the following statement}{Ich fand hier in Leipzig den Vordruck folgender Erklärung vor}, which I signed without hesitation (as did probably all docents): "\?{We honestly and freely pledge ourselves to the peoples' state of Bavaria while remaining true to our beliefs and convictions, in the interest of putting the full product of our labor at its disposal}{Wir verpflichten uns dem Volksstaat Bayern unter Wahrunf unserer Gesinnung und Überzeugung freiwillig und aufrichtig im Interesse der Gesammtheit unsere Arbeitskrsft zur Verfügung zu stellen}. We arrive at this decision with a view to the Fatherland, which now more than ever needs \?{manpower}{Arbeitskräfte}." (From 11.11.18)

\textit{The military}. On Wednesday (11/12) I went to the war ministry with Hans Meyerhof and \?{showed the Vilna card}{zeigte den Wilnaee Auweis vor}, accodding to which the NCO Kl. was to report to the war ministry in Munich. The orderly pointed me directly to the minister's antechamber. There -- where images of battles from other times still hung and where there was incessant coming and going, a young man askes me whether I insist on speaking to the minister himself. I said not necessarily. So then I should go to the adjutant. A young lieutenant, with the title "Herr Doctor", knew my name from the university and gave me a friendly welcome. I explaines that I had been sent from Vilna to Munich, and that I was still without any answer to a request for leave from Leipzig. He dictated to a typist: The NCO Kl., \?{whose military membership is no longer precisely discernable}{dessen militärische Zugehörigkeit nicht mehr genau feststellbar} and whose papers are to be requested from the Vilna press office, is transferred to the 7th Bavarian F.A. regiment, \?{reserve division}{Ersatzabteilung}. The war minister. By the order of Lt ***. With that I moved to the Nymphenburger neighborhood, and have now been running around for three days in the well-known huts of the barracks and the Alfons school. In addition to the white-blue flag, the barracks flies the red flag and is decorates with fir sprigs for the expected return of a regiment. The crew does no work at all, a soldiers' council reigns, a lieutenant colonel \?{goes around in civilian clothes}{läuft in Civil herum}. It was moving to be recognized and greeted with joy by many field-comrades whose names I had forgotten long ago. Finally I met the fresh young Schlosser, who I got along with well in the Plouichferme, and who was already at the time a \?{better}{frischer tüchtiger} soldier. He looked radiant and sharp, is an NCO and wore the EK I in addition to the usual other ribbons. He told me about my 6th battery. They had participated in Montdidier and the 2nd battle of the Marne, then the retreat, and was almost completely wiped out. The student sergeant, with whom I had bet at the time that we would have peace by May 1916, had, like many others, fallen, Ruhl, my friend and \?{arbiter of my fate}{Schicksalbestimmer}, was badly wounded, but lives...strange, how this regiment, which was thought of as a \?{peace formation}{Ruheformation}, had been thrown into the heavy offensive battles...in the division I found Sergeant Laibach, who I knew from Hans M, and who looked after me, also wrote all kinds of \?{private recomendations}{Privatempfehlungen} for me \?{to the relevant authorities}{an die in Betracht kommenden Stellen}. (Now, after the revolution, he is an adjutant and holds an important position). In these letters of recommendation he described me embarassingly and strangely exageratedly as needy. I was an outsider without position and would now seek out contacts with various universities in order to obtain a post, I needed medical leave to recuperate, etc. Namely, it was like this: I could be released immediately -- then I get nothing. I could apply to remain with the troup for another 4 months or until I find a position, then I get all kinds of pay and fees and remittances, but would then have to serve a some bureau or the like; I could demand leave, then I get free board, rather than nothing; finally, I could obtain medical leave, and get the back pay -- I had been released from Vilna in mid-November -- 200M. Hence Munich-Berlin tickets (?{I would have business to do there}{dort hätte ich geschäftlich zu tun}!) and Berlin-Munich, also permission to get the very cheap military tickets on the Leipzig-Kipsdotf line. Thus I have made provisions for my Christmas trave and for a visit in Berlin. Of course there might still be some money troubles, since the remittances for days in Leipzig are hardly in order. Incidentally I have received double pay for some time in October, mobile in Vilna and immobile from a Leipzig territorial company. I try without scruples to get out of them whatever is possible. The doctor who must approve the medical leave did so without a glance and a friendly word. I simply said that I would have to rest before the start of my lectures. \?{I would almost be happy using the second class}{Fast wäre mir auch die Benutzung der zweiten Klasse geglückt}. The doctor wanted to note on my \?{letter}{Billet} that I complained of constant back pain, but I haven't seen him since. There was an amiable, energetic young man. Just before him stood a red-haired, broad-shouldered \?{farmworker}{Bauernknecht} who trembeld audibly -- I don't know if it was real or faked. The doctor said: "\?{You are stubborn, I am stubborn. With jitters and rage I achieve nothing at all. That gives your heart a shock, please be decent to me rather than rude, then give your leave. So, dear friend, consider this...}{Haben Sie an Dickkopf, habe i aa an Dickkopf. Mit Zittern und Wüten erreichens gar nix von mir. Geben's Ihrem Herzen an Stoß, bitten's mich anständig, statt unverschämt zu sein, dann bekommen's Ihren Urlaub. Also, lieber Freund überlegen's sich...}." I don't know what became of that, but the scene was very characteristic of the new regime. Powerlessness of the formerly mighty, \?{self-help}{Selbsthilfe}, completely transformed way of operating...I had \?{sent}{?} one particular letter to the \?{chamber sergeant}{Kammersergeanten} Langermayr, who I had known from earlier. He received me as an old friend, \?{and so I also got along with him well}{und so schnitt ich auch bei ihm gut ab}. He gave me some new long pants, which I must however \?{pack away}{einpacken} ("\?{otherwise they will be stolen, we have so few new things here}{sonst reißens's Ihnen weg, wir haben so weneg neues Zeug}!"), the loveliest new lace-up boots \WTF{zu dauerndem Besitz}, thick shoestrings of \?{real string}{echtem Bindfaden}, two pair woolen socks. I held cold-bloodedly to protocol, that the riding boots were my own property, that I had no more laundry, and other such things, that I did not need to h
nd anything over because of my release besides the \?{oldest uniform with the scraps of my coat}{Uniformgarintur mit dem Mantelfetzen}, which I had been given in Driburg. Also, Langermayr gave me two more portions of bread and \?{tickets}{Anweisungen} to the barracks canteen. The first time I ate in the \?{squadron}{Mannschaftssaal} hall, a man lent me his infantry mess tin and his spoon, and I hastily guzzled down the unsalted food. The other time I sat (with effort) in the NCO canteen, where there were proper clean porcelain plates (blue inscription: 7th F.A.R. "PRL"). There they had marmelade soup and pasta, which was of course \WTF{stale}{semmeltrocken}. Here, at noon on Friday, I met \?{the locksmith}{der Schlosser}. This NCO canteen was the object of my desire in 1915; one entered as a corporal, pulled out of thr ranks, the tumult and the noise of the squadron area. \?{At the time, I had not yet attained the rank of corporal}{Ich habe es damals nicht bis zum Gefreiten gebracht}. Now I believe would have been the end of the NCO-canteen; another comrade who recognized me took me there...the little Saxon art dealer Trautscholl from the 5th battery, who I had last seen in 1916 in the Thomas Church, \?{greeted me as a corporal}{traf ich als Gefreiten} in an office in the Alfons school. Strangely enough, I was uncomfortable moving throughout all these rooms of the barracks and the school, in chambers, bureaus, canteens, cellars, courtyards, upstairs, downstairs. I have suffered everywhere there. Now I went around as a free man. Nothing could happen to me, at worst a refusal of some money. If something does not suit me, I only need to demand immediate dismissal. \?{My tour of duty is over}{Mein Jarhgang ist längst fällig}. And if I leave without all of the formalities, it would probably not cost my head as it would have earlier. There was a certain delight in this sojourn. And I remained there longer than strictly necessary. The dismissal will go quickly in February. I am regarding myself as being in these places for the last time. And the tremendous upheaval, the red flag on the roof, the officers in civilian clothes...there was some humor in the whole affair.

% wce