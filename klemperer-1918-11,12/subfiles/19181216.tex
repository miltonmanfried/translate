\textit{(16/12)}

At the entrance of the Alfons school there is a stone bench eigt up against the wall. I sat there in July 1915 \WTF{??? during a prisoner drill???}{in Sträflingsdrillich} with Eva, who brought me my email in the evening and greeted me. I was not allowed out, I was literally a prisoner, since I couldn't even say hello. I, the 33-year-old professor, the husband, the war volunteer! I know, discipline is essential, and the salute is a necessary evil, I know how it is going now without this discipline, I am however in a "counterrevolutionary" mood, I broke out laughing when Theodore Wolff is now demanding "clearly visible military might", when the Ebert government is forming a "voluntary Peoples' Defense", into which they only recruit soldiers who are over 24 years old and have had a long service at the front, and so precisely the best human material that formed the old army! And yet: when I think of the bench, or of the day the lunatic captain Berghausen bellowed at me in his tropical frenzy that I was to be confined to the barracks since I had forgotten to click my heels while "standing at attention", then it is like an all-too heavy stone on my chest, like a hand squeezing my throat; \?{I then think of my bitterest doubts and understand each of them, but also of the senseless of the revolution}{ich gedenke dann meiner bittersten Verzweiflungen und verstehe jede, aber auch jede Sinnlosigkeit der Revolution}. --

I has adjusted myself to the fuller enjoyment and unscrupulousness of the \?{revolutionary upheaval}{Revolutionstreiben} in Munich, insofar as it is meant getting a little personal gain. I \?{tallied}{von meiner ... berichtete} my tickets and fees, boots, pants, socks and shoestrings. \?{But then I took it too far}{Sodann trieb ich es sehr bunt} with \textit{meal tickets}. There is a place in Munich at the station. There \?{you get marks for your leave-pass}{bekommt man Marken auf seinen Urlaubsschein}, and \?{meals for (fantastically little) money with your travel ticket}{Naturalverpflegung gegen Geld auf seinen Fahrschein}. I had nothing to trade, since it was marked on my leave pass that I had entered Leipzig on the 15th of XII and was in Munich from 10-13 XII. On Hans M's advice I took the \?{letter}{Ausweiszettel} whereby the Vilna press office transferred me to tge Bavarian war ministry, gave it to the ticket-window clerk and said that I would have to do two weeks' service at the war ministry. He stamped the ticket and gave me marks for 14 days. Later I went with the new leave pass to Leipzig and Berlin and Kipsdorf to the ticket window, said, I would still stay in M and got marks for two days. At the \WTF{???}{Naturalschalter} I got a big chunk of sausage and some bread (which they were stingy with) for 80pf.

Finally, in the barracks, where I had no actual claim, I ate lunch twice and ordered more bread rations (very stingy portions - there wasno longer any bread!). So, I could replace much of Meyerhof's, and still bring much home as a pure subsidy to our inventory there. Hans said that zi was a very apt student.

With \textit{Hans Meyerhof} it is still always the same. These people live on the \?{outer edge}{in größter Enge}, cook in the oven without a range, do their washing in a tiny bowl, are extremely crampes with everyone in their two rooms, have all types of antiques and ornaments and objets d'at gathered around them there. Elena has almost entirely lost her hearing and is aided by an ear-trumpet. Hans is still always making his dubious business transactions, sells provisions on the black market, etc, has nominally bought a candy business which (along with other firms) closed because of too-high prices and is now entitled to a 5\% share in the profits. With all this he still is the purest ideologue and dreamer. His hospitality and readiness to help is without bounds. I would again and again have coffee and supper with him, \?{he was upset}{man war gekrankt} when I skipped a lunch in which the evening before he stuffed a goose -- today, a goose! -- with apples. He worries about poor artists, an old sculptor, Ballabene, hangs around him, he is hospitable in grand style. In the evening after the day at the mental workers' council Elena wants to make tea for the three of us. There is a knock at the window, H goes out and comes back after a while with two ladies and two gentlemen, Bohemian types, some of whom he did not know;\?{he was supposed to go see them}{er habe mit zu ihnen gesollt}, but he had a guest, so they had to stay with him. So the 7 of us drank our tea, and even got cakes, and talked politics and aesthetics. (I am now on the "counterrevolutionary side"!) But now there are greater tests for H's ideology than those of his private life. He knew Eisner and some of the other leaders personally. He would have obtained a post, if not a ministry, at least one high up, he would with be able to reach the soldiers' council with ease with his handy connections to all the \?{shady}{anrüchigen} soldiers -- \?{and he has done nothing for himself}{und hat nichts für sich getan}. Hence he was involved constantly in politics. As the newspapers in Munich were stormed, it was also said that the banks were occupied. He had warned a bankers' meeting, had been treated as a blackmailer, should have been rewarded and was turned away. He should be free! \WTF{But he felt some organizational talent in himself}; when the Herren wanted to support him later...! That's how he told it to me, and it could not have been much different. He felt himself lucky in all the \?{extremity}{der Enge}, hurriedness, poverty, \?{adventurousness}{Abenteuerlichkeit}. Perhaps the wife has settled him down somewhat, otherwise he would be quite satisfied \?{??}{??}. The whole current atmosphere is perfect for him. There he really lives. He does not complain about any want, has enough to eat, is always smoking, is always in motion and considers the adventurous uncertainty of his existence as a particular piece of luck. Once he told me so outright. There is no rest in his rooms: traders, bohemians, advice-seekers, friends constantly come and go. Mealtimes are hasty and seldom taken alone. Business, art, politics, friendship, charitity -- everything blurs together. Elena, almost cut off bybher deafness, strained by cooking and washing, nevertheless \?{remains obedient}{hält doch brav mit}. I would not like to live a week like that, and would not let my wife live a day like that. And yet we are \?{kindred spirits}{verwandte Naturen}, we both shrink before the everyday. But he has no ambition at all and much more courage than I, \?{as a rule he is much more ideological than I}{er ist der Allgemeinheit gegenüber viel ideologischer als ich}, and in relation to his family he is much more neglectful of his duties, much less scrupulous. He rushes around more than I, and yet is much lazier than I.

I have achieved something for Eva's \textit{music}. \?{First I got her a brochure from the musical academy}{Ich holte ihr erst eine Prospekt der Musikakademie} and was then with the professor \textit{Kellermann}, \?{the Wolzogen powerhouse}{dem Wolzogenschen Kraftmeyer}, who we knew from Urfeld. He sits on the directory of the academy, where he lectures on music history. He promised to take care of E. Either he would get her permission to practice in a church, and then she could take private lessons, or she could enter as a student in the academy, and will be freed from all the \?{distraction of unrelated subjects}{Brimborium der Nebenfächer} and could then concentrate completely on playing the organ. The semester begins there on the 15th of II, and if we \?{move into a furnished house now}{nun weiter möbliert hausen}, then Eva will indeed have no worries about moving or finances and can likewise proceed with her studies.

I met Kellermann on Friday afternoon in the little house by the streetcar depot in the Nymphenburgerstr. We chatted long into the night; he will soon give his answer in writing.

As for \textit{politics} I have already \?{described}{mitgeschildert} all kinds of details, since politics now plays as much a role in everything, and almost even more so, than before the war. But the necessity of making money brought me above all to two meetings. On Tuesday (10/XII) I went with M to the "Political Council of Mental Workers", which H naturally belonged to and which had a meeting at the Bavarian court. A large, elegant room with stucco and paintings and decorations and a big podium. We got there very early, because Elena wanted to sit way up in the front. Very elegant audience, \?{literati and Jewry}{Literatur und Judentum} -- separate from the people. I found this \?{fearful sycophantic pandering}{speichelleckerische angstvolle Anbiederung}, which is already manifest in the name "mental workers". You should and must be proletarian! On the stage a table with a committee of 6-8 Herren, literary types. In the room the elegance of more artistic and studious women predominates. Several hundred people -- "Volk" are entirely absent. From the table on the stage a young man, more sturdy than slim, gets up, clean-shaven, very strong broad lower jaw, \?{giant mouth}{Riesenmund}, severe gray-blue eyes, thin blonde hair parted to one side, going bald, elegant, firm in tone and attitude, manly like an officer - but not \?{a simpleton}{der Simplicissimussorte}. \textit{Bruno Frank}, whose fine "Sisters and the foreigners" we saw here in the theater. He \textit{read} his lecture "Revolution and brotherly love", but he read it with good, simple, forceful intonation and is from time to time warm, so his brittle unsentimental organ seems good and attractive. He underlined that he wantes to offer practicalities, be touches upon economic issues and many times stresses that now everything is going into \?{the purses of the bourgeois}{Bürgerportemonnaie}, that we would now, \WTF{??}{so furchtbar schwer uns das auch fallen werde}, have to do without the benefits of property, the \?{beautiful things of life}{Lebens in Schönheit}. We must voluntarily take up, must want to, it can no longer be avoided, the control over the community, we must practice neighbourly love. All very nice, with fine remarks and quotes, among which a section from V Hugo's speech to Voltaire stood powerfully -- but overall it remained in generalities and \?{blurriness}{Verschwommenen}; of course it cemented in me the impression that in peoples' assemblies it \textit{must always remain} at the level of generalities, blurriness and trivialities.

% a goose!