\header{Thursday morning. 9/January 19}

\missing

I do very little work; I only rarely get up early, I am only briefly in the Univ. library, where I still always make notes on Rotrou's Saint-Genest. --

Thuringer Hof and Merkur, as always; frequently together with Harms. I told about the Précieuses (yesterday I had just done notes on Somaize) and compared them with the \?{state-of-the-art}{Modernsten}, specifically with a crazy poem that just appeared in the Frankfurter. He asked for a short 50-line article, as he has already often asked for the like. I accepted. In the further discussion, he came again, as he often does, to this: \textit{that I should be sent to Munich to report, above all politically}. It would be quite a stoke of fate if I now, as a university lecturer, could get into real journalism, which I had so longed for in my early years, and which would have completely changed my life (and perhaps for the better). By real journalism I mean (1) politics and (2) \WTF{the reporter's relationship to a paper or the editorial post}{das Berichterstatten-Verhältniszu einem Blatt oder den Redacteurposten}. I have only ever authored individual articles about literary things as an outsider. That cost me much effort and was no more certain or comfortable way of life...I now often fantasize about turning back to journalism, often also about merging it with the teaching position and perhaps succeeding in the \?{geat ocean of politics}{große politische Fahrwasser} (the Reichstag - no, "Volkshaus").

Everything unchanged in Berlin: battles, negotiations, deficiencies, \?{barricades}{Schranken}, chaos. Harms sees the thing with gallows humor. I believe that it personally comforts him that \?{things are going so badly with the B.T.}{man dem B.T. so übel mitspielt}. It is also comical how exactly the revolution-friendly, anti-militaristic Th. Wolff \?{is now gaining practical experience}{jetzt Erfahrungen sammelt}. --

We now have several circles in the Merkur which soon join, soon part. (1) the N.N. and Stettenheim. (2) Scherner and \?{Father Christmas, Popper}{den Weihnachtsmann Popper}, with accompanying ladies. (3) \?{The conservatory people}{die Conservatoristen}. Hempel is a man who, \?{as I get to know him better, is always winning}{der bei näherer Bekanntschaft ständig \?Qgewinnt}: mature, multi-faceted interests, without any \WTF{pretensions to being a genius}{Geniespielerei}. He was a war propagandist, and has \?{still never been able to get a proper job}{nie eine richtige Wirtschaft führen können}, has a wife with the in-laws in Magdeburg...Fraulein Bölike is strange, mid-twenties, fine brunette, apparently has bad lungs, spends all day in the cafe. She's escaping the scoldung of her tyrannical petit-bourgeois mother, \?{who begrudges her her studies}{die ihr das Studium verdenkt}. Their family is rich, the mother gives her several hundred marks a month for private studies alone, had sent her a severak-thousand-mark piano for Christmas. Peculiar contradictory behavior. Several of the girl's sisters are dead, in part due to lung disease, it seems that she also feels death behind her. But there is no noticeable sexual excitement nor other types of bohemian exuberance; she sits quietly and harmlessly, chatting in the cafe; it is her refuge from the squabbles with her mother. --

% Christmas man