\header{Monday morning. 20 January 19.}

Around 8 on Saturday I went to "Ceres", wher everything was like the old days. Opposite from it was the Volkshaus, which belongs to the independent party. The street was completely dark, packed full of people. Right in front of the Volkshaus one saw a smoldering pyre, whose red lit up the narrowest surrounding circle. Someone was speaking loudly, then there was the sound of commandos, and one could see the people marching off...burning of opposition election ballots, flyers. The whole day I saw miserable demonstrations: nothing but hooting 16-year-olds, \?{and weathered old faces among them}{und wüste ältere Gesichte darunter}, also soldiers. They had stormed into the Merkur since they had \?{thought to find a few epaulets to remove there}{dort ein paar zu entfernende Ächselstücke vermutet hatten}. I would have liked to put the knout to these "politicians". They still often print ghastly and violent flyers in the Neuesten Nachrichten, scatter the typefaces, shut down the operation...after eating, to Harms's. I persuaded him to pick the ladies up from the opera. We found the city completely calm under cold mist. On the way Harms explained to me again that he voted for the government socialists and that no one was less suited foe the job than the Liberals. There is definitely personal bitterness at play with H. His wife naively parrots his arguments...Yesterday the election day went off very calmly. Our landlady, who went to vote early -- a pub on Colonnadestrasse -- reported a huge crowd and high turnout; but when we went around 12:30, there were few people present. \?{The individuals who went into the booth ???? }{Das Einzeln die Kabine betreten nahm man nicht genau}. Eva stood just behind the curtain, as I myself just put my ballot in the box there (Inscription: "Saxon Republic"). We both voted for the Democratic list, quickly and without feeling. In the cafe it is said that tumult will break out if this evening it turns out the Independents were defeated. We had arranged with Harms and Kopke (already on Saturday) a visit to the Neuesten Nachrichten for the late evening on election day...there (Petersteinweg) from Scherners with Frau Harms around 11:30. The door, which had recently violently stormed, was closed. It was opened for us. The Neuesten have their building way at the rear, behind a streetside restaurant. The editorial rooms above were nothing new to me. I recognized it from the Berloner Tageblatt. Results from the election were still almost completely lacking; it seems in Leipzig the Independents had secured the great preponderance. Then Kopke showed us the operation of the machines. My absolute ignorance in these things always depresses me -- in the newspaper business especially, since I would still like to work my way in here. The long rows of typesetting machines \?{which add the lines to the cast metal}{die zur Screibmaschine sogleich den Metalguße der Zeile fügen} ("triple decker" with three dofferent types of lines), the peculiar cylinders, on which the entire page \?{appears together}{vereint erscheint}, the ovens for melting down the used metal. On the other side the enormous actual printimg machines...if in 1903 I had been led through the machine room of the "Français" with the Alliance Française in Paris, the typesetting machines and the \?{printing heads poured into the cylinders}{su Cylindern gegossenen Druckseiten}, there were probably none of these. But fundamentally, the machinery was a equally a mystery to me at the time as it was yesterday, only then the mystery did not depress me, and now it depresses me very much...there were some sailors with \?{ammunition belts}{umgeschnallten Patronentasch} and weapons as guards -- whose reliabiloty is very doubtful. Incidentally, friendly decent faces. Throuh snow and rain we got home around 1:00 and \?{made do with a candle}{behalfen uns mit einem Lichtstumpf}. --

\textit{Afternoon}...the delivery of the program for the "{six months of war trouble}{Kriegsnothalbjahr}". It came out that my classes overlapped with those of Jordan. That will certainly cause friction, \?{especially since J might not like me anyway}{zumal J mir sowieso nicht grün sein dürfte}. In my annoyance I said that it was because I didn't live in Munich, that no one saw me there -- that precisely Eva's antipathy against M, the fact that her interests diverge from mine, is hurting me. That was a lot of injustice with a tiny bit of justice mixed in, and caused and causes tremendous displeasure and weighs heavily on Eva and me. Today I have offered her again -- since Dr. Luther has again given up this room here -- to remain here at least until Easter and to further her  studies, until things have cleared up in Munich. --

At the cafe yesterday afternoon. (I have foregone lunch in favor of a rich breakfast; my most recent nutrition was \?{donkey liver sausage}{Eselsleberwurst}, that cost 4M; today the vendor confided in me that the donkey was a horse; tasted expensive and cost just 4M, while rabbit-sausage cost 12M). After that to a church concert in the university church. Organ and violin, wonderful. The violinist played \?{pure and full}{rein und voll}, not screeching like Scherner often is. Among others,? Mendelsohn... Then to Scherner's. \?{Grieg}{Grieg} and supper, but a less fresh mood and good atmosphere than usual (we still contributed some.) \WTF{Frau Köhler was still outside}{Noch war Frau Köhler draußen}, a fresh, powerful creature, gymnasium pupil, was an operaring room nurse, wants to be a medic. Later Frau Harms came.

A new cafe-acquaintance is the fresh young Petersen-Buckel couple. He in his early 20s, landlord and merchant, \?{???}{hier in Stellung}; she, the Buckel, same age, halfway his fiancée, studying singing here. We expect the two of them here this afternoon. Unfortunately, since I get to work so little. Hempel and Frau Bölke have been separated under the strain of their brilliance. The girl is a hysteric, has bad lungs, \?{apparently running from her early childhood}{scheinbar dicht ans Dirnenhaft Streifende}; what she has told us about her relationships is supposedly all lies; she lives, Petersen says, on the money of her cafe friends. Which was presently Hempel, who has entirely fallen for her despite a young wife (\?{married during the war}{kriegsangetrauten}) in Magdeburg. Bölke pursued him, Petersen, with requests for money for her lungs and wrote on this occasion \WTF{to his Buckolina}{an seine Buckolina}. Compared to all these \?{young people}{Leutchen} without affectation, I feel so wistfully old...

\missing

\?{For the first time in 3 days, I wore my civilian clothes}{Seit 3 Tagen trage ich endgiltig Civil}. I have already been ashamed for the longest time of always going around in uniform. A word in the cafe drove me to the decision. Someone said essentially that only the unemployed would wear uniforms...I would have taken it off anyway in 14 days. I have been thrifty for long enough. --


% Kannst du blow mir wo die Pampers ist?