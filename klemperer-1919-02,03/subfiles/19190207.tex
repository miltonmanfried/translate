\header{Friday morning 7/II.}

Yesterday a message appeared on the telegram board at the N.N.: "the demonstration will not occur, since the negioiations for the unemployed have taken a favorable turn." \?{There had been a big deal planned}{Es muß Großes geplant gewesen sein}; then as we were to inquire at the station as to how much the Berlin D-Train would be delayed this time (heavy snowfall!) -- it could however no longer be estimated -- , in the exterior hall \?{a powerful machine gun arrived}{stand in der Außenhalle ein mächtiges Maschinengewehr aufgefahren} and a sign was posted: "At 10 o'clock all \?{those not travelling leave the station}{verlasseb alle Nichtreisende den Bahnhof}." --

\header{Evening 10:45. 7/II}

I held my first lecture from 3-4, room 147, before about 30 students, Eva, and a lady. One moment of stage fright, then everything went well and smoothly. Introduction about the "pulse of the \?{state}{des Staatlichen}" and filled it out with an overview of Aubigné. The works and the readings of the reître noir remain for Monday.

Tomorrow we finally move into the pension on Schellingstrasse 1. Across from the university. \missing -- We ate a very good in the afternoon and in the evening at the Old Pinakothek, went for coffee and in the evening for tea to Hans M, discussed with \textit{Weckerle}, certainly a fine person, but totally obstinate Independent, in between chatted with \textit{Pontius} in the Stephanie around 5, to whom I owe the news about Levien. The man is married, wife and child in Alsace. German-Russian, \?{a pharmacist in Zurich}{stand vor dem Züricher chemischen Doctor}, well-off father. On the 31st of July 1914 his son was born, on August 2nd \?{he had to go into the field}{mußte er als Leiber ins Feld}, now is father is impoverished, he did not return to his studies, he is separated from the wife, who lives the occupied territories. Fine tall blonde man, Weckerle's more resolute opponent. -- Some time in the working room in the morning and afternoon. The D'Urfé is still not yet complete.

I recently noticed in Berlin: (1) officers with white armbands, (2) \?{closed ranks of troops}{geschlossene Trupps} with white and red bands marching to music (\?{changing of the guard}{ausziehende Wache}), (3) an automobile on the Potsdamerplatz with \?{some kind of cargo}{irgendwelcher Lasten}; the soldiers standing on it with storm helmets and weapons in hand, (4) placards for the electoral struggle for the Prussian National Assembly, which has just finished, on all houses, also posted on the memorials on the Potsdamerplatz: the cover of one, "The German Democratic Party is the fig-leave of those who don't dare to \?{commit to the International}{sich zur Internationale zu bekennen} even though they are Social Democrats!" A black heart, from which three thick red drops childishly fall. "What heals the Prussian heart? The German National Party!" A red \?{death}{Todesgerippe} on a golden triumphal carriage, tossing rosses from it. Very effective. -- A handsome little youth: "Mother, think of me, vote Social Democratic!" --

\?{On the long train trip, the following \textit{types} stuck with me}{Von der langen Bahnfahrerei blieben mir diese Typus}: A nice, young, sensible Bavarian married couple. The husband admitted, half-ashamed, to being an officer. He had had an administrative post in Berlin, and had now asked to be relieved since \?{the business had become too stressfuk for him}{ihm das Treiben zu bunt geworden}...An apparently academic sergeant, who told of the chaotic business with the Spartakus people, the sorrows and the embitterment of the affected population in Berlin, and found the \?{lynching}{Lynchgericht} of Rosa and Liebknecht to be understandable. A younger Austrian medic, returning from French captivity over Bordeaux and Seeweg (Sweden), imprisoned at the Skagerrak, a Jew, son of a Hungarian cafétier in Vienna, constantly speaking, also telling jokes, \?{cajoling}{zuthunkich}, good-natured, irrepressible. "\?{Der Papa hat g'sagt, wann keiner zurückkommt -- der kommt zurück!}{...}" -- 

%b