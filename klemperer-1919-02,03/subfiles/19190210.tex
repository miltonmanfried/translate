\header{Monday 10/2. 19. Morning, 8:30.}

The say in Berlin, so the 29th of January, was the first day of heavy frost. I was cold in my summer jacket (which Eva had worn for so long) as I went to the station early at 4:30, I was cold all day in Berlin. Then as we got to Leipzig towards 4 o'clock in the evening, the windows were so thick with frost that I could see nothing. A huge crowd stormed onto the train and blocked the aisles. I didn't know how to behave with Eva, who I had reserved a second seat next to me. Finally I got onto the platform, and I did not see her. I thought: \?{perhaps she is on her way, perhaps she will come tomorrow}{vielleicht ist sie mitgekommen, vielleicht kommt sie morgen nach}. There would probably be leeway in Plauen so that the train could be searched. I had to give up the reserved place. The train was moving for hardly $\frac{1}{4}$ hour when Eva appeared, who was more resolute than I, despite heavier luggage, \?{she kept going}{ihren Weg gebahnt hatte} until she found me. I gave her my seat and stood for several hours. -- In Munich, my first concern was release from the army. It went very quickly. On Friday afternoon \?{I informed them in a few minutes}{informierte ich mich in wenigen Minuten}, on Sunday morning I/31 I finished everything in a few hours. I brought no clothes to hand over, and so of course got no money for clothes, but only 50M for severance. It is different in Saxony. \?{One is medically examined for infectious diseases}{Man wird ärztlich auf ansteckende Krankheiten untersuchen}. I feared a long procedure and told the doctor that Georg had found me healthy a few days ago. "One moment", he replied. "In order to satisfactorily fill out the form, take out your genitals!" -- "Please!" At the same time, without looking: "So, thank you, here is the signature." I met a comrade from the field, corporal, will become a gendarme. He complained bitterly that he was supposed to out together the watch, but it never happened since the people had no desire for it. I ate once again in the lower-officer's canteen. -- --

I was a soldier from July 15, 1915 to January 31, 1919. --

At tea at Muncker's (he himself was absent) on Friday afternoon (31/I), I befriended Kutscher, who seems more reasonable, manly ($\beta o\tilde{\omega}\pi\iota\zeta$) than I had imagined, big strapping man, graying black locks, somewhat hard of hearing. Then I was held up by an old doctor who seemed \?{mentally off}{geistig angeknickten}, Dr Tesskopf or something similar. \?{He accosted me about a brochure}{Er redete in mich hinein über eine Broschüre} in which he proved that Wilhelm II had a long-existing mental illness, and which the newspapers wanted to know nothing about. Later he also spoke about a study about Perrault, which he wrote. Eclectic, not confidence-inspiring.

\?{The lady Muncker}{Die Munckerin} \?{cautiously told}{erzählte schonend} about the death of old Crusius. He seems to have come home quite drunk, and then suffered a heart attack at night. Stefl told me yesterday that Crusius was apparently a \?{drinker}{Potator} and had \?{gotten clean}{steril gewesen} 10 years ago. He had moved Rehm her as a second Greek scholar, to be sure to have a collegue that didn't \?{outpace him}{über den Kopf wachse}. Stefl gave many "in my opinion"s with much knowledge and without médisance. --

Those are probably all remaining recollections from the trip. The frost has meanwhile not subsided. Everything is covered in snow, and bitter cold.

\?{This morning I am going to Wetsch}{Heute Vormittag steht Wetsch vor}, where I will fetch a winter coat and a warm military jacket for the house. A piano and \?{luggage}{Pachtkoffer} are also to be sent for. -- I am calm about by lectutes. I only ever work on one to two lectures in advance, in order to have the material fresh on my mind. In between, the lecture on Astrée is being planned.

\missing

The apartment is continually making good and also unfortunately \WTF{nutritious}{nahrhaften} impressions. Up to now, a small circle: landlady a sprightly grandmother of perhaps 50-55, Frau Conradine Berg; a lovely granddaughter, about 7; two students, one a lieutenant and a student at the technical school, the other a very young jurist, three girls who study and look slightly Balkan, a middle-aged lady.

\textit{Evening}.

Grim cold. Outside and in. An abundance of exasperation at Wetsch. One of the trunks from Leipzig was broken, the \?{civilian clothes}{Civilkleidung} in a big chest where they have been simce 1915, totally smashed, the price for transporting just the piano was 40M. --

\missing

