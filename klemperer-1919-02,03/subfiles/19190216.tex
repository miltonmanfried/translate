\header{Sunday evening, 16/II 10 O'clock}

Today an all-around bad day for work. I only put the Descates Vita together after many \?{drafts}{Vorbildeern}, and yet still haven't filled tomorrow's lecture. So in this respect I am always living awkwardly from hand to mouth. From where am I to take the time for scientific literary work?

This morning a funeral service on the Ludwig church for the fallwn students. Archbishop Faulhaber presided. I got nothing from his sermon, \?{I saw right through his pantomime}{seine Pantomime sah ich genau mit an}. \?{Great splendor of the student body}{Großer Prunk des Studentisches}. I squeezed out a few lines for the Leipziger N.N. about that, and about the crowd and about the importance of colors in Bavarian folk-life (and thus undercover about the importance of the Catholic church). As I said, a miserable day.

Meanwhile Eva tortures herself excessively with her theoretical tasks and 2x in vain visited Kopke's friend \textit{Pidoll}, the composer who I met in Leipzig, and \?{from whom she wanted to borrow a harmony lesson}{von dem sie eine Harmonielehre ausliehen wollte}.

After eating briefly (after a long pause) with Hans M, after supper very briefly in the smokey \?{newspaperless}{zeitungsarmen} cafe Stephanie. There sat Pontius and played chess with Levien, who wore elegant civilian clothes. The whole while he complained still complained about Levien in the most violent tones, and \?{repeated coarse jokes about him}{ihm Gemeinstes nachsagt}. --

At \textit{Vossler's} on Friday evening, it was in interesting in many regards. Lerch, the incorrigible, had his sister-in-law with him. A fraulien Rabinowotsch, very similar to the late Sonja, hateful, speaking with a strong Jargon accent, studying philosophy and music theory here, seems to be anti-Eisner in politics. Does Lerch want to take the sister of the dead wife, as in the old testament? -- I brought up his words: with newer literary history one does not become a professor; Vossler disagreed. I put forward my  Corneille thesis: the individual vs the state. Vigorous protest from Vossler and Lerch: Corneille recognizes no individuals, but rather only ideas which struggle with one another. -- Vossler read aloud some of the critique which he wrote in an upcoming compilation on my Montsequieu: I viewed the man too romantically, too softly, I read into him a personality, a heart, which he did not possess. I seemed to have intended the same with Corneille...It was sad for me to recognize how much Lerch had \?{surpassed me}{über den Kopf gewachsen} in the meantime. He is now coming out with two syntactical books: one on the methods and one (Samson-Stiftung prize) on the Latin future tense. He told of how a student helped him with it...And, as hard as I try, I cannot extraordinarily prize such work. \WTF{One is bound to have an idea in very great gaps which to prove one then spends months collecting material}{Man ist dazu verpflichtet, in sehr großen Zwischenräumen einen Gedanken zu haben, den zu belegen man dann Monatelang Stoff sammelt}. Rather: nous verrons...a \textit{suggestive future tense}. Now I look everywhere for suggestive future tenses, and \?{the content of what I read for this purpose is irrelevant}{...inhaltlich einerlei}. -- Perhaps Winter shall allow the sleeping German-Latin monthlies to arise again, and hence necessitate a new editor. Vossler told me that I could apply for it myself, I would not be unsuitable. Perhaps I'll do that. -- There was a sharp debate over Prussia. \?{I passionately defended it from attacks on all sides}{Ich nahm das von allen Seiten angegriffene mit Leidenschaft in Schutz}. 

-- I have just written the letter to Winter.

\missing

