\header{Tuesday afternoon 25/II.}

Idiot visit with the professors. Yesterday afternoon outside with \textit{Rehm}, in a villa on Parzivalstrasse; I had already been there once, met him at the door and then obtained my lectureship at the tram station. (I am now very worried about the payment.) We met the wife, still youthful, talked with her for a while about the political chaos. This morning we \?{browsed}{grasten...ab} the Wiedenmayerstrasse. At the Isar, very beautiful; sleepy pre-Spring weather. No one at home, only left cards for Otto, at Wölfflin and at the widow Crusius. We are finished for a while.

-- I read (leafed through parts) the \textit{Lanson/Corneille} to the end today and now do not know whether I should write the study on \?{the State in Corneille}{das Staatloche Element in Corneille} that hovers before me or not. It would cost me all manner of lectures. In any case I will work some more on lectures today.

Then in the morning I readied a short article for Harms: on the uncertainty of the situation here, the struggle between soviet snd democratic ideas, the ringing of bells for Eisner. It is the third time that I have ventured to send such an Eisner article to Leipzig. I don't know, does he need it there, did it come too late, is it superfluous? One is far too isolated here. It is just this feeling of helplessness that I am always trying to give expression to. I must gamble on this journalistic experiment, it might perhaps lead me out of the misery of my current existence. Since payment for the preparatory courses has again become uncertain, the money issue severely depresses me again. --

\textit{12:30 in the evening} We were together with "the Lerches" in the Stephanie. She is incidentally a divorcee and has a ten-year-old child. We went to the "police commissary" on Amalienstrasse, where we two private docents rather clumsily but pleasantly presented the permits; for me he had to first telephone the pension to inquire whether I actually live there. Really there are only permits for doctors and midwives; but since we had a scientific conference...we decided to use our papers to take the ladies with us to Vossler's. At home I worked first on the Corneille lecture. Then I was consumed by the idea of a little column on the "Right question" wherein I pithily and forcefully attack the jesuitical mischief of the suppression (lifted today) of the bourgeois papers, which appeared as de facto Spartacus organs, excepting only the sharply-oppositional "Post". I wrote the few lines, and in the joy over them committed to read aloud from the wonderfully pointed Bagatelle Pontio...At supper Eva was very tired, also had very poor eyesight. I left her at home so that she could sleep and I alone went with the Lerches, who picked us up, to Vosslers. The Ludwigstrasse was deserted, ther was more life on Ludwigstrasse. At Vossler's first there was violent political grumbling; neither V nor Lerch were sad about Eisner's death. (Incidentally, as I put my column in the box, Lerch also threw a fat couvert in there: a column about Eisner for the B.T. He equated Eisner and Arco-Valley as political fanatics.) Then literature: I again saw myself attacked with my Corneille idea. Vossler and Lerch regard literary history as pure \?{Formalism}{Formlehre} and aesthetics, and spoke rather contemptuously of my "psychology"...we left at 11:30. At the Victory Gate we saw a large group of people and avoided them, went along the academy to the Amalienstrasse. There a guard and a soldier asked for our papers. We two men brought them out, they did not inquire further about Lerch's sister in law. But they warned us good-naturedly: there was a patrol of 60 men on the way and already \?{picked up several people}{führte schon mehrere Leute mit sich}. We should be on our guard. Pleasant! After we dropped off the Lerch girl, we went once again to the Ludwigstrasse. It was now entirely empty, though bright and shiny.