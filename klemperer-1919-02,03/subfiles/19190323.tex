\header{Sunday morning 8:45. 23/March 19.}

Yesterday Lerch presented me with his just-published little book on \?{Modi}{Modi} in French. In it he already points towards his "future tense" which will be published as a prize article, also announces a historical syntax. How quickly he will become a professor and ordinarius with all of that! And I stand aside, totally outdates, totally "unscientific". I am quite uneasy...At Lerch's urging I took part yesterday in a \textit{Sitting of the Private Lecturers and Extraordinarius Association}. The chair, old-philology Maurenbrecher, is to provide reports on \WTF{achievements and efforts in scientific reform}{wissenschaftlicher Reformhinsicht Erreichte und Anzustrebende}. After an hour I was left embitteres and deathly bored; one could certainly find more logic in a servants' association than in this. It would have been material for Molière, this jumble of polite phrases, \WTF{pretentiousness}{Salbadern}, childish chit-chat and time-wasting. About 20 little men present, among them a wide range of ages, the oldest lector (among other things) Prof Hartmann with white fringe of hair and a completely bald head, a novelist since middle-age, a gruesome memento for me. Maurenbrecher, middle-aged, with a big powerful nose, gave the report. I.e. he said in spirited, general phrases that he had negotiated with the senate that the ordinarien were entitled to everything that does not infringe on \textit{their} rights and what the government had to pay, but not however what infringed on the ordinarien rights; we would have to go on to the government and finally to the Landtag, he asked for a resolution that we were committed to "holding out". And so on for unending speakers \WTF{with much talking over one another}{woran man sich aber nicht hielt, vielmehr redete man durcheinander}. \?{Holding out was roundly rejected}{Man könne kein Durchhalten in Bausch und Bogen beschließen}, they only wanted to know details point for point, they were not informed, etc etc. \missing

I left, swearing to myself never to come again. This useless waffling is a pure waste of time. \missing

On Wednesday an old student came to see me, Keller, sent by Muncker.  He had asked M for a doctoral topic from in the French-German area. Muncker advised: Montsequieu/Herder, and then sent to me for further instructions. I danced around it a little without being able to say anything, then recommended finding a topic betweem Hugo and the Young Germans -- in a sudden idea tertio loco -- the vanishing of French loan-words in German between Sturm und Drang and the Napoleanic wars. That seemed to at least please the lad -- much reading, little thinking, philology, true science! My first advice to a doctoral student. --

\missing

Only the rarest news from mother, otherwise no more news from Berlin at all. I am completely isolated.

\missing
