\header{Monday morning 21/April.}

Yesterday I wrote a "revolutionary journal" article, and I now feel like all of the news and possibilities for the situation have been exhausted and that this can remain the last article unless the most fundamental upheaval intervenes. After that

 made some notes on the still quite-weak \textit{"Weltkindern" by P. u. V. Margueritte}, also read some Zola. He makes machines and animals into people, people into machines. Materialistic pantheism. And every \WTF{mounted}{besprungene} cow and every machine in action becomes a symbol to him. In the evening, the strange, \WTF{well-endowed}{an guten Dingen reiche} Frau Dr Sobat and Dr Ritter sang here. \?{If they both could only sing a bit!}
 
But the new development in recent days is that we -- I think the first time since February 1st! -- took a walk \WTF{"outside"}{ins "Freie"}, i.e. for an hour or so down to a bench by tue Monopteros. It has truly become Spring: new leaves, green willows. On top of that the full and rapidly-running oily white-green brook. I was suddenly overtaken by a yearning for nature and wandering. Eva, who this yearning has probably never left, was totally sorrowful. We sit locked in this multifaceted captivity. Eva's leg always fails after prolonged usage, the trains are open only for workers, also it would be too expensive to pay the rent here and to live elsewhere. We are going to wait until the Summer break and then go to Tubingen, where a piano, a university library and countrside will have to be found. But anything can happen before August! I am yet again pursued by the hope that the Leipziger NN could offer me the post of Parisian reporter. That would give me the most difficult inner conflict. A great journalistic opportunity or a lectern -- which entices me more?

Thus, yesterday afternoon a proper Easter walk, two steps before the door, cioè before the veterinary hospital. For the first time in many, many months I remembered that there is Nature. And that it is not always Winter. --

Today got further on Corneille.

% Herr Drumpf wird kampfen fur das Proletariat