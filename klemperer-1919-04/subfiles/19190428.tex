\header{Monday afternoon 3 o'clock 28/4 19.}

I got up at 6 o'clock, worked on the \textit{Corneille study} until about 9, and then dead-tired and weary with despair \?{packed it up}{couvertierte sie}, although I knew that there were still some citations missing in the Polyeucte section. I couldn't do any more. Despite Eva's pleading I then took the manuscripts to the post office. There, it was refused -- the post is only running as far as Dachau! I took it as the will of the angels, went back home, and \?{somewhat grudgingly}{trug einiges nach} put the last section right with cutting and pasting. At noon I was finally completely ready. Whether the work stands on its own? $\frac{2}{3}$ \WTF{applied Lanson very well}{wenden Lanson sehr gut an}, the last third, the idea of the state, is probably my own. The Korff recension and the Corneille study are now the only harvest from the Easter holiday. Now I must work on the lectures, first of all inserting my notes on Bête humaine and \?{reading more deeply}{lese etwas reichlicher} into La Terre.

On the way I first heard from Pontius, then from the friseur, that Levien has been toppled (he is said to have flown in an airplane to Hungary with 140,000M in support for disabled veterans) and that a provisional government has just been formed, which will immediately negotiate with Hoffmann and the troops. What is true in that remains to be seen, but it seems that at least the worst times have passed here, and \?{I am thankful}{hab ich mir allweil denkt} that it will not come to tragic bloodletting. The people are are too worn out with everything. \?{It becomes ever more muddled}{Es wurstelt immer weiter}.

Yesterday morning I read the Polyeucte until 11, yesterday afternoon I wrote, in the last moments before eating I read the end to Eva, after eating there was again some music. But between 11 and 1 it went festively here. First, only the studious Kaeser was here, \?{who was asked here this evening}{der auf heute abend gebeten wurde}. But then, returning our visit, (a) Geheimrat Schick, small, gray-haired agile and very lively with his English wife and (b) as the Schicks \?{arrived}{aufbrachen}, Lehmann and his wife. We wer already intimate with them, since there was already some gossipping, and the wife told of the customs and \?{ancient traditions}{Zopfigkeiten} of the faculty community, at least before the war. \?{Docents make visits everywhere with all docents on the faculty}{Dozenten machen überall Besuch, bei allen Dozenten der Fakultät}. From time to time they are invited by the Ordinarien; ante bellum there were luxurious dinners. \?{But if a docent himself invited [someone?] to a dinner}{Hätte aber ein Dozent selber zum Diner eingeladen}, then that was taken to be "bold" and held to be inapropriate; a docent can only invite his most intimates to tea. \WTF{We laughed a lot}{Wir haben sehr in uns hereingelacht}. How the revolution changes things. Earlier the rector wrote: To the Herren Professors and docents. Recently, regarding the fact that none of us become strikebreakers against the "revolutionary higher education council", it is now: "Colleagues".

The \textit{Sarason} continues. Like the first day 8 weeks ago I had a clash with the loutish and comically puffed-up Jew, who asserted that German merchants always cheat. Since then I've kept my peace when he declaimed, did not immediately involve myself in the farces to which he increasingly fell prey. He waved a knife around, spoke with his hands and feet, bellowed, told of his progress in riding. Dr Ritter especially teased him a lot, and he finally withdrew back to his room for dinner. But now he constantly complains about too much music, can be seen in the corridor in his undershirt, sitting in the closet with the door open. The day before yesterday he posted an unsigned note: he only allowed music between 12 and 3 in the afternoon and during his absence, which he will make known by a white sheet on his door. I wrote on the note: "Such things are not posted here. One closes closet doors and behaves like a civilized European." Even Ritter left a note, and we sent the sheet back. Afterwards an open letter from the Herr of the health council and publisher of the medical yearly course to me: he would have to ask me, but he would not challenge such people to a duel, he appreciated my brother Georg, I had no similarities with Georg, I was (three times) unashamed, childishly uneducated...and he will move out, in order to no longet disturb the homogenaity of the building! After long reflection, I replied as Eva suggested: "Since you boast of being acquainted with my brother, and since we find ourselves in the same building, I will spare you the \?{fisticuffs demanded by your letter}{Ihren Brief gebührende handgreifliche Antwort}. Incidentally, I have taken note of the fact that you have left the house."

% 