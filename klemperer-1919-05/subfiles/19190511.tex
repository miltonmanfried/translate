\header{Sunday morning, May 11th, towards 8 o'clock}

From 9-10:45 yesterday I stood in the full auditorium maximum by the lectern, with Muncker, Rehm, etc with me. Essentialky it was a recruiting meeting for Epp. Then it was made known what exam- and financial relief for war veterans was decided upon by the senate. What I did not mention in my report was that \?{the young student and Epp officer was also ?as agains his will? in a single word, antisemitic}{der junge Student und Eppoffizier im Mannschaftanzug auch wie gegen seinen Willen in einem einzigen Wort antisemitisch wurde}. It must not again happen that \?{foreign}{land- uns rassefremde} students should close the university to us without \?{being tossed out}{niedergeschlagen zu werden}...This and the apparently serious atrocities of the soldiers put me off very much. The essence of the assembly was the \textit{Schill-ian tone}. This appeal, these "führer reserves" apply not merely to Munich and the revolution -- after the peace deal, certainly not merely the inland. At 10:45, as the details of the university-relief for the war veterans \?{continued with no end in sight}{als...kein Ende nehmen}, I left. Met Lerch and was again drawn into an unpleasant discussion about his sister in law...Then in the newspaper room I saw that the LNN, under May 6th, still only had telegrams from Augsburg about Munich, nothing directly out of Munich. It can be assumed that no more lines from my revolution diary will appear. At least I got the manuscripts back.

At home in the afternoon I wrote a new article on the situation here, short and very pessimistic. For the first time in a long time, I sent it out un-registered. I am very proud of a sentence in it: "I always fear that \?{the speaker could slip up}{der Redner könnte sich versprechen} and instead of Epp, say Schill". It signifies a great deal, and \?{explains nothing that cannot be explained today}{Damit ist...nichts ausgeführt, was heute nicht ausgeführt werden darf}...Towards evening at Hans M's, who is recovering slowly from his operation and seems politically calmer since Weckerle fled. He has constant house-searches -- but nothing is found...after eating Ritter, Sobat and the now-back from Salzburg Fraulein Gehmacher came to see us. Eva, because of her bile, was on the sofa. --

\textit{Evening towards 11 o'clock.} The spokesman for the soldiers is named Hemmeter and despite his crew uniform is a first-lieutenant. The people get the \?{spirit of their rank}{Gehalt ihrs Ranges} and form closed officer-companies -- where does the equality and sacrifice remain there? ... Another uniformed spokesman was unpleasant for me, who revealed himself in his first words to be an aesthete and a phrasemonger: "In these unrhythmic days...", by which he meantvthe revolution.

After eating, with Eva to Meyerhof's, where Behle was as well. Italian reminiscences, peaceful. Then walked here with Hans and the others. I read aloud one of my articles from the LNN, Eva played the piano. Later I read the La Fontaine dissertation to the end and started the first lines of the recension. Also read some Zola aloud, which I would like to continue... We (+ Fraulein Gehmacher) just came from the Stephanie; Dr Ritter is again laid up sick. 

\textit{Küchler} wrote me, he wants to bring out the Corneille study, but wants to know whether I could shorten it. I will ask him to wait on the printing until he can bring out the whole thing. Midday \textit{Dr Borcherdt} paid us a visit with his powerful, Christlike, Saxon wife. By brave student, Fraulein Idler, lives with him.


% u. nregistered