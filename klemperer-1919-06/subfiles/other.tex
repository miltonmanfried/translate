June 23, 1919
{Saturday, trip back from mother's funeral in Berlin}
On Saturday morning already at the station at 6:45, though the train only departs an hour later. Nevertheless, only standing room and always squished closer, so that I could hardly shift my feet. From Nuremberg on, however, sitting room. The people on the railroads in these days: old good-natured master craftsman, organized craftsman between SPD and USPS; a Saxon jurist my age with an adorable youth and a lovely wife; Insurance-Reichsrat, pianist and Liszt-enthusiast, crippled by a shot to the hand; an independent worker, sworn to the world revolution and the soviet republic, a petite-bourgeois widow full of anxiety about her wealth, etc, politics permeates everything, everyone for peace and more or less convincedq of Germany's guilt; {more in tune with the socialists than conservatives}, most are indifferent to foreign policy.

{...}
July 3, 1919

{...}
Because of the war-relief I was with the current dean Frhrn von Bissing, who as an all-German is well-known and much-hated, who during the bad hostage days kept himself hidden; a small, skinny, elegant gentleman with blond mustache, blue eyes, nervously rolling his head and neck back and forth. I had never personally spoken to him, knew him only from faculty meetings (in later times he had indeed always been represented by Rehm). Waiting outside, I asked a younger lecturer, the religious history lecturer Heiler, whether one said "Herr Dean" or "Herr Professor". "He likes best to hear 'Herr Baron', but I say 'Herr Professor'!"
I {used many 'Herr Baron's}, and we understood eachother very well. Also political. Also university politics, where I was against hasty reforms and {Futterneid}, and was pleased with the uncommon character, which in these {times don't grow on trees}. i played a role, but it wasn't difficult for me, because I stand a bit closer to the right than the extreme left. Bissing showed me a dented bullet, which had flown into his dean's office on May 1, and as {some bullet in the university} certainly was intended for him personally, he named it "Tell's shot". He mentioned his father in Brussels (the late Gouverneur), he stressed that he lectured unpaid - I don't know why - in Munich, that he maintained his conservative opinions without personal gain. We understood eachother very well and I shared my {concerns}. That I need to have the title of professor, and would like to have it quickly. On this he (who incidentally wore a medal in his buttonhole) said: he had {processed my Genter affair}, knew my {merits}(!) and my particular bad luck, he would ask that I as soon as possible would become professor. On Friday there would be a faculty meeting. 
{...}
In the evening we we to Stephanie. Pontius was there, yesterday Meyerhof too - but the main thing bleak times...the conclusion of the peace of Saturday {leaves} one dull and nearly without words. First, the ceremony was only worth silence; today a B.P. article by Paul Fock is circulating, {which has been seen in Versailles}. The general feeling is basically: {wordless} extortion and disarmament, so that there is much anxiety and distraction about the domestic situation. We are probably close to a new explosion.

July 15, 1919
Tuesday evening towards 12:00

Two extensive letters from E, things are apparently good with her....after eating with Meyerhof, only very boring tiresome people there, Hamburg Jews, who of course know my local relations - awkward! Afterwards however very lively alone with M and the {decent (naturally also donating) long} Hemecher. After a while - Elena had made us coffee - there appeared a young fresh handsome man of perhaps 20 years in an officer's uniform without epaulettes {with EKII, Saxon August band}, wounded- and aviator badges. {The discussion was eclectic}: stockings, suspenders, soap, cigarettes, of a trip to Darmstadt amd Frankfurt. Best behavior, confidence-inspiring. Which troop did he belong to, I asked. "None at all" was the calm response, even though the wearing of uniforms is forbidden. A moment later I heard that the youth was credentialed by Noske (I looked at the papers once more later. "8th Hussar regiment. Holder is a member of the Freikorps. Government troops, Berlin. Signed Noske. He may carry weapons. All authorities are requested not to hinder him...and to support him." Three stamps. Space for signature and photograph.) With that, one could move freely in all of Germany, can travel to Berlin via military car (and thus for 4 or 5M), is {allowed} in barracks and can eat there. {Good for trading and agitation. The 80M certainly will be good for a couple of trips already. There is no high-treason in it. Incidentally Hans M has already given me a holiday pass for 25M, perhaps one for Driburg}. As the youth left, he said that he was one of a company of 20, behind which stood Berlin donors; they bought contraband in large quantities for Berlin! Nowhere have I seen so deeply into the uprooting of our public life, in his boundless Russification, as with Hans. Against the state we all become amoral. Indeed there is no state any more. Went home through incessantly and powerfully pouring rain.

{...}
August 9, 1919
Saturday after 6, lecturer's newspaper room

{...}
Yesterday evening the speech councilor / professor Beyerle, member of the constitutional committee in Weimar, Centrum man, was equally bad in form, gesture and content. The man, middle-aged, in a frock coat, paced the width of the podium of the Aud. Max. back and forth like a caged tiger, with great arm-motions, perpetually screaming and declaiming in hackneyed phrases - Anchored! In defense of dee people, in defense of dee national assembly, etc., acted {as if he would have to celebrate the king's birthday in a country club} and had almost completely forgotten about all of the actual statement which determined the new constitution. This afternoon my new table-mate, the Ordinarius of history, Hirsch, from Prague, asked how the lecture had been. I tore into it violently. He said that he was acquainted with Bayerle, {...}.
Hirsch (Catholic), formerly an Extraordinarius in Vienna, said awful things about Ph. A. Becker, he was the joke of the faculty and had been the worst {Radau-}anti-semite, had himself been subjected to the most violent antisemitism! (thus he was clueless a about me, as I was clueless to him!)

August 10, 1919
Sunday evening, 11:30

The time of my lonesomeness has run out. I anxiously ask myself whether Eva missed me, as I had her. She has developed spiritually above me, is more self-reliant than I. She is perhaps also more sensually sedate than I. {...} I love her more than she loves me seems to be the right formula; {and in any case that is not to my credit}. {...}

I was this morning, in very strong heat, with Hommel. Old, mobile, talkative man, Orientalist. A little house out {past} the end of Leopoldstr. Showed me {a self-written college tome}, introduction in Romance philology with Ebert in Leipzig 1878, showed me his garden, in which young ducks waddled around, gave me a couple of flowers to give to her, asked me to come together with her in Autumn. [...] Prof. Hirsch, the Prague historian with whom I often talk shop, told me yesterday that I was very obviously suffering from "Privatdozentitus". Even Vossler spoke recently of the Privatdozent-sickness! I must see, perhaps better to hold the tongue. {Why is Lerch silent?} What do I know about his prospects in Cologne, in Prague? And I make myself ridiculous before all the world with Dresden.

August 15, 1919
Friday afternoon, 3:00
Driburg garden, Müller pension

i sit in the garden of the Müller pensions as in 1917 and 1916, the same goats, the same whistling trains, the same landlady, the same room -- only the veranda and the large dining room are now rented, and only I myself am different. I intrude into Eva's tranquility like a {troublemaker}, same erotic agonizing, where she needs rest and enjoyment and hardly has the urge for erotic excitement, same chilling worries about money and the future, same dim moods. I am recently so overwhelmingly tired and spent, that only just now started preparing my university lectures for print (which now the International Newspaper for Technology amd Culture wants to print) {...} This weighs on my mood.

The trip was all too exhausting. True, I almost always had a porter or other help, but a few times I had nevertheless to {ernst zugreifen}, and the long drive, the heat and the {lack of sleep-overnightness} made the smallest load too heavy for my heart. The seating was most uncomfortable, there were also much quarreling and bustle.\WTF{In Munich, one was forced to press against ladies in the narrow passage.}{In München, drückte man Frauen an die Barrière um den schmalen Durchgang zu forcieren}.

August 16, 1919
Saturday afternoon

Eva left around 6 to go to the fountain, and I slept until 8. Now after eating I slept on the lawn of the garden from 2-3. We also ate in the garden at noon. Complete relaxation, eating, sleep, slow promenade, reading aloud; no excess of eroticism after the first violent welcome. But the inner satisfaction is lacking. The future is sad,  and compared to Eva I feel superfluous and almost burdensome. We are affectionate with one another, but I still have the feeling of only disturbing her peace here. {...}

August 21, 1919
Thursday morning
Driburg

Almost a week here; I rarely leave the area of the house or garden; only towards evening I accompany Eva to the fountain. I start my day early at 6:00; I get up as soon as Eva is up (to bath and the fountain) and work a little. The lectures on the French University was ready for presses yesterday, I've just read the Snobbuch by Curtius. Then towards 9 breakfast with Eva. After the mud bath she lies in bed, or in the hammock. I do much reading aloud, we are already on the second volume of the Count of Monte Cristo. After eating I also sleep, in the garden on the plaid. And again reading aloud, and to the fountain (after reading). Then supper, usually in the garden, ample. A few people around, above all the always-speechifying Krupp engineer Bröhl. And so begins the torture of the lightless evening. There is only light in a shared room. Once Eva played Halma and I wrote the lecture; often we went for walks on the dark country road{...}. Most difficult heat, today for the first time ceaseless rain. I don't feel good. I can hardly work here, and the rest gives me little enjoyment. I am like a {fly in Eva's vaseline?}{foreign body in Eva's cure}. We don't always understand eachother very well, there are quarrels about trifles. I will return to Munich at the start of September, in order to get back to my work. It's a shame that I still haven't dreamed the Dresden dream.

August 30, 1919
Saturday morning towards 6:00
Driburg

Today we want to make a trip to Weser. But this will be the first time that we leave our narrow confines. {...}

Heiss wrote me the day before yesterday, that my prospects had "deteriorated" on the way from the commission to the senate. Now they are probably essentially buried. (And yet one always hopes in the innermost corner of the heart). But even before this sad news my heart was already heavy. I can no longer find any courage for the future {Zukunftsmut}. I don't always strike the right tone with Eva, who on her part is depressed by the chronic failure of her knee and often speaks of incurability.
{...}
Twice there was a military concert at the fountain for the benefit of the convalescent home, which is still always occupied, and is now apparently limited to one's own house. There one could see the miserable petit-bourgeois-ness of the local visitors. They contrast strangely with the elegance of the park and its stately half-timbered buildings. And it was also strange where these people get the money. {..} when I see a military band or something else that reminds me of our army, I become ever more bitter. One consolation is that I am never able to take the current situation for anything even approximately final. It seems to be an evil intermezzo of sickness, which must pass...\WTF{It occured to me, what recently moved me towards Hans M, who is actually quite mad}{Mir fällt ein, was mich neulich an Hans, dem wirklich sehr reichlich verrückten, rührte}. The Spartacist and Internationalist is said to have given his business-friend Pojero in Palmero information about the German situation. Since he wrote in a business letter in proper Italian, Pojero wanted this verse by Heine translated by someone knowledgable in German: "Denknich an Deutschland in der Nacht, so bin ich um den Schlaf gebracht"....

August 31, 1919
Sunday after lunch, 2:00
Driburg, pension garden

The main gain from yesterday's trip was the Carlshafen-Hoxler water-ride. 
{...}

September 16, 1919
Tuesday morning, 7:00
{...}
New guests: a factory worker from Essen with an English wife and two well-raised little kids. The wife betrays herself by her pronunciation as an Englishwoman, was embarrassed when I asked her about it, was pleased that she was received well by all of us, took part with childlike jubilant joy in Stöpselspiel.  How is one to hate individual foreigners as enemies?? The french can do it, Ella Doehring could too.

The most appalling thing in the current German state of affairs is that I can find no party that could speak for me. The Nationals push the antisemitism ever more disgustingly and repellently. it is a frightful shame and at the same time downright comical that the Jews should have all the blame: for the war and the revolution. To the nationals, the traitors to the state and Bolsheviks (Leviné!), to the revolutionaries the "capitalists" and warmongers. No one sympathizes with them, no one accepts them as German.

Afternoon: 
