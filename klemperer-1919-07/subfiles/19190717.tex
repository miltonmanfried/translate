\header{Thursday morning 17 July 19.}

Yesterday Otto Kl sent me what he found in his father's library and shall "present" to me, a \textit{Lexicon medicum} in 8 languages, German, English, French, Italian, Spanish, Russian, Hungarian and Japanese. Very amusing. --

The one-time \?{war-veterans' payment}{Kriegsteuerungszulage} of 500M was granted to me. --

\missing

The whole morning I prepared the seminar and \textit{exams}. The whole seminar on cultural history is going well...For the exam in the little room (at the same time a mathematician was taking an exam) six people showed up, who incidentally as veterans did not need to. Overall \?{it went well}{uns sehr brav geantwortet}, only one told me that he had never heard the name D'Aubigné and then added that he hadn't been present in the first classes. Another asked me to "examine him quite extensively" so that he could get a good score, he needed the stipend. I was not completely fair. I could not give entirely outstanding scores and found in the brief exam no criteria for differentiating them. \?{So the grades were given out a bit arbitrarily}{So kriegten die Leute ein bißchen aufs Geratewohl}, one 0.5, some 1s and some 2s. It is said that \?{the stipends are awarded not only to the 0.5 and 1, to the "excellent" and "very good"}{das Stipendium falle nicht nur der 0,5 und 1, dem "ausgezeichnet" und "sehr gut" zu}. --

After this strenuous day's work I was quite worn-out. From the work room I wrote to Eva, who had sounded downright sad in her brief card to still not have had any mail from me for 5 days after Sunday, and who also complained seriously about her knee again...then I read a bit in the newspaper room. --

At home I found the 500M grant. But a bigger surprise was awaiting. A "Grand gentleman with a \?{windblown}{Wehendem} mustache" had been here, and without leaving a name left, said Frau Berg at lunch. \?{Just the \textit{Vossler} rang}{Gleich darauf klingekte Vossler an}. (In previous days he had telephoned as to the date of my habilitation, \WTF{since he is now making a request because of the A.O. professor}{da er jetzt die Eingabe wegen des A. O. Prof.s macht}. The faculty has already decided. Io, Lerch, Frankl, Gallimathias Majr, etc. I no longer believe so strongly in it, and \?{it no longer makes me happy enough}{es beglückt mich auch nicht mehr ernsthaft genug}). This time: "\textit{Heiss} from Berlin is called to Freiburg." I: he has fantastic luck, it also goes with modern literature and "representation". Why does he depress me so? He: "Shall I \?{lift you up}{Sie aufrichten}? My wife is in theater, come see me." I went, many hopes running through my mind on the way. Vossler said: Heiss had written him, afterwards he had "sulked" for three years, since he, Vossler, had brought Würzburg to Kuechler instead of Heiss. Now however he stood up for H in Jena, who then got Schulz-Gora. He, V, was not asked about Freiburg. Heiss believed that he, V, was to blame about Freiburg. Vossler immediately answered that he was not to blame for Freiburg, but he accepted thanks for Jena. And now Heiss could thank \textit{me}. And Vossler thought that I would be a candidate there for the first position. I would already have to be supported by Walzel and Heiss \?{among the workforce in the Athenaion affair}{um der Mitarbeiterschaft an der Athenaionsache}, modern literature is needed there, even Becker was very favorable. And if I would go there, then I would also go further. I would have plenty of time there to write books. -- Certainly, I would go further, since everything happens \?{through personal connections}{durchs Persönliche}, and Vossler (we also spoke very much about it) might now come to Berlin, though he bristled a bit against it, \?{but he could be persuaded}{läßt sich aber doch gern zureden}, and from Berlin on he will be Distributor Cathedrarum. And suddenly his opinion towards me is again milder. He laughed: he baited, made snide remarks, he exaggerated -- with the best intentions. I proposed to him: I will work on my first literary installment (Petrarca). After that in the Winter an Astrée monograph. And in the Summer I will come out with a language course: French in the 16th century, from which I will certainly develop some study or another. He was exceptionally satisfied with this plan, and the reconcilliation was completed. Everything is just fine with him. He sees that Heiss, who only wrote one Balzac and was sent to Berlin "for disciplinary reasons", who obtained the lovely Freiburg Ordinariat, it occurred to him that I was very probably Heiss's successor, that I had already once had a call to Gent -- and suddenly everything is different. If the evening brought me nothing but this reconcilliation, then it would already be a very good evening...After Vossler I have only an honest \?{broker}{Contrahenten} in Dresden: \textit{Curtius, Bonn}. Who habilitated together with Heyss, who has just written a little book about the \?{all-modern}{allermodernsten} French, who \?{instead of me}{statt meiner} got the call to Posen and -- declined. How I now already recognize the course of things, how some totally unknown third will come to Berlin. But I struggle against it so hard: the hope again carries me as in the days of Posen and Gent. It naturally also makes me incapable of doing any work...Dresden would be strange; then our destiny would remain bound to Saxony...Vossler lent me his Habilitation paper: \textit{Poetic theory in the early Italian renaissance}. \WTF{He intended to include}{Er meint zu..in der..} Morf's Overview in the Teubner collection: yet \WTF{it only came out in}{nur aneinander gereihte} in the little Goeschen volume. With such overviews one should have the courage \textit{to only bring out the whole thing}. I will do that: Petrarca, Boccaccio, Machiavelli, Ariost, Tasso -- that will be my Italian, everything else must go in with them. --

\missing


%q