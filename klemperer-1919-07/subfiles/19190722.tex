\header{Tuesday night after 12 o'clock. 22/7}

The whole day very productive. Morning and afternoon made notes for Vossler. 50 of 87 pages are ready; in the notes the matter has become quite clear and has been very enriching. The greatest enjoyment: the ability to work. I succeeded today, but it is so rare; I am usually worn out...At noon I went to the \textit{Süddeutschen Monatsheften}. There were difficulties on the telephone, request for enrollment in writing; \WTF{I then showed in writing}{da zeigte ich dann gleich schtiftlich wegen...an} the university lectures and the Curtius book. And instead I made two professor-visits. At the psychologist Becker's (Schackstrasse) I only left a card, at \textit{Weymann}'s I got a friendly reception (by him alone). An older classical philologist in a shabby old house on Amaliemstrasse, in a rather shabby library. House-slippers, \?{sagging white socks}{herunteegerutschte weißliche Strümpfe}, \?{hitched-up}{heraufgerutschte} pants, between them underpants of filthy, crumpled Alpaca. But a lovable rather \WTF{tussled}{sorgenvoller} graybeard. Advised me to work less and to think more of my health. --
Afternoon until 5:30 worked some more, then a letter to Eva. When that was ready Vossler came, read by a letter from Heiss. I should send my curriculum vitae. But Curtius seemed to occupy the foremost position. His book \WTF{fell through}{gefällt}. Lommatzsch also seems to be under consideration, but seems to be too big for Dresden. Neubert is named, Glaser and Winkler, unknown to me. Curtius will be the victor...I discussed with Vossler only reading until Aug 15; he was in agreement. He requested I come to him in the evening, the Lerch would be there -- without the sister-in-law! After eating I wrote very quickly, perhaps too quickly, the CV for Heiss (the many German studies in it will discredit me -- I only named Wilbrandt 1906 -- , the Dr. phil. Wilhelm Klemperer and evangelistic confession disgusts me); then I went to Vossler. There only the Lerch couple. Frau Vossler is in Landshut. The whole evening and also the whole way home (late) we discussedthe university reform and Lerch's hobby-horse, the \?{minimum for existence}{Existenzminimum}. What was interesting to me was Vossler's \?{measured}{stimmungsmäßige} and brilliant opposition: talent pushes through all \?{obstacles}{Minima}, genius always gets the necessary money, if the state worries about everything then life will become boring, etc etc. Naturally for all that nothing came out of it. I was very tired and very glad when we finally parted.

A more joyful report from Eva today. She had finally received at least one letter from me, he now has hopes for Dresden. But Curtius will get it. --

\missing


%