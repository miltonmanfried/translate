\documentclass[a4paper,11pt]{article}
\usepackage{amsmath}
\usepackage{amsfonts}
\usepackage{titling}
\usepackage{graphicx}


\newcommand{\WTF}[1]{\textbf{???}\textit{#1}\textbf{???}}
\newcommand{\?}[2]{#1\footnote{\textsc{Translator note}: #2}}
\newcommand{\nequ}[2]{\begin{align*}\tag{#1}#2\end{align*}}
\newcommand{\uequ}[1]{\begin{align*}#1\end{align*}}
\newcommand{\unit}[1]{\text{#1}}
\providecommand{\operatorfont}[1]{\texttt{#1}}
%\newcommand{\operatorfont}[1]{}
\newcommand{\grad}{\operatorfont{grad}}
\renewcommand{\div}{\operatorfont{div}}
\newcommand{\curl}{\operatorfont{curl}}
\newcommand{\rot}{\,\operatorfont{rot}\,}
\renewcommand{\exp}[1]{e^{#1}}
\newcommand{\pXpY}[2]{\frac{\partial #1}{\partial #2}}
\newcommand{\ppXpYY}[2]{\frac{\partial^2 #1}{\partial {#2}^2}}
\newcommand{\dXdY}[2]{\frac{d{#1}}{{d{#2}}}}
\newcommand{\ddXdYY}[2]{\frac{d^2{#1}}{d{#2}^2}}
\newcommand{\mf}[1]{\mathfrak{#1}}
\newcommand{\Nth}[1]{{#1}^\text{th}}

\newcommand{\citeauthor}[1]{\textsc{#1}}
\newcommand{\citetitle}[1]{\textit{#1}}
\newcommand{\citepub}[1]{#1}
\newcommand{\citevol}[1]{\textbf{#1}}
\newcommand{\citepage}[1]{#1}
\newcommand{\citedate}[1]{#1}
\newcommand{\citeyear}[1]{#1}

\newcommand{\El}[1]{\text{#1}}
\newcommand{\mnEl}[3]{{}^{#1}_{#2}{\El{#3}}}

\newcommand{\publication}[1]{%
    \gdef\puB{#1}}
\newcommand{\puB}{}
\renewcommand{\maketitlehooka}{%
    \par\noindent \puB}


\newcommand{\location}[1]{%
    \gdef\loB{#1}}
\newcommand{\loB}{}
\renewcommand{\maketitlehooka}{%
    \par\noindent \loB}

%\newcommand{\dX}[2]{\frac{d#1}{{#2}}}
%\newcommand{\dY}[1]{{d#1}}
%\newcommand{\pX}[1]{\frac{\partial{#1}}}
%\newcommand{\pY}[1]{{\partial{#1}}}

% \newcommand{\original}[1]{}
\newenvironment{translation}[0]{
}

\newenvironment{original}{
\renewcommand{\footnote}[1]{\small{(Footnote: ##1)}}
(\textit{Begin original text}:
}{
\textit{-- end original text})
}

\newenvironment{letter}[1]{
	\newcommand{\letternumber}{#1}
	\newenvironment{header}[0]{
		\newcommand{\from}[1]{\newcommand{\headerfrom}{####1}}
		\renewcommand{\to}[1]{\newcommand{\headerto}{####1}}
		\renewcommand{\date}[1]{\newcommand{\headerdate}{####1}}
		\renewcommand{\location}[1]{\newcommand{\headerlocation}{####1}}
		\newcommand{\note}[1]{\newcommand{\headernote}{####1}}
		\newcommand{\makeheader}{
		\begin{tabular}{|p{0.9\textwidth}|}
			\hline
			Letter \# \letternumber\\
			From: \headerfrom \\
			To: \headerto \\
			Date: \headerdate \\
			\ifdefined\headerlocation
				Location: \headerlocation \\
			\fi
			\ifdefined\headernote
				Notes: \headernote \\
			\fi
			 \hline
		 \end{tabular}\\
		}
	}{

	}
	\renewcommand{\d}[1]{d##1}
	\newcommand{\figw}[2]{
		\begin{figure}[h]
			\begin{center}
			\includegraphics[width=##2]{##1}
			\end{center}
		\end{figure}
	}
	\newcommand{\fig}[1]{\figw{##1}{150pt}}
	
}{
 \quad  \\------------------------------------\\
}
%new environment, define _ to append to denom and ^ to append to num
%in after, \frac{\num}{\den} — _/^ should be undefined automatically

%\providecommand{\n}{}
%\providecommand{\d}{}
%\newcommand{\numer}{}
%\newcommand{\denom}{}
%
%\newenvironment{derivative}[1]{
%	%\renewcommand{\numer}{}
%	\renewcommand{\n}[1]{\let\numer{\numer #1##1}}
%	%\renewcommand{\denom}{}
%	\renewcommand{\d}[1]{
%		\let\ndenom{\denom #1##1}
%		\renewcommand{\denom}{\ndenom}
%	}
%}{
%\numer\\
%\ensuremath{\frac{(\numer)}{(\denom)}}\\
%\denom\\
%}
%
%\newcommand{\D}[1]{
%\begin{derivative}{d}
%	#1
%\end{derivative}
%}

\newcommand{\D}{\mathrm{d}}

\newcommand{\derX}[2]{
\renewcommand{\dY}[1]{\ensuremath{\frac{#1 #2}{#1 ##1}}}
\renewcommand{\pY}[1]{\dY{##1}}
}

\newcommand{\dderX}[2]{
	\renewcommand{\dY}[1]{
		\renewcommand{\dY}[1]{
			\ensuremath{\frac{#1^2 #2}{ {#1 ##1}\, {#1 ####1}}}
		}
	}
	\renewcommand{\ddY}[1]{
		\ensuremath{\frac{#1^2 #2}{{#1^2 ##1}}}
	}
	\renewcommand{\pY}[1]{\dY{##1}}
	\renewcommand{\ppY}[1]{\ddY{##1}}
}

\newcommand{\dY}[1]{\text{ERROR: dY without dX}}
\newcommand{\pY}[1]{\text{ERROR: pY without pX}}
\newcommand{\ddY}[1]{\text{ERROR: ddY without ddX}}
\newcommand{\ppY}[1]{\text{ERROR: ppY without ppX}}
\newcommand{\ddX}[1]{\dderX{\D}{#1}}
\newcommand{\ppX}[1]{\dderX{\partial}{#1}}
\newcommand{\dX}[1]{\derX{\D}{#1}}
\newcommand{\pX}[1]{\derX{\partial}{#1}}

\begin{document}

\begin{letter}{80}
\begin{header}
\from{Pauli}
\to{Land\'e}
\date{1925/01/05}
\location{Vienna}
\note{Postcard}

\makeheader

\end{header}

Dear Herr Land\'e!

Best thanks for your last two cards and your friendly invitation to stay with you. I arrive on the 9th at 2:07 in T\"ubingen (assuming the correctness of my information from the travel bureau).

With best greetings, until I see you again.

Yours,

W. Pauli

\end{letter}
\begin{letter}{81}
\begin{header}
\from{Bohr}
\to{Pauli}
\date{1925/01/10}
\location{Copenhagen}
\note{Typed}

\makeheader

\end{header}

Dear Pauli!

I am here sending back your lovely paper. If you can spare a copy of the same later, I would be very happy to get one. After many delays I am now getting into preparing the note on the Rydberg-Ritz formula. I do not believe it, but perhaps it would be desirable to refer to your paper with a word. Please write me a couple of lines, whether that would be alright.

It was an old arrangement that you were to come here during your father's visit to Copenhagen, and both I and Kramers, who you ridicule, have already long enjoyed arguing with you again. Later I will write you again on the practical details of your visit.

With best wishes for the new year and many greetings from us all.

PS. Jeg opdager pludselig, at jeg efter gammel Vane har skrevet dette Brev med Kramers' Understøttelse paa Tysk. At jeg med Heisenbergs Hjælp har skrevet de forrige Breve paa dansk er vel en Yttring for en Reciprocitetssætning af dybtliggende Natur, hvor Kampen mellem Vane og Forbedring spiller en lignende Rolle som i Atomsvindelen.

Jeg opdager pludselig(plotzlich), at jeg efter(after) gammel Vane har skrevet (schreibt) dette (diese) Brev (Brief) med (mit) Kramers' Understøttelse (Unterstutzung) paa Tysk. At jeg med Heisenbergs Hjælp har skrevet de forrige (vorige) Breve paa dansk er vel en Yttring for en Reciprocitetssætning af (of) dybtliggende Natur, hvor Kampen (Kampfen) mellem Vane og Forbedring spiller (spiele) en lignende Rolle som i Atomsvindelen (Atomswindel).


I suddenly notice that, as in old times, I have written this letter with Kramers' support in German. That, with Heisenberg's Help, I have written the previous letters in Danish, it is an expression of a reciprocity of deep nature, where the struggle between sense and improvement plays a similar role as in the atomic fraud.

\end{letter}
\begin{letter}{82}
\begin{header}
\from{Pauli}
\to{Land\'e}
\date{1925/01/15}
\location{Hamburg}
\note{Postcard}

\makeheader

\end{header}

Dear Herr Land\'e!

I have just sent off my paper, and it it I have taken into account all of the wishes of Herrn Back that you shared with me in your card from the 12th. After the theoretical calculations and results on the paper published by Grotrian I wrote: "The previously unpublished measurements of the Zeeman effect \WTF{einiger Bleilinien} by E \textit{Back} make it further very likely that the first-named four $p$-terms get the $j$-values $(2,2,1,0)$ and that the $g$-values of these terms are in agreement with the theoretically-expected values.*"

Footnote to this *: "These results were made possible by the cooperation of Herrn Black, who in the most friendly way permitted me to view his measurements even before they were published. For this I would like here to express my thanks to him. Herr \textit{Back}

\end{letter}

\end{document}