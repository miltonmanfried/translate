\begin{thebibliography}{14}
\bibitem{1} \textsc{G. Mie}, Ann. d. Phys. \textbf{37}, 512, \textbf{39}, 1 and \textbf{40}, 1 (1912-1913); \textsc{H. Weyl}, Raum-Zeit-Materie 1920. \textsc{A. Einstein} and \textsc{W. Mayer}, Sitz. Ber. der Preuss. Akad. d. Wiss. 1931.
\bibitem{2} \textsc{W. Pauli} and \textsc{V. Weisskopf}, Helv. Phys. Acta \textbf{7}, 709 (1934).
\bibitem{3} \textsc{W. Heisenberg}, Zs. f. Phys. \textbf{90}, 209 (1934).
\bibitem{4} \textsc{B. L. van der Waerden}, Gruppentheoret. Methode i. d. Quantenmechanik, Berlin 1932.
\bibitem{5} \textsc{L. de Broglie}, Une nouvelle conception de la lumi\`ere, Act. Scint. Hermann, Paris 1934.
\bibitem{6} \textsc{G. Wentzel}, Zs. f. Phys. \textbf{92}, 337 (1934); \textsc{P. Jordan}, Zs. f. Phys. \textbf{93}, 464 (1935), \textbf{98}, 709 and 759 (1936); \textsc{R. de L. Kronig}, Physica \textbf{2}, 491, 854, 968 (1935); \textsc{O. Scherzer}, Zs. f. Phys. \textbf{97}, 725 (1935).
\bibitem{7} \textsc{M. Born} and \textsc{L. Infeld}, Proc. Roy. Soc. A. \textbf{144}, 423 (1934) and following papers.
\bibitem{8} \textsc{P. Jordan}, loc. cit., Zs. f. Phys \textbf{98}, 761 \WTF{and following}{u. ff.}. (1936).
\bibitem{9} \textsc{E. Fermi}, Zs. f. Phys. \textbf{88}, 161 (1934).
\bibitem{10} \textsc{E. Majorana}, Zs. f. Phys. \textbf{82}, 137, (1933).
\bibitem{11} \textsc{P. Jordan} and \textsc{E. Wigner}, Zs. f. Phys. \textbf{47}, 631 (1928), cf also \textsc{W. Heisenberg}, Physikalische Prinzipien der Quantenmechanik, Leipzig 1930.
\bibitem{12} See e.g. \textsc{W. Heisenberg}, loc. cit.\cite{11}, pages 109 and 113.
\bibitem{13} \textsc{E. J. Konopinski} and \textsc{G. E. Uhlenbeck}, Phys. Rev. \textbf{48}, 7 and 107 (1935)
\bibitem{14} c.f. W. Heisenberg, Zeemanfestschrift, Haag 1935.
\end{thebibliography}