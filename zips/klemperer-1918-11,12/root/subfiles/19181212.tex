\header{Munich. Thursday morning 7:30. December 12th 1918}

Pension houses, Bayerstr., as in the last Munich days in October.

Munich makes a \?{livelier}{belebteren} impression on me than Leilzig. Perhaps only because it was rather more tranquil and had no street-life at all. Perhaps it is also the festive \?{display of flags}{Fahnenschmuck} that makes the difference. It appears very colorful and cheerful, with lots of white-blue, not uncommonly red-white-gold, frequently black-yellow, occasionally red. Occasionally and in-between it doesn't look aesthetically bad. Isolated, it blows proudly in front of the residence, but in the small size, corner of Ludwigstr and Hofgarten. The \?{Field commander's hall}{Feldherrnhalle} is free of disruptive colors, in front of it stand two \?{obelisks intertwined with trees}{Tannenumwundene Obelisken} with the inscription 1914-1918, but there is nevertheless a strong hint of revolutionary life here: at many places pamphlets are sold, or revolutionary newspapers, numerous posters -- where before the army reports were stuck up! -- threaten the people with "ruthless use of firearms" if the order here were to be disrupted from the right or left (which?!). From the \?{posters}{Litfaßsäulen} an appeal \?{to stop the wild joyrides with women in military vehicles}{die wilden Spazierfahrten mit Damen in Heeresautomobilen zu verhindern} because of a lack of supplies, reports of political assemblies. And all of those had a large audience, important groups and gatherings. Leipzig carries on with its usual lively business life, Munich is in festive excitement...the soldiers are no longer conspicuous here. They are brightly adorned; they wear on their field hats (which are bolder than Saxon ones) \textit{red and light-blue} ornaments (ribbons, bands, flowers), they carry \WTF{reservist canes}{Reservistenstöcke} with colorful stripes. They have kept the Bavarian cockade, not the German; one no longer sees any red-white-black. Very amusing to see rows of scruffy soldiers standing before a shoeshine having their boots cleaned. \?{One must know how the soldiers had to constantly struggle with shoe-shining earlier in the barracks}{Man muß wissen, wie sich der Soldat mit dem Stiefelputzen früher in der Garnison ständig zu quälen hatte}! They must give over the penny with the highest delight. There is a boot-sonnet by Heyse: "Whoever has a Soldo can be cleaned, and wheoever can be cleaned is a Signore..." In Leipzig, one occasionally sees squads of Russian prisoners of war. They are wretched and modest and don't attract attention. The peculiarity in the streetscape here in Munich is the French flaneur in a fine uniform. The bright red hose, the cap of the alpine hunter looks as bold as the cap of the Bavarian soldiers, they walk around victoriously in little groups. I bought a satirical (incidentally lame) pamphlet "Red Hand" (not far removed from Glassbrenner): it points very specifically at the indignity of the men and women of Munich by the French.

I have put these brief and more g
neral remarks first, since I will probably only later summarize and annotate the greater part of{} these very full and exhausting days.

I arrived here on Tuesday the 10th at 1:45, instead of the planned 10:45. Since then I have never had even a bit of rest, but rather have always been in violent motion. And so it shall probably remain until tomorrow evening, when I make my return trip. Three matters need to be resolved here: (1) the university and Vossler, (2) the military business, (3) apartment searching. The first two are proceedimg very satisfactorily, on the third I failed completely. I did not think such a housing shortage was possible. The rental bureaus (Lion on Ludwigstr, the homeowners' bank on Sonnenstr, the city housing office on Karlatr) \?{had hardly a half-dozen listings}{legen kaum 1/2 Dtz Zettel vor}. Among them I found one three-room apartment for 1800, the remainder for 2000-3000M. I don't need to see that first, I can't pay it. Then I arrived at the idea -- naturally via Meyerhof -- of looking at \?{studios with adjoining rooms}{Ateliers mit Nebenräumen}. Two were listed, Leopold- and Ungererstr (?{far outside}{in dieser weit draußen}). When I got there, they had already been rented long ago. There is said to be something available for 800M in Unterhaching. But, on the Deisenhofen line. \?{Thus one would first have to go to the East Station with the tram, then wait for the train, which seldom runs}{Man müßte also erst zur Ostbahn mit der Tram, und dann auf den Eisenbahnzug warten, der selten gehen wird.} Impossible, especially since the trams at the moment already stop running at 7 O'clock in the evening. So I have still not even actually looked at one apartment. Frau Vossler offered me in all earnestness two of her rooms: by law here people with large homes must quarter others with them; it has already happened with several other families, Frau V wanted to at least know who she would have with her. \WTF{Vossler was quietly embarrassed, and I was as well}{Voßler vir schwieg betreten und ich auch}. There is still the university housing office, essentially for individual rooms. I saw the \?{placards}{Anschläge} yesterday -- amusing, and harmonizing with what I heard from Vossler -- are the names of the renters: professors, privy councillors, \?{?}{Oberstudienräte}, etc. Incidentally, what does "large home" mean? The V's have 8 rooms, but also 3 children (one grown, one adolescent son and a small daughter). They are supposed to give up two rooms. I want to inquire today at the university office. But I can see it coming, that I shall depart with the matter unsettled. Then Hans [Meyerhof] will have to find something for us in the middle of January, which must not cost over 900M. \WTF{Since with such small objects a months' notice is usual}{Da bei so kleinen Objekten monatliche Kündigung üblich}, -- according to the new law, essentially \textit{only} the tenant needs to give notice! -- \WTF{giving such power to Hans can have no completely catastrophic consequences for us}{so kann solche Vollmacht an Hans schließlich keine allzu katastrophalen Folgen für uns haben.} I cannot even say that I am very \?{upset}{verzweifelt} by this shortage. I accept the situation as fate, against which I am powerless. I do not know what shall be come of us financially; somehow we must go on living, and indeed probably well. And in Munich during the revolution one must allow for much which earlier was not befitting one's social status. --

Now back to the barracks, and then it will probably be night before I can sit down again. \WTF{I try to hold the day's events to one page of keywords}{Ich halte den Tagesverlauf in Stichworten auf einem Zettel fest}. To the three points mentioned above are added the Meherhofs and politics. On Tuesday I was in the "Political Council of Mental Workers" and hope today to attend an Eisner-meeting.

% keep they heads ringin 