\header{17/XII}

\missing After Frank, a series of \?{discussions}{Discussionsredner}, various interesting types. One drew a thick prepared manuscript from a bag and began a lecture on the \?{damage to the university's constitution}{Schäden der Universitätsverfassung}. He was interrupted and had to stop. A man with narrow eyes behind glasses claimed to have been a proletarian and a worker, spoke to thunderous applause against the outrageous behavior of those who sit idle in the coffee-houses when the proletarians go to work or return home dead-tired. One energetically demanded more specific projects instead of more general meaningless words, but himself gave only phrases. Some reproached Frank for writing war poems in 1914, many for only haven spoken aesthetically today. He, full of bitterness: he was not at all aware of having "spoken so frightfully beautifully", and it was not a betrayal of one's views if he had been enthusiastic in 1914, if at the time he still believed in the innocence of Germany and has now discovered otherwise! A young man with an awkward stereotypical smile, but not the \?{brazenness}{Unverschämtheit} which he was accused, but rather obviously the embarrassment, had the courage to go against the general mood: the people don't want to learn anything at all from the educated, \?{he had done very badly as a volunteer teacher at the peoples' courses}{er sei als freiwillig Lehrer bei Volkskursen sehr schlecht gefahren}. Raging indignation. A genial-looking elder with long white hair, \?{the painter}{der Maler}, and, as Hans told me, paid Russian agent Stuckkols, spoke an incomprehensible broken German in the most unashamed tone for the Bolsheviks in Russia. He was interrupted often, he went over the time fixed for the discussion, and was not to be silenced. A very dumb blonde brought a comical intermezzo into the awkward monotony by revealing herself \?{with ??? naivite}{mit gretchenhafter Naiviät} to be a librarian in a peoples' reading room, spoke of her experiences, and began to read verses \?{that she thought were relevant}{die aie zur Sache gemacht habe}. Frightful \?{lyre verses}{Leierverse}, so that I was embarassed for the poor creature. Some laughed, some cried "stop", others bellowed "freedom of speech", yet others: "but no \?{freedom to be dense}{Dichtfreiheit}" -- and since the woman kept droning on, finally the clapping of the whole room was so loud and sustained that it drowned out everything and forced her to stop...overall, this assembly made a wretched impression in me. Hack writers, literati, \?{phrasemongers}{Phraseure}, \?{turncoats}{Manteldreher}, cowards! What does that then mean: you must give them your money, \WTF{become one if the masses}{ihr müßt Masse werden}, work with them, submit yourself? How can ther be art, where want reigns? The artist, the educated must be free of worries and \?{bread}{Brod}. Whether the artist comes from the noble or proletarian circles -- he simply cannot create if he is \?{financially hobbled}{geldlich gelähmt}. All these hacks - one was there, \?{constantly quoting Glassbrenner}{ständig ein Glassbrennersches Wort citierend}: "\?{The cursed fellow, the ghost, we will catch him!}{Den verfluchten Kerl, den Geist, wollen wir schon kriegen!}". That was a perfect caricature of a hack writer made flesh, \?{sat on the board of Gulbransson}{saß im Vorstand an dem Gulbrannsontisch} and stood at the speakers' platform -- all these hacks \?{know just as well as I}{wissen das so genau wie ich}, but for now they are "mental workers", proletarians, tomorrow again exclusive and the day after to become something else again. (\?{So like Scherl's week there}{So wie Scherla Woche dort}, where the obligatory image of the Kaiser once stood, there is now an image of Wilson). The meeting seemed large and \?{turbulent}{heftig bewegt}. But it was only an aesthetic \?{diversion}{Spielerei} compared to Thursday (12/12).

The two electoral assembles of the Independent Social Democrats were announced. We went to \?{Treffler}{Treffler}, where the \?{festivities}{Carnevalsredouten} were to take place, on Sonnenstr. We, i.e. Hans, Elena, Merz and I. Towards 7 O'clock we found the hall closed by the police because of overcrowding. Hans lead us through the neighborhood. A \?{guard}{Wärter} already prevented several people from entering illegally. Hans said that he had to personally speak to the prime minister and slipped through. I wanted to follow him and was \?{held back by my arms}{an den Armen zurückgehalten}. I said emphatically that I absolutely must go directly to Eisner, I had a letter from the war minister in my pocket -- and actually got past with that. At the same time, Merz and Elena squeezed in. We went through small \?{lounges}{Buffeträume} into the \?{great hall}{Riesensaal}, directly underneath the big podium. The room was oppressively hot and smoky. Thousands of people packed together here and on the surrounding \?{galleries}{Tribünen}. \?{But on the rignt was a round table at which there was drinking, and even indeed eating}{Es standen aber rechts runde Tische, an denen getrunken, ja sogar gespeist wurde}. On the left rows of seats were assigned. All of the space between the benches and tables and in the \?{aisles}{Gänge} were packed with people. And waitresses were constantly pushing themselves through this crowd with 6, 8, 10 beer mugs in hand! In was an \?{incredible skill}{unerklärte Kraftleistung}. The heat was soon so bad that one saw open tunics. On the large podium sat a colorful young group, probably the \?{kinfolk}{Verwandtschaft} of the soldiers' council etc. The steps from the podium to the hall were likewise crowded full. It was made especially picturesque by a soldier with long colorful ribbons on his breast, who was at first busily running around, then sat on the edge of the podium, his legs dangling in the hall. I recognized the wild blonde youth particularly by a few scars on his neck. We had travelled together from Leipzig to Munich. He wore boots which he \?{boasted}{sich rühmte} to have forcibly taken off of an officer, he said that he had once \?{spent}{abgebüßt} in a fortress, since he "\WTF{}{eene jeklotzt}" a Lieutenant, he said to the controlling officer that his billet in Leipzig had been taken away, laughed at the request to get out and afterwards triumphantly showed me a ticket running to Breslau...this revolutionary hero -- incidentally a mature personality compared to the 17-year-olds demonstrating in Berlin for the abolition of \?{corporal punishment}{Züchtigungsrechtes} and for the right to vote for 18-year-olds!! Sub serio and taken seriously! -- Who thus particularly \?{decorated}{schmückte} the podium. Front left a lectern, front right a conference table, at which a women with pince-nez and \?{without physical charm}{ohne plastische Reize} presided. She did it matter-of-factly, without shouting and \?{too many bells}{allzu vieles Klingeln}; only sometimes, when her voice no longer cut through the pandemonium, a man spoke for her. \WTF{The mininister for social affairs, Unterleitner, also received word}{Das Wort erhielt gleich der Minister für Soziales, Unterleitner}. A thirty-something, brown-haired with a little moustache, slim, not especially distinctive.
In order to make himself understood in the enormous room, he shouted and \?{broke up the individual words}{hackte die einzelnen Worte ab}, just like Lipinski here recently in Leipzig. But he spoke warmly, compellingly, matter-of-factly. Of course, he also spoke in generalities. He defended the Independents, who are apparently more moderate in Southern Germany than they are here, \?{to the right and to the left}{nach links und rechts}. He was warm when he spoke of the women who are now \?{comin into their own}{zu ihrem Recht kommen sollten}. (Once it seemed to me that he was alluding to Sonja Lerch's sad fate). Warmer when he rejected the "lie" that the Entente did not want to negotiate with the workers' and soldiers' councils. They have again installed the soldiers' council in the palace. (Here there is wild approval, wild \?{shouts of phooey}{Pfui-Rufen} against the "lying press" etc.) However he defended our workers' and soldiers' councils against the comparison with the Russians. These would have created chaos, those would create order. He finally reached a fanatical tone when he spoke about Eisner. "Kurt Eisner is the sword of the revolution"..."I am a Social Democrat, and I stand by our brilliant leader, who has toppled not one, but all twenty-two thrones in Germany, and the path to Kurt Eisner goes only over my dead body!" Wild applause...The discussion was introduced by the question of freedom or limitation of speech. \?{Freedom of speech was adopted, but not always granted}{Man nahm Redefreiheit an. Um sie durchaus nicht immer zu gewähren}. Someone began to softly read aloud from a beautifully-worded manuscript - a counterpart to the verses of the poet at the mental workers' council - and therr was shouting and clamor until he had to give up. (There was a funny interjection: "\WTF{Have it printed and hand it out}{Druckenlassen und verteilen}!") A gaunt, gentle, hoarse man spoke in favor of thr Majority Socialists, and was drowned in the roaring anger of the crowd. "I am a socialist!" -- "Bourgeois! Bourgeois! Bourgeois! He seemed to be somehow personally known by the people \?{and to have been mistaken for someone else}{als doppeldeutig bekannt zu sein}. I saw snarling faces menacing him. To his protest that he was nonetheless a socialist came the occasional loud reply: "\WTF{You are rubbish}{A Schmarrnkibi bist}!" There were also piercing whistles. He had to back down...The strongest impression was made by a speaker during the discussion, a Doctor Levin, from the Spartakusbund (the people stress Spartákus), banned from Berlin. He is said to be from the Baltics, perhaps he is a blonde Jew. A cold, blonde, unabashed Roman beauty. Thin and young in a gray field uniform with a high collar, beardless face, large commanding gray eyes, blonde mane, \textit{not} \?{long and flowing}{langwallend}, commanding, penetrating tone, wide powerful arm-motions: "Comdades, \?{Party members}{Genossen}, Citizens and Citizenesses!!" He raged against the bloodhounds Ebert and Scheidemann in Berlin, he raged against the lies that have been used to disparage the noble Bolsheviks (\?{Oxytonon}{Oxytonon}) to us, he urges onward to the gentle Eisner government, he suspects Auer, he is not for \?{unity}{die Einigkeit} that Unterleitner preaches, he coldly brushes off all the heckling, applause and disapproval ("were you in Russia?" -- "I was born in Moscow!"...but he had to add later that he was not there during the war and subsequent revolution! -- "He is no Bavarian" -- "This is a Bavarian \?{kilt}{Rock}!" And he shakes his uniform...), he put on a big show with wild gestures and \?{a commanding stance}{befehlendem Dastehen} and cutting language. \?{Nothing makes as strong an impression as this unpleasant and theatrical one}{Keine Gestalt prägt sich so stark ein wie diese widerwärtige und schauspielerische}.
... There is constant noise in the stuffy room. There are always shouts and counter-shouts somewhere, one would think that at any moment violence is just around the corner. (Prof. Escherich, who I went back with Friday night, told me in all seriousness that blood would flow if we still had the old beer, but the war-beer is too weak.) All at once, it becomes still. Everyone looks at the side-door, where a little pushing and shoving has broken out. Someone whispers: "Eisner, Eisner is there!" \?{He speaks, breaks off}{Der gerade redet, bricht ab}, and a spontaneous horray for Eisner goes up. It is a soldiers' council; he \?{puts in with}{er setzt an mit} "Hurr...", interrupts himself, and shouts "hooray, hooray, hooray!". The ranks stormily join in. Eisner goes right past me, and as he goes by I am almost within "touching distance". A delicate, frail, tiny bent little man. Bald head, not very imposing. On the back there is dirty-gray hair. The beard is reddish, dirty gray, the severe eyes look murky-gray through glasses, nothing brilliant, nothing venerable, nothing heroic in the whole form. An ailing, \?{worn}{verbrauchter}, mediocre old man, who is at least 65 years old. According to the introduction of a book that was purchased in the hall, he was only born in 1867. He did not look very Jewish, even less German or Bavarian. Afterwards, as he was milling around the podium -- not behind the lectern -- he reminded me of the portrait of Herrn Wippchen...they say that he has to rest for a few minutes, meanwhile there is further discussion. That goes on for a while, then he steps forward. He spoke softly, but nonetheless everything was comprehensible, since everyone was reverentially silent. He was \?{ailing}{leidend}, but he was also not here the whole evening, so he could deny and renounce everything since he hadn't heard it. \WTF{This is the first joke of many, he almost always replaces pathos with with jokes, and he is always thankfully cheered}{Dies ist der erste Witz von vielen, der Witz ersetzt ihm fast immer das Pathos und wird ihm immer dankbar bejubelt...} "If I am pushed forward, I am not afraid; I am ahead of all the pushers, since I am a dreamer, passionate, a poet!!" Frenzied applause... "I speak not as minister-president, I speak as an Independent and \WTF{traitor}{Verräter}...I should request that you vote Independent, but I don't - follow your opinion - and let us be united!" Yet again wild applause. "Give me only a little time; I would like to be able to work as your minister-president for just few more days!" A voice from the gallery: "A hundred years!" Eisner bows, with further arm motions: "I will endeavor to comply with this request!" Wild approval. "\WTF{??}{Aber ich wollte ja nichts sagen und fühle, daß ich ins Reden komme}..." A \?{chattering jornalist}{feuilletonistischer Plauderer} among friends who \?{cheer him}{ihm bestimmt zujubeln}, he says what he wants, a much-loved president of a bowling league, a successful comedian who occasionally strikes a moralistic note. So finally, the whole literary invitation to come to the national theater on Sunday, where Andreas Latzko, \?{who wrote}{der Mann der} "Men in War", the best book about war besides Barbusse, will speak; \?{then everything comes to the renewal of the "soul"}{denn auf die Erneuerung der "Seelen" komme alles an}...Eisner is a mystery to me: how can this columnist, this \WTF{?}{Wippchenatur} without heroic, dictatorial gestures affect the people, and now even the Bavarians? But one thing has become certain to me: he rules in Bavaria, he is \?{rooted}{verankert} in the people, who venerate him like a god. Perhaps he will soon fall, but for now he is definitely supported by the people.

After this Eisner-evening I was completely exhausted from the heat, there was not one sentence that I hadn't seen in 100 out of 101 newspapers; nevertheless, I have \?{received}{empfangen} the greatest impression of the revolution. Unterleitner, Levin, Eisner, the masses - what a strange, enigmatic, unpredictable, illogical affair!

On the \textit{train ride} Friday night, I happened to find a place to sit. Riding in the car with me there was a Hungarian tailor with wife and child -- they were living in Berlin, he had been in the military hospital as a soldier, she went to fetch him --,  foreman from Reinhardt, an incoming Bavarian student and a gentleman, mid-forties with a small goatee and some scars, \?{enjoying himself}{vergnügt}, \?{somewhat lascivious towards}{ein bisschen lasciv gegen} the small Hungarian wife, but good-natured and affectionate. We got to talking, he was a dean of the Munich university, Ordinarius for some \WTF{Forstschädlingslehre}{?}, Esterich. He said that he knew Eisner and one of is advisors, Prof Bonn. Eisner had a totally dependent nature. He could not undertake anything without directives, earlier as an editor for Vorwärts he had not finished any articles. The rising star in Bavaria, if not its president, was Dr Heim, the genius organizer of the \?{Peasants' League}{Bauerbbundes}, who had started as a gymnasium teacher and a \?{new philologist}{Neuphilologe}. He was, despite being a member of the Centrum, not a \?{black}{Schwarzer} (as the peasantry was also no longer clerical!); there are also jews and protestants in his strongly-growing Volksbund...I asked the professor \?{whether King Ludwig was really so entirely uprooted}{was eigentlich den König Udwog so ganz entwurzelt habe}. Answer: in the last years of the war, he had been possesses by the idea of becoming "Ludwig the \?{Greater}{Mehrer}". He had wanted to have Strausberg for Bavaria, it would be awarded to him, so far as he participated in Prussian politics. Since then he had been Prussian / all-German, and the people have felt betrayed, sold by their king to the hated Berlin. A long series of diatribes against Prussians and Prussiandom and Prussian militarism and Prussian suppression of Bavaria ante bellum. \?{The Berliners also received their fat}{Auch die Berliner erhielten ihr Fett}, yet the professor said that it was not so bad in Berlin itself -- only the \?{out of towners make it hateful}{der auswärtige Berliner mache sich verhaßt} (there is some truth in this). "If we had won in 1914, Bavaria would have been lost, enslaved by Prussia!" -- Professor Escherich, who got off at Regensburg, to have a rest in the quiet "\?{ossified middle-ages}{erstarrten Mittelalter}", was the most striking \?{feature}{Gestalt} of the trip back. On the trip there I joked to that youth that I then saw again as a steward at the Eisner assembly. An impression was also made by a sturdy Landsturm man, who explained in a pathetically cocksure manner that it was just that Germany was punished, since it had "\?{sinfully indulged itself}{gesündigt}". Finally, a rather boastful, half-educated interpreter for the Russians, who came from the Ukraine, and merged republicanism with antisemitism. At the end of the trip, there was a comedic scene. A white-haired Swabian uncle had picked up two teenage girls from the boarding school in Harz, and travelled on 3 second-class tickets for nore than 200M. He stowed his luggage \WTF{im Abort}{???} and had to watch it closely. Meanwhole the unwatched teenagers befriended soldiers and smoked cigarttes. The old guy cried, grumbled, was helpless. ... I don't know whether the train was fuller on the trip there or back. Naturally in Leipzig I found 1 place to stand by climbing over all kinds of limbs. In Plauen I was able to sit on the little bench, next to a broken window. Only in Hof did I find a seat. Once I went out into the fresh air. Could not go back in through the door. Two seamen lifted me in through the window with \?{loud commands}{lautem Verladecommando}. \?{By the same method}{Auf gleichem Wege} I arrived Saturday morning on the soil of Leipzig. About three hours late both times...Before I left here the other day, the last hours at home between cafe and departure -- the cafe closed at 10:30, the train left at 1:00 --, I was at first unbearably angry about the need to travel and the dark forebodings; but then we found ourselves \?{in good spirits together}{im besten Trost zusammen}. --

The results of these days: \?{put the military situation into order}{Regelung der Militärsache}, announcment of the lectures, \textit{no} place to live, great political impressions.


\textit{Leipzig, Tuesday afternoon and later, 17/XII 18.}

Since I've been back from Munich, the whole of my work consists in this diary. Only now, towards 2, have I finished with the Munich trip. Now I want to throw myself into the Astrée, at least in order to \?{consolidate}{unter Dach zu bringen} the first three volumes; after that, it should go to the "17th and 18th centuries". All these days I was exhausted and in pieces, and often distracted; so the \WTF{Notizenkram}{???} was long stretched out. Also, Eva was and is unwell...

\missing

On Saturday afternoon to the Thomas Motet. A quidam Günther Ramin played a fugue; he is 24 years old and has already applied to be Straub's successor. I don't understand anything aboutbthe finesse. Eva said \WTF{she could do it as well, or at least she could so it}{so gut könne sie's auch oder werde sie's doch können}... One time Kopke went through a Bruckner symphony with Eva here. \?{I did not let it prevent me from writing in my diary}{Ich ließ mich im Tagebuchschreiben nicht abhalten}. --

We were often together with the Harms. Saturday evening they came to the cafe; Sunday afternoon we were amazed at the progress of their little daughter, who is now already $\frac{5}{4}$ years old (\?{torture}{eine Qual}!), and Monday afternoon we were with them in \WTF{Scherner's night duty}{Scherners Nachtdienst}, which went off very pleasantly with strong pharmaceutical schnapps. Harms occasionally apoke of Byzantinism and the asocial essence of the \textit{Berliner Tageblatt}. The Kaiser was never able to be criticized, in Ulk Frau Engel \?{idolized the Hohenzollern until tz}{bis zum tz verhimmelt}, the workers at the B.T. were poorly treated. Harms added to this yet another particular bitterness: when some M. d. R. \?{produced a fifth article out of four}{aus 4 Artikeln den fünften herstellte}, he got 100M for it, but he, Harms, only received 40M for each extra article.

On Sunday morning, the 107th regiment \?{marched in here}{zog...hier ein}. We went to the market around 11 and in the Grimmaisch Strasse we \?{could only still see the procession of the very long train}{sahen nur noch den Durchzug des sehr langen Trains}. Wretched, exhausted shaggy horses, miserable wagons, small trench mortars. Children alongside the flower-bedecked soldiers, fir branchez were thrown from the windows, one man passed \?{the travellers}{den Fahrenden} some wine in a glass. At the Rathaus banner and a special tapestry that in its time was embroiderd by the citizenry; every stitch was paid for, the most expensive was the red \?{emblem}{Wappentierzunge}. They would have probably imagined the entry very differently at the time. The regiment is marcning in \textit{armed} and had only handed their weapons over in the barracks. The Workers' and Soldiers' Council let their demand for a weaponless entry fall.

Much time lost and I will lose yet more with worries about the \?{leave stamps and goods}{Urlaubenmarken und -Waren}. It is torture. Got a ticket for a garment shop for Eva, got some of Harms's \?{bread}{Commisbrod}. And this afternoon, Christmas shopping an the teeming Althoff. Nothingbfor Eva, I don't think I can get anything at all for her, and that depresses me very much. Only for Harms and Scherner; \?{the former}{von jenen} invited us to New Years', \?{the latter}{von diesen} to Christmas, \?{and we have committed to both}{beiden sehr verpflichtet}. With Harms, we stuck to the child; it got a construction set and a jigsaw puzzle of colorful pieces. That cost us 13M. A good deal, since toys are very expensive. For Scherner we have so far only bought a nice porcelain vase for 10M, \?{but it won't stay that way}{dürfen dabei aber nicht stehenbleiben}. And nothing for Eva!

Across from the Thuringer Hof is a "\WTF{Salvage books store}{Bücherramsschhalle}", in which there seem to be all kinds of valuables to be had. Yesterday I bought six old Kladderadatsch calendars from the years 1854-1872 for 10pf each there (Scholz and Kalisch).

Eva felt so miserable yesterday that for the first time in a long time I read aloud again. I started the "Corbeaux" by Becque. But she longed for her organ, and I imagined my works with orror. I don't think that we will get very far in the Corbeaux.

Now this entry is up to date, and I can now above all get back into the Astrée.

%n, e lies
% zigber von frrundschadt