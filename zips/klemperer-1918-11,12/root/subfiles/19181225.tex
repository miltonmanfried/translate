\header{Wednesday morning, 10 o'clock, 25th XII 18. Leipzig.}

We only came home from Connewitz after 2 O'clock through clear, mild moonlight, where we celebrated Christmas with Scherner, and even then \?{before much love}{vor vieler Liebe} \?{didn't go straight to bed}{nicht gleich zur Ruhe}. We were Sch's only guests and were most warmly welcomed.
A luxuriant supper, wine, punch, later on coffee, charming trifles under the lovely Christmas tree, such as bonbons, liqueur, cigars, for Eva -- a touching thought -- the framed picture of her professor Heynsen with the fine thin skeptical face. Scherner played "silent night" and other Christmas songs on the violin with Eva's accompaniment, and we chatted until late into the night.
He spoke, as he often does, of the great suffering he experienced in his parents' house. The mother, a cold, almost criminal nature, domineering, loveless, hypocritical fanatical Catholic, the father good, totally without energy, \?{narrowly Catholic}{katholisch eng} (a professor), the sisters bigoted in their Catholicism. The mother often told and wrote to him that she would rather see him dead than an apostate from the spiritual life, she now writes similarly to a younger brother, \?{since he has a girlfriend}{weil der ein Mädel habe}...
Yesterday afternoon a \textit{Christmas Motet in the Thomas Church}. The silent night, holy night was sung by various choirs with wonderful harmonizing voices. Otherwise the choir and organ rather too monotonously whimpering con sordino. On the whole the concert sounded flat. I listened standing on the steps of the organ gallery. As I came in, there was an amazing bit of impertinence. The church was over-full, and a policeman was turning away \?{streams of people}{neu Zuströmende}. I said that I was with the organists, and since the policeman demanded legitimation, I slowly and calmly pulled the library entry pass from the inner pocket of the uniform; he studied it and let me through. This card has already blessed me with the most wicked surprises...We bought ourselves a little potted tree in the market hall, but we will probably only light it tomorrow, since we will probably be at Scherner's today from 5 o'clock on. We are to bring potatoes and some fat and eat there again, Kopke is coming as well, and there will be music. (Regarding the cultural history of these days: yesterdat Eva took the rest of our Wilna fat to Scherner's to leave it there, otherwise the jealousy of the landlady would be aroused.) Eva is getting nothing from me this time but the twenty Bach Canatas that Kopke can get rather cheaply \?{through his editorship}{durch seine Redaction}. Our presents to Scherner, the vase and the leather bag which Eva has had for years and \?{liked}{zierlich findet}, \?{went over well}{wirkten glücklicherweise gediegeb}.The Scherners are, despite all of the kindheartedness and refinement, petit bourgeois. \?{How they are now peacefully and openly (even the daughter!) indulge themselves in the joy of their inheritance}{Wie sie jetzt ruhig und offen-sachlich (auch die Tochter) sich der Freude an jedem Erbstück hingeben}. How they now bought as the loveliest addition an electric lamp with batik-work shade (the highest fashion!), and for that simple piece paid the fantasic price of 170M. How they are together with us every day, and how we nonetheless just this morning obtained their printed thank-you \?{for our condolences}{für das erwiesene Beileid}! How we had to lie to them and did so well with it, obviously well, that Kopke threw his condolence-letter in the box together with ours. (He had forgotten to write, so we \?{threw something together}{wir pauken ihn somit heraus}!) --

On Sunday afternoon, the 22nd of XII (that date is significant to me), just after finishing the third volume of Astreé, I started \?{working on the lectures}{die Arbeit am Kolleg}. It will of course be something totally superficial, it is a lie, but I think -- after initial doubts I now think it is -- that it should be extraordinarily successful. Of course the first that I saw was being able to write something new about Astrée. If I could only once again be productive in some field. That alone could \?{give me back my footing}{Mir neuen innerlichen Halt geben}, so that I could assert myself with respect to Eva again, the I wouldn't be jealous of her any longer or keep torturing her, then I would be able to more easily manage the everyday privations. And above all, I would be a man again. In all these recent times I haven't been, and Eva has suffered much because of me. Since yesterday, when I found the first points of light for the lectures, things have been a bit better.

The library is closed from 23/XII - 1/I; so I went to \textit{Becker}, and he helped me out with his magnificent library. He gave me the 17th century of \textit{Lotheissen}, a very nimble, rather flat journalistic work ("instructive", says Becker), but which will help me very much with its substantive breadth and citations, indeed it made the whole project possible. Lanson is indeed so deep and compressed that I only ever understand him when I have read the authors beimg discussed. (And I have still not read the majority of them.) I also took a very nice American anthology on the \WTF{Preciösenkreis}{???} with: \textit{La Société française au Dix-septième Siècle} by \textit{Thomas Frederick Crane}. The English introduction was mostly easy for me. If only I understood the pronunciation better. It was strange how I had to comfort the old gentleman. First he, whose household always appeared rather humble to me, complained about the financial situation. He will "never eqt his fill again", will not be able to leaving anything behind for his children. Then he \?{stirred}{rührte} me more, \?{his lectures would fail}{seine Collegien mißlängen}, he would soon have to retire, he could no longer even produce, \?{it was permanently gone}{es sei dauernd vorbei}. He often stood around for hours without thinking -- "how much they expect of you used to be rather skillful!" At that he stuck out a rolled-up tongue for a moment, like a child. He had overall something childlike about him, a little \?{manchild}{Männchen} with great gray hair, red nose, good-natured blue eyes, constant stutter. He complains about exactly what I complain about. But he is 20 years older, and has a significant life of work behind him, and is an Ordinarius. And I am nothing and have created nothing...I \?{told him he was wrong}{widersprach ihm}, praised the freshness of his lectures (recently, the Hugo-lecture), consoled him. Also spoke confidently on the future of Germany, without quite believing in it. --

In Berlin, chaos. Yesterday, the heaviest, most muddled battles around the castle. Guards against the sailors in the castle. I have the impression that part of the victorious government guards have gone over to the side of the Spartacists, or at least have become neutral. But nothing can be seen clearly. The invasion of the Entente will probably still yet come. --

% scurvis