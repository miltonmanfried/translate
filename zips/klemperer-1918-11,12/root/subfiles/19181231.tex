\header{Tuesday afternoon, 5 o'clock. Leipzig 31/XII 18.}

The papers brought news of the death of old \textit{Crusius}. Stroke, 61 years old. I had always thoight of the man as an arrogant charlatan. Then in October we met in a Greek restaurant in Munich. There he was so good-natured, \?{wittily cajoling}{Witzig zuthunlich}. But he looked sick, ragged, \?{beaten}{verelendet}. He made the same impression on me when we recently met in the faculty assembly hall. He wore a nice black coat, but was hunched over and dragged one leg...we had planned to visit Crusius in Munich. Now he is dead...also, Paul is still hardly counted among the living. And how long will Muncker hold? Entirely egotistical: I will have to seek contact in order to not be entirely alienated.

-- Eva remained sitting in the cafe; she still has an errand to do, and Kopke will accompany her. It has been arranged to meet back up again at the Merkur -- since the Scherners are there. Then later in the evening we will celebrate with Harms. \?{We will eat beforehand at Scherners}{Ev. essen wir vorher bei Scherners}; since the Thuringer Hof and many other restaurants have closed -- they fear unrest, and wild rumors are going around. It seems, sailors who had marched out of Berlin yesterday were attacked -- but who, for what reason, for what purpose?? Now, everything is confusion and darkness, everything is possible, it has never looked so chaotic in Germany.

\missing

\textit{Resumé of 1918}

The beginning: the theater season, the social life with Tovote and Frau Öhlmann lies far, far away. Then came the first stay in Munich, Easter. Then came the paper on Astrée, through that continual progress of the \?{wintry social life}{winterlichen Gesellschaftslebens}. Already at Easter the skin-disease which has made my life more difficult had started, only slowly, it reached its high point in October, on the 1st of November in Wilna perhaps saved my life, and then ended just 14 days later; still, I'll carry the scars into the new year. -- The command came to go to the front. Weeks of uncertainty followed. The call to Gent came. \?{The best luck, the port, the target}{Das höchste Glück, der Hafen, das Ziel} -- then the bitterest reversal followed. I went with Eva to Munich, went back to Leipzig and was again back in Munich on leave. We were in Urfeld, in Munich, in Füssen, in Landsberg (where we had already spent a week in Easter). The leave ran out. Now the Leipzig-Wilna-Brussels-Urfeld trip. The ceasedire, the crushing of my hopes...then I did two weeks service in Wilna. I've been here since mid-November, I was again in Munich for a few days in mid-December.

So, this year has tossed me around more than most of the rest (Munich, Landsberg, Urfeld, Füssen, Leipzig, Wilna, Berlin, Brussels), it has dropped me in the middle of the war and revolution, it has fulfilled one of my fondest wishes \?{via the Gent professorship}{durch die Genter Professur} and then smashed it.

At the start of the new year I am very poor. Financially ruined, mentally wrecked, without hope, without security in any respect, homeless, since Bavarians seems to me a hostile foreign land, and since we've never found an apartment...I am half-numb, half-disgusted with the immense upheaval, not democratic in the least. -- Eva has taken a pause in her organ study, and I am always depressed by her art, often envious and frequently alienated from her. My path has all and all gone far backwards this year.

I have not been able to write or produce since 1915. Also few results from study. Early in the year probably some Victor Hugo or some other French, I don't know anymore. Later with many breaks only the Astrée. For months, absolutely nothing. \?{I don't want to count on casting about with old French work}{Sofern ich ein Herumtasten am Altfranzösischen nicht rechnen will}.
% Schlaganfang