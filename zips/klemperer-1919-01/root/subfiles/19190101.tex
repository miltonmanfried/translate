\header{Wednesday January 1st 1919. Leipzig. 10 o'clock in the morning.}

New Year's Eve yesterday was a great disappointment. First we ate with Scherners in the "Reichsecke", a pub with guitar music on Reichsstrasse -- since the Thur. Hof was closed. It was (and remains) quiet in the city, the rumors had again been wrong, I needlessly had the little sidearm in the coat-pocket of my \?{Civil clothes}{Civilkleidung} (which only very seldom allow myself, I am already so very ashamed of the uniform. But from where are the new Civil paid, when the old is gone?). I arrived at Harms' \?{chastized and badly sated}{kasteit und schlecht gesättigt}! Is it, like Eva thinks, the fault of the \?{rather stiff free salon semicircle}{etwas steifen freien Salonhalbkreises}, which doesn't have a table (replaced my small writing desks), that has left all warmth and feeling from Harms' conversation? I recall having once been very bored there, although the two Harms \?{rather than by themselves}{überall anders als bei sich selber} are entertaining enough. \?{Or are they just lacking in material things}{Oder fehlte es allzusehr am Materiellen}? First there was tea without cake, then one hot bowl of punch which tasted very flat (sugar and alcohol replaced by cloves) and with that a tiny little piece of cake, also, the room was badly heated. So the atmosphere \?{was not lively at all}{kam denn gar keine Stimmung auf}, and the conversation dragged on, and time dragged. During the day, Eva complained for the first time in a long time about her bile, and that also made me less \?{festive}{freier}. Only it was surprising (almost petit-bourgeois, like the birthday speech of the Volks-school teacher at Scherner's) how in front of this tightly-knit group of people (Scherners, Kopke, us) towards 12 Harms gave a festive and well-prepared toast and \WTF{delivered}{schwang} a totally earnest speech. With totally general, hackneyed ideas. \?{In it he compared the spirit of the formerly-great Abfütterungen with the present typically more insubstantial entertainment}{Er vergloch dabei wohl im Geiste die einstigen großen Abfütterungen mit den jetzt üblichen mehr unkörperlichen Bewirtungen}. But (1) I have already read that at least once and (2) yesterday each of us would have much, much rather been feed more. Afterwards Harms went over to the most common \WTF{???}{Leitartiklerton}. Our Volk could not perish, and the time shall come when the heroism of these 4 years of war will yet bear fruit. To which then as suddenly as the jump from part 1 to part 2, the "Happy New Year" was added to the speech. There was cautious clinking of glasses, Frau Harms had a tear in her eye, Scherner said enthusiastically that Herr Dr Harms should have been a pastor, Kopke, delighted, earnestly said no, he wouod have had to be a lecturer, I hypocritically endorse both compliments, and Harms was honestly touched...\?{we poured lead in the kitchen}{man goß Blei in der Küche} (in prepared shapes, money bags, children, crowns), kt remained frosty. Then Kopke took a photograph. Then I would have very much liked to leave, but the guests sat according to the law of inertia. We only got home at 3. I fell upon the edibles. There were still petty quarrels. At 3:30 we went to sleep. Now, after 10:30 in the morning, I still haven't had breakfast.

Our landlady gave me two numbers of an Anti-Simplicissimus, the anti-radical satirical paper in Simplicissimus form which were sent to her by relatives in Munich: "Phosphor", published by Fredrich Freksa. Shiny drawings, also sharp humor and verse. I brought it with me to Harms, a he liked (rightly so) a poem to Hindenburg so much that he wants to publish it in the N.N. Since I couldn't leave that number there, I have just copied down the verse. -- Harms' essay "Between belief and knowledge", Türmerjahrbuch 1906, seemed to me to rather sophistically pass over the fact that belief in a scientific hypothesis and belief in god are two very different things. Belief has two meanings. Sometimes it means "I am not certain, I only believe." Other times: "I am more than certain, I believe. The actual scientist knows the hypothesis only as regards the first "I believe" = dubito, not at all the second, the "credo".

% l