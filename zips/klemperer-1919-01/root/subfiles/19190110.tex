\header{Friday evening towards 9 o'clock. 10 January 19.}

Yesterday morning from the library, \WTF{where I left my clothes}{wo ich meine Garderobe ließ}, over to the \textit{Gewandhaus}. It was the third time during my whole stay in Leipzig that I had been to a Gewandhaus concert. \WTF{It did not bring me great enjoyment}{Großen Genuß bereitete es mir nicht}. During the "tragic symphony" by Schubert I couldn't hold my thoughts together; they were in the Berlin struggles, in my studies, and overall, the music didn't seem overpowering to me. After that I was delighted by the first part of a \?{klavier concerto}{Klavierkonzertes} by Beethoven. An enormously energetic, warlike, voluntaristic theme mixed with the most tender, melancholy. Corneille meets Racine. In between technical fiddling about, but full of melody. The young pianist who was always shaking of his blonde hair, Willy Kempff, got great applause. Nikisch raised his fists high and soothed with his left hand, the little finger \WTF{???}{wollüstig gelöst abspreizend}...for the following sections I was again already too tired, and I couldn't at all follow the Leonore overture at the conclusion. I was just missing the musical understanding...also today there was yet again a fruitless debate with Eva on the \?{slow separation between us that the organ has brought about}{leise trennende Element, das die Orgel zwischen uns gebracht}...

Every free moment from the Thuringer Hof and Merkur I read on the 17th century, part here, part at the library. Yesterday La Fontaine, this morning Malherbe, in the afternoon Racine's Iphigenie. I still haven't taken any notes since Rotrou and Somaize. \?{I want to bring the century to life inside me}{Ich will das Jahrhundert nur in mir lebendig werden lassen}. Indeed I have to teach it in a very short time. Then I can only ever make notes from class to class. --

Eva was just at a klavier concert with Mitja Nikisch. I took her there and will now pick her up. I took the opportunity to go to Augustusplatz, where machine guns were said to be lined up for the demonstations announced for tomorrow (sympathy strike for Spartakus in Berlin); I found nothing but peaceful fairground booths. Overall, Leipzig is still perfectly calm. Only in Leutzsch has blood been spilled; trains with government troops heading to Berlin were stopped and disarmed. "Government troops" -- \?{how that sounds!}{wie sich das anhört}! And in Berlin, constant waxing and waning of battles, around train stations, around newspaper buildings, around the Brandenburg gate. Artillery in the streets, machine guns on roofs, pedestrians shot, flame- and mine-throwers, hand grenades...it is fantastically gruesome, and they apathetically accept it as natural, sit in concerts, in cafes, in the library...\?{The actual ability to live with it, to go along with it, is unlimited}{Das wirkliche Miterleben-, Mitschwingkönnen ist ungemein begrenzt}.

 % librarBohemian exuberance