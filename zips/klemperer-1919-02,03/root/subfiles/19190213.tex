\header{Thursday morning, 13/II 19.}

Yesterday, with a certain cheekiness only worked on Astrée, read IV to the end and took notes,though I have only worked out lecture material for up to today...the idea of a study of the State in Corneille's dramas occured to me...this morning I got the lectureship for 8 preparatory courses, 4 upper-level French courses, 4 lower-level German courses. More details on that as soon as I \?{get through orientation}{orientiert bin}. The work-load is increasing, the journalism recedes into the background. The finances are better, the strain increases. A pitiful letter from Frau Schlemmer; we sent her 50M. Eva's \?{organ situation}{Orgelsverhältnisse} is so bad that she is taking a theoretical major besides the organ, that she has to seek practice hours outside of the academy. Such unpleasantness with Hans M that we spend most of the time sitting in the cafe (where one constantly sees the recently-arrested and again-freed Dr Levien). All in all: we will spend 3000M a month, and we only have 2000M at our disposal from Georg. But the income lets us stretch out our Sebba reserves.

When possible I work in the warm and comfortable working room. Was also in the splendidly-equipped newspaper room for a moment. There the Lepizig papers are hung up, along with all the bigger German papers. The Corriere and the Osservatore are also there. \missing

\textit{Evening after 10 o'clock}. With Wetsch (there was trouble with him and the rather \?{angry}{heftigen} Eva because the piano shipment was delayed)...in the afternoon my lecture (Astrée) went over extraordinarily well; I was also pleased that the number of \?{students}{Inscribierten} increased to 29...Answer from Behren ("Schwan-Behren") Giessen, I may take over the Voltaire recension. There I now have the first \?{topical connection}{fachwissenschaftlich Verbindung}. With what newspaper incidentally? I don't know it.

After the lecture, a little while in Stephanie. Meanwhile, Eva got two tickets for the German Theater from Elena, where Einser gave his \?{Bern incident}{Berner Auftreten}. It is supposed to be in the minister's seats. Then as we arrived there in the evening, we got caught in a life-threatening crowd and couldn't get in. I asked for an usher or a guard. Answer: the guards had been crowded out, any order had disappeared long ago. On the way we heard people gossiping and hissing at one another. The mood is agitated, since E is supposed to have said in Bern that the French were right in keeping our prisoners; we had acted the same way in Belgium. Eisner claimed (according to Weckerle) that this report was a lie.

\missing

After our Eisner-visit fell through, we had supper here. Pontius told of the battle of Radonvillers, which he took part in. Later we went with him to Stephanie; now he spoke of Brest, where he was an interpreter for the peace negotiations. Embittered about the German Ludendorff system, about the privation in the ranks and the officers' feasts, about the brutality which was shown to the Russians...Pontius himself is a somewhat mysterious personality; he is rather poor; he hopes to get a post from Eisner \?{follows him closely}{schilt sehr auf ihn}. \WTF{He spoke to me of Levien, his old friend, with disgust, and they greeted eachother as friends}{Er spricht mit Abscheu von Levien, seinem Duzfreund, und begrüßt sich freundlich mit ihm}...Levien appeared with a bigger \?{corona}{?} around 10. He was apparently coming from the Eisner meeting. A youthful, extremely impertinent face. Rigid big blue eyes, no beard, long straight dark-blond hair, field uniform. -- 

 % lsiabset's dispel once and for all this fiction