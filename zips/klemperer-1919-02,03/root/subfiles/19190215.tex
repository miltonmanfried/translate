\header{Sunday evening 15/II 10 o'clock.}

\missing

This morning, in the working room: \?{read Descartes in the Windelband}{Decartes im Windelband gelesen}; there I found (comparing with the lectures of 1913) that what I admired most about Vossler's lecture was literally derived from Windelband. \?{I have fabulously soured on these philosophical lectures}{Mir wird diese philosophische Lectür fabelhaft sauer}...briefly at the cafe in the afternoon, then I toiled away here in despair to make some preparations for the afternoon. The problem is this: every week I have to give two two-hour prepatory courses in German for people
returning home from the field who are to take their abitur in the Summer, \?{who are already studying}{dabei aber schon studieren dürfen}. Today I find out that they are mostly medics, who went into the field as \?{undergraduates}{Unterprimaner} with an \?{emergency university certificate}{Not-Reifezeugnis} and who still need to take another exam. I am to have 26 students in the course, 12 of whom were there for the first time today, as I said mostly medics, also a dentist and a veterinarian, a political scientist and a theologist. I am to bring them German literature from Sturm and Drang up to Goethe's death, have them write essays and read "Räuber" and "Carlos" with them. I took some \?{outlines}{Literaturabriß} from the library, as well as \?{Schiller Minors}{?} and Brahm's, \?{but I was not to find anything worthwhile to read}{kam aber nicht dazu, etwas Ernstliches zu lesen}. Nevertheless, the first two hours went glowingly. I asked the names of the students, ensured them that I was hostile to all schoolmasterliness, developed a program. They will give a weekly talk about a classical piece or the like. At the same time, the literary-historical instruction and repetition will be tied in. So then we will discuse die Räuber, from which they will have to read a few scenes at home: finally, they will have to write a monthly essay. \WTF{Then the second hour}{Dem galt heute die zweite Stunde}. I put forward the topic: "Freiheit, die ich meine", explained the deductive and inductive methods, and inductively developed the topic to the present. I asked: Who did you vote for! -- "Centrum!" I connected that from politics to religion, morality, etc and arrived at \?{conformity to natural law}{Gesetzmäßigkeit}. Finally I developed the same topic on Schiller's drama. The hours flew by. I will likewise fill two first hours of the other German course. Other than that, I still have a 4-hour upper-level French course -- i.e. in front of people who have already taken their abitur and are only \WTF{to repeat}{nur repetieren sollen}. There I will read The Misanthrope and translate from an exercise book. My hourly plans:

Mon: 6-7 German, Tue: 4-5 French, Wed 6-7 French, Thur 5-7 French, Fri 6-7 German, Sat 5-7 German, and on top of that Mon Tue Thur Fri 3-4 readings. That is actually enough. The German lower-level course goes through the 2 semester until July, the French upper-level course only through the \WTF{War-emergency semester}{Kriegsnktsemester} until Easter. Honorarium of 220 to 250M per hour and semester. --

A letter from Frau Harms, from which it seems that the Neuesten printed my article.

Eva is completely in pieces today: she has to solve difficult theoretical problems for the academy's exam for Monday. Her piano came.

%ch meine} w