\header{Friday morning 8 o'clock 4/April 19}

We had some music at our place on Wednesday evening. Ritter, Steffl, the young, fresh Austrian Gehmacher, who came out so bravely in my lecture -- on top of that, as a newcomer the student \textit{Kaeser}, a faithful attendee, who introduced himself to me some time ago. The man, not exactly sympathetic from the look of him, rather stern and gloomy, is already 29 years old, was doing a doctorate \?{on a 14th century manuscript}{über eine Notenhandschrift des 14. Jh's} for his habilitation as a musical theorist when the war broke out, \?{lost the use of an arm as a lieutenant in the war}{ist im Kriege als Leutnant um die Beweglichkeit eines Arms gekommen} so that he can no longer play and is now \?{saddled in on Romance languages}{sattelt zur Romanistik über}. He considers the thing very earnestly and self-evidenttly, very eager, perhaps in all rather too inclined to pedantry and fanatisism; I still don't see his charachter too clearly, but would not like to have him as a bosom buddy. In Berlin, he said, the revolutionary days had almost cost him his life; his people had set upon him, "but one still has to stick by them". Across from Kaeser there are two inseparable blonde youths who are always in my lecture, always come to the French preparatory course, where they know astonishingly little but give great effort, \textit{Lenz and Merk}, harmless children. Up to now, those are my faithful students. It will be seen in May how many hearts I have won. -- I have now reached Racine. The lecture is still always excitement and exhaustion.

\textit{Vossler} came to a fable in his talk yesterday; it is broad, often precious, many times tiring, entirely and completely \WTF{beyond the common man}{unvolkstümlich}. He applies himself conpletely to his university friends and recited a Lafontaine study to them. He sent me a recension that he had written in 1905 on a Corneille book (Steinweg) and from which his masterful bur rigid standpoint emerges. \?{He has also written to Neumann, to get me recensions in his paper}{Er hat auch an Neumann geschrieben, mir in dessen Zeitschrift Rescensionen zu verschaffen}. --

Today a letter from Seebass; the "Historie véritable" will probably come into question in his "fruit bowl". Eva rcentky spoke about it too. It is only questionable whether the possibility and permission for translation can be found.

%whichnad si Am Markt stehen