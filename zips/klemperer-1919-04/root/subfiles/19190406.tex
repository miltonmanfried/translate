\header{Sunday morn. 6  April 19.}

On Friday evening I went directly from the university to a democratic assembly at the endling gate in which \textit{Quidde} spoke; I took very extensive notes. It went off very peacefully and when I got home I saw in the paper that the declaration of the Räterepublik was immanent and the Landtag would again \?{be during the day}{einmal am Tagen verhindert sei}. With Quidde the most interesting thing was how he spared the Bavarian "national feeling", the hatred against Berlin had to be fanned. The Bavarian with the s-t-pronounciation...the bourgeois are thus once again clueless. Yesterday then a day of most anxious expectation and tension. In the morning the Zentralrat met and the proclamation of the soviet republic was expected any minute. Which however has not happened. So then the general strike was threatened. And again the dreaful feeling of knowing nothing at all, of being in the dark. Suddenly -- I was in the work room -- the hurried announcement of a meeting of the broader philosophy faculty for today at 12 o'clock. They want to take a resolution against the planned \?{interventions and changes}{Eingriffe und Umkrempelungen}. Then the anxiety over the closing of the library. I quickly got myself some volumes of Victor Hufo and Lettres Persanes, and also ordered V Hugo biographies from the city library. In the evening I bet \textit{Stefl} in Stephanie (where we only rarely go, since we have gotten our own coffee); he promised me to get the books for me even if thr city library closed...Monday evening Landauer spoke on the university reforms. All of this might once again provide me with material for an article for the Leipziger N.N; I might after all not let myself be discouraged by the fact that my last reports were lost in the general strike. But naturally: this new political \?{situation}{Inanspruchnahme} throws me again out of my area of expertise: the Voltaire book has still not been sorted out. -- Yesterday afternoon there was music here: Seebass played the fiddle. Kaeser was also here and brought books on musical literature of the 12th-14th century. -- This morning was very indecent, very enjoyable, a total waste of time.

Two new fashionable words: "\textit{Tension}". Tension between Voltaire and German literature. \textit{Anchoring}. In the state of mind, in literary history. I myself have already been infected by this word. I realized that it was a fashionable word from a remark by Quiddes, "as one now likes to say". -- \WTF{We have missed the boat on the latter}{Das letzte was uns von der Flotte geblieben}.

I made a frightfully bad joke. The new slogan is: "\WTF{???}{Räte sich wer kann}".

I read "Weltkinder" by P. u. V. Margueritte aloud to the end and will make notes on it. Much good, much repulsive, some theatrics. Next comes a Zola (French).

-- Expression of the unrest yesterday: Dr Janentzki explained to me, he cancells his lectures the past week and is travelling to the countryside immediately, before the general strike blocks the tracks; there would be hunger here since the countryside would not supply the Munich soviet. -- The banks are going to be overrun by \?{people making withdraws}{Abhebenden}, the Deutsche Bank is said to only cash 200M checks. I have 350M at home and my rent is paid until May 1st, so I can wait and see for a while.

Here, all recruiting for the Freikorps and the Grenzschutz Ost is forbidden. I was sent the appeal and recruiting letter of a Berliner Bund, foolishly conservative and provocative, \WTF{which is now again not the right}{was nun auch wieder nicht das Rechte ist} -- \WTF{???}{anzi}.

\textit{Towards 7 o'clock in the evening}

The day is lost for work. From 12-2:45 a fruitless meeting in the little hall, \WTF{eating afterwards}{mit dem Essen nachexerciert}, Stephanie and a little walk on Ludwigstrasse with Eva and Ritter. (Peaceful Sunday in the loveliest Spring weather.) Started reading \textit{La bête humaine} aloud; almost fell asleep from weariness. There is to be music in the evening!

The meeting was again as pointless as all of these undertakings. Words, prattling, helplessness. It had been called suddenly, since it has become known that the Zentralrat will close the university, fire the lecturers, and found a new school. Should we strike? What will that help. Should we declare solidarity? Solidarity with whom? "In the case on an infringement of academic freedom!" But if the new state fires some, others should not serve it! After many, many verbose speeches one such solidarity resolution was adopted, which will only be published in case of emergency. About 50 people were present. The old gray stooped Grauert spoke as a Cunctator. To him it seemed that there was much time, while the new situation could come at any moment. A reform resolution from socialist students became known, which is supposed to have many supporters in the Zentralrat, on the firing of all lecturers, on the transformation into a kind of Peoples' School. I also excitedly spoke a few words, incited by Vossler: leave aside the ideal generalities and put the question bluntly: who does the Centralrat serve when it fires a bunch of professors? --

% niemals