\header{Tuesday morning 7:30 April 15.}

Cut off from the world, general strike. Levien de facto reigns over the city, the officers who are stuggling for Hoffmann are turned away at the train station. I heard about it after eating at Hans M's through Weckerle, who is now Landauer's secretary and \?{who informed me mockingly}{mich spöttisch überlegen belehrte}. The Zentralrats themselves have been scattered by the putsch and keep themselves hidden. Munich can still only be rescued from the outside. Even Weckerle says: perhaps we must mellow for now -- then we can come back in the Autumn. Victorious over all of Germany. But \textit{Spendig} says no. In Berlin, in the North of Germany there are masses of iron-solid troops behind Weimar: active officers and NCOs, Prussian peasants. They had shown what they are capable of at the last Spartakus uprising in Berlin.

Spendig, a small, pale-blonde, goateed little man, deaf and charming, is the editorial director of Athenaion. He telephoned me in the afternoon, at three he came to see me. \WTF{He came from Konstanz}{Er war von Konstanz hergekommen}, he wanted to try to escape from Munich that very same day. He wantes to get to Dresden, there was a train to Stuttgart -- he went in the evening to Stuttgart. I wanted to give him a letter to take with him for Leipzig. It was however in the rush too short and colorless. I had already gotten to Odeonsplatz to bring him the letter when I turned around and went back.

The contract has, I believe, after some negotiations on my part, become very good. Since there are many illustrations, I demanded and obtained 25 instead of 20 printed pages. The first two to be handed in on the 1st of Nov 1919, the next 5 on the 1st of May 1920, the whole on October 1, 1921. Print run of 3000. \WTF{Pay for the pages, 250M}{Honorar für den Bogen 250M}. 40M for each further thousand. Spendig wants to immediately print 5000, then I would get a sum of 8,225M. In any case, it is enormous progress compares to my earlier published work. What makes me less joyous is only that I have to treat such an enormous area in a few 100 lines; since one cannot say all too much, and yet there is much complication. Of course it will also not be such a huge undertaking. I don't believe at all that I will still be busy on 1/X/21. But I will allow myself time and will single out things from the large area for special investigation.

In the morning I wrote more in the Voltair referat. Also again read aloud from "Bête humaine".