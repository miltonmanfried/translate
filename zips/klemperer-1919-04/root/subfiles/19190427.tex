\header{Sunday morning 7 o'clock 27/4 19}

Yesterday evening once again in the Stephanie, for this first time in a long, long time. The Berlin papers from the 18th of April \?{were the newest from the Reich}{waren das Neueste aus dem Reich}. From Vienna, a paper from 4/24. We read that with the highest interest. They all wrote of the imminent fall of Munich, which is besieged by many government troops (Lettow-Vorbeck) -- and we have noticed none of this. Of course, we notice few of the horrors found outside. But yet we are besieges from without. The conviction of the imminent fall of those presently in power is widespread everywhere here. -- 

\textit{The Sarason affair}, a disagreeable counterpart to the Rabinowitz affair, will be briefly summarized after its total conclusion. Today I hope to finish off the Corneille study, which increasingly oppresses me; \?{I cannot decide to end it}{sie nahm kein Ende}, causes great difficulties and yet can't be broken off. It has become much longer than I had intended, it comes to probably over 16 pages in \?{large format}{großen Aktenformates}, 9 in my unlined small handwriting. Cinna caused me the greatest effort. Today still "Polyeucte". I read am always re-reading the piece before I write.

-\textit{Zola}- reading totally forced onto the back burner; yesterday after a long pause a few pages; finished the first lart of "Terre". Up to now more cultural history than novel, more \?{dull}{unbelebter}, unconcentrated than Bête humaine.

% til the king is born in tupelo